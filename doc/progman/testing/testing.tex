The \PSIfour\ test suite is designed to maximize code reuse and
provide testing in \$prefix before the \PSIfour\
executables have been installed. The configure script in \$PSI4 
will take all the necessary files in \$PSI4/tests
with the .in stub: Makefile.in, MakeRules.in, MakeVars.in,
and runtest.pl.in, replace variables with system specific parameters,
and copy/create the testing files and directories in \$prefix/tests.
The tests should be run in the object directory before installation.

If you have just added a new module for performing, say multireference 
coupled cluster, and you would like to add a test case to the current 
test suite, here is what you should do.  
\begin{enumerate}
\item Copy one of the existing test case directories to an 
      appropriately named directory for the new test case.

\item Create an appropriate input file for running the new module. 
      Then, if your program produced the correct data, rename
      the output files to *.ref. Follow the convention of the 
      existing test cases. Make sure you add a descriptive comment to the
      input file, stating what the calculation type is.  Use the special comment
      marker ``\%!'' to do this, so that the comment is inserted into the user's
      manual.

\item If the test case is small, add the directory name to the list
      in \$PSI4/tests/Makefile.in.  If the test is particularly tricky,
      see the psi\_start or rhf-stab test cases as an example.

\item All the testing functionality is located in the perl library
      \file{runtest.pl.in}. If you are testing for a quantity that
      is not searched for currently, then add a function to the 
      library following the format of the functions already available.
      If you have added functionality to the \PSIfour\ driver,
      make sure to update the appropriate functions in \file{runtest.pl.in}.

\item Add the location of the Makefile for the new test case
      to the configure script in \$PSI4.

\end{enumerate}

Please contact one of the authors of \PSIfour\ before making any
major changes or if you have a problem adding a new test case.
Remember, if all else fails, read the source code.

