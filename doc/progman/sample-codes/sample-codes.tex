In this chapter we demonstrate how to write code for \PSIfour\ using some simple
examples.  It can be quite daunting to develop code in an unfamiliar programming
environment, so a good starting point is to write code that requires little
knowledge of the existing code structure.  In this regard, the modular nature
of \PSIthree\ made it a perfect development platform, with each ``module''
existing as a standalone code.  With the transition to a single executable
paradigm in \PSIfour, developing new code is still a simple process; in fact
the availability of library functions to perform most tasks makes it even
easier to get started with programming in \PSIfour, as we will demonstrate.
