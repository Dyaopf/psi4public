As noted previously, we want to start from a code that's not too tightly
integrated with the \PSIfour\ code itself, so we beginwith a Makefile that will
allow us to write a standalone code that includes all requisite \PSI\
libraries.  We're going to write a small sample code that generates integrals,
which involves two just two source files.  We begin by defining a Makefile that
will include all of the \PSIfour\ libraries and header files, so that we can
take full advantage of the wide range of features implemented without having to
worry about the details of their implementation.
\newpage
\includesource{sample-codes/integrals/Makefile}{make}
Only a few lines of this makefile need to be modified to utilize it for other
programming projects; we'll concentrate on them.  On the second line, we define
the name of the executable to be generated, in this example we opt for the
unimaginative title of {\tt integrals}.  Line 4 provides the list of source files
that the project comprises; these will be detailed below.  The top source
directory for the \PSIfour\ installation and the top object directory (where
\PSIfour\ was compiled) should be provided on lines 6 and 8, respectively.
Lines 10 and 11 describe the flags needed to link in the {\tt BLAS} and {\tt
LAPACK} libraries and might need a combination of ``-Lfolder\_name'' and
``-llibrary\_name'', depending on your system's setup.  Finally, the compiler
and flags are detailed on lines 12--17.  It's a good idea to use the flags
described on line 16 for development; they speed up code compilation and
provide lots of information for standard debugging tools.  As noted in the
Makefile itself, nothing below line 17 should require modification for any
\PSIfour\ project.

The \PSIfour\ driver program provides a lot of functionality that we forgo in
writing a standalone code; this is instead emulated in the {\tt main.cc} file,
shown below.  
\includesource{sample-codes/integrals/main.cc}{C++}

All modules in \PSIfour\ must have the argument list and return type shown on
line 13.  The possible return types, defined by an enumeratable constant are
documented in {\tt psi4-dec.h}.  Notice that all of the code must live in it's
own namespace within the {\tt psi} namespace, in this case it's in the {\tt
psi::integrals} namespace.  Without this nesting, functions belonging to
different parts of the code, but having the same name, would cause conflicts.
The {\tt read\_options} function is responsible for setting up the {\tt
Options} object, which contains the list of user-provided options.  Lines
25--32 are important - these provide the list of keywords expected by the code,
their types, and their default values (if any).  This part of the code will be
inserted into the \PSIfour\ driver when the module is ready for merging with
the \PSIfour\ distribution; this process will be detailed later in the chapter.
Notice the special formats of the comments on lines 27 and 30.  These are still
valid {\tt C++} comments, but the extra hyphens inside are essential in this
context.  Whenever adding any options for any module, you must comment them as
shown - this will ensure that the keywords are automatically inserted into the
\PSIfour\ users' manual.  The {\tt main} function does a little setting up of
the \PSI\ input and output environments, before calling the module code we're
developing (on line 53) and shutting down the \PSIfour\ I/O systems.

The module we're developing is in the following source file.
\includesource{sample-codes/integrals/integrals.cc}{C++}

Given the extensive documentation within the code, we'll not describe this file
line-by-line; however, some points warrant elaboration.  Notice that the entire
module is encapsulated in the {\tt psi::integrals} namespace (lines 6 and 92).
This simple exmple has only one function body, which lives in a single source
file - if more functions and/or source files were added, these too would have
to live in the {\tt psi::integrals} namespace.  On lines 29 and 31 of {\tt
main.cc} we told the parser which keywords to expect, and provided default
values in case the user omited them from the input.  This makes retrieving
these options very clean and simple ({\it c.f.} lines 11 and 12 of {\tt
integrals.cc}).  Each \PSIfour\ module will have to initialize its own local
{\tt PSIO} and {\tt Chkpt} objects to perform I/O and to retrieve information
from previously run modules.  Notice that these objects are created within
smart pointers (see section XXX for more information) so that they are
automatically deleted when they go out of scope, thus reducing the burden on
the programmer.  Likewise, the basis sets, matrices and integral objects are
allocated using smart pointers.


