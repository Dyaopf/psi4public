\newcommand{\hiddensubsubsection}[1]{
\stepcounter{subsubsection}
\subsubsection*{\arabic{section}.\arabic{subsection}.\arabic{subsubsection}\hspace{1em}{#1}}}

\section{Python Wrappers\label{pywrappers}}

The Python foundations of the \PSIfour driver and Psithon syntax permit
many commonly performed post-processing procedures to be integrated into
the \PSIfour suite.
The wrappers below can be called in combination according to Table \ref{table:wrapintercalls}
below, so that \texttt{db(opt(cbs(energy()))} is permitted while
\texttt{db(cp(energy()))} is not.
Note that the options documented for these wrappers are placed in the command 
that calls the wrapper, not in the set globals or set locals options section.
Table not yet validated for calls with cp().

\begin{table}[!htbp]
\begin{footnotesize}
\caption{Permitted intercalls between python wrappers.} \label{table:wrapintercalls}
\parsep 10pt
\begin{center}
\begin{tabular}{lccccc}
\hline\hline
Caller & \multicolumn{5}{c}{Callee} \\
\cline{2-6}
& \texttt{cp} & \texttt{db} & \texttt{opt} & \texttt{cbs} & \texttt{energy} \\
\hline
\texttt{cp}     &    & -- & Y  & Y  & Y \\
\texttt{db}     & -- &    & Y  & Y  & Y \\
\texttt{opt}    & -- & -- &    & Y  & Y \\
\texttt{cbs}    & -- & -- & -- &    & Y \\
\texttt{energy} & -- & -- & -- & -- &   \\
\hline\hline
\end{tabular}
\end{center}
\end{footnotesize}
\end{table}

%\subsection{\texttt{counterpoise\_correct}\label{wrapcp}}
%\subsection{\texttt{database}\label{wrapdb}}
\subsection{\texttt{complete\_basis\_set}\label{wrapcbs}}

The \texttt{complete\_basis\_set()} or \texttt{cbs()} wrapper defines a
multistage energy method from combinations of basis set extrapolations 
and delta corrections and condenses the components into a minimum
number of calculations.

As represented in the equation below, a CBS energy method is defined in 
four sequential stages (scf, corl, delta, delta2) covering treatment of 
the reference total energy, the correlation energy, a delta 
correction to the correlation energy, and a second delta correction. 
Each is activated by a stage\_wfn keyword and is only allowed if all 
preceding stages are already active.

At present, \texttt{WFN} and \texttt{BASIS} options in the global and 
local options sections are not considered by this wrapper.

%$E_{tot}^{\infty} = E_{\text{tot}}^{\text{SCF}/\infty} + E_{\text{tot}}^{\text{SCF}}\rvert_{\infty} + E_{\text{corl}}^{corl\_wfn/\infty} + 
%\underbrace{\delta_{delta\_wfn\_lesser}^{delta\_wfn}}_{E_{\text{corl}}^{delta\_wfn} - E_{\text{corl}}^{delta\_wfn\_lesser}}$
%
%$E_{tot}^{\text{CBS}} = 
%   E_{\text{tot}}^{\text{SCF}}\rvert_{scf\_basis} + 
%   E_{\text{corl}}^{corl\_wfn}\rvert_{corl\_basis} + 
%   \underbrace{\delta_{delta\_wfn\_lesser}^{delta\_wfn}\rvert_{delta\_basis}}_{E_{\text{corl}}^{delta\_wfn}\rvert_{delta\_basis} - 
%       E_{\text{corl}}^{delta\_wfn\_lesser}\rvert_{delta\_basis}} +
%   \underbrace{\delta2_{delta2\_wfn\_lesser}^{delta2\_wfn}\rvert_{delta2\_basis}}_{E_{\text{corl}}^{delta2\_wfn}\rvert_{delta2\_basis} - 
%       E_{\text{corl}}^{delta2\_wfn\_lesser}\rvert_{delta2\_basis}}$

%\begin{align*}
%E_{\text{tot}}^{\text{CBS}} = & \; E_{\text{tot}}^{\text{SCF}}\rvert_{scf\_basis} \; + \\
%    & E_{\text{corl}}^{corl\_wfn}\rvert_{corl\_basis} \; + \\
%    & \left. \delta_{delta\_wfn\_lesser}^{delta\_wfn}\rvert_{delta\_basis} \; + 
%        \qquad \right \} E_{\text{corl}}^{delta\_wfn}\rvert_{delta\_basis} - E_{\text{corl}}^{delta\_wfn\_lesser}\rvert_{delta\_basis} \\
%    & \left. \delta_{delta2\_wfn\_lesser}^{delta2\_wfn}\rvert_{delta2\_basis} 
%        \qquad \right \} E_{\text{corl}}^{delta2\_wfn}\rvert_{delta2\_basis} - E_{\text{corl}}^{delta2\_wfn\_lesser}\rvert_{delta2\_basis}
%\end{align*}

%\begin{align*}
%E_{\text{tot}}^{\text{CBS}} = \;
%    & \mathcal{F}_{scf\_scheme} \left(E_{\text{tot}}^{\text{SCF}}\rvert_{scf\_basis}\right) \; + \\
%    & \mathcal{F}_{corl\_scheme} \left(E_{\text{corl}}^{corl\_wfn}\rvert_{corl\_basis}\right) \; + \\
%    & \left. \delta_{delta\_wfn\_lesser}^{delta\_wfn} \; + 
%        \qquad \right \} \mathcal{F}_{delta\_scheme} \left(E_{\text{corl}}^{delta\_wfn}\rvert_{delta\_basis}\right) - 
%        \mathcal{F}_{delta\_scheme} \left(E_{\text{corl}}^{delta\_wfn\_lesser}\rvert_{delta\_basis}\right) \\
%    & \left. \delta_{delta2\_wfn\_lesser}^{delta2\_wfn}
%        \qquad \right \} \mathcal{F}_{delta2\_scheme} \left(E_{\text{corl}}^{delta2\_wfn}\rvert_{delta2\_basis}\right) - 
%        \mathcal{F}_{delta2\_scheme} \left(E_{\text{corl}}^{delta2\_wfn\_lesser}\rvert_{delta2\_basis}\right)
%\end{align*}

%\begin{align*}
%E_{total}^{\text{CBS}} = \;
%    & \mathcal{F}_{\texttt{SCF\_SCHEME}} \left(E_{total}^{\text{SCF}}\rvert_{\texttt{SCF\_BASIS}}\right) \; + \\
%    & \mathcal{F}_{\texttt{CORL\_SCHEME}} \left(E_{corl}^{\texttt{CORL\_WFN}}\rvert_{\texttt{CORL\_BASIS}}\right) \; + \\
%    & \left. \delta_{\texttt{DELTA\_WFN\_LESSER}}^{\texttt{DELTA\_WFN}} \; + 
%        \qquad \right \} \mathcal{F}_{\texttt{DELTA\_SCHEME}} \left(E_{corl}^{\texttt{DELTA\_WFN}}\rvert_{\texttt{DELTA\_BASIS}}\right) - 
%        \mathcal{F}_{\texttt{DELTA\_SCHEME}} \left(E_{corl}^{\texttt{DELTA\_WFN\_LESSER}}\rvert_{\texttt{DELTA\_BASIS}}\right) \\
%    & \left. \delta_{\texttt{DELTA2\_WFN\_LESSER}}^{\texttt{DELTA2\_WFN}}
%        \qquad \right \} \mathcal{F}_{\texttt{DELTA2\_SCHEME}} \left(E_{corl}^{\texttt{DELTA2\_WFN}}\rvert_{\texttt{DELTA2\_BASIS}}\right) - 
%        \mathcal{F}_{\texttt{DELTA2\_SCHEME}} \left(E_{corl}^{\texttt{DELTA2\_WFN\_LESSER}}\rvert_{\texttt{DELTA2\_BASIS}}\right)
%\end{align*}

\begin{align*}
E_{total}^{\text{CBS}} = \;
    & \mathcal{F}_{\texttt{SCF\_SCHEME}} \left(E_{total,\; \text{SCF}}^{\texttt{SCF\_BASIS}}\right) \; + \\
    & \mathcal{F}_{\texttt{CORL\_SCHEME}} \left(E_{corl,\; \texttt{CORL\_WFN}}^{\texttt{CORL\_BASIS}}\right) \; + \\
    & \left. \delta_{\texttt{DELTA\_WFN\_LESSER}}^{\texttt{DELTA\_WFN}} \; + 
        \qquad \right \} \mathcal{F}_{\texttt{DELTA\_SCHEME}} \left(E_{corl,\; \texttt{DELTA\_WFN}}^{\texttt{DELTA\_BASIS}}\right) - 
        \mathcal{F}_{\texttt{DELTA\_SCHEME}} \left(E_{corl,\; \texttt{DELTA\_WFN\_LESSER}}^{\texttt{DELTA\_BASIS}}\right) \\
    & \left. \delta_{\texttt{DELTA2\_WFN\_LESSER}}^{\texttt{DELTA2\_WFN}}
        \qquad \right \} \mathcal{F}_{\texttt{DELTA2\_SCHEME}} \left(E_{corl,\; \texttt{DELTA2\_WFN}}^{\texttt{DELTA2\_BASIS}}\right) - 
        \mathcal{F}_{\texttt{DELTA2\_SCHEME}} \left(E_{corl,\; \texttt{DELTA2\_WFN\_LESSER}}^{\texttt{DELTA2\_BASIS}}\right)
\end{align*}

\hiddensubsubsection{Required Arguments}

The \texttt{cbs()} wrapper has no particular required arguments. 
However, at a minimum, the SCF basis (for a SCF wavefunction) or the wavefunction 
and basis (for a correlated wavefunction) must be specified. 
The following keyword combinations will launch some basic calculations.
\begin{Snippet}
cbs(name='mp2', corl_basis='cc-pVDZ')
cbs(scf_basis='cc-pVDZ')
cbs(corl_wfn='mp2', corl_basis='cc-pVDZ')
cbs(corl_wfn='mp2', corl_basis='cc-pV[DT]Z', corl_scheme=corl_xtpl_helgaker_2)
cbs(scf_basis='cc-pV[DTQ]Z', scf_scheme=scf_xtpl_helgaker_3)
\end{Snippet}

\hiddensubsubsection{Optional Arguments}

Default values for keywords are indicated by italics.

\begin{itemize}
\item Energy Methods

The presence of a stage\_wfn keyword is the indicator to incorporate (and check for 
stage\_basis and stage\_scheme keywords) and compute that stage in defining the CBS energy.
\begin{itemize}
\item[] \texttt{CORL\_WFN} = \texttt{`mp2'} \textbar\; \texttt{`ccsd(t)'} \textbar\; etc. \\
Indicates the energy method for which the correlation energy is to be obtained. Can also be specified with `name'.
\item[] \texttt{DELTA\_WFN} = \texttt{`ccsd'} \textbar\; \texttt{`ccsd(t)'} \textbar\; etc. \\
Indicates the (superior) energy method for which a delta correction to the correlation energy is to be obtained.
\item[] \texttt{DELTA\_WFN\_LESSER} = \texttt{\textit{`mp2'}} \textbar\; \texttt{`ccsd'} \textbar\; etc. \\
Indicates the inferior energy method for which a delta correction to the correlation energy is to be obtained.
\item[] \texttt{DELTA2\_WFN} = \texttt{`ccsd'} \textbar\; \texttt{`ccsd(t)'} \textbar\; etc. \\
Indicates the (superior) energy method for which a second delta correction to the correlation energy is to be obtained.
\item[] \texttt{DELTA2\_WFN\_LESSER} = \texttt{\textit{`mp2'}} \textbar\; \texttt{`ccsd'} \textbar\; etc. \\
Indicates the inferior energy method for which a second delta correction to the correlation energy is to be obtained.
\end{itemize}

\item Basis Sets

Currently, the basis sets set in set globals have no influence in a cbs() calculation.
\begin{itemize}
\item[] \texttt{SCF\_BASIS} = \texttt{\textit{CORL\_BASIS}} \textbar\; \texttt{`cc-pV[TQ]Z'} \textbar\; \texttt{`jun-cc-pv[tq5]z'} \textbar\; \texttt{`6-31G'} \textbar\; etc. \\
Indicates the sequence of basis sets employed for the reference energy. If any correlation method is specified, 'scf\_basis' can default to 'corl\_basis'.
\item[] \texttt{CORL\_BASIS} = \texttt{`cc-pV[TQ]Z'} \textbar\; \texttt{`jun-cc-pv[tq5]z'} \textbar\; \texttt{`6-31G'} \textbar\; etc. \\
Indicates the sequence of basis sets employed for the correlation energy.
\item[] \texttt{DELTA\_BASIS} = \texttt{`cc-pV[TQ]Z'} \textbar\; \texttt{`jun-cc-pv[tq5]z'} \textbar\; \texttt{`6-31G'} \textbar\; etc. \\
Indicates the sequence of basis sets employed for the delta correction to the correlation energy.
\item[] \texttt{DELTA2\_BASIS} = \texttt{`cc-pV[TQ]Z'} \textbar\; \texttt{`jun-cc-pv[tq5]z'} \textbar\; \texttt{`6-31G'} \textbar\; etc. \\
Indicates the sequence of basis sets employed for second delta correction to the correlation energy.
\end{itemize}

\item Schemes

Transformations of the energy through basis set extrapolation for each stage of the CBS definition. 
A complaint is generated if number of basis sets in stage\_basis does not exactly satisfy 
requirements of stage\_scheme. 
An exception is the default, \texttt{`highest\_1'}, which uses the best basis set available without error.
\begin{itemize}
\item[] \texttt{SCF\_SCHEME} = \texttt{\textit{`highest\_1'}} \textbar\; \texttt{`scf\_xtpl\_helgaker\_3'} \\
Indicates the basis set extrapolation scheme to be applied to the reference energy.
\item[] \texttt{CORL\_SCHEME} = \texttt{\textit{`highest\_1'}} \textbar\; \texttt{`corl\_xtpl\_helgaker\_2'} \\
Indicates the basis set extrapolation scheme to be applied to the correlation energy.
\item[] \texttt{DELTA\_SCHEME} = \texttt{\textit{`highest\_1'}} \textbar\; \texttt{`corl\_xtpl\_helgaker\_2'} \\
Indicates the basis set extrapolation scheme to be applied to the delta correction to the correlation energy.
\item[] \texttt{DELTA2\_SCHEME} = \texttt{\textit{`highest\_1'}} \textbar\; \texttt{`corl\_xtpl\_helgaker\_2'} \\
Indicates the basis set extrapolation scheme to be applied to the second delta correction to the correlation energy.
\end{itemize}
\end{itemize}

\hiddensubsubsection{Examples}

\begin{Snippet}
# DT extrapolated mp2 correlation energy atop a T reference 
cbs(corl_wfn='mp2',corl_basis='cc-pv[dt]z',corl_scheme=corl_xtpl_helgaker_2) 

# a DT extrapolated coupled-cluster correction atop a TQ extrapolated mp2 correlation energy atop a Q reference 
cbs(corl_wfn='mp2',corl_basis='aug-cc-pv[tq]z',corl_scheme=corl_xtpl_helgaker_2,delta_wfn='ccsd(t)',delta_basis='aug-cc-pv[dt]z',delta_scheme=corl_xtpl_helgaker_2) 

# cbs() coupled with database() 
database('mp2','BASIC',subset=['h2o','nh3'],symm='on',func=cbs,corl_basis='cc-pV[tq]z',corl_scheme=corl_xtpl_helgaker_2,delta_wfn='ccsd(t)',delta_basis='sto-3g')
\end{Snippet}

\hiddensubsubsection{Extrapolation Schemes}

\begin{itemize}

\item \texttt{corl\_xtpl\_helgaker\_2}\cite{Halkier:1998:CBS}

   \begin{itemize}
   \item[] Suitable Methods: correlated
   \item[] Suitable Basis Sets: two adjacent $\zeta$-level bases
   \item[] Governing Equation: $E_{corl}^{X} = E_{corl}^{\infty} + \beta X^{-3}$
   %# Solution equation in LaTeX:  $E_{corl}^{\infty} = \frac{E_{corl}^{X} X^3 - E_{corl}^{X-1} (X-1)^3}{X^3 - (X-1)^3}$
   %# Solution equation in LaTeX:  $\beta = \frac{E_{corl}^{X} - E_{corl}^{X-1}}{X^{-3} - (X-1)^{-3}}$
   \item[] Optional Arguments: none
   \end{itemize}

\item \texttt{scf\_xtpl\_helgaker\_2}

   \begin{itemize}
   \item[] Suitable Methods:
   \item[] Suitable Basis Sets:
   \item[] Governing Equation:
   \item[] Optional Arguments:
   \end{itemize}

\item \texttt{highest\_1}

   \begin{itemize}
   \item[] Suitable Methods: all
   \item[] Suitable Basis Sets: single basis or highest $\zeta$ among array of bases
   \item[] Governing Equation: $E_{total}^{X} = E_{total}^{X}$
   \item[] Optional Arguments: none
   \end{itemize}

\end{itemize}

%\item \texttt{scf\_xtpl\_helgaker\_2}
%
%   \begin{itemize}
%   \item[] Suitable Methods:
%   \item[] Suitable Basis Sets:
%   \item[] Governing Equation:
%   \item[] Optional Arguments:
%   \end{itemize}



