\newcommand{\hiddensubsubsection}[1]{
\stepcounter{subsubsection}
\subsubsection*{\arabic{section}.\arabic{subsection}.\arabic{subsubsection}\hspace{1em}{#1}}}

\section{Python Wrappers\label{sec:pywrappers}}
\renewcommand{\optionname}[2]{\texttt{\nameref{op-#2-#1}}}

The Python foundations of the \PSIfour driver and Psithon syntax permit
many commonly performed post-processing procedures to be integrated into
the \PSIfour suite.
Among these are automated computations of interaction energies through
\texttt{cp()}, of a model chemistry applied to a database of systems through
\texttt{db()}, and of several model chemistries together approximating greater
accuracy through \texttt{cbs()}, all of which are discussed below.
Note that the options documented for these wrappers are placed as arguments
in the command that calls the wrapper (formatted in this manual as
\pyoptionname{option = } \pyoptionval{value}),
not in the \texttt{set globals} block or with any other \texttt{set} command.

\subsection{\texttt{counterpoise\_correct}\label{sec:wrapcp}}

The \texttt{counterpoise\_correct()} or \texttt{cp()} wrapper computes
counterpoise-corrected two-body interaction energies for complexes composed of
arbitrary numbers of monomers.

\hiddensubsubsection{Required Arguments}
\begin{itemize}
\item \pyoptionname{name} = \pyoptionval{\qq scf\qq} \textbar\; \pyoptionval{\qq ccsd(t)\qq} \textbar\; etc. \\
First argument, usually unlabeled. Indicates the computational method to be applied to the database.
Most any argument to \texttt{energy()} can be supplied here; however, SAPT is not appropriate.
\end{itemize}


\subsection{\texttt{database}\label{sec:wrapdb}}

The \texttt{database()} or \texttt{db()} wrapper facilitates access to
the molecule objects and reference energies of popular chemical databases.

\hiddensubsubsection{Required Arguments}
\begin{itemize}
\item \pyoptionname{name} = \pyoptionval{\qq scf\qq} \textbar\; \pyoptionval{\qq sapt0\qq} \textbar\; \pyoptionval{\qq ccsd(t)\qq} \textbar\; etc. \\
First argument, usually unlabeled. Indicates the computational method to be applied to the database.
Any argument to \texttt{energy()} can be supplied here, though the more exotic choices have not been tested.

\item \pyoptionname{db\_name} = \textsf{\qq BASIC\qq} \textbar\; \textsf{\qq S22\qq} \textbar\; \textsf{\qq HTBH\qq} \textbar\; etc. \\

Second argument, usually unlabeled. String of the requested database name, matching the name of a
python file in \texttt{\$PSIDATADIR/lib/databases} . Consult that directory for available databases
and literature citations.
\end{itemize}


\hiddensubsubsection{Optional Arguments}
\begin{itemize}
\item \pyoptionname{mode} = \textit{\pyoptionval{\qq continuous\qq}} \textbar\; \pyoptionval{\qq sow\qq} \textbar\; \pyoptionval{\qq reap\qq} \\
Indicates whether the calculation required to complete the database are to be run in one
file (\pyoptionname{\qq continuous\qq}) or are to be farmed out in an embarrassingly 
parallel fashion (\pyoptionname{\qq sow\qq}/\pyoptionname{\qq reap\qq}).
For the latter, run an initial job with \pyoptionname{\qq sow\qq} and follow instructions in its output file.

\item \pyoptionname{cp} = \pyoptionval{\qq on\qq} \textbar\; \textit{\pyoptionval{\qq off\qq}} \\
Indicates whether counterpoise correction is employed in computing interaction energies.
Use this option and NOT the \texttt{cp()} wrapper for BSSE correction in the \texttt{database()} wrapper.
Option valid only for databases consisting of bimolecular complexes.

\item \pyoptionname{rlxd} = \pyoptionval{\qq on\qq} \textbar\; \textit{\pyoptionval{\qq off\qq}} \\
Indicates whether correction for the deformation energy is employed in computing interaction energies.
Option valid only for databases consisting of bimolecular complexes with non-frozen monomers, e.g., HBC6.

\item \pyoptionname{symm} = \textit{\pyoptionval{\qq on\qq}} \textbar\; \pyoptionval{\qq off\qq} \\
Indicates whether the native symmetry of the database molecules is employed (\pyoptionval{\qq on\qq}) or whether
it is forced to c1 symmetry (\pyoptionval{\qq off\qq}). Some computational methods (e.g., SAPT) require no symmetry,
and this will be set by the database() wrapper.

\item \pyoptionname{zpe} = \pyoptionval{\qq on\qq} \textbar\; \textit{\pyoptionval{\qq off\qq}} \\
Indicates whether zero-point-energy corrections are appended to single-point energy values. Option
valid only for certain thermochemical databases. Disabled until Hessians ready.

\item \pyoptionname{benchmark} = \textit{\pyoptionval{\qq default\qq}} \textbar\; etc. \\
Indicates whether a non-default set of reference energies, if available, are employed for the 
calculation of error statistics.

\item \pyoptionname{tabulate} = \textit{\pyoptionval{[]}} \textbar\; {\pyoptionval{[\qq scf total energy\qq, \qq mp2 correlation energy\qq]}} \textbar\; etc. \\
Indicates whether to form tables of variables other than the primary requested energy. 
Available for any \PSI\ variable.

\item \pyoptionname{subset} \\
Indicates a subset of the full database to run. This is a very flexible option and can be used in
three distinct ways, outlined below. Note that two take a string, and the last takes a list.
\begin{itemize}
\item \pyoptionname{subset} = \pyoptionval{\qq small\qq} \textbar\; \pyoptionval{\qq large\qq} \textbar\; \pyoptionval{\qq equilibrium\qq} \\
Calls predefined subsets of the requested database, either \pyoptionval{\qq small\qq}, a few
of the smallest database members, \pyoptionval{\qq large\qq}, the largest of the database members,
or \pyoptionval{\qq equilibrium\qq}, the equilibrium geometries for a database composed of
dissociation curves.
\item \pyoptionname{subset} = \textsf{\qq BzBz\_S\qq} \textbar\; \textsf{\qq ArNe\qq} \textbar\; etc. \\
For databases composed of dissociation curves, individual curves can be called by name.
Consult the database python files for available molecular systems.
\item \pyoptionname{subset} = \pyoptionval{[1,2,5]} \textbar\; \pyoptionval{[\qq 1\qq, \qq 2\qq, \qq 5\qq]} \textbar\; \textsf{[\qq BzMe-3.5\qq, \qq MeMe-5.0\qq]} \textbar\; etc. \\
Specify a list of database members to run. Consult the database python files for availables
molecular systems.
\end{itemize}
\end{itemize}

\hiddensubsubsection{Examples}
A database job requires, at a minimum, the basis set to be set in its input file. The following are valid calls to the wrapper.
\begin{Snippet}
# [1] Two-stage SCF calculation on short, equilibrium, and long helium dimer
db('scf','RGC10',cast_up='sto-3g',subset=['HeHe-0.85','HeHe-1.0','HeHe-1.5'], \
  tabulate=['scf total energy','natom'])

# [2] Counterpoise-corrected interaction energies for three complexes in S22
#     Error statistics computed wrt an old benchmark, S22A
database('dfmp2','S22',cp=1,subset=[16,17,8],benchmark='S22A')

# [3] SAPT0 on the neon dimer dissociation curve
db('sapt0',subset='NeNe',cp=0,symm=0,db_name='RGC10')

# [4] Optimize system 1 in database S22, producing tables of scf and mp2 energy
db('mp2','S22',db_func=optimize,subset=[1], \
  tabulate=['mp2 total energy','current energy'])

# [5] CCSD on the smallest systems of HTBH, a hydrogen-transfer database
database('ccsd','HTBH',subset='small', \
  tabulate=['ccsd total energy', 'mp2 total energy'])
\end{Snippet}

\hiddensubsubsection{Output}

At the beginning of a \texttt{db()} job is printed a listing of the individual system 
calculations which will be performed. The output snippet below is from the example job [1]
above. It shows each reagent required for the subset of database reactions requested.
Note that this is an un-counterpoise-corrected example, and the wrapper is smart enough
to compute only once the monomer whose energy will be subtracted from each of the three dimers.
\begin{verbatim}
                    RGC1-HeHe-0.85-dimer
                    RGC1-He-mono-unCP
                    RGC1-HeHe-1.0-dimer
                    RGC1-HeHe-1.5-dimer
\end{verbatim}

At the end of the job, the Requested Energy table is printed that gives the total
energies for the requested model chemistry for each reagent and each reaction, as
well as the stoichoimetric weights by which the reagent energies are transfromed
into the reaction energy. In this case, the dimer is $+1$ and the monomer is $-2$,
indicating the the interaction energy is computed from dimer less first monomer
less second (identical) monomer. Error statistics are computed with respect to the reference
energies stored in the database. One of these, the mean absolute deviation, is 
returned by the wrapper as an ordinary Python variable. (For databases
without a stored reference energy, e.g., BASIC, large and meaningless numbers are
printed for error.) The other two tables tabulate the \PSI variables requested
through \pyoptionname{tabulate}, in this case the total SCF energy and the number
of atoms in each reagent.

\begin{verbatim}
==> Scf Total Energy <==

-----------------------------------------------------------------------------------
         Reaction          Reaction Value              Reagent 1       Reagent 2
                                                        Value Wt        Value Wt
-----------------------------------------------------------------------------------
   RGC1-HeHe-0.85              0.00011520         -5.71020576  1  -2.85516048 -2
    RGC1-HeHe-1.0              0.00000153         -5.71031943  1  -2.85516048 -2
    RGC1-HeHe-1.5             -0.00000000         -5.71032096  1  -2.85516048 -2
-----------------------------------------------------------------------------------

==> Natom <==

-----------------------------------------------------------------------------------
         Reaction          Reaction Value              Reagent 1       Reagent 2
                                                        Value Wt        Value Wt
-----------------------------------------------------------------------------------
   RGC1-HeHe-0.85              0.00000000          2.00000000  1   1.00000000 -2
    RGC1-HeHe-1.0              0.00000000          2.00000000  1   1.00000000 -2
    RGC1-HeHe-1.5              0.00000000          2.00000000  1   1.00000000 -2
-----------------------------------------------------------------------------------

==> Requested Energy <==

-----------------------------------------------------------------------------------
         Reaction     Reaction Energy      Error       Reagent 1       Reagent 2
                         Ref     Calc [kcal/mol]          [H] Wt          [H] Wt
-----------------------------------------------------------------------------------
   RGC1-HeHe-0.85     0.0376   0.0723     0.0347  -5.71020576  1  -2.85516048 -2
    RGC1-HeHe-1.0    -0.0219   0.0010     0.0228  -5.71031943  1  -2.85516048 -2
    RGC1-HeHe-1.5    -0.0029  -0.0000     0.0029  -5.71032096  1  -2.85516048 -2
-----------------------------------------------------------------------------------
      Minimal Dev                         0.0029
      Maximal Dev                         0.0347
  Mean Signed Dev                         0.0201
Mean Absolute Dev                         0.0201
          RMS Dev                         0.0240
-----------------------------------------------------------------------------------
\end{verbatim}


\subsection{\texttt{complete\_basis\_set}\label{sec:wrapcbs}}

The \texttt{complete\_basis\_set()} or \texttt{cbs()} wrapper defines a
multistage energy method from combinations of basis set extrapolations
and delta corrections and condenses the components into a minimum
number of calculations.

As represented in the equation below, a CBS energy method is defined in
four sequential stages (scf, corl, delta, delta2) covering treatment of
the reference total energy, the correlation energy, a delta
correction to the correlation energy, and a second delta correction.
Each is activated by a stage\_wfn keyword and is only allowed if all
preceding stages are already active.

At present, the \optionname{BASIS}{MINTS} keyword in the global and
local options sections are not considered by this wrapper.

%\begin{align*}
%E_{total}^{\text{CBS}} = \;
%    & \mathcal{F}_{\pyoptionname{SCF\_SCHEME}} \left(E_{total,\; \text{SCF}}^{\pyoptionname{SCF\_BASIS}}\right) \; + \\
%    & \mathcal{F}_{\pyoptionname{CORL\_SCHEME}} \left(E_{corl,\; \pyoptionname{CORL\_WFN}}^{\pyoptionname{CORL\_BASIS}}\right) \; + \\
%    & \left. \delta_{\pyoptionname{DELTA\_WFN\_LESSER}}^{\pyoptionname{DELTA\_WFN}} \; +
%        \qquad \right \} \mathcal{F}_{\pyoptionname{DELTA\_SCHEME}} \left(E_{corl,\; \pyoptionname{DELTA\_WFN}}^{\pyoptionname{DELTA\_BASIS}}\right) -
%        \mathcal{F}_{\pyoptionname{DELTA\_SCHEME}} \left(E_{corl,\; \pyoptionname{DELTA\_WFN\_LESSER}}^{\pyoptionname{DELTA\_BASIS}}\right) \\
%    & \left. \delta_{\pyoptionname{DELTA2\_WFN\_LESSER}}^{\pyoptionname{DELTA2\_WFN}}
%        \qquad \right \} \mathcal{F}_{\pyoptionname{DELTA2\_SCHEME}} \left(E_{corl,\; \pyoptionname{DELTA2\_WFN}}^{\pyoptionname{DELTA2\_BASIS}}\right) -
%        \mathcal{F}_{\pyoptionname{DELTA2\_SCHEME}} \left(E_{corl,\; \pyoptionname{DELTA2\_WFN\_LESSER}}^{\pyoptionname{DELTA2\_BASIS}}\right)
%\end{align*}

\begin{equation*}
E_{total}^{\text{CBS}} = \mathcal{F}_{\pyoptionname{SCF\_SCHEME}} \left(E_{total,\; \text{SCF}}^{\pyoptionname{SCF\_BASIS}}\right) \; +
    \mathcal{F}_{\pyoptionname{CORL\_SCHEME}} \left(E_{corl,\; \pyoptionname{CORL\_WFN}}^{\pyoptionname{CORL\_BASIS}}\right) \; +
    \delta_{\pyoptionname{DELTA\_WFN\_LESSER}}^{\pyoptionname{DELTA\_WFN}} \; +
    \delta_{\pyoptionname{DELTA2\_WFN\_LESSER}}^{\pyoptionname{DELTA2\_WFN}}
\end{equation*}
Here, $\mathcal{F}$ is an energy or energy extrapolation scheme, and the following also hold.
\begin{equation*}
    \delta_{\pyoptionname{DELTA\_WFN\_LESSER}}^{\pyoptionname{DELTA\_WFN}} \; =
        \mathcal{F}_{\pyoptionname{DELTA\_SCHEME}} \left(E_{corl,\; \pyoptionname{DELTA\_WFN}}^{\pyoptionname{DELTA\_BASIS}}\right) -
        \mathcal{F}_{\pyoptionname{DELTA\_SCHEME}} \left(E_{corl,\; \pyoptionname{DELTA\_WFN\_LESSER}}^{\pyoptionname{DELTA\_BASIS}}\right)
\end{equation*}
\begin{equation*}
    \delta_{\pyoptionname{DELTA2\_WFN\_LESSER}}^{\pyoptionname{DELTA2\_WFN}} \; =
        \mathcal{F}_{\pyoptionname{DELTA2\_SCHEME}} \left(E_{corl,\; \pyoptionname{DELTA2\_WFN}}^{\pyoptionname{DELTA2\_BASIS}}\right) -
        \mathcal{F}_{\pyoptionname{DELTA2\_SCHEME}} \left(E_{corl,\; \pyoptionname{DELTA2\_WFN\_LESSER}}^{\pyoptionname{DELTA2\_BASIS}}\right)
\end{equation*}


A translation of this ungainly equation to example [2] below is as follows. In words, this is a double- and \mbox{triple-$\zeta$} 2-point Helgaker-extrapolated CCSD(T) coupled-cluster correlation correction appended to a triple- and \mbox{quadruple-$\zeta$} 2-point Helgaker-extrapolated MP2 correlation energy appended to a SCF/aug-cc-pVQZ reference energy.
\begin{equation*}
E_{total}^{\text{CBS}} = \mathcal{F}_{\text{highest\_1}} \left(E_{total,\; \text{SCF}}^{\text{aug-cc-pVQZ}}\right) \; +
    \mathcal{F}_{\text{corl\_xtpl\_helgaker\_2}} \left(E_{corl,\; \text{MP2}}^{\text{aug-cc-pV[TQ]Z}}\right) \; +
    \delta_{\text{MP2}}^{\text{CCSD(T)}}
\end{equation*}
\begin{equation*}
    \delta_{\text{MP2}}^{\text{CCSD(T)}} \; =
        \mathcal{F}_{\text{corl\_xtpl\_helgaker\_2}} \left(E_{corl,\; \text{CCSD(T)}}^{\text{aug-cc-pV[DT]Z}}\right) -
        \mathcal{F}_{\text{corl\_xtpl\_helgaker\_2}} \left(E_{corl,\; \text{MP2}}^{\text{aug-cc-pV[DT]Z}}\right)
\end{equation*}

\hiddensubsubsection{Required Arguments}

All wrappers require an (optionally) unnamed first argument \pyoptionname{name}
that specifies the quantum chemical method. This isn't exactly suitable for \texttt{cbs()},
which deals with multiple levels of theory through the \pyoptionname{stage\_wfn} keywords
below, so this first argument can be used in place of \pyoptionname{CORL\_WFN} for
correlated calculations or with value \pyoptionval{\qq{scf}\qq} otherwise. Specifying
\pyoptionname{CORL\_WFN} outright renders the first argument defunct. To run the
\texttt{cbs()} wrapper, at a minimum, the SCF basis (for a SCF wavefunction) or the wavefunction
and basis (for a correlated wavefunction) must be specified.
The following keyword combinations will launch some basic calculations.
\begin{Snippet}
cbs(name='mp2', corl_basis='cc-pVDZ')
cbs('scf', scf_basis='cc-pVDZ')
cbs('mp2', corl_basis='cc-pVDZ')
cbs('mp2', corl_basis='cc-pV[DT]Z', corl_scheme=corl_xtpl_helgaker_2)
cbs('scf', scf_basis='cc-pV[DTQ]Z', scf_scheme=scf_xtpl_helgaker_3)
\end{Snippet}

\hiddensubsubsection{Optional Arguments}

Default values for keywords are indicated by italics.

\begin{itemize}
\item Energy Methods

The presence of a \pyoptionname{stage\_wfn} keyword is the indicator to incorporate (and check for
\pyoptionname{stage\_basis} and \pyoptionname{stage\_scheme} keywords) and compute that stage in defining the CBS energy.
\begin{itemize}
\item[] \pyoptionname{CORL\_WFN} = \pyoptionval{\qq{mp2}\qq} \textbar\; \pyoptionval{\qq{ccsd(t)\qq}} \textbar\; etc. \\
Indicates the energy method for which the correlation energy is to be obtained. Can also be specified with \pyoptionname{name}
or as the unlabeled first argument to the wrapper.
\item[] \pyoptionname{DELTA\_WFN} = \pyoptionval{\qq{ccsd}\qq} \textbar\; \pyoptionval{\qq{ccsd(t)}\qq} \textbar\; etc. \\
Indicates the (superior) energy method for which a delta correction to the correlation energy is to be obtained.
\item[] \pyoptionname{DELTA\_WFN\_LESSER} = \textit{\pyoptionval{\qq{mp2}\qq}} \textbar\; \pyoptionval{\qq{ccsd}\qq} \textbar\; etc. \\
Indicates the inferior energy method for which a delta correction to the correlation energy is to be obtained.
\item[] \pyoptionname{DELTA2\_WFN} = \pyoptionval{\qq{ccsd}\qq} \textbar\; \pyoptionval{\qq{ccsd(t)\qq}} \textbar\; etc. \\
Indicates the (superior) energy method for which a second delta correction to the correlation energy is to be obtained.
\item[] \pyoptionname{DELTA2\_WFN\_LESSER} = \textit{\pyoptionval{\qq{mp2}\qq}} \textbar\; \pyoptionval{\qq{ccsd}\qq} \textbar\; etc. \\
Indicates the inferior energy method for which a second delta correction to the correlation energy is to be obtained.
\end{itemize}

\item Basis Sets

Currently, the basis sets set in set globals have no influence in a \texttt{cbs()} calculation.
\begin{itemize}
\item[] \textbf{\pyoptionname{SCF\_BASIS}} = \textit{\pyoptionname{CORL\_BASIS}} \textbar\; \pyoptionval{\qq{cc-pV[TQ]Z}\qq} \textbar\; \pyoptionval{\qq{jun-cc-pv[tq5]z}\qq} \textbar\; \pyoptionval{\qq{6-31G}\qq} \textbar\; etc. \\
Indicates the sequence of basis sets employed for the reference energy. If any correlation method is specified, \pyoptionname{scf\_basis} can default to \pyoptionname{corl\_basis}.
\item[] \pyoptionname{CORL\_BASIS} = \pyoptionval{\qq{cc-pV[TQ]Z}\qq} \textbar\; \pyoptionval{\qq{jun-cc-pv[tq5]z}\qq} \textbar\; \pyoptionval{\qq{6-31G}\qq} \textbar\; etc. \\
Indicates the sequence of basis sets employed for the correlation energy.
\item[] \pyoptionname{DELTA\_BASIS} = \pyoptionval{\qq{cc-pV[TQ]Z}\qq} \textbar\; \pyoptionval{\qq{jun-cc-pv[tq5]z}\qq} \textbar\; \pyoptionval{\qq{6-31G}\qq} \textbar\; etc. \\
Indicates the sequence of basis sets employed for the delta correction to the correlation energy.
\item[] \pyoptionname{DELTA2\_BASIS} = \pyoptionval{\qq{cc-pV[TQ]Z}\qq} \textbar\; \pyoptionval{\qq{jun-cc-pv[tq5]z}\qq} \textbar\; \pyoptionval{\qq{6-31G}\qq} \textbar\; etc. \\
Indicates the sequence of basis sets employed for second delta correction to the correlation energy.
\end{itemize}

\item Schemes

Transformations of the energy through basis set extrapolation for each stage of the CBS definition.
A complaint is generated if number of basis sets in \pyoptionname{stage\_basis} does not exactly satisfy
requirements of \pyoptionname{stage\_scheme}.
An exception is the default, \pyoptionval{\qq{highest\_1}\qq}, which uses the best basis set available.
\begin{itemize}
\item[] \pyoptionname{SCF\_SCHEME} = \textit{\pyoptionval{\qq{highest\_1}\qq}} \textbar\; \pyoptionval{\qq{scf\_xtpl\_helgaker\_3}\qq} \\
Indicates the basis set extrapolation scheme to be applied to the reference energy.
\item[] \pyoptionname{CORL\_SCHEME} = \textit{\pyoptionval{\qq{highest\_1}\qq}} \textbar\; \pyoptionval{\qq{corl\_xtpl\_helgaker\_2}\qq} \\
Indicates the basis set extrapolation scheme to be applied to the correlation energy.
\item[] \pyoptionname{DELTA\_SCHEME} = \textit{\pyoptionval{\qq{highest\_1}\qq}} \textbar\; \pyoptionval{\qq{corl\_xtpl\_helgaker\_2}\qq} \\
Indicates the basis set extrapolation scheme to be applied to the delta correction to the correlation energy.
\item[] \pyoptionname{DELTA2\_SCHEME} = \textit{\pyoptionval{\qq{highest\_1}\qq}} \textbar\; \pyoptionval{\qq{corl\_xtpl\_helgaker\_2}\qq} \\
Indicates the basis set extrapolation scheme to be applied to the second delta correction to the correlation energy.
\end{itemize}
\end{itemize}

\hiddensubsubsection{Examples}

\begin{Snippet}
# [1] DT-zeta extrapolated mp2 correlation energy atop a T-zeta reference
cbs('mp2', corl_basis='cc-pv[dt]z', corl_scheme=corl_xtpl_helgaker_2)

# [2] a DT-zeta extrapolated coupled-cluster correction atop a TQ-zeta 
#     extrapolated mp2 correlation energy atop a Q-zeta reference
cbs('mp2', corl_basis='aug-cc-pv[tq]z', corl_scheme=corl_xtpl_helgaker_2, \
  delta_wfn='ccsd(t)', delta_basis='aug-cc-pv[dt]z', delta_scheme=corl_xtpl_helgaker_2)

# [3] a D-zeta ccsd(t) correction atop a DT-zeta extrapolated ccsd cluster correction 
#     atop a TQ-zeta extrapolated mp2 correlation energy atop a Q-zeta reference
cbs('mp2', corl_basis='aug-cc-pv[tq]z', corl_scheme=corl_xtpl_helgaker_2, \
  delta_wfn='ccsd', delta_basis='aug-cc-pv[dt]z', delta_scheme=corl_xtpl_helgaker_2, \
  delta2_wfn='ccsd(t)', delta2_wfn_lesser='ccsd', delta2_basis='aug-cc-pvdz')

# [4] cbs() coupled with database()
database('mp2', 'BASIC', subset=['h2o','nh3'], symm='on', func=cbs, \
  corl_basis='cc-pV[tq]z', corl_scheme=corl_xtpl_helgaker_2, \
  delta_wfn='ccsd(t)', delta_basis='sto-3g')
\end{Snippet}

\hiddensubsubsection{Output}

At the beginning of a \texttt{cbs()} job is printed a listing of the individual energy
calculations which will be performed. The output snippet below is from the example job [2]
above. It shows first each model chemistry
needed to compute the aggregate model chemistry requested through \texttt{cbs()}. Then,
since, for example, an \texttt{energy(\qq{ccsd(t)}\qq)} yields CCSD(T), CCSD, MP2, and SCF
energy values, the wrapper condenses this task into the second list of minimum number of
calculations which will actually be run.

\begin{verbatim}
    Naive listing of computations required.
            scf / aug-cc-pvqz              for  SCF TOTAL ENERGY
            mp2 / aug-cc-pvtz              for  MP2 CORRELATION ENERGY
            mp2 / aug-cc-pvqz              for  MP2 CORRELATION ENERGY
        ccsd(t) / aug-cc-pvdz              for  CCSD(T) CORRELATION ENERGY
        ccsd(t) / aug-cc-pvtz              for  CCSD(T) CORRELATION ENERGY
            mp2 / aug-cc-pvdz              for  MP2 CORRELATION ENERGY
            mp2 / aug-cc-pvtz              for  MP2 CORRELATION ENERGY

    Enlightened listing of computations required.
            mp2 / aug-cc-pvqz              for  MP2 CORRELATION ENERGY
        ccsd(t) / aug-cc-pvdz              for  CCSD(T) CORRELATION ENERGY
        ccsd(t) / aug-cc-pvtz              for  CCSD(T) CORRELATION ENERGY
\end{verbatim}

At the end of a \texttt{cbs()} job is printed a summary section like the one below. First,
in the components section, are listed the results for each model chemistry available, whether
required for the cbs job (*) or not. Next, in the stages section, are listed the results for
each extrapolation. The energies of this section must be dotted with the weightings in column Wt
to get the total cbs energy. Finally, in the CBS section, are listed the results for each stage
of the cbs procedure. The stage energies of this section sum outright to the total cbs energy.

\begin{verbatim}
==> Components <==

----------------------------------------------------------------------------------
               Method / Basis            Rqd   Energy [H]   Variable
----------------------------------------------------------------------------------
                  scf / aug-cc-pvqz        *  -1.11916375   SCF TOTAL ENERGY
                  mp2 / aug-cc-pvqz        *  -0.03407997   MP2 CORRELATION ENERGY
                  scf / aug-cc-pvdz           -1.11662884   SCF TOTAL ENERGY
                  mp2 / aug-cc-pvdz        *  -0.02881480   MP2 CORRELATION ENERGY
              ccsd(t) / aug-cc-pvdz        *  -0.03893812   CCSD(T) CORRELATION ENERGY
                 ccsd / aug-cc-pvdz           -0.03893812   CCSD CORRELATION ENERGY
                  scf / aug-cc-pvtz           -1.11881134   SCF TOTAL ENERGY
                  mp2 / aug-cc-pvtz        *  -0.03288936   MP2 CORRELATION ENERGY
              ccsd(t) / aug-cc-pvtz        *  -0.04201004   CCSD(T) CORRELATION ENERGY
                 ccsd / aug-cc-pvtz           -0.04201004   CCSD CORRELATION ENERGY
----------------------------------------------------------------------------------

==> Stages <==

----------------------------------------------------------------------------------
 Stage         Method / Basis             Wt   Energy [H]   Scheme
----------------------------------------------------------------------------------
   scf            scf / aug-cc-pvqz        1  -1.11916375   highest_1
  corl            mp2 / aug-cc-pv[tq]z     1  -0.03494879   corl_xtpl_helgaker_2
 delta        ccsd(t) / aug-cc-pv[dt]z     1  -0.04330347   corl_xtpl_helgaker_2
 delta            mp2 / aug-cc-pv[dt]z    -1  -0.03460497   corl_xtpl_helgaker_2
----------------------------------------------------------------------------------

==> CBS <==

----------------------------------------------------------------------------------
 Stage         Method / Basis                  Energy [H]   Scheme
----------------------------------------------------------------------------------
   scf            scf / aug-cc-pvqz           -1.11916375   highest_1
  corl            mp2 / aug-cc-pv[tq]z        -0.03494879   corl_xtpl_helgaker_2
 delta  ccsd(t) - mp2 / aug-cc-pv[dt]z        -0.00869851   corl_xtpl_helgaker_2
 total            CBS                         -1.16281105
----------------------------------------------------------------------------------
\end{verbatim}

\hiddensubsubsection{Extrapolation Schemes}

\begin{itemize}

\item \pyoptionval{highest\_1}

   \begin{itemize}
   \item[] Suitable Methods: all
   \item[] Suitable Basis Sets: single basis or highest $\zeta$ among array of bases
   \item[] Governing Equation: $E_{total}^{X} = E_{total}^{X}$
   \item[] Optional Arguments: none
   \end{itemize}

\item \pyoptionval{scf\_xtpl\_helgaker\_3}

   \begin{itemize}
   \item[] Suitable Methods: reference
   \item[] Suitable Basis Sets: three adjacent $\zeta$-level bases
   \item[] Governing Equation: $E_{total}^{X} = E_{total}^{\infty} + \beta e^{-\alpha X}$
   \item[] Optional Arguments: none
   \end{itemize}

\item \pyoptionval{scf\_xtpl\_helgaker\_2}

   \begin{itemize}
   \item[] Suitable Methods: reference
   \item[] Suitable Basis Sets: two adjacent $\zeta$-level bases
   \item[] Governing Equation: $E_{total}^{X} = E_{total}^{\infty} + \beta e^{-\alpha X}$
   \item[] Optional Arguments: none (someday will be allowed to set $\alpha$ through argument)
   \end{itemize}

\item \pyoptionval{corl\_xtpl\_helgaker\_2}\cite{Halkier:1998:CBS}

   \begin{itemize}
   \item[] Suitable Methods: correlated
   \item[] Suitable Basis Sets: two adjacent $\zeta$-level bases
   \item[] Governing Equation: $E_{corl}^{X} = E_{corl}^{\infty} + \beta X^{-3}$
   \item[] Optional Arguments: none
   \end{itemize}

\end{itemize}

%\item \texttt{scf\_xtpl\_helgaker\_2}
%
%   \begin{itemize}
%   \item[] Suitable Methods:
%   \item[] Suitable Basis Sets:
%   \item[] Governing Equation:
%   \item[] Optional Arguments:
%   \end{itemize}


\subsection{Compound Wrapper Calls}\label{sec:wrapcompound}

The wrappers below can be called in combination according to Table \ref{table:wrapintercalls}
below, so that db(opt(cbs(energy())) is permitted while
db(cp(energy())) is not. Note also that the command
db(opt(cbs(energy())) is actually expressed
as \texttt{db($\dots$, db\_func=opt, opt\_func=cbs)}. The perhaps expected final
argument of \texttt{cbs\_func=energy} is not necessary since the \texttt{energy()}
is always the function called by default. Also, the outermost internal function call
(\texttt{db\_func} above) can be called as just \texttt{func}. Several examples of intercalls
between wrappers can be found in sample input \file{pywrap\_all}.
Table not yet validated for calls with cp().

\begin{table}[!htbp]
\begin{footnotesize}
\caption{Permitted intercalls between python wrappers.} \label{table:wrapintercalls}
\parsep 10pt
\begin{center}
\begin{tabular}{lccccc}
\hline\hline
Caller & \multicolumn{5}{c}{Callee} \\
\cline{2-6}
& \texttt{cp} & \texttt{db} & \texttt{opt} & \texttt{cbs} & \texttt{energy} \\
\hline
\texttt{cp}     &    & -- & Y  & Y  & Y \\
\texttt{db}     & -- &    & Y  & Y  & Y \\
\texttt{opt}    & -- & -- &    & Y  & Y \\
\texttt{cbs}    & -- & -- & -- &    & Y \\
\texttt{energy} & -- & -- & -- & -- &   \\
\hline\hline
\end{tabular}
\end{center}
\end{footnotesize}
\end{table}

A few miscellaneous notes.
documented in the Appendix \ref{kw-DFMP2}). Some basic recommendations are included below:
\begin{itemize}
\item It is very easy to make a database from a collection of xyz files using the script
\texttt{\$PSIDATADIR/databases/ixyz2database.pl}. Instructions are available at 
\htmladdnormallink{http://sirius.chem.vt.edu/trac/wiki/WrapperDatabase}{http://sirius.chem.vt.edu/trac/wiki/WrapperDatabase}.
\item All python keyword arguments are passed along to each function traversed in the
python driver, so don't be concerned about separating them, grouping them or
designating them for a nested job, e.g., db(opt(energy())). Where the same keyword is
used my multiple python wrappers, the identity of the wrapper is prefixed to the keyword
name, e.g., \pyoptionname{db\_mode} and \pyoptionname{opt\_mode}.
\item Boolean keywords can be specified with any of \pyoptionval{\qq yes\qq}, 
\pyoptionval{\qq true\qq}, \pyoptionval{\qq on\qq}, \pyoptionval{1}, \pyoptionval{\qq no\qq}, 
\pyoptionval{\qq false\qq}, \pyoptionval{\qq off\qq}, \pyoptionval{0} .
\item Similarly, the derivative level for \texttt{optimize()} and \texttt{frequencies()} wrappers
can be specified with any of \pyoptionval{0}, \pyoptionval{\qq none\qq}, \pyoptionval{\qq energy\qq},
\pyoptionval{1}, \pyoptionval{\qq first\qq}, \pyoptionval{\qq gradient\qq},
\pyoptionval{2}, \pyoptionval{\qq second\qq}, \pyoptionval{\qq hessian\qq}.
\end{itemize}

