\section{Basis Sets}

Basis sets in \PSIfour are Gaussian functions (not Slater-type functions or plane waves), 
all-electron [no effective core potentials (ECPs)], of segmented contraction (not general contraction), 
and of Gaussian94 format (for ease of export from \htmladdnormallink{EMSL}{https://bse.pnl.gov/bse/portal}). 
Both spherical harmonic (5D/7F) and Cartesian (6D/10F) Gaussian functions are supported, 
but their mixtures are not, neither within a basis set (\textit{e.g.}, 6D/7F) nor within a calculation
(\textit{e.g.}, cartesian for the orbital basis and spherical for the fitting basis).
For built-in basis sets, the correct spherical/cartesian value is set internally from the orbital basis.


\subsection{Built-in Basis Sets}
Tables \ref{table:basisPopleOrbital} (Pople), \ref{table:basisDunningOrbital}--\ref{table:basisDunningDK} (Dunning), 
and \ref{table:basisOther} (other) summarize the orbital basis sets available in \PSIfour. 
These tables are arranged so that columns indicate degree of augmentation by diffuse functions 
(generally necessary for anions, excited states, and noncovalent interactions)
and DTQ56 indicate the X$\;=\zeta$ levels available.
Several intermediate levels of diffuse space between the customary non-augmented and augmented versions have been
supplied for each basis set, including heavy-augmented and Truhlar's\cite{Papajak:2011:10} calendar truncations 
described in Table \ref{table:basisMonths}.
Fitting bases JKFIT (Table \ref{table:basisDunningJKFIT}), RI (Table \ref{table:basisDunningMP2FIT}), and
DUAL (Table \ref{table:basisDunningDUAL}) are available for methods incorporating density-fitting or
dual-basis approximations.
Citations for basis sets can be found in their definition files at psi4/lib/basis in the source.
For basis set availability by element and the default value for keyword \texttt{PUREAM}, consult Appendix \ref{basisElement}.

%Methods where density-fitting or dual-basis approximations are employed use the appropriate stub-JKFIT
%(Table \ref{table:basisDunningJKFIT}), stub-RI (Table \ref{table:basisDunningMP2FIT}), or
%stub-DUAL (Table \ref{table:basisDunningDUAL}) fitting bases if they are defined for orbital basis stub.
%For basis set availability by element and the default value for keyword \texttt{PUREAM}, consult Appendix \ref{basisElement}.


\begin{table}[!htbp]
\begin{footnotesize}
\caption{Summary of Pople-style orbital basis sets available in \PSIfour.} \label{table:basisPopleOrbital}
\parsep 10pt
\begin{center}
\begin{tabular}{llllll}
\hline\hline
\multicolumn{2}{c}{no diffuse} & \multicolumn{2}{c}{heavy-augmented} & \multicolumn{2}{c}{augmented} \\
Basis Set & [Alias] & Basis Set & [Alias] & Basis Set & [Alias] \\
\hline
STO-3G           &            &                   &             &                    &              \\
3-21G            &            &                   &             &                    &              \\

6-31G            &            & 6-31+G            &             & 6-31++G            &              \\
6-31G(d)         & [6-31G*]   & 6-31+G(d)         & [6-31+G*]   & 6-31++G(d)         & [6-31++G*]   \\
6-31G(d\_p)      & [6-31G**]  & 6-31+G(d\_p)      & [6-31+G**]  & 6-31++G(d\_p)      & [6-31++G**]  \\

6-311G           &            & 6-311+G           &             & 6-311++G           &              \\
6-311G(d)        & [6-311G*]  & 6-311+G(d)        & [6-311+G*]  & 6-311++G(d)        & [6-311++G*]  \\
6-311G(d\_p)     & [6-311G**] & 6-311+G(d\_p)     & [6-311+G**] & 6-311++G(d\_p)     & [6-311++G**] \\
6-311G(2d)       &            & 6-311+G(2d)       &             & 6-311++G(2d)       &              \\
6-311G(2d\_p)    &            & 6-311+G(2d\_p)    &             & 6-311++G(2d\_p)    &              \\
6-311G(2d\_2p)   &            & 6-311+G(2d\_2p)   &             & 6-311++G(2d\_2p)   &              \\
6-311G(2df)      &            & 6-311+G(2df)      &             & 6-311++G(2df)      &              \\
6-311G(2df\_p)   &            & 6-311+G(2df\_p)   &             & 6-311++G(2df\_p)   &              \\
6-311G(2df\_2p)  &            & 6-311+G(2df\_2p)  &             & 6-311++G(2df\_2p)  &              \\
6-311G(2df\_2pd) &            & 6-311+G(2df\_2pd) &             & 6-311++G(2df\_2pd) &              \\
6-311G(3df)      &            & 6-311+G(3df)      &             & 6-311++G(3df)      &              \\
6-311G(3df\_p)   &            & 6-311+G(3df\_p)   &             & 6-311++G(3df\_p)   &              \\
6-311G(3df\_2p)  &            & 6-311+G(3df\_2p)  &             & 6-311++G(3df\_2p)  &              \\
6-311G(3df\_2pd) &            & 6-311+G(3df\_2pd) &             & 6-311++G(3df\_2pd) &              \\
6-311G(3df\_3pd) &            & 6-311+G(3df\_3pd) &             & 6-311++G(3df\_3pd) &              \\
\hline\hline
\end{tabular}
\end{center}
Note: Absolutely no commas are allowed in basis set specification. Use the underscore character instead.
\end{footnotesize}
\end{table}


\begin{table}[!htbp]
\begin{footnotesize}
\caption{Levels of truncation for diffuse functions in standard basis sets. \label{table:basisMonths}}
\parsep 10pt
\begin{center}
\begin{tabular}{rllrrr}
\hline\hline
\multicolumn{1}{c}{Augmentation Level} &
\multicolumn{2}{c}{Angular Momenta in the Diffuse Space$^a$} &
\multicolumn{3}{c}{Valid Basis Sets} \\
\cline{2-3}\cline{4-6}
&
\multicolumn{1}{c}{Li--Kr Main Group} &
\multicolumn{1}{c}{H \& He} &
\multicolumn{1}{c}{D$\zeta$} &
\multicolumn{1}{c}{T$\zeta$} &
\multicolumn{1}{c}{Q$\zeta$} \\
\hline
aug-cc-pVXZ             & s, p, $\cdots\,$, $\ell_{max}-2$, $\ell_{max}-1$, $\ell_{max}$ & s, p, $\cdots\,$, $\ell_{max}-1$ &  aDZ &  aTZ &  aQZ \\
$^b$heavy-aug-cc-pVXZ   & s, p, $\cdots\,$, $\ell_{max}-2$, $\ell_{max}-1$, $\ell_{max}$ &                                  & haDZ & haTZ & haQZ \\
jun-cc-pVXZ             & s, p, $\cdots\,$, $\ell_{max}-2$, $\ell_{max}-1$               &                                  & jaDZ & jaTZ & jaQZ \\
may-cc-pVXZ             & s, p, $\cdots\,$, $\ell_{max}-2$                               &                                  &      & maTZ & maQZ \\
$\cdots$                & s, p                                                           &                                  &      &      & aaQZ \\
cc-pVXZ                 &                                                                &                                  &   DZ &   TZ &   QZ \\
\hline\hline
\end{tabular}
\end{center}
$^a\quad$ \mbox{D$\zeta$} has $\ell_{max}=2$ or d. \mbox{T$\zeta$} has $\ell_{max}=3$ or f. \mbox{Q$\zeta$} has $\ell_{max}=4$ or g, \textit{etc}. \\
$^b\quad$ The heavy-aug-cc-stub and jul-cc-stub basis sets are identical.
\end{footnotesize}
\end{table}


\begin{table}[!htbp]
\begin{footnotesize}
\caption{Summary of Dunning orbital basis sets available in \PSIfour.} \label{table:basisDunningOrbital}
\parsep 10pt
\begin{center}
\begin{tabular}{llllllllll} 
\hline\hline
Basis Set            & no diffuse & feb & mar & apr & may & jun & heavy-aug$^a$ & aug & d-aug \\
\hline
cc-pVXZ              & DTQ56 & --{}--{}--{}--6 & --{}--{}--56 & --{}--Q56 & --TQ56 & DTQ56 & DTQ56 & DTQ56 & DTQ56 \\
cc-pV(X+d)Z          & DTQ56 & --{}--{}--{}--6 & --{}--{}--56 & --{}--Q56 & --TQ56 & DTQ56 & DTQ56 & DTQ56 & DTQ56 \\
cc-pCVXZ             & DTQ56 & --{}--{}--{}--6 & --{}--{}--56 & --{}--Q56 & --TQ56 & DTQ56 & DTQ56 & DTQ56 & DTQ56 \\
cc-pCV(X+d)Z         & DTQ56 & --{}--{}--{}--6 & --{}--{}--56 & --{}--Q56 & --TQ56 & DTQ56 & DTQ56 & DTQ56 & DTQ56 \\
cc-pwCVXZ            & DTQ5  &                 & --{}--{}--5  & --{}--Q5  & --TQ5  & DTQ5  & DTQ5  & DTQ5  & DTQ5  \\
cc-pwCV(X+d)Z        & DTQ5  &                 & --{}--{}--5  & --{}--Q5  & --TQ5  & DTQ5  & DTQ5  & DTQ5  & DTQ5  \\
\hline\hline
\end{tabular}
\end{center}
$^a\quad$ The heavy-aug-cc-stub and jul-cc-stub basis sets are identical.
\end{footnotesize}
\end{table}


\begin{table}[!htbp]
\begin{footnotesize}
\caption{Summary of Dunning Douglas-Kroll basis sets available in \PSIfour.} \label{table:basisDunningDK}
\parsep 10pt
\begin{center}
\begin{tabular}{llllllllll}
\hline\hline
Basis Set            & no diffuse & feb & mar & apr & may & jun & heavy-aug$^a$ & aug & d-aug \\
\hline
cc-pVXZ-DK           & DTQ5  &  &  &  &  &  & DTQ5  & DTQ5  &  \\
cc-pV(X+d)Z-DK       \\
cc-pCVXZ-DK          & DTQ5  &  &  &  &  &  & DTQ5  & DTQ5  &  \\
cc-pCV(X+d)Z-DK      \\
cc-pwCVXZ-DK         & --TQ5 &  &  &  &  &  & --TQ5 & --TQ5 &  \\
cc-pwCV(X+d)Z-DK     \\
\hline\hline
\end{tabular}
\end{center}
$^a\quad$ The heavy-aug-cc-stub and jul-cc-stub basis sets are identical.
\end{footnotesize}
\end{table}


\begin{table}[!htbp]
\begin{footnotesize}
\caption{Summary of Dunning JK-fitting basis sets available in \PSIfour.} \label{table:basisDunningJKFIT}
\parsep 10pt
\begin{center}
\begin{tabular}{llllllllll} 
\hline\hline
Basis Set            & no diffuse & feb & mar & apr & may & jun & heavy-aug$^a$ & aug & d-aug \\
\hline
cc-pVXZ-JKFIT$^b$    & DTQ5 &  & --{}--{}--5 & --{}--Q5 & --TQ5 &  DTQ5  & DTQ5  & DTQ5  &  \\
cc-pV(X+d)Z-JKFIT    & DTQ5 &  & --{}--{}--5 & --{}--Q5 & --TQ5 &  DTQ5  & DTQ5  & DTQ5  &  \\
cc-pCVXZ-JKFIT$^b$   \\
cc-pCV(X+d)Z-JKFIT   \\
cc-pwCVXZ-JKFIT$^b$  \\
cc-pwCV(X+d)Z-JKFIT  \\
\hline\hline
\end{tabular}
\end{center}
$^a\quad$ The heavy-aug-cc-stub and jul-cc-stub basis sets are identical. \\
$^b\quad$ The JKFIT basis sets are designed in the cc-stub(X+d)Z framework that includes an additional set of
$d$-fuctions for second-row $p$-block elements. Identical basis sets with the cc-stubXZ-JKFIT label are provided
for convenience.
\end{footnotesize}
\end{table}


\begin{table}[!htbp]
\begin{footnotesize}
\caption{Summary of Dunning MP2-fitting basis sets available in \PSIfour.} \label{table:basisDunningMP2FIT}
\parsep 10pt
\begin{center}
\begin{tabular}{llllllllll} 
\hline\hline
Basis Set            & no diffuse & feb & mar & apr & may & jun & heavy-aug$^a$ & aug & d-aug \\
\hline
cc-pVXZ-RI           & DTQ56 & --{}--{}--{}--6 & --{}--{}--56 & --{}--Q56 & --TQ56 & DTQ56 & DTQ56 & DTQ56 &  \\
cc-pV(X+d)Z-RI       & DTQ56 & --{}--{}--{}--6 & --{}--{}--56 & --{}--Q56 & --TQ56 & DTQ56 & DTQ56 & DTQ56 &  \\
cc-pCVXZ-RI          \\
cc-pCV(X+d)Z-RI      \\
cc-pwCVXZ-RI         & DTQ5  &                 & --{}--{}--5  & --{}--Q5  & --TQ5  & DTQ5  & DTQ5  & DTQ5  &  \\
cc-pwCV(X+d)Z-RI     & DTQ5  &                 & --{}--{}--5  & --{}--Q5  & --TQ5  & DTQ5  & DTQ5  & DTQ5  &  \\
\hline\hline
\end{tabular}
\end{center}
$^a\quad$ The heavy-aug-cc-stub and jul-cc-stub basis sets are identical.
\end{footnotesize}
\end{table}


\begin{table}[!htbp]
\begin{footnotesize}
\caption{Summary of Dunning dual-basis helper basis sets available in \PSIfour.} \label{table:basisDunningDUAL}
\parsep 10pt
\begin{center}
\begin{tabular}{llllllllll}
\hline\hline
Basis Set            & no diffuse & feb & mar & apr & may & jun & heavy-aug$^a$ & aug & d-aug \\
\hline
cc-pVXZ-DUAL         & --TQ &  &  &  &  &  & --TQ & DTQ &  \\
cc-pV(X+d)Z-DUAL     \\
cc-pCVXZ-DUAL        \\
cc-pCV(X+d)Z-DUAL    \\
cc-pwCVXZ-DUAL       \\
cc-pwCV(X+d)Z-DUAL   \\
\hline\hline
\end{tabular}
\end{center}
$^a\quad$ The heavy-aug-cc-stub and jul-cc-stub basis sets are identical.
\end{footnotesize}
\end{table}


\begin{table}[!htbp]
\begin{footnotesize}
\caption{Summary of other orbital basis sets available in \PSIfour.} \label{table:basisOther}
\parsep 10pt
\begin{center}
\begin{tabular}{lll}
\hline\hline
\multicolumn{2}{c}{Karlsruhe} & \multicolumn{1}{c}{Other} \\
no diffuse & augmented & \\
\hline
def2-SV(P)   &             & DZP   \\
def2-SVP     & def2-SVPD   & TZ2P  \\
def2-TZVP    & def2-TZVPD  & TZ2PF \\
def2-TZVPP   & def2-TZVPPD &       \\
def2-QZVP    & def2-QZVPD  &       \\
def2-QZVPP   & def2-QZVPPD &       \\
\hline\hline
\end{tabular}
\end{center}
\end{footnotesize}
\end{table}


\subsection{User-Specified Basis Sets}

There are three routes by which a basis set in G94 format can be introduced to \PSIfour's notice.

\begin{itemize}

\item Install new basis set file into \PSIfour basis library.

Copy the basis set definitions for all elements into a blank file. Exclamation points denote comments.
As the first line of the file, add the word \texttt{spherical} or \texttt{cartesian} to indicate
whether the basis set will run in (5D/7F) or (6D/10F).
Name the file with the name of the basis set and a \texttt{.gbs} extension, after applying the following transformations.
\begin{itemize}
\item[] All letters lowercase
\item[] Replace all * with s
\item[] Replace all + with p
\item[] Replace all ( ) , with \_ \quad (underscores replace parentheses and commas)
\end{itemize}
For example, basis 6-31++G** is stored in \texttt{6-31ppgss.gbs}, and cc-pV(D+d)Z is stored in \texttt{cc-pv\_dpd\_z.gbs}.
Only one basis set may be specified per file.
Copy the new basis set file into \texttt{\$PSIDATADIR/basis}.

\item Use new basis set file in arbitrary location.

Copy the basis set definitions for all elements into a blank file. Exclamation points denote comments.
As the first line of the file, add the basis set name in brackets.
As the second line of the file, add the word \texttt{spherical} or \texttt{cartesian} to indicate
whether the basis set will run in (5D/7F) or (6D/10F). 
The combination of [ basis name ], \texttt{PUREAM} value, and element basis set specifications 
forms a section, like the one shown below.
Multiple basis sets can be specified in the same file by adding additional sections to the file.
Specify the location of the new basis set file in a \PSIfour input file with the command 
\texttt{basis file path/to/basis.file}, where the path can be either relative or absolute.
\begin{Snippet}
[ sto-3g ]
cartesian
****
H     0 
S   3   1.00
      3.42525091             0.15432897       
      0.62391373             0.53532814       
      0.16885540             0.44463454       
****
O     0 
S   3   1.00
    130.7093200              0.15432897       
     23.8088610              0.53532814       
      6.4436083              0.44463454       
SP   3   1.00
      5.0331513             -0.09996723             0.15591627       
      1.1695961              0.39951283             0.60768372       
      0.3803890              0.70011547             0.39195739       
****
\end{Snippet}

\item Include new basis set in input file.

Construct for a basis set a section like the one above that includes [ basis name ], \texttt{PUREAM}
value, and element basis set specifications. Hash signs denote comments.
Copy the section into a \PSIfour input file and surround it with the command \texttt{basis \{\}}, 
as shown below.
Multiple basis sets can be specified by adding additional sections within the surrounding brackets.
\begin{Snippet}
basis {
# basis set section like in snippet above goes here
# additional basis set sections follow
}
\end{Snippet}
\end{itemize}


%\subsection{Calling Built-in Basis Sets}
%\subsection{Calling Custom Basis Sets}

