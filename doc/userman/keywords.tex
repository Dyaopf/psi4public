{
 \footnotesize

\subsection{ADC}

{\normalsize Performs Algebraic-Diagrammatic Construction (ADC) propagator computations for excited states.}\\
\begin{tabular*}{\textwidth}[tb]{c}
	  \\ 
\end{tabular*}
\begin{tabular*}{\textwidth}[tb]{p{0.1\textwidth}p{0.9\textwidth}}
	 CACHELEVEL\\ 

	 & How to cache quantities within the DPD library \\ 
\end{tabular*}
\begin{tabular*}{\textwidth}[tb]{p{0.3\textwidth}p{0.35\textwidth}p{0.35\textwidth}}
	   & {\bf Type:} integer &  {\bf Default:} 2\\
	 & & \\
\end{tabular*}
\begin{tabular*}{\textwidth}[tb]{p{0.1\textwidth}p{0.9\textwidth}}
	 MEMORY\\ 

	 & The amount of memory available (in Mb) \\ 
\end{tabular*}
\begin{tabular*}{\textwidth}[tb]{p{0.3\textwidth}p{0.35\textwidth}p{0.35\textwidth}}
	   & {\bf Type:} integer &  {\bf Default:} 1000\\
	 & & \\
\end{tabular*}
\begin{tabular*}{\textwidth}[tb]{p{0.1\textwidth}p{0.9\textwidth}}
	 NEWTON\_CONVERGENCE\\ 

	 & The convergence criterion for pole searching step. See the note at the beginning of Section \ref{keywords}. \\ 
\end{tabular*}
\begin{tabular*}{\textwidth}[tb]{p{0.3\textwidth}p{0.35\textwidth}p{0.35\textwidth}}
	   & {\bf Type:} double &  {\bf Default:} 1e-7\\
	 & & \\
\end{tabular*}
\begin{tabular*}{\textwidth}[tb]{p{0.1\textwidth}p{0.9\textwidth}}
	 NORM\_TOLERANCE\\ 

	 & The cutoff norm of residual vector in SEM step. See the note at the beginning of Section \ref{keywords}. \\ 
\end{tabular*}
\begin{tabular*}{\textwidth}[tb]{p{0.3\textwidth}p{0.35\textwidth}p{0.35\textwidth}}
	   & {\bf Type:} double &  {\bf Default:} 1e-6\\
	 & & \\
\end{tabular*}
\begin{tabular*}{\textwidth}[tb]{p{0.1\textwidth}p{0.9\textwidth}}
	 NUM\_AMPS\_PRINT\\ 

	 & Number of components of transition amplitudes printed \\ 
\end{tabular*}
\begin{tabular*}{\textwidth}[tb]{p{0.3\textwidth}p{0.35\textwidth}p{0.35\textwidth}}
	   & {\bf Type:} integer &  {\bf Default:} 5\\
	 & & \\
\end{tabular*}
\begin{tabular*}{\textwidth}[tb]{p{0.1\textwidth}p{0.9\textwidth}}
	 POLE\_MAXITER\\ 

	 & Maximum iteration number in pole searching \\ 
\end{tabular*}
\begin{tabular*}{\textwidth}[tb]{p{0.3\textwidth}p{0.35\textwidth}p{0.35\textwidth}}
	   & {\bf Type:} integer &  {\bf Default:} 20\\
	 & & \\
\end{tabular*}
\begin{tabular*}{\textwidth}[tb]{p{0.1\textwidth}p{0.9\textwidth}}
	 PR\\ 

	 & Do use the partial renormalization scheme for the ground state wavefunction? \\ 
\end{tabular*}
\begin{tabular*}{\textwidth}[tb]{p{0.3\textwidth}p{0.35\textwidth}p{0.35\textwidth}}
	   & {\bf Type:} boolean &  {\bf Default:} false\\
	 & & \\
\end{tabular*}
\begin{tabular*}{\textwidth}[tb]{p{0.1\textwidth}p{0.9\textwidth}}
	 REFERENCE\\ 

	 & The Reference \\ 
\end{tabular*}
\begin{tabular*}{\textwidth}[tb]{p{0.3\textwidth}p{0.35\textwidth}p{0.35\textwidth}}
	   & {\bf Type:} string &  {\bf Default:} No Default\\
	 & & \\
\end{tabular*}
\begin{tabular*}{\textwidth}[tb]{p{0.1\textwidth}p{0.9\textwidth}}
	 SEM\_MAXITER\\ 

	 & Maximum iteration number in simultaneous expansion method \\ 
\end{tabular*}
\begin{tabular*}{\textwidth}[tb]{p{0.3\textwidth}p{0.35\textwidth}p{0.35\textwidth}}
	   & {\bf Type:} integer &  {\bf Default:} 30\\
	 & & \\
\end{tabular*}
\begin{tabular*}{\textwidth}[tb]{p{0.1\textwidth}p{0.9\textwidth}}
	 STATES\_PER\_IRREP\\ 

	 & The poles per irrep vector \\ 
\end{tabular*}
\begin{tabular*}{\textwidth}[tb]{p{0.3\textwidth}p{0.35\textwidth}p{0.35\textwidth}}
	   & {\bf Type:} array &  {\bf Default:} No Default\\
	 & & \\
\end{tabular*}

\subsection{CCDENSITY}

{\normalsize Computes the coupled cluster density matrices, and is called whenever CC properties and/or gradients are required.}\\
\begin{tabular*}{\textwidth}[tb]{c}
	  \\ 
\end{tabular*}
\begin{tabular*}{\textwidth}[tb]{p{0.1\textwidth}p{0.9\textwidth}}
	 AO\_BASIS\\ 

	 & The algorithm to use for the $\left<VV||VV\right>$ terms \\ 

	  & {\bf Possible Values:} NONE, DISK, DIRECT \\ 
\end{tabular*}
\begin{tabular*}{\textwidth}[tb]{p{0.3\textwidth}p{0.35\textwidth}p{0.35\textwidth}}
	   & {\bf Type:} string &  {\bf Default:} NONE\\
	 & & \\
\end{tabular*}
\begin{tabular*}{\textwidth}[tb]{p{0.1\textwidth}p{0.9\textwidth}}
	 CACHELEVEL\\ 

	 & The amount of cacheing of data to perform \\ 
\end{tabular*}
\begin{tabular*}{\textwidth}[tb]{p{0.3\textwidth}p{0.35\textwidth}p{0.35\textwidth}}
	   & {\bf Type:} integer &  {\bf Default:} 2\\
	 & & \\
\end{tabular*}
\begin{tabular*}{\textwidth}[tb]{p{0.1\textwidth}p{0.9\textwidth}}
	 GAUGE\\ 

	 & The type of gauge to use for properties \\ 
\end{tabular*}
\begin{tabular*}{\textwidth}[tb]{p{0.3\textwidth}p{0.35\textwidth}p{0.35\textwidth}}
	   & {\bf Type:} string &  {\bf Default:} LENGTH\\
	 & & \\
\end{tabular*}
\begin{tabular*}{\textwidth}[tb]{p{0.1\textwidth}p{0.9\textwidth}}
	 INTS\_TOLERANCE\\ 

	 & Minimum absolute value below which integrals are neglected. See the note at the beginning of Section \ref{keywords}. \\ 
\end{tabular*}
\begin{tabular*}{\textwidth}[tb]{p{0.3\textwidth}p{0.35\textwidth}p{0.35\textwidth}}
	   & {\bf Type:} double &  {\bf Default:} 1e-14\\
	 & & \\
\end{tabular*}
\begin{tabular*}{\textwidth}[tb]{p{0.1\textwidth}p{0.9\textwidth}}
	 OPDM\_RELAX\\ 

	 & Do relax the one-particle density matrix? \\ 
\end{tabular*}
\begin{tabular*}{\textwidth}[tb]{p{0.3\textwidth}p{0.35\textwidth}p{0.35\textwidth}}
	   & {\bf Type:} boolean &  {\bf Default:} false\\
	 & & \\
\end{tabular*}
\begin{tabular*}{\textwidth}[tb]{p{0.1\textwidth}p{0.9\textwidth}}
	 PROP\_ALL\\ 

	 & Do compute all relaxed excited states? \\ 
\end{tabular*}
\begin{tabular*}{\textwidth}[tb]{p{0.3\textwidth}p{0.35\textwidth}p{0.35\textwidth}}
	   & {\bf Type:} boolean &  {\bf Default:} false\\
	 & & \\
\end{tabular*}
\begin{tabular*}{\textwidth}[tb]{p{0.1\textwidth}p{0.9\textwidth}}
	 PROP\_ROOT\\ 

	 &  \\ 
\end{tabular*}
\begin{tabular*}{\textwidth}[tb]{p{0.3\textwidth}p{0.35\textwidth}p{0.35\textwidth}}
	   & {\bf Type:} integer &  {\bf Default:} 1\\
	 & & \\
\end{tabular*}
\begin{tabular*}{\textwidth}[tb]{p{0.1\textwidth}p{0.9\textwidth}}
	 PROP\_SYM\\ 

	 & The symmetry of states \\ 
\end{tabular*}
\begin{tabular*}{\textwidth}[tb]{p{0.3\textwidth}p{0.35\textwidth}p{0.35\textwidth}}
	   & {\bf Type:} integer &  {\bf Default:} 1\\
	 & & \\
\end{tabular*}
\begin{tabular*}{\textwidth}[tb]{p{0.1\textwidth}p{0.9\textwidth}}
	 REFERENCE\\ 

	 & Reference wavefunction type \\ 
\end{tabular*}
\begin{tabular*}{\textwidth}[tb]{p{0.3\textwidth}p{0.35\textwidth}p{0.35\textwidth}}
	   & {\bf Type:} string &  {\bf Default:} RHF\\
	 & & \\
\end{tabular*}
\begin{tabular*}{\textwidth}[tb]{p{0.1\textwidth}p{0.9\textwidth}}
	 STATES\_PER\_IRREP\\ 

	 & The number of electronic states to computed, per irreducible representation \\ 
\end{tabular*}
\begin{tabular*}{\textwidth}[tb]{p{0.3\textwidth}p{0.35\textwidth}p{0.35\textwidth}}
	   & {\bf Type:} array &  {\bf Default:} No Default\\
	 & & \\
\end{tabular*}
\begin{tabular*}{\textwidth}[tb]{p{0.1\textwidth}p{0.9\textwidth}}
	 XI\\ 

	 & Do compute Xi? \\ 
\end{tabular*}
\begin{tabular*}{\textwidth}[tb]{p{0.3\textwidth}p{0.35\textwidth}p{0.35\textwidth}}
	   & {\bf Type:} boolean &  {\bf Default:} false\\
	 & & \\
\end{tabular*}
\begin{tabular*}{\textwidth}[tb]{p{0.1\textwidth}p{0.9\textwidth}}
	 ZETA\\ 

	 & Do ? \\ 
\end{tabular*}
\begin{tabular*}{\textwidth}[tb]{p{0.3\textwidth}p{0.35\textwidth}p{0.35\textwidth}}
	   & {\bf Type:} boolean &  {\bf Default:} false\\
	 & & \\
\end{tabular*}

\subsection{CCENERGY}

{\normalsize Computes coupled cluster energies, and is called as part of any coupled cluster computation.}\\
\begin{tabular*}{\textwidth}[tb]{c}
	  \\ 
\end{tabular*}
\begin{tabular*}{\textwidth}[tb]{p{0.1\textwidth}p{0.9\textwidth}}
	 ABCD\\ 

	 &  \\ 

	  & {\bf Possible Values:} NEW, OLD \\ 
\end{tabular*}
\begin{tabular*}{\textwidth}[tb]{p{0.3\textwidth}p{0.35\textwidth}p{0.35\textwidth}}
	   & {\bf Type:} string &  {\bf Default:} NEW\\
	 & & \\
\end{tabular*}
\begin{tabular*}{\textwidth}[tb]{p{0.1\textwidth}p{0.9\textwidth}}
	 ANALYZE\\ 

	 & Do ? \\ 
\end{tabular*}
\begin{tabular*}{\textwidth}[tb]{p{0.3\textwidth}p{0.35\textwidth}p{0.35\textwidth}}
	   & {\bf Type:} boolean &  {\bf Default:} false\\
	 & & \\
\end{tabular*}
\begin{tabular*}{\textwidth}[tb]{p{0.1\textwidth}p{0.9\textwidth}}
	 AO\_BASIS\\ 

	 & The algorithm to use for the $\left<VV||VV\right>$ terms \\ 

	  & {\bf Possible Values:} NONE, DISK, DIRECT \\ 
\end{tabular*}
\begin{tabular*}{\textwidth}[tb]{p{0.3\textwidth}p{0.35\textwidth}p{0.35\textwidth}}
	   & {\bf Type:} string &  {\bf Default:} NONE\\
	 & & \\
\end{tabular*}
\begin{tabular*}{\textwidth}[tb]{p{0.1\textwidth}p{0.9\textwidth}}
	 BRUECKNER\_ORBS\_R\_CONVERGENCE\\ 

	 & Convergence criterion for Breuckner orbitals. See the note at the beginning of Section \ref{keywords}. \\ 
\end{tabular*}
\begin{tabular*}{\textwidth}[tb]{p{0.3\textwidth}p{0.35\textwidth}p{0.35\textwidth}}
	   & {\bf Type:} double &  {\bf Default:} 1e-5\\
	 & & \\
\end{tabular*}
\begin{tabular*}{\textwidth}[tb]{p{0.1\textwidth}p{0.9\textwidth}}
	 CACHELEVEL\\ 

	 & Cacheing level for libdpd governing the storage of amplitudes, integrals, and intermediates in the CC procedure. A value of 0 retains no quantities in cache, while a level of 6 attempts to store all quantities in cache. For particularly large calculations, a value of 0 may help with certain types of memory problems. The default is 2, which means that all four-index quantites with up to two virtual-orbital indices (e.g., <ij|ab> integrals) may be held in the cache. \\ 
\end{tabular*}
\begin{tabular*}{\textwidth}[tb]{p{0.3\textwidth}p{0.35\textwidth}p{0.35\textwidth}}
	   & {\bf Type:} integer &  {\bf Default:} 2\\
	 & & \\
\end{tabular*}
\begin{tabular*}{\textwidth}[tb]{p{0.1\textwidth}p{0.9\textwidth}}
	 CACHETYPE\\ 

	 & Selects the priority type for maintaining the automatic memory cache used by the libdpd codes. A value of LOW selects a "low priority" scheme in which the deletion of items from the cache is based on pre-programmed priorities. A value of LRU selects a "least recently used" scheme in which the oldest item in the cache will be the first one deleted. \\ 

	  & {\bf Possible Values:} LOW, LRU \\ 
\end{tabular*}
\begin{tabular*}{\textwidth}[tb]{p{0.3\textwidth}p{0.35\textwidth}p{0.35\textwidth}}
	   & {\bf Type:} string &  {\bf Default:} LOW\\
	 & & \\
\end{tabular*}
\begin{tabular*}{\textwidth}[tb]{p{0.1\textwidth}p{0.9\textwidth}}
	 CC\_OS\_SCALE\\ 

	 &  \\ 
\end{tabular*}
\begin{tabular*}{\textwidth}[tb]{p{0.3\textwidth}p{0.35\textwidth}p{0.35\textwidth}}
	   & {\bf Type:} double &  {\bf Default:} 1.27\\
	 & & \\
\end{tabular*}
\begin{tabular*}{\textwidth}[tb]{p{0.1\textwidth}p{0.9\textwidth}}
	 CC\_SS\_SCALE\\ 

	 &  \\ 
\end{tabular*}
\begin{tabular*}{\textwidth}[tb]{p{0.3\textwidth}p{0.35\textwidth}p{0.35\textwidth}}
	   & {\bf Type:} double &  {\bf Default:} 1.13\\
	 & & \\
\end{tabular*}
\begin{tabular*}{\textwidth}[tb]{p{0.1\textwidth}p{0.9\textwidth}}
	 DIIS\\ 

	 & Do use DIIS extrapolation to accelerate convergence? \\ 
\end{tabular*}
\begin{tabular*}{\textwidth}[tb]{p{0.3\textwidth}p{0.35\textwidth}p{0.35\textwidth}}
	   & {\bf Type:} boolean &  {\bf Default:} true\\
	 & & \\
\end{tabular*}
\begin{tabular*}{\textwidth}[tb]{p{0.1\textwidth}p{0.9\textwidth}}
	 LOCAL\\ 

	 & Do simulate the effects of local correlation techniques? \\ 
\end{tabular*}
\begin{tabular*}{\textwidth}[tb]{p{0.3\textwidth}p{0.35\textwidth}p{0.35\textwidth}}
	   & {\bf Type:} boolean &  {\bf Default:} false\\
	 & & \\
\end{tabular*}
\begin{tabular*}{\textwidth}[tb]{p{0.1\textwidth}p{0.9\textwidth}}
	 LOCAL\_CUTOFF\\ 

	 & Value (always between one and zero) for the Broughton-Pulay completeness check used to contruct orbital domains for local-CC calculations. See J. Broughton and P. Pulay, J. Comp. Chem. 14, 736-740 (1993) and C. Hampel and H.-J. Werner, J. Chem. Phys. 104, 6286-6297 (1996). \\ 
\end{tabular*}
\begin{tabular*}{\textwidth}[tb]{p{0.3\textwidth}p{0.35\textwidth}p{0.35\textwidth}}
	   & {\bf Type:} double &  {\bf Default:} 0.02\\
	 & & \\
\end{tabular*}
\begin{tabular*}{\textwidth}[tb]{p{0.1\textwidth}p{0.9\textwidth}}
	 LOCAL\_METHOD\\ 

	 & Type of local-CCSD scheme to be simulated. WERNER selects the method developed by H.-J. Werner and co-workers, and AOBASIS selects the method developed by G.E. Scuseria and co-workers (currently inoperative). \\ 

	  & {\bf Possible Values:} WERNER, AOBASIS \\ 
\end{tabular*}
\begin{tabular*}{\textwidth}[tb]{p{0.3\textwidth}p{0.35\textwidth}p{0.35\textwidth}}
	   & {\bf Type:} string &  {\bf Default:} WERNER\\
	 & & \\
\end{tabular*}
\begin{tabular*}{\textwidth}[tb]{p{0.1\textwidth}p{0.9\textwidth}}
	 LOCAL\_PAIRDEF\\ 

	 &  \\ 

	  & {\bf Possible Values:} BP, RESPONSE \\ 
\end{tabular*}
\begin{tabular*}{\textwidth}[tb]{p{0.3\textwidth}p{0.35\textwidth}p{0.35\textwidth}}
	   & {\bf Type:} string &  {\bf Default:} BP\\
	 & & \\
\end{tabular*}
\begin{tabular*}{\textwidth}[tb]{p{0.1\textwidth}p{0.9\textwidth}}
	 LOCAL\_WEAKP\\ 

	 & Desired treatment of "weak pairs" in the local-CCSD method. A value of NEGLECT ignores weak pairs entirely. A value of NONE treats weak pairs in the same manner as strong pairs. A value of MP2 uses second-order perturbation theory to correct the local-CCSD energy computed with weak pairs ignored. \\ 

	  & {\bf Possible Values:} NONE, NEGLECT, MP2 \\ 
\end{tabular*}
\begin{tabular*}{\textwidth}[tb]{p{0.3\textwidth}p{0.35\textwidth}p{0.35\textwidth}}
	   & {\bf Type:} string &  {\bf Default:} NONE\\
	 & & \\
\end{tabular*}
\begin{tabular*}{\textwidth}[tb]{p{0.1\textwidth}p{0.9\textwidth}}
	 MAXITER\\ 

	 & Maximum number of iterations to solve the CC equations \\ 
\end{tabular*}
\begin{tabular*}{\textwidth}[tb]{p{0.3\textwidth}p{0.35\textwidth}p{0.35\textwidth}}
	   & {\bf Type:} integer &  {\bf Default:} 50\\
	 & & \\
\end{tabular*}
\begin{tabular*}{\textwidth}[tb]{p{0.1\textwidth}p{0.9\textwidth}}
	 MP2\_AMPS\_PRINT\\ 

	 & Do print the MP2 amplitudes which are the starting guesses for RHF and UHF reference functions? \\ 
\end{tabular*}
\begin{tabular*}{\textwidth}[tb]{p{0.3\textwidth}p{0.35\textwidth}p{0.35\textwidth}}
	   & {\bf Type:} boolean &  {\bf Default:} false\\
	 & & \\
\end{tabular*}
\begin{tabular*}{\textwidth}[tb]{p{0.1\textwidth}p{0.9\textwidth}}
	 MP2\_OS\_SCALE\\ 

	 &  \\ 
\end{tabular*}
\begin{tabular*}{\textwidth}[tb]{p{0.3\textwidth}p{0.35\textwidth}p{0.35\textwidth}}
	   & {\bf Type:} double &  {\bf Default:} 1.20\\
	 & & \\
\end{tabular*}
\begin{tabular*}{\textwidth}[tb]{p{0.1\textwidth}p{0.9\textwidth}}
	 MP2\_SS\_SCALE\\ 

	 &  \\ 
\end{tabular*}
\begin{tabular*}{\textwidth}[tb]{p{0.3\textwidth}p{0.35\textwidth}p{0.35\textwidth}}
	   & {\bf Type:} double &  {\bf Default:} 1.0/3.0\\
	 & & \\
\end{tabular*}
\begin{tabular*}{\textwidth}[tb]{p{0.1\textwidth}p{0.9\textwidth}}
	 NEW\_TRIPLES\\ 

	 & Do ? \\ 
\end{tabular*}
\begin{tabular*}{\textwidth}[tb]{p{0.3\textwidth}p{0.35\textwidth}p{0.35\textwidth}}
	   & {\bf Type:} boolean &  {\bf Default:} true\\
	 & & \\
\end{tabular*}
\begin{tabular*}{\textwidth}[tb]{p{0.1\textwidth}p{0.9\textwidth}}
	 NUM\_AMPS\_PRINT\\ 

	 & Number of important $t_1$ and $t_2$ amplitudes to print \\ 
\end{tabular*}
\begin{tabular*}{\textwidth}[tb]{p{0.3\textwidth}p{0.35\textwidth}p{0.35\textwidth}}
	   & {\bf Type:} integer &  {\bf Default:} 10\\
	 & & \\
\end{tabular*}
\begin{tabular*}{\textwidth}[tb]{p{0.1\textwidth}p{0.9\textwidth}}
	 NUM\_THREADS\\ 

	 & Number of threads \\ 
\end{tabular*}
\begin{tabular*}{\textwidth}[tb]{p{0.3\textwidth}p{0.35\textwidth}p{0.35\textwidth}}
	   & {\bf Type:} integer &  {\bf Default:} 1\\
	 & & \\
\end{tabular*}
\begin{tabular*}{\textwidth}[tb]{p{0.1\textwidth}p{0.9\textwidth}}
	 PAIR\_ENERGIES\_PRINT\\ 

	 & Do print MP2 and CCSD pair energies for RHF references? \\ 
\end{tabular*}
\begin{tabular*}{\textwidth}[tb]{p{0.3\textwidth}p{0.35\textwidth}p{0.35\textwidth}}
	   & {\bf Type:} boolean &  {\bf Default:} false\\
	 & & \\
\end{tabular*}
\begin{tabular*}{\textwidth}[tb]{p{0.1\textwidth}p{0.9\textwidth}}
	 PROPERTY\\ 

	 &  \\ 

	  & {\bf Possible Values:} POLARIZABILITY, ROTATION, MAGNETIZABILITY, ROA, ALL \\ 
\end{tabular*}
\begin{tabular*}{\textwidth}[tb]{p{0.3\textwidth}p{0.35\textwidth}p{0.35\textwidth}}
	   & {\bf Type:} string &  {\bf Default:} POLARIZABILITY\\
	 & & \\
\end{tabular*}
\begin{tabular*}{\textwidth}[tb]{p{0.1\textwidth}p{0.9\textwidth}}
	 REFERENCE\\ 

	 & Reference wavefunction type \\ 
\end{tabular*}
\begin{tabular*}{\textwidth}[tb]{p{0.3\textwidth}p{0.35\textwidth}p{0.35\textwidth}}
	   & {\bf Type:} string &  {\bf Default:} RHF\\
	 & & \\
\end{tabular*}
\begin{tabular*}{\textwidth}[tb]{p{0.1\textwidth}p{0.9\textwidth}}
	 RESTART\\ 

	 & Do restart the coupled-cluster iterations from old $t_1$ and $t_2$ amplitudes? \\ 
\end{tabular*}
\begin{tabular*}{\textwidth}[tb]{p{0.3\textwidth}p{0.35\textwidth}p{0.35\textwidth}}
	   & {\bf Type:} boolean &  {\bf Default:} true\\
	 & & \\
\end{tabular*}
\begin{tabular*}{\textwidth}[tb]{p{0.1\textwidth}p{0.9\textwidth}}
	 R\_CONVERGENCE\\ 

	 & Convergence criterion for wavefunction (change) in CC amplitude equations. See the note at the beginning of Section \ref{keywords}. \\ 
\end{tabular*}
\begin{tabular*}{\textwidth}[tb]{p{0.3\textwidth}p{0.35\textwidth}p{0.35\textwidth}}
	   & {\bf Type:} double &  {\bf Default:} 1e-7\\
	 & & \\
\end{tabular*}
\begin{tabular*}{\textwidth}[tb]{p{0.1\textwidth}p{0.9\textwidth}}
	 SCSN\_MP2\\ 

	 & Do ? \\ 
\end{tabular*}
\begin{tabular*}{\textwidth}[tb]{p{0.3\textwidth}p{0.35\textwidth}p{0.35\textwidth}}
	   & {\bf Type:} boolean &  {\bf Default:} false\\
	 & & \\
\end{tabular*}
\begin{tabular*}{\textwidth}[tb]{p{0.1\textwidth}p{0.9\textwidth}}
	 SCS\_CCSD\\ 

	 & Do ? \\ 
\end{tabular*}
\begin{tabular*}{\textwidth}[tb]{p{0.3\textwidth}p{0.35\textwidth}p{0.35\textwidth}}
	   & {\bf Type:} boolean &  {\bf Default:} false\\
	 & & \\
\end{tabular*}
\begin{tabular*}{\textwidth}[tb]{p{0.1\textwidth}p{0.9\textwidth}}
	 SCS\_MP2\\ 

	 & Do ? \\ 
\end{tabular*}
\begin{tabular*}{\textwidth}[tb]{p{0.3\textwidth}p{0.35\textwidth}p{0.35\textwidth}}
	   & {\bf Type:} boolean &  {\bf Default:} false\\
	 & & \\
\end{tabular*}
\begin{tabular*}{\textwidth}[tb]{p{0.1\textwidth}p{0.9\textwidth}}
	 SEMICANONICAL\\ 

	 & Convert ROHF MOs to semicanonical MOs \\ 
\end{tabular*}
\begin{tabular*}{\textwidth}[tb]{p{0.3\textwidth}p{0.35\textwidth}p{0.35\textwidth}}
	   & {\bf Type:} boolean &  {\bf Default:} true\\
	 & & \\
\end{tabular*}
\begin{tabular*}{\textwidth}[tb]{p{0.1\textwidth}p{0.9\textwidth}}
	 SPINADAPT\_ENERGIES\\ 

	 & Do print spin-adapted pair energies? \\ 
\end{tabular*}
\begin{tabular*}{\textwidth}[tb]{p{0.3\textwidth}p{0.35\textwidth}p{0.35\textwidth}}
	   & {\bf Type:} boolean &  {\bf Default:} false\\
	 & & \\
\end{tabular*}
\begin{tabular*}{\textwidth}[tb]{p{0.1\textwidth}p{0.9\textwidth}}
	 T2\_COUPLED\\ 

	 & Do ? \\ 
\end{tabular*}
\begin{tabular*}{\textwidth}[tb]{p{0.3\textwidth}p{0.35\textwidth}p{0.35\textwidth}}
	   & {\bf Type:} boolean &  {\bf Default:} false\\
	 & & \\
\end{tabular*}
\begin{tabular*}{\textwidth}[tb]{p{0.1\textwidth}p{0.9\textwidth}}
	 T3\_WS\_INCORE\\ 

	 & Do ? \\ 
\end{tabular*}
\begin{tabular*}{\textwidth}[tb]{p{0.3\textwidth}p{0.35\textwidth}p{0.35\textwidth}}
	   & {\bf Type:} boolean &  {\bf Default:} false\\
	 & & \\
\end{tabular*}

\subsection{CCEOM}

{\normalsize Performes equation-of-motion (EOM) coupled cluster excited state computations.}\\
\begin{tabular*}{\textwidth}[tb]{c}
	  \\ 
\end{tabular*}
\begin{tabular*}{\textwidth}[tb]{p{0.1\textwidth}p{0.9\textwidth}}
	 ABCD\\ 

	 &  \\ 

	  & {\bf Possible Values:} NEW, OLD \\ 
\end{tabular*}
\begin{tabular*}{\textwidth}[tb]{p{0.3\textwidth}p{0.35\textwidth}p{0.35\textwidth}}
	   & {\bf Type:} string &  {\bf Default:} NEW\\
	 & & \\
\end{tabular*}
\begin{tabular*}{\textwidth}[tb]{p{0.1\textwidth}p{0.9\textwidth}}
	 CACHELEVEL\\ 

	 &  \\ 
\end{tabular*}
\begin{tabular*}{\textwidth}[tb]{p{0.3\textwidth}p{0.35\textwidth}p{0.35\textwidth}}
	   & {\bf Type:} integer &  {\bf Default:} 2\\
	 & & \\
\end{tabular*}
\begin{tabular*}{\textwidth}[tb]{p{0.1\textwidth}p{0.9\textwidth}}
	 CACHETYPE\\ 

	 &  \\ 

	  & {\bf Possible Values:} LOW, LRU \\ 
\end{tabular*}
\begin{tabular*}{\textwidth}[tb]{p{0.3\textwidth}p{0.35\textwidth}p{0.35\textwidth}}
	   & {\bf Type:} string &  {\bf Default:} LRU\\
	 & & \\
\end{tabular*}
\begin{tabular*}{\textwidth}[tb]{p{0.1\textwidth}p{0.9\textwidth}}
	 CC3\_FOLLOW\_ROOT\\ 

	 & Do ? \\ 
\end{tabular*}
\begin{tabular*}{\textwidth}[tb]{p{0.3\textwidth}p{0.35\textwidth}p{0.35\textwidth}}
	   & {\bf Type:} boolean &  {\bf Default:} false\\
	 & & \\
\end{tabular*}
\begin{tabular*}{\textwidth}[tb]{p{0.1\textwidth}p{0.9\textwidth}}
	 COLLAPSE\_WITH\_LAST\\ 

	 & Do ? \\ 
\end{tabular*}
\begin{tabular*}{\textwidth}[tb]{p{0.3\textwidth}p{0.35\textwidth}p{0.35\textwidth}}
	   & {\bf Type:} boolean &  {\bf Default:} true\\
	 & & \\
\end{tabular*}
\begin{tabular*}{\textwidth}[tb]{p{0.1\textwidth}p{0.9\textwidth}}
	 COMPLEX\_TOLERANCE\\ 

	 & See the note at the beginning of Section \ref{keywords}. \\ 
\end{tabular*}
\begin{tabular*}{\textwidth}[tb]{p{0.3\textwidth}p{0.35\textwidth}p{0.35\textwidth}}
	   & {\bf Type:} double &  {\bf Default:} 1e-12\\
	 & & \\
\end{tabular*}
\begin{tabular*}{\textwidth}[tb]{p{0.1\textwidth}p{0.9\textwidth}}
	 EOM\_GUESS\\ 

	 &  \\ 

	  & {\bf Possible Values:} SINGLES, DISK, INPUT \\ 
\end{tabular*}
\begin{tabular*}{\textwidth}[tb]{p{0.3\textwidth}p{0.35\textwidth}p{0.35\textwidth}}
	   & {\bf Type:} string &  {\bf Default:} SINGLES\\
	 & & \\
\end{tabular*}
\begin{tabular*}{\textwidth}[tb]{p{0.1\textwidth}p{0.9\textwidth}}
	 EOM\_REFERENCE\\ 

	 &  \\ 

	  & {\bf Possible Values:} RHF, ROHF, UHF \\ 
\end{tabular*}
\begin{tabular*}{\textwidth}[tb]{p{0.3\textwidth}p{0.35\textwidth}p{0.35\textwidth}}
	   & {\bf Type:} string &  {\bf Default:} RHF\\
	 & & \\
\end{tabular*}
\begin{tabular*}{\textwidth}[tb]{p{0.1\textwidth}p{0.9\textwidth}}
	 E\_CONVERGENCE\\ 

	 & Convergence criterion for excitation energy (change) in the Davidson algorithm for CC-EOM. See the note at the beginning of Section \ref{keywords}. \\ 
\end{tabular*}
\begin{tabular*}{\textwidth}[tb]{p{0.3\textwidth}p{0.35\textwidth}p{0.35\textwidth}}
	   & {\bf Type:} double &  {\bf Default:} 1e-8\\
	 & & \\
\end{tabular*}
\begin{tabular*}{\textwidth}[tb]{p{0.1\textwidth}p{0.9\textwidth}}
	 FULL\_MATRIX\\ 

	 & Do ? \\ 
\end{tabular*}
\begin{tabular*}{\textwidth}[tb]{p{0.3\textwidth}p{0.35\textwidth}p{0.35\textwidth}}
	   & {\bf Type:} boolean &  {\bf Default:} false\\
	 & & \\
\end{tabular*}
\begin{tabular*}{\textwidth}[tb]{p{0.1\textwidth}p{0.9\textwidth}}
	 LOCAL\\ 

	 & Do ? \\ 
\end{tabular*}
\begin{tabular*}{\textwidth}[tb]{p{0.3\textwidth}p{0.35\textwidth}p{0.35\textwidth}}
	   & {\bf Type:} boolean &  {\bf Default:} false\\
	 & & \\
\end{tabular*}
\begin{tabular*}{\textwidth}[tb]{p{0.1\textwidth}p{0.9\textwidth}}
	 LOCAL\_CUTOFF\\ 

	 &  \\ 
\end{tabular*}
\begin{tabular*}{\textwidth}[tb]{p{0.3\textwidth}p{0.35\textwidth}p{0.35\textwidth}}
	   & {\bf Type:} double &  {\bf Default:} 0.02\\
	 & & \\
\end{tabular*}
\begin{tabular*}{\textwidth}[tb]{p{0.1\textwidth}p{0.9\textwidth}}
	 LOCAL\_DO\_SINGLES\\ 

	 & Do ? \\ 
\end{tabular*}
\begin{tabular*}{\textwidth}[tb]{p{0.3\textwidth}p{0.35\textwidth}p{0.35\textwidth}}
	   & {\bf Type:} boolean &  {\bf Default:} true\\
	 & & \\
\end{tabular*}
\begin{tabular*}{\textwidth}[tb]{p{0.1\textwidth}p{0.9\textwidth}}
	 LOCAL\_FILTER\_SINGLES\\ 

	 & Do ? \\ 
\end{tabular*}
\begin{tabular*}{\textwidth}[tb]{p{0.3\textwidth}p{0.35\textwidth}p{0.35\textwidth}}
	   & {\bf Type:} boolean &  {\bf Default:} true\\
	 & & \\
\end{tabular*}
\begin{tabular*}{\textwidth}[tb]{p{0.1\textwidth}p{0.9\textwidth}}
	 LOCAL\_GHOST\\ 

	 &  \\ 
\end{tabular*}
\begin{tabular*}{\textwidth}[tb]{p{0.3\textwidth}p{0.35\textwidth}p{0.35\textwidth}}
	   & {\bf Type:} integer &  {\bf Default:} -1\\
	 & & \\
\end{tabular*}
\begin{tabular*}{\textwidth}[tb]{p{0.1\textwidth}p{0.9\textwidth}}
	 LOCAL\_METHOD\\ 

	 &  \\ 

	  & {\bf Possible Values:} WERNER, AOBASIS \\ 
\end{tabular*}
\begin{tabular*}{\textwidth}[tb]{p{0.3\textwidth}p{0.35\textwidth}p{0.35\textwidth}}
	   & {\bf Type:} string &  {\bf Default:} WERNER\\
	 & & \\
\end{tabular*}
\begin{tabular*}{\textwidth}[tb]{p{0.1\textwidth}p{0.9\textwidth}}
	 LOCAL\_PRECONDITIONER\\ 

	 &  \\ 

	  & {\bf Possible Values:} HBAR, FOCK \\ 
\end{tabular*}
\begin{tabular*}{\textwidth}[tb]{p{0.3\textwidth}p{0.35\textwidth}p{0.35\textwidth}}
	   & {\bf Type:} string &  {\bf Default:} HBAR\\
	 & & \\
\end{tabular*}
\begin{tabular*}{\textwidth}[tb]{p{0.1\textwidth}p{0.9\textwidth}}
	 LOCAL\_WEAKP\\ 

	 &  \\ 

	  & {\bf Possible Values:} NONE, MP2, NEGLECT \\ 
\end{tabular*}
\begin{tabular*}{\textwidth}[tb]{p{0.3\textwidth}p{0.35\textwidth}p{0.35\textwidth}}
	   & {\bf Type:} string &  {\bf Default:} NONE\\
	 & & \\
\end{tabular*}
\begin{tabular*}{\textwidth}[tb]{p{0.1\textwidth}p{0.9\textwidth}}
	 MAXITER\\ 

	 & Maximum number of iterations \\ 
\end{tabular*}
\begin{tabular*}{\textwidth}[tb]{p{0.3\textwidth}p{0.35\textwidth}p{0.35\textwidth}}
	   & {\bf Type:} integer &  {\bf Default:} 80\\
	 & & \\
\end{tabular*}
\begin{tabular*}{\textwidth}[tb]{p{0.1\textwidth}p{0.9\textwidth}}
	 NEW\_TRIPLES\\ 

	 & Do ? \\ 
\end{tabular*}
\begin{tabular*}{\textwidth}[tb]{p{0.3\textwidth}p{0.35\textwidth}p{0.35\textwidth}}
	   & {\bf Type:} boolean &  {\bf Default:} true\\
	 & & \\
\end{tabular*}
\begin{tabular*}{\textwidth}[tb]{p{0.1\textwidth}p{0.9\textwidth}}
	 NUM\_AMPS\_PRINT\\ 

	 & Number of important CC amplitudes to print \\ 
\end{tabular*}
\begin{tabular*}{\textwidth}[tb]{p{0.3\textwidth}p{0.35\textwidth}p{0.35\textwidth}}
	   & {\bf Type:} integer &  {\bf Default:} 5\\
	 & & \\
\end{tabular*}
\begin{tabular*}{\textwidth}[tb]{p{0.1\textwidth}p{0.9\textwidth}}
	 NUM\_THREADS\\ 

	 & Number of threads \\ 
\end{tabular*}
\begin{tabular*}{\textwidth}[tb]{p{0.3\textwidth}p{0.35\textwidth}p{0.35\textwidth}}
	   & {\bf Type:} integer &  {\bf Default:} 1\\
	 & & \\
\end{tabular*}
\begin{tabular*}{\textwidth}[tb]{p{0.1\textwidth}p{0.9\textwidth}}
	 PROP\_ROOT\\ 

	 & Root number (within its irrep) for computing properties. Defaults to highest root requested. \\ 
\end{tabular*}
\begin{tabular*}{\textwidth}[tb]{p{0.3\textwidth}p{0.35\textwidth}p{0.35\textwidth}}
	   & {\bf Type:} integer &  {\bf Default:} 0\\
	 & & \\
\end{tabular*}
\begin{tabular*}{\textwidth}[tb]{p{0.1\textwidth}p{0.9\textwidth}}
	 PROP\_SYM\\ 

	 & Symmetry of the state to compute properties. Defaults to last irrep for which states are requested. \\ 
\end{tabular*}
\begin{tabular*}{\textwidth}[tb]{p{0.3\textwidth}p{0.35\textwidth}p{0.35\textwidth}}
	   & {\bf Type:} integer &  {\bf Default:} 1\\
	 & & \\
\end{tabular*}
\begin{tabular*}{\textwidth}[tb]{p{0.1\textwidth}p{0.9\textwidth}}
	 REFERENCE\\ 

	 & Reference wavefunction type \\ 

	  & {\bf Possible Values:} RHF, ROHF, UHF \\ 
\end{tabular*}
\begin{tabular*}{\textwidth}[tb]{p{0.3\textwidth}p{0.35\textwidth}p{0.35\textwidth}}
	   & {\bf Type:} string &  {\bf Default:} RHF\\
	 & & \\
\end{tabular*}
\begin{tabular*}{\textwidth}[tb]{p{0.1\textwidth}p{0.9\textwidth}}
	 RESTART\_EOM\_CC3\\ 

	 & Do ? \\ 
\end{tabular*}
\begin{tabular*}{\textwidth}[tb]{p{0.3\textwidth}p{0.35\textwidth}p{0.35\textwidth}}
	   & {\bf Type:} boolean &  {\bf Default:} false\\
	 & & \\
\end{tabular*}
\begin{tabular*}{\textwidth}[tb]{p{0.1\textwidth}p{0.9\textwidth}}
	 RHF\_TRIPLETS\\ 

	 & Do ? \\ 
\end{tabular*}
\begin{tabular*}{\textwidth}[tb]{p{0.3\textwidth}p{0.35\textwidth}p{0.35\textwidth}}
	   & {\bf Type:} boolean &  {\bf Default:} false\\
	 & & \\
\end{tabular*}
\begin{tabular*}{\textwidth}[tb]{p{0.1\textwidth}p{0.9\textwidth}}
	 R\_CONVERGENCE\\ 

	 & Convergence criterion for norm of the residual vector in the Davidson algorithm for CC-EOM. See the note at the beginning of Section \ref{keywords}. \\ 
\end{tabular*}
\begin{tabular*}{\textwidth}[tb]{p{0.3\textwidth}p{0.35\textwidth}p{0.35\textwidth}}
	   & {\bf Type:} double &  {\bf Default:} 1e-6\\
	 & & \\
\end{tabular*}
\begin{tabular*}{\textwidth}[tb]{p{0.1\textwidth}p{0.9\textwidth}}
	 SCHMIDT\_ADD\_RESIDUAL\_TOLERANCE\\ 

	 & Minimum absolute value above which a guess vector to a root is added to the Davidson algorithm. See the note at the beginning of Section \ref{keywords}. \\ 
\end{tabular*}
\begin{tabular*}{\textwidth}[tb]{p{0.3\textwidth}p{0.35\textwidth}p{0.35\textwidth}}
	   & {\bf Type:} double &  {\bf Default:} 1e-3\\
	 & & \\
\end{tabular*}
\begin{tabular*}{\textwidth}[tb]{p{0.1\textwidth}p{0.9\textwidth}}
	 SEMICANONICAL\\ 

	 & Convert ROHF MOs to semicanonical MOs \\ 
\end{tabular*}
\begin{tabular*}{\textwidth}[tb]{p{0.3\textwidth}p{0.35\textwidth}p{0.35\textwidth}}
	   & {\bf Type:} boolean &  {\bf Default:} true\\
	 & & \\
\end{tabular*}
\begin{tabular*}{\textwidth}[tb]{p{0.1\textwidth}p{0.9\textwidth}}
	 SINGLES\_PRINT\\ 

	 & Do print information on the iterative solution to the single-excitation EOM-CC problem used as a guess to full EOM-CC? \\ 
\end{tabular*}
\begin{tabular*}{\textwidth}[tb]{p{0.3\textwidth}p{0.35\textwidth}p{0.35\textwidth}}
	   & {\bf Type:} boolean &  {\bf Default:} false\\
	 & & \\
\end{tabular*}
\begin{tabular*}{\textwidth}[tb]{p{0.1\textwidth}p{0.9\textwidth}}
	 SS\_E\_CONVERGENCE\\ 

	 & Convergence criterion for excitation energy (change) in the Davidson algorithm for the CIS guess to CC-EOM. See the note at the beginning of Section \ref{keywords}. \\ 
\end{tabular*}
\begin{tabular*}{\textwidth}[tb]{p{0.3\textwidth}p{0.35\textwidth}p{0.35\textwidth}}
	   & {\bf Type:} double &  {\bf Default:} 1e-6\\
	 & & \\
\end{tabular*}
\begin{tabular*}{\textwidth}[tb]{p{0.1\textwidth}p{0.9\textwidth}}
	 SS\_R\_CONVERGENCE\\ 

	 & Convergence criterion for norm of the residual vector in the Davidson algorithm for the CIS guess to CC-EOM. See the note at the beginning of Section \ref{keywords}. \\ 
\end{tabular*}
\begin{tabular*}{\textwidth}[tb]{p{0.3\textwidth}p{0.35\textwidth}p{0.35\textwidth}}
	   & {\bf Type:} double &  {\bf Default:} 1e-6\\
	 & & \\
\end{tabular*}
\begin{tabular*}{\textwidth}[tb]{p{0.1\textwidth}p{0.9\textwidth}}
	 SS\_SKIP\_DIAG\\ 

	 & Do ? \\ 
\end{tabular*}
\begin{tabular*}{\textwidth}[tb]{p{0.3\textwidth}p{0.35\textwidth}p{0.35\textwidth}}
	   & {\bf Type:} boolean &  {\bf Default:} false\\
	 & & \\
\end{tabular*}
\begin{tabular*}{\textwidth}[tb]{p{0.1\textwidth}p{0.9\textwidth}}
	 SS\_VECS\_PER\_ROOT\\ 

	 &  \\ 
\end{tabular*}
\begin{tabular*}{\textwidth}[tb]{p{0.3\textwidth}p{0.35\textwidth}p{0.35\textwidth}}
	   & {\bf Type:} integer &  {\bf Default:} 5\\
	 & & \\
\end{tabular*}
\begin{tabular*}{\textwidth}[tb]{p{0.1\textwidth}p{0.9\textwidth}}
	 STATES\_PER\_IRREP\\ 

	 & Number of excited states per irreducible representation for EOM-CC and CC-LR calculations. Irreps denote the final state symmetry, not the symmetry of the transtion. \\ 
\end{tabular*}
\begin{tabular*}{\textwidth}[tb]{p{0.3\textwidth}p{0.35\textwidth}p{0.35\textwidth}}
	   & {\bf Type:} array &  {\bf Default:} No Default\\
	 & & \\
\end{tabular*}
\begin{tabular*}{\textwidth}[tb]{p{0.1\textwidth}p{0.9\textwidth}}
	 T3\_WS\_INCORE\\ 

	 & Do ? \\ 
\end{tabular*}
\begin{tabular*}{\textwidth}[tb]{p{0.3\textwidth}p{0.35\textwidth}p{0.35\textwidth}}
	   & {\bf Type:} boolean &  {\bf Default:} false\\
	 & & \\
\end{tabular*}
\begin{tabular*}{\textwidth}[tb]{p{0.1\textwidth}p{0.9\textwidth}}
	 VECS\_CC3\\ 

	 &  \\ 
\end{tabular*}
\begin{tabular*}{\textwidth}[tb]{p{0.3\textwidth}p{0.35\textwidth}p{0.35\textwidth}}
	   & {\bf Type:} integer &  {\bf Default:} 10\\
	 & & \\
\end{tabular*}
\begin{tabular*}{\textwidth}[tb]{p{0.1\textwidth}p{0.9\textwidth}}
	 VECS\_PER\_ROOT\\ 

	 &  \\ 
\end{tabular*}
\begin{tabular*}{\textwidth}[tb]{p{0.3\textwidth}p{0.35\textwidth}p{0.35\textwidth}}
	   & {\bf Type:} integer &  {\bf Default:} 12\\
	 & & \\
\end{tabular*}

\subsection{CCHBAR}

{\normalsize Assembles the coupled cluster effective Hamiltonian, and is called whenever CC properties and/or gradients are required.}\\
\begin{tabular*}{\textwidth}[tb]{c}
	  \\ 
\end{tabular*}
\begin{tabular*}{\textwidth}[tb]{p{0.1\textwidth}p{0.9\textwidth}}
	 CACHELEVEL\\ 

	 &  \\ 
\end{tabular*}
\begin{tabular*}{\textwidth}[tb]{p{0.3\textwidth}p{0.35\textwidth}p{0.35\textwidth}}
	   & {\bf Type:} integer &  {\bf Default:} 2\\
	 & & \\
\end{tabular*}
\begin{tabular*}{\textwidth}[tb]{p{0.1\textwidth}p{0.9\textwidth}}
	 EOM\_REFERENCE\\ 

	 &  \\ 
\end{tabular*}
\begin{tabular*}{\textwidth}[tb]{p{0.3\textwidth}p{0.35\textwidth}p{0.35\textwidth}}
	   & {\bf Type:} string &  {\bf Default:} RHF\\
	 & & \\
\end{tabular*}
\begin{tabular*}{\textwidth}[tb]{p{0.1\textwidth}p{0.9\textwidth}}
	 T\_AMPS\\ 

	 & Do compute the Tamplitude equation matrix elements? \\ 
\end{tabular*}
\begin{tabular*}{\textwidth}[tb]{p{0.3\textwidth}p{0.35\textwidth}p{0.35\textwidth}}
	   & {\bf Type:} boolean &  {\bf Default:} false\\
	 & & \\
\end{tabular*}
\begin{tabular*}{\textwidth}[tb]{p{0.1\textwidth}p{0.9\textwidth}}
	 WABEI\_LOWDISK\\ 

	 & Do use the minimal-disk algorithm for Wabei? It's VERY slow! \\ 
\end{tabular*}
\begin{tabular*}{\textwidth}[tb]{p{0.3\textwidth}p{0.35\textwidth}p{0.35\textwidth}}
	   & {\bf Type:} boolean &  {\bf Default:} false\\
	 & & \\
\end{tabular*}

\subsection{CCLAMBDA}

{\normalsize Solves for the Lagrange multipliers, which are needed whenever coupled cluster properties or gradients are requested.}\\
\begin{tabular*}{\textwidth}[tb]{c}
	  \\ 
\end{tabular*}
\begin{tabular*}{\textwidth}[tb]{p{0.1\textwidth}p{0.9\textwidth}}
	 ABCD\\ 

	 &  \\ 
\end{tabular*}
\begin{tabular*}{\textwidth}[tb]{p{0.3\textwidth}p{0.35\textwidth}p{0.35\textwidth}}
	   & {\bf Type:} string &  {\bf Default:} NEW\\
	 & & \\
\end{tabular*}
\begin{tabular*}{\textwidth}[tb]{p{0.1\textwidth}p{0.9\textwidth}}
	 AO\_BASIS\\ 

	 & The algorithm to use for the $\left<VV||VV\right>$ terms \\ 

	  & {\bf Possible Values:} NONE, DISK, DIRECT \\ 
\end{tabular*}
\begin{tabular*}{\textwidth}[tb]{p{0.3\textwidth}p{0.35\textwidth}p{0.35\textwidth}}
	   & {\bf Type:} string &  {\bf Default:} NONE\\
	 & & \\
\end{tabular*}
\begin{tabular*}{\textwidth}[tb]{p{0.1\textwidth}p{0.9\textwidth}}
	 CACHELEVEL\\ 

	 &  \\ 
\end{tabular*}
\begin{tabular*}{\textwidth}[tb]{p{0.3\textwidth}p{0.35\textwidth}p{0.35\textwidth}}
	   & {\bf Type:} integer &  {\bf Default:} 2\\
	 & & \\
\end{tabular*}
\begin{tabular*}{\textwidth}[tb]{p{0.1\textwidth}p{0.9\textwidth}}
	 DIIS\\ 

	 & Do use DIIS extrapolation to accelerate convergence? \\ 
\end{tabular*}
\begin{tabular*}{\textwidth}[tb]{p{0.3\textwidth}p{0.35\textwidth}p{0.35\textwidth}}
	   & {\bf Type:} boolean &  {\bf Default:} true\\
	 & & \\
\end{tabular*}
\begin{tabular*}{\textwidth}[tb]{p{0.1\textwidth}p{0.9\textwidth}}
	 LOCAL\\ 

	 & Do ? \\ 
\end{tabular*}
\begin{tabular*}{\textwidth}[tb]{p{0.3\textwidth}p{0.35\textwidth}p{0.35\textwidth}}
	   & {\bf Type:} boolean &  {\bf Default:} false\\
	 & & \\
\end{tabular*}
\begin{tabular*}{\textwidth}[tb]{p{0.1\textwidth}p{0.9\textwidth}}
	 LOCAL\_CPHF\_CUTOFF\\ 

	 &  \\ 
\end{tabular*}
\begin{tabular*}{\textwidth}[tb]{p{0.3\textwidth}p{0.35\textwidth}p{0.35\textwidth}}
	   & {\bf Type:} double &  {\bf Default:} 0.10\\
	 & & \\
\end{tabular*}
\begin{tabular*}{\textwidth}[tb]{p{0.1\textwidth}p{0.9\textwidth}}
	 LOCAL\_CUTOFF\\ 

	 &  \\ 
\end{tabular*}
\begin{tabular*}{\textwidth}[tb]{p{0.3\textwidth}p{0.35\textwidth}p{0.35\textwidth}}
	   & {\bf Type:} double &  {\bf Default:} 0.02\\
	 & & \\
\end{tabular*}
\begin{tabular*}{\textwidth}[tb]{p{0.1\textwidth}p{0.9\textwidth}}
	 LOCAL\_FILTER\_SINGLES\\ 

	 & Do ? \\ 
\end{tabular*}
\begin{tabular*}{\textwidth}[tb]{p{0.3\textwidth}p{0.35\textwidth}p{0.35\textwidth}}
	   & {\bf Type:} boolean &  {\bf Default:} true\\
	 & & \\
\end{tabular*}
\begin{tabular*}{\textwidth}[tb]{p{0.1\textwidth}p{0.9\textwidth}}
	 LOCAL\_METHOD\\ 

	 &  \\ 
\end{tabular*}
\begin{tabular*}{\textwidth}[tb]{p{0.3\textwidth}p{0.35\textwidth}p{0.35\textwidth}}
	   & {\bf Type:} string &  {\bf Default:} WERNER\\
	 & & \\
\end{tabular*}
\begin{tabular*}{\textwidth}[tb]{p{0.1\textwidth}p{0.9\textwidth}}
	 LOCAL\_PAIRDEF\\ 

	 &  \\ 
\end{tabular*}
\begin{tabular*}{\textwidth}[tb]{p{0.3\textwidth}p{0.35\textwidth}p{0.35\textwidth}}
	   & {\bf Type:} string &  {\bf Default:} No Default\\
	 & & \\
\end{tabular*}
\begin{tabular*}{\textwidth}[tb]{p{0.1\textwidth}p{0.9\textwidth}}
	 LOCAL\_WEAKP\\ 

	 &  \\ 
\end{tabular*}
\begin{tabular*}{\textwidth}[tb]{p{0.3\textwidth}p{0.35\textwidth}p{0.35\textwidth}}
	   & {\bf Type:} string &  {\bf Default:} NONE\\
	 & & \\
\end{tabular*}
\begin{tabular*}{\textwidth}[tb]{p{0.1\textwidth}p{0.9\textwidth}}
	 MAXITER\\ 

	 & Maximum number of iterations \\ 
\end{tabular*}
\begin{tabular*}{\textwidth}[tb]{p{0.3\textwidth}p{0.35\textwidth}p{0.35\textwidth}}
	   & {\bf Type:} integer &  {\bf Default:} 50\\
	 & & \\
\end{tabular*}
\begin{tabular*}{\textwidth}[tb]{p{0.1\textwidth}p{0.9\textwidth}}
	 NUM\_AMPS\_PRINT\\ 

	 &  \\ 
\end{tabular*}
\begin{tabular*}{\textwidth}[tb]{p{0.3\textwidth}p{0.35\textwidth}p{0.35\textwidth}}
	   & {\bf Type:} integer &  {\bf Default:} 10\\
	 & & \\
\end{tabular*}
\begin{tabular*}{\textwidth}[tb]{p{0.1\textwidth}p{0.9\textwidth}}
	 PROP\_ALL\\ 

	 & Do ? \\ 
\end{tabular*}
\begin{tabular*}{\textwidth}[tb]{p{0.3\textwidth}p{0.35\textwidth}p{0.35\textwidth}}
	   & {\bf Type:} boolean &  {\bf Default:} false\\
	 & & \\
\end{tabular*}
\begin{tabular*}{\textwidth}[tb]{p{0.1\textwidth}p{0.9\textwidth}}
	 PROP\_ROOT\\ 

	 &  \\ 
\end{tabular*}
\begin{tabular*}{\textwidth}[tb]{p{0.3\textwidth}p{0.35\textwidth}p{0.35\textwidth}}
	   & {\bf Type:} integer &  {\bf Default:} 1\\
	 & & \\
\end{tabular*}
\begin{tabular*}{\textwidth}[tb]{p{0.1\textwidth}p{0.9\textwidth}}
	 PROP\_SYM\\ 

	 &  \\ 
\end{tabular*}
\begin{tabular*}{\textwidth}[tb]{p{0.3\textwidth}p{0.35\textwidth}p{0.35\textwidth}}
	   & {\bf Type:} integer &  {\bf Default:} 1\\
	 & & \\
\end{tabular*}
\begin{tabular*}{\textwidth}[tb]{p{0.1\textwidth}p{0.9\textwidth}}
	 RESTART\\ 

	 & Do ? \\ 
\end{tabular*}
\begin{tabular*}{\textwidth}[tb]{p{0.3\textwidth}p{0.35\textwidth}p{0.35\textwidth}}
	   & {\bf Type:} boolean &  {\bf Default:} false\\
	 & & \\
\end{tabular*}
\begin{tabular*}{\textwidth}[tb]{p{0.1\textwidth}p{0.9\textwidth}}
	 R\_CONVERGENCE\\ 

	 & Convergence criterion for wavefunction (change) in CC lambda-amplitude equations. See the note at the beginning of Section \ref{keywords}. \\ 
\end{tabular*}
\begin{tabular*}{\textwidth}[tb]{p{0.3\textwidth}p{0.35\textwidth}p{0.35\textwidth}}
	   & {\bf Type:} double &  {\bf Default:} 1e-7\\
	 & & \\
\end{tabular*}
\begin{tabular*}{\textwidth}[tb]{p{0.1\textwidth}p{0.9\textwidth}}
	 SEKINO\\ 

	 & Do ? \\ 
\end{tabular*}
\begin{tabular*}{\textwidth}[tb]{p{0.3\textwidth}p{0.35\textwidth}p{0.35\textwidth}}
	   & {\bf Type:} boolean &  {\bf Default:} false\\
	 & & \\
\end{tabular*}
\begin{tabular*}{\textwidth}[tb]{p{0.1\textwidth}p{0.9\textwidth}}
	 STATES\_PER\_IRREP\\ 

	 &  \\ 
\end{tabular*}
\begin{tabular*}{\textwidth}[tb]{p{0.3\textwidth}p{0.35\textwidth}p{0.35\textwidth}}
	   & {\bf Type:} array &  {\bf Default:} No Default\\
	 & & \\
\end{tabular*}
\begin{tabular*}{\textwidth}[tb]{p{0.1\textwidth}p{0.9\textwidth}}
	 ZETA\\ 

	 &  \\ 
\end{tabular*}
\begin{tabular*}{\textwidth}[tb]{p{0.3\textwidth}p{0.35\textwidth}p{0.35\textwidth}}
	   & {\bf Type:} boolean &  {\bf Default:} false\\
	 & & \\
\end{tabular*}

\subsection{CCRESPONSE}

{\normalsize Performs coupled cluster response property computations.}\\
\begin{tabular*}{\textwidth}[tb]{c}
	  \\ 
\end{tabular*}
\begin{tabular*}{\textwidth}[tb]{p{0.1\textwidth}p{0.9\textwidth}}
	 ABCD\\ 

	 &  \\ 
\end{tabular*}
\begin{tabular*}{\textwidth}[tb]{p{0.3\textwidth}p{0.35\textwidth}p{0.35\textwidth}}
	   & {\bf Type:} string &  {\bf Default:} NEW\\
	 & & \\
\end{tabular*}
\begin{tabular*}{\textwidth}[tb]{p{0.1\textwidth}p{0.9\textwidth}}
	 ANALYZE\\ 

	 & Do ? \\ 
\end{tabular*}
\begin{tabular*}{\textwidth}[tb]{p{0.3\textwidth}p{0.35\textwidth}p{0.35\textwidth}}
	   & {\bf Type:} boolean &  {\bf Default:} false\\
	 & & \\
\end{tabular*}
\begin{tabular*}{\textwidth}[tb]{p{0.1\textwidth}p{0.9\textwidth}}
	 CACHELEVEL\\ 

	 & Cacheing level for libdpd \\ 
\end{tabular*}
\begin{tabular*}{\textwidth}[tb]{p{0.3\textwidth}p{0.35\textwidth}p{0.35\textwidth}}
	   & {\bf Type:} integer &  {\bf Default:} 2\\
	 & & \\
\end{tabular*}
\begin{tabular*}{\textwidth}[tb]{p{0.1\textwidth}p{0.9\textwidth}}
	 DIIS\\ 

	 & Do use DIIS extrapolation to accelerate convergence? \\ 
\end{tabular*}
\begin{tabular*}{\textwidth}[tb]{p{0.3\textwidth}p{0.35\textwidth}p{0.35\textwidth}}
	   & {\bf Type:} boolean &  {\bf Default:} true\\
	 & & \\
\end{tabular*}
\begin{tabular*}{\textwidth}[tb]{p{0.1\textwidth}p{0.9\textwidth}}
	 GAUGE\\ 

	 & Gauge for optical rotation \\ 
\end{tabular*}
\begin{tabular*}{\textwidth}[tb]{p{0.3\textwidth}p{0.35\textwidth}p{0.35\textwidth}}
	   & {\bf Type:} string &  {\bf Default:} LENGTH\\
	 & & \\
\end{tabular*}
\begin{tabular*}{\textwidth}[tb]{p{0.1\textwidth}p{0.9\textwidth}}
	 LINEAR\\ 

	 & Do Bartlett size-extensive linear model? \\ 
\end{tabular*}
\begin{tabular*}{\textwidth}[tb]{p{0.3\textwidth}p{0.35\textwidth}p{0.35\textwidth}}
	   & {\bf Type:} boolean &  {\bf Default:} false\\
	 & & \\
\end{tabular*}
\begin{tabular*}{\textwidth}[tb]{p{0.1\textwidth}p{0.9\textwidth}}
	 LOCAL\\ 

	 & Do simulate local correlation? \\ 
\end{tabular*}
\begin{tabular*}{\textwidth}[tb]{p{0.3\textwidth}p{0.35\textwidth}p{0.35\textwidth}}
	   & {\bf Type:} boolean &  {\bf Default:} false\\
	 & & \\
\end{tabular*}
\begin{tabular*}{\textwidth}[tb]{p{0.1\textwidth}p{0.9\textwidth}}
	 LOCAL\_CPHF\_CUTOFF\\ 

	 &  \\ 
\end{tabular*}
\begin{tabular*}{\textwidth}[tb]{p{0.3\textwidth}p{0.35\textwidth}p{0.35\textwidth}}
	   & {\bf Type:} double &  {\bf Default:} 0.10\\
	 & & \\
\end{tabular*}
\begin{tabular*}{\textwidth}[tb]{p{0.1\textwidth}p{0.9\textwidth}}
	 LOCAL\_CUTOFF\\ 

	 &  \\ 
\end{tabular*}
\begin{tabular*}{\textwidth}[tb]{p{0.3\textwidth}p{0.35\textwidth}p{0.35\textwidth}}
	   & {\bf Type:} double &  {\bf Default:} 0.01\\
	 & & \\
\end{tabular*}
\begin{tabular*}{\textwidth}[tb]{p{0.1\textwidth}p{0.9\textwidth}}
	 LOCAL\_FILTER\_SINGLES\\ 

	 & Do ? \\ 
\end{tabular*}
\begin{tabular*}{\textwidth}[tb]{p{0.3\textwidth}p{0.35\textwidth}p{0.35\textwidth}}
	   & {\bf Type:} boolean &  {\bf Default:} false\\
	 & & \\
\end{tabular*}
\begin{tabular*}{\textwidth}[tb]{p{0.1\textwidth}p{0.9\textwidth}}
	 LOCAL\_METHOD\\ 

	 &  \\ 
\end{tabular*}
\begin{tabular*}{\textwidth}[tb]{p{0.3\textwidth}p{0.35\textwidth}p{0.35\textwidth}}
	   & {\bf Type:} string &  {\bf Default:} WERNER\\
	 & & \\
\end{tabular*}
\begin{tabular*}{\textwidth}[tb]{p{0.1\textwidth}p{0.9\textwidth}}
	 LOCAL\_PAIRDEF\\ 

	 &  \\ 
\end{tabular*}
\begin{tabular*}{\textwidth}[tb]{p{0.3\textwidth}p{0.35\textwidth}p{0.35\textwidth}}
	   & {\bf Type:} string &  {\bf Default:} NONE\\
	 & & \\
\end{tabular*}
\begin{tabular*}{\textwidth}[tb]{p{0.1\textwidth}p{0.9\textwidth}}
	 LOCAL\_WEAKP\\ 

	 &  \\ 
\end{tabular*}
\begin{tabular*}{\textwidth}[tb]{p{0.3\textwidth}p{0.35\textwidth}p{0.35\textwidth}}
	   & {\bf Type:} string &  {\bf Default:} NONE\\
	 & & \\
\end{tabular*}
\begin{tabular*}{\textwidth}[tb]{p{0.1\textwidth}p{0.9\textwidth}}
	 MAXITER\\ 

	 & Maximum number of iterations to converge perturbed amplitude equations \\ 
\end{tabular*}
\begin{tabular*}{\textwidth}[tb]{p{0.3\textwidth}p{0.35\textwidth}p{0.35\textwidth}}
	   & {\bf Type:} integer &  {\bf Default:} 50\\
	 & & \\
\end{tabular*}
\begin{tabular*}{\textwidth}[tb]{p{0.1\textwidth}p{0.9\textwidth}}
	 NUM\_AMPS\_PRINT\\ 

	 &  \\ 
\end{tabular*}
\begin{tabular*}{\textwidth}[tb]{p{0.3\textwidth}p{0.35\textwidth}p{0.35\textwidth}}
	   & {\bf Type:} integer &  {\bf Default:} 5\\
	 & & \\
\end{tabular*}
\begin{tabular*}{\textwidth}[tb]{p{0.1\textwidth}p{0.9\textwidth}}
	 OMEGA\\ 

	 & Energy of applied field for dynamic polarizabilities \\ 
\end{tabular*}
\begin{tabular*}{\textwidth}[tb]{p{0.3\textwidth}p{0.35\textwidth}p{0.35\textwidth}}
	   & {\bf Type:} array &  {\bf Default:} No Default\\
	 & & \\
\end{tabular*}
\begin{tabular*}{\textwidth}[tb]{p{0.1\textwidth}p{0.9\textwidth}}
	 PROPERTY\\ 

	 &  \\ 
\end{tabular*}
\begin{tabular*}{\textwidth}[tb]{p{0.3\textwidth}p{0.35\textwidth}p{0.35\textwidth}}
	   & {\bf Type:} string &  {\bf Default:} POLARIZABILITY\\
	 & & \\
\end{tabular*}
\begin{tabular*}{\textwidth}[tb]{p{0.1\textwidth}p{0.9\textwidth}}
	 REFERENCE\\ 

	 & Reference wavefunction type \\ 
\end{tabular*}
\begin{tabular*}{\textwidth}[tb]{p{0.3\textwidth}p{0.35\textwidth}p{0.35\textwidth}}
	   & {\bf Type:} string &  {\bf Default:} RHF\\
	 & & \\
\end{tabular*}
\begin{tabular*}{\textwidth}[tb]{p{0.1\textwidth}p{0.9\textwidth}}
	 RESTART\\ 

	 & Do restart from on-disk amplitudes? \\ 
\end{tabular*}
\begin{tabular*}{\textwidth}[tb]{p{0.3\textwidth}p{0.35\textwidth}p{0.35\textwidth}}
	   & {\bf Type:} boolean &  {\bf Default:} true\\
	 & & \\
\end{tabular*}
\begin{tabular*}{\textwidth}[tb]{p{0.1\textwidth}p{0.9\textwidth}}
	 R\_CONVERGENCE\\ 

	 & Convergence criterion for wavefunction (change) in perturbed CC equations. See the note at the beginning of Section \ref{keywords}. \\ 
\end{tabular*}
\begin{tabular*}{\textwidth}[tb]{p{0.3\textwidth}p{0.35\textwidth}p{0.35\textwidth}}
	   & {\bf Type:} double &  {\bf Default:} 1e-7\\
	 & & \\
\end{tabular*}
\begin{tabular*}{\textwidth}[tb]{p{0.1\textwidth}p{0.9\textwidth}}
	 SEKINO\\ 

	 & Do Sekino-Bartlett size-extensive model-III? \\ 
\end{tabular*}
\begin{tabular*}{\textwidth}[tb]{p{0.3\textwidth}p{0.35\textwidth}p{0.35\textwidth}}
	   & {\bf Type:} boolean &  {\bf Default:} false\\
	 & & \\
\end{tabular*}

\subsection{CCSORT}

{\normalsize Sorts integrals for efficiency, and is called before (non density-fitted) MP2 and coupled cluster computations.}\\
\begin{tabular*}{\textwidth}[tb]{c}
	  \\ 
\end{tabular*}
\begin{tabular*}{\textwidth}[tb]{p{0.1\textwidth}p{0.9\textwidth}}
	 AO\_BASIS\\ 

	 & The algorithm to use for the $\left<VV||VV\right>$ terms \\ 

	  & {\bf Possible Values:} NONE, DISK, DIRECT \\ 
\end{tabular*}
\begin{tabular*}{\textwidth}[tb]{p{0.3\textwidth}p{0.35\textwidth}p{0.35\textwidth}}
	   & {\bf Type:} string &  {\bf Default:} NONE\\
	 & & \\
\end{tabular*}
\begin{tabular*}{\textwidth}[tb]{p{0.1\textwidth}p{0.9\textwidth}}
	 CACHELEVEL\\ 

	 &  \\ 
\end{tabular*}
\begin{tabular*}{\textwidth}[tb]{p{0.3\textwidth}p{0.35\textwidth}p{0.35\textwidth}}
	   & {\bf Type:} integer &  {\bf Default:} 2\\
	 & & \\
\end{tabular*}
\begin{tabular*}{\textwidth}[tb]{p{0.1\textwidth}p{0.9\textwidth}}
	 EOM\_REFERENCE\\ 

	 &  \\ 
\end{tabular*}
\begin{tabular*}{\textwidth}[tb]{p{0.3\textwidth}p{0.35\textwidth}p{0.35\textwidth}}
	   & {\bf Type:} string &  {\bf Default:} RHF\\
	 & & \\
\end{tabular*}
\begin{tabular*}{\textwidth}[tb]{p{0.1\textwidth}p{0.9\textwidth}}
	 INTS\_TOLERANCE\\ 

	 & Minimum absolute value below which integrals are neglected. See the note at the beginning of Section \ref{keywords}. \\ 
\end{tabular*}
\begin{tabular*}{\textwidth}[tb]{p{0.3\textwidth}p{0.35\textwidth}p{0.35\textwidth}}
	   & {\bf Type:} double &  {\bf Default:} 1e-14\\
	 & & \\
\end{tabular*}
\begin{tabular*}{\textwidth}[tb]{p{0.1\textwidth}p{0.9\textwidth}}
	 KEEP\_OEIFILE\\ 

	 & Do retain the input one-electron integrals? \\ 
\end{tabular*}
\begin{tabular*}{\textwidth}[tb]{p{0.3\textwidth}p{0.35\textwidth}p{0.35\textwidth}}
	   & {\bf Type:} boolean &  {\bf Default:} false\\
	 & & \\
\end{tabular*}
\begin{tabular*}{\textwidth}[tb]{p{0.1\textwidth}p{0.9\textwidth}}
	 KEEP\_TEIFILE\\ 

	 & Do retain the input two-electron integrals? \\ 
\end{tabular*}
\begin{tabular*}{\textwidth}[tb]{p{0.3\textwidth}p{0.35\textwidth}p{0.35\textwidth}}
	   & {\bf Type:} boolean &  {\bf Default:} false\\
	 & & \\
\end{tabular*}
\begin{tabular*}{\textwidth}[tb]{p{0.1\textwidth}p{0.9\textwidth}}
	 LOCAL\\ 

	 & Do ? \\ 
\end{tabular*}
\begin{tabular*}{\textwidth}[tb]{p{0.3\textwidth}p{0.35\textwidth}p{0.35\textwidth}}
	   & {\bf Type:} boolean &  {\bf Default:} false\\
	 & & \\
\end{tabular*}
\begin{tabular*}{\textwidth}[tb]{p{0.1\textwidth}p{0.9\textwidth}}
	 LOCAL\_CORE\_CUTOFF\\ 

	 &  \\ 
\end{tabular*}
\begin{tabular*}{\textwidth}[tb]{p{0.3\textwidth}p{0.35\textwidth}p{0.35\textwidth}}
	   & {\bf Type:} double &  {\bf Default:} 0.05\\
	 & & \\
\end{tabular*}
\begin{tabular*}{\textwidth}[tb]{p{0.1\textwidth}p{0.9\textwidth}}
	 LOCAL\_CPHF\_CUTOFF\\ 

	 &  \\ 
\end{tabular*}
\begin{tabular*}{\textwidth}[tb]{p{0.3\textwidth}p{0.35\textwidth}p{0.35\textwidth}}
	   & {\bf Type:} double &  {\bf Default:} 0.10\\
	 & & \\
\end{tabular*}
\begin{tabular*}{\textwidth}[tb]{p{0.1\textwidth}p{0.9\textwidth}}
	 LOCAL\_CUTOFF\\ 

	 &  \\ 
\end{tabular*}
\begin{tabular*}{\textwidth}[tb]{p{0.3\textwidth}p{0.35\textwidth}p{0.35\textwidth}}
	   & {\bf Type:} double &  {\bf Default:} 0.02\\
	 & & \\
\end{tabular*}
\begin{tabular*}{\textwidth}[tb]{p{0.1\textwidth}p{0.9\textwidth}}
	 LOCAL\_DOMAIN\_MAG\\ 

	 & Do ? \\ 
\end{tabular*}
\begin{tabular*}{\textwidth}[tb]{p{0.3\textwidth}p{0.35\textwidth}p{0.35\textwidth}}
	   & {\bf Type:} boolean &  {\bf Default:} false\\
	 & & \\
\end{tabular*}
\begin{tabular*}{\textwidth}[tb]{p{0.1\textwidth}p{0.9\textwidth}}
	 LOCAL\_DOMAIN\_POLAR\\ 

	 & Do ? \\ 
\end{tabular*}
\begin{tabular*}{\textwidth}[tb]{p{0.3\textwidth}p{0.35\textwidth}p{0.35\textwidth}}
	   & {\bf Type:} boolean &  {\bf Default:} false\\
	 & & \\
\end{tabular*}
\begin{tabular*}{\textwidth}[tb]{p{0.1\textwidth}p{0.9\textwidth}}
	 LOCAL\_DOMAIN\_SEP\\ 

	 & Do ? \\ 
\end{tabular*}
\begin{tabular*}{\textwidth}[tb]{p{0.3\textwidth}p{0.35\textwidth}p{0.35\textwidth}}
	   & {\bf Type:} boolean &  {\bf Default:} false\\
	 & & \\
\end{tabular*}
\begin{tabular*}{\textwidth}[tb]{p{0.1\textwidth}p{0.9\textwidth}}
	 LOCAL\_FILTER\_SINGLES\\ 

	 & Do ? \\ 
\end{tabular*}
\begin{tabular*}{\textwidth}[tb]{p{0.3\textwidth}p{0.35\textwidth}p{0.35\textwidth}}
	   & {\bf Type:} boolean &  {\bf Default:} false\\
	 & & \\
\end{tabular*}
\begin{tabular*}{\textwidth}[tb]{p{0.1\textwidth}p{0.9\textwidth}}
	 LOCAL\_METHOD\\ 

	 &  \\ 
\end{tabular*}
\begin{tabular*}{\textwidth}[tb]{p{0.3\textwidth}p{0.35\textwidth}p{0.35\textwidth}}
	   & {\bf Type:} string &  {\bf Default:} WERNER\\
	 & & \\
\end{tabular*}
\begin{tabular*}{\textwidth}[tb]{p{0.1\textwidth}p{0.9\textwidth}}
	 LOCAL\_PAIRDEF\\ 

	 &  \\ 
\end{tabular*}
\begin{tabular*}{\textwidth}[tb]{p{0.3\textwidth}p{0.35\textwidth}p{0.35\textwidth}}
	   & {\bf Type:} string &  {\bf Default:} BP\\
	 & & \\
\end{tabular*}
\begin{tabular*}{\textwidth}[tb]{p{0.1\textwidth}p{0.9\textwidth}}
	 LOCAL\_WEAKP\\ 

	 &  \\ 
\end{tabular*}
\begin{tabular*}{\textwidth}[tb]{p{0.3\textwidth}p{0.35\textwidth}p{0.35\textwidth}}
	   & {\bf Type:} string &  {\bf Default:} NONE\\
	 & & \\
\end{tabular*}
\begin{tabular*}{\textwidth}[tb]{p{0.1\textwidth}p{0.9\textwidth}}
	 OMEGA\\ 

	 & Energy of applied field [au] for dynamic properties \\ 
\end{tabular*}
\begin{tabular*}{\textwidth}[tb]{p{0.3\textwidth}p{0.35\textwidth}p{0.35\textwidth}}
	   & {\bf Type:} array &  {\bf Default:} No Default\\
	 & & \\
\end{tabular*}
\begin{tabular*}{\textwidth}[tb]{p{0.1\textwidth}p{0.9\textwidth}}
	 PROPERTY\\ 

	 &  \\ 
\end{tabular*}
\begin{tabular*}{\textwidth}[tb]{p{0.3\textwidth}p{0.35\textwidth}p{0.35\textwidth}}
	   & {\bf Type:} string &  {\bf Default:} POLARIZABILITY\\
	 & & \\
\end{tabular*}
\begin{tabular*}{\textwidth}[tb]{p{0.1\textwidth}p{0.9\textwidth}}
	 REFERENCE\\ 

	 & Reference wavefunction type \\ 
\end{tabular*}
\begin{tabular*}{\textwidth}[tb]{p{0.3\textwidth}p{0.35\textwidth}p{0.35\textwidth}}
	   & {\bf Type:} string &  {\bf Default:} RHF\\
	 & & \\
\end{tabular*}
\begin{tabular*}{\textwidth}[tb]{p{0.1\textwidth}p{0.9\textwidth}}
	 SEMICANONICAL\\ 

	 & Convert ROHF MOs to semicanonical MOs \\ 
\end{tabular*}
\begin{tabular*}{\textwidth}[tb]{p{0.3\textwidth}p{0.35\textwidth}p{0.35\textwidth}}
	   & {\bf Type:} boolean &  {\bf Default:} true\\
	 & & \\
\end{tabular*}

\subsection{CCTRIPLES}

{\normalsize Computes the triples component of CCSD(T) energies (and gradients, if necessary).}\\
\begin{tabular*}{\textwidth}[tb]{c}
	  \\ 
\end{tabular*}
\begin{tabular*}{\textwidth}[tb]{p{0.1\textwidth}p{0.9\textwidth}}
	 NUM\_THREADS\\ 

	 & Number of threads \\ 
\end{tabular*}
\begin{tabular*}{\textwidth}[tb]{p{0.3\textwidth}p{0.35\textwidth}p{0.35\textwidth}}
	   & {\bf Type:} integer &  {\bf Default:} 1\\
	 & & \\
\end{tabular*}
\begin{tabular*}{\textwidth}[tb]{p{0.1\textwidth}p{0.9\textwidth}}
	 REFERENCE\\ 

	 & Reference wavefunction type \\ 
\end{tabular*}
\begin{tabular*}{\textwidth}[tb]{p{0.3\textwidth}p{0.35\textwidth}p{0.35\textwidth}}
	   & {\bf Type:} string &  {\bf Default:} RHF\\
	 & & \\
\end{tabular*}
\begin{tabular*}{\textwidth}[tb]{p{0.1\textwidth}p{0.9\textwidth}}
	 SEMICANONICAL\\ 

	 & Convert ROHF MOs to semicanonical MOs \\ 
\end{tabular*}
\begin{tabular*}{\textwidth}[tb]{p{0.3\textwidth}p{0.35\textwidth}p{0.35\textwidth}}
	   & {\bf Type:} boolean &  {\bf Default:} true\\
	 & & \\
\end{tabular*}

\subsection{CIS}

{\normalsize Performs configuration interaction singles (CIS) computations, but is currently unused in Psi4.}\\
\begin{tabular*}{\textwidth}[tb]{c}
	  \\ 
\end{tabular*}
\begin{tabular*}{\textwidth}[tb]{p{0.1\textwidth}p{0.9\textwidth}}
	 DIAG\_METHOD\\ 

	 &  \\ 

	  & {\bf Possible Values:} DAVIDSON, FULL \\ 
\end{tabular*}
\begin{tabular*}{\textwidth}[tb]{p{0.3\textwidth}p{0.35\textwidth}p{0.35\textwidth}}
	   & {\bf Type:} string &  {\bf Default:} DAVIDSON\\
	 & & \\
\end{tabular*}
\begin{tabular*}{\textwidth}[tb]{p{0.1\textwidth}p{0.9\textwidth}}
	 DOMAINS\\ 

	 &  \\ 
\end{tabular*}
\begin{tabular*}{\textwidth}[tb]{p{0.3\textwidth}p{0.35\textwidth}p{0.35\textwidth}}
	   & {\bf Type:} array &  {\bf Default:} No Default\\
	 & & \\
\end{tabular*}
\begin{tabular*}{\textwidth}[tb]{p{0.1\textwidth}p{0.9\textwidth}}
	 DOMAIN\_PRINT\\ 

	 & Do ? \\ 
\end{tabular*}
\begin{tabular*}{\textwidth}[tb]{p{0.3\textwidth}p{0.35\textwidth}p{0.35\textwidth}}
	   & {\bf Type:} boolean &  {\bf Default:} false\\
	 & & \\
\end{tabular*}
\begin{tabular*}{\textwidth}[tb]{p{0.1\textwidth}p{0.9\textwidth}}
	 LOCAL\\ 

	 & Do ? \\ 
\end{tabular*}
\begin{tabular*}{\textwidth}[tb]{p{0.3\textwidth}p{0.35\textwidth}p{0.35\textwidth}}
	   & {\bf Type:} boolean &  {\bf Default:} false\\
	 & & \\
\end{tabular*}
\begin{tabular*}{\textwidth}[tb]{p{0.1\textwidth}p{0.9\textwidth}}
	 LOCAL\_AMPS\_PRINT\_CUTOFF\\ 

	 &  \\ 
\end{tabular*}
\begin{tabular*}{\textwidth}[tb]{p{0.3\textwidth}p{0.35\textwidth}p{0.35\textwidth}}
	   & {\bf Type:} double &  {\bf Default:} 0.60\\
	 & & \\
\end{tabular*}
\begin{tabular*}{\textwidth}[tb]{p{0.1\textwidth}p{0.9\textwidth}}
	 LOCAL\_CUTOFF\\ 

	 &  \\ 
\end{tabular*}
\begin{tabular*}{\textwidth}[tb]{p{0.3\textwidth}p{0.35\textwidth}p{0.35\textwidth}}
	   & {\bf Type:} double &  {\bf Default:} 0.02\\
	 & & \\
\end{tabular*}
\begin{tabular*}{\textwidth}[tb]{p{0.1\textwidth}p{0.9\textwidth}}
	 LOCAL\_GHOST\\ 

	 &  \\ 
\end{tabular*}
\begin{tabular*}{\textwidth}[tb]{p{0.3\textwidth}p{0.35\textwidth}p{0.35\textwidth}}
	   & {\bf Type:} integer &  {\bf Default:} -1\\
	 & & \\
\end{tabular*}
\begin{tabular*}{\textwidth}[tb]{p{0.1\textwidth}p{0.9\textwidth}}
	 LOCAL\_METHOD\\ 

	 &  \\ 

	  & {\bf Possible Values:} AOBASIS, WERNER \\ 
\end{tabular*}
\begin{tabular*}{\textwidth}[tb]{p{0.3\textwidth}p{0.35\textwidth}p{0.35\textwidth}}
	   & {\bf Type:} string &  {\bf Default:} WERNER\\
	 & & \\
\end{tabular*}
\begin{tabular*}{\textwidth}[tb]{p{0.1\textwidth}p{0.9\textwidth}}
	 LOCAL\_WEAKP\\ 

	 &  \\ 

	  & {\bf Possible Values:} MP2, NEGLECT, NONE \\ 
\end{tabular*}
\begin{tabular*}{\textwidth}[tb]{p{0.3\textwidth}p{0.35\textwidth}p{0.35\textwidth}}
	   & {\bf Type:} string &  {\bf Default:} MP2\\
	 & & \\
\end{tabular*}
\begin{tabular*}{\textwidth}[tb]{p{0.1\textwidth}p{0.9\textwidth}}
	 MAXITER\\ 

	 & Maximum number of iterations \\ 
\end{tabular*}
\begin{tabular*}{\textwidth}[tb]{p{0.3\textwidth}p{0.35\textwidth}p{0.35\textwidth}}
	   & {\bf Type:} integer &  {\bf Default:} 500\\
	 & & \\
\end{tabular*}
\begin{tabular*}{\textwidth}[tb]{p{0.1\textwidth}p{0.9\textwidth}}
	 REFERENCE\\ 

	 & Reference wavefunction type \\ 

	  & {\bf Possible Values:} RHF, ROHF, UHF \\ 
\end{tabular*}
\begin{tabular*}{\textwidth}[tb]{p{0.3\textwidth}p{0.35\textwidth}p{0.35\textwidth}}
	   & {\bf Type:} string &  {\bf Default:} RHF\\
	 & & \\
\end{tabular*}
\begin{tabular*}{\textwidth}[tb]{p{0.1\textwidth}p{0.9\textwidth}}
	 R\_CONVERGENCE\\ 

	 & Convergence criterion for CIS wavefunction. See the note at the beginning of Section \ref{keywords}. \\ 
\end{tabular*}
\begin{tabular*}{\textwidth}[tb]{p{0.3\textwidth}p{0.35\textwidth}p{0.35\textwidth}}
	   & {\bf Type:} double &  {\bf Default:} 1e-7\\
	 & & \\
\end{tabular*}
\begin{tabular*}{\textwidth}[tb]{p{0.1\textwidth}p{0.9\textwidth}}
	 STATES\_PER\_IRREP\\ 

	 &  \\ 
\end{tabular*}
\begin{tabular*}{\textwidth}[tb]{p{0.3\textwidth}p{0.35\textwidth}p{0.35\textwidth}}
	   & {\bf Type:} array &  {\bf Default:} No Default\\
	 & & \\
\end{tabular*}

\subsection{CLAG}

{\normalsize Solves for the CI Lagrangian, and is called whenver CI properties or gradients are requested.}\\
\begin{tabular*}{\textwidth}[tb]{c}
	  \\ 
\end{tabular*}
\begin{tabular*}{\textwidth}[tb]{p{0.1\textwidth}p{0.9\textwidth}}
	 CAS\_FILES\_WRITE\\ 

	 & Do write the OEI, TEI, OPDM, TPDM, and Lagrangian files in canonical form, Pitzer order? \\ 
\end{tabular*}
\begin{tabular*}{\textwidth}[tb]{p{0.3\textwidth}p{0.35\textwidth}p{0.35\textwidth}}
	   & {\bf Type:} boolean &  {\bf Default:} false\\
	 & & \\
\end{tabular*}
\begin{tabular*}{\textwidth}[tb]{p{0.1\textwidth}p{0.9\textwidth}}
	 FOLLOW\_ROOT\\ 

	 & Root to get OPDM \\ 
\end{tabular*}
\begin{tabular*}{\textwidth}[tb]{p{0.3\textwidth}p{0.35\textwidth}p{0.35\textwidth}}
	   & {\bf Type:} integer &  {\bf Default:} 1\\
	 & & \\
\end{tabular*}

\subsection{DCFT}

{\normalsize Performs Density Cumulant Functional Theory computations}\\
\begin{tabular*}{\textwidth}[tb]{c}
	  \\ 
\end{tabular*}
\begin{tabular*}{\textwidth}[tb]{p{0.1\textwidth}p{0.9\textwidth}}
	 ALGORITHM\\ 

	 & The algorithm to use for lambda and orbital updates \\ 

	  & {\bf Possible Values:} TWOSTEP, SIMULTANEOUS \\ 
\end{tabular*}
\begin{tabular*}{\textwidth}[tb]{p{0.3\textwidth}p{0.35\textwidth}p{0.35\textwidth}}
	   & {\bf Type:} string &  {\bf Default:} SIMULTANEOUS\\
	 & & \\
\end{tabular*}
\begin{tabular*}{\textwidth}[tb]{p{0.1\textwidth}p{0.9\textwidth}}
	 AO\_BASIS\\ 

	 & The algorithm to use for the $\left<VV||VV\right>$ terms \\ 

	  & {\bf Possible Values:} NONE, DISK, DIRECT \\ 
\end{tabular*}
\begin{tabular*}{\textwidth}[tb]{p{0.3\textwidth}p{0.35\textwidth}p{0.35\textwidth}}
	   & {\bf Type:} string &  {\bf Default:} NONE\\
	 & & \\
\end{tabular*}
\begin{tabular*}{\textwidth}[tb]{p{0.1\textwidth}p{0.9\textwidth}}
	 CACHELEVEL\\ 

	 & How to cache quantities within the DPD library \\ 
\end{tabular*}
\begin{tabular*}{\textwidth}[tb]{p{0.3\textwidth}p{0.35\textwidth}p{0.35\textwidth}}
	   & {\bf Type:} integer &  {\bf Default:} 2\\
	 & & \\
\end{tabular*}
\begin{tabular*}{\textwidth}[tb]{p{0.1\textwidth}p{0.9\textwidth}}
	 DAMPING\_PERCENTAGE\\ 

	 & The amount (percentage) of damping to apply to the initial SCF procedures 0 will result in a full update, 100 will completely stall the update. A value around 20 (which corresponds to 20\% of the previous iteration's density being mixed into the current iteration) can help in cases where oscillatory convergence is observed. \\ 
\end{tabular*}
\begin{tabular*}{\textwidth}[tb]{p{0.3\textwidth}p{0.35\textwidth}p{0.35\textwidth}}
	   & {\bf Type:} double &  {\bf Default:} 0.0\\
	 & & \\
\end{tabular*}
\begin{tabular*}{\textwidth}[tb]{p{0.1\textwidth}p{0.9\textwidth}}
	 DIIS\_MAX\_VECS\\ 

	 & Maximum number of error vectors stored for DIIS extrapolation \\ 
\end{tabular*}
\begin{tabular*}{\textwidth}[tb]{p{0.3\textwidth}p{0.35\textwidth}p{0.35\textwidth}}
	   & {\bf Type:} integer &  {\bf Default:} 6\\
	 & & \\
\end{tabular*}
\begin{tabular*}{\textwidth}[tb]{p{0.1\textwidth}p{0.9\textwidth}}
	 DIIS\_MIN\_VECS\\ 

	 & Minimum number of error vectors stored for DIIS extrapolation \\ 
\end{tabular*}
\begin{tabular*}{\textwidth}[tb]{p{0.3\textwidth}p{0.35\textwidth}p{0.35\textwidth}}
	   & {\bf Type:} integer &  {\bf Default:} 3\\
	 & & \\
\end{tabular*}
\begin{tabular*}{\textwidth}[tb]{p{0.1\textwidth}p{0.9\textwidth}}
	 DIIS\_START\_CONVERGENCE\\ 

	 & DIIS starts when the RMS lambda and SCF errors are less than $10^{-diis\_start}$ \\ 
\end{tabular*}
\begin{tabular*}{\textwidth}[tb]{p{0.3\textwidth}p{0.35\textwidth}p{0.35\textwidth}}
	   & {\bf Type:} double &  {\bf Default:} 1e-3\\
	 & & \\
\end{tabular*}
\begin{tabular*}{\textwidth}[tb]{p{0.1\textwidth}p{0.9\textwidth}}
	 IGNORE\_TAU\\ 

	 & Don't include the tau terms? \\ 
\end{tabular*}
\begin{tabular*}{\textwidth}[tb]{p{0.3\textwidth}p{0.35\textwidth}p{0.35\textwidth}}
	   & {\bf Type:} boolean &  {\bf Default:} false\\
	 & & \\
\end{tabular*}
\begin{tabular*}{\textwidth}[tb]{p{0.1\textwidth}p{0.9\textwidth}}
	 INTS\_TOLERANCE\\ 

	 & Minimum absolute value below which integrals are neglected. See the note at the beginning of Section \ref{keywords}. \\ 
\end{tabular*}
\begin{tabular*}{\textwidth}[tb]{p{0.3\textwidth}p{0.35\textwidth}p{0.35\textwidth}}
	   & {\bf Type:} double &  {\bf Default:} 1e-14\\
	 & & \\
\end{tabular*}
\begin{tabular*}{\textwidth}[tb]{p{0.1\textwidth}p{0.9\textwidth}}
	 LAMBDA\_MAXITER\\ 

	 & Maximum number of lambda iterations per macro-iteration \\ 
\end{tabular*}
\begin{tabular*}{\textwidth}[tb]{p{0.3\textwidth}p{0.35\textwidth}p{0.35\textwidth}}
	   & {\bf Type:} integer &  {\bf Default:} 50\\
	 & & \\
\end{tabular*}
\begin{tabular*}{\textwidth}[tb]{p{0.1\textwidth}p{0.9\textwidth}}
	 LOCK\_OCC\\ 

	 & Do force the occupation to be that of the SCF starting point? \\ 
\end{tabular*}
\begin{tabular*}{\textwidth}[tb]{p{0.3\textwidth}p{0.35\textwidth}p{0.35\textwidth}}
	   & {\bf Type:} boolean &  {\bf Default:} true\\
	 & & \\
\end{tabular*}
\begin{tabular*}{\textwidth}[tb]{p{0.1\textwidth}p{0.9\textwidth}}
	 MAXITER\\ 

	 & Maximum number of iterations \\ 
\end{tabular*}
\begin{tabular*}{\textwidth}[tb]{p{0.3\textwidth}p{0.35\textwidth}p{0.35\textwidth}}
	   & {\bf Type:} integer &  {\bf Default:} 40\\
	 & & \\
\end{tabular*}
\begin{tabular*}{\textwidth}[tb]{p{0.1\textwidth}p{0.9\textwidth}}
	 MO\_RELAX\\ 

	 & Do relax the orbitals? \\ 
\end{tabular*}
\begin{tabular*}{\textwidth}[tb]{p{0.3\textwidth}p{0.35\textwidth}p{0.35\textwidth}}
	   & {\bf Type:} boolean &  {\bf Default:} true\\
	 & & \\
\end{tabular*}
\begin{tabular*}{\textwidth}[tb]{p{0.1\textwidth}p{0.9\textwidth}}
	 R\_CONVERGENCE\\ 

	 & Convergence criterion for residuals (RMS error) in density cummulant equations. See the note at the beginning of Section \ref{keywords}. \\ 
\end{tabular*}
\begin{tabular*}{\textwidth}[tb]{p{0.3\textwidth}p{0.35\textwidth}p{0.35\textwidth}}
	   & {\bf Type:} double &  {\bf Default:} 1e-10\\
	 & & \\
\end{tabular*}
\begin{tabular*}{\textwidth}[tb]{p{0.1\textwidth}p{0.9\textwidth}}
	 SCF\_D\_CONVERGENCE\\ 

	 & Convergence criterion for the SCF density (RMS error). See the note at the beginning of Section \ref{keywords}. \\ 
\end{tabular*}
\begin{tabular*}{\textwidth}[tb]{p{0.3\textwidth}p{0.35\textwidth}p{0.35\textwidth}}
	   & {\bf Type:} double &  {\bf Default:} 1e-8\\
	 & & \\
\end{tabular*}
\begin{tabular*}{\textwidth}[tb]{p{0.1\textwidth}p{0.9\textwidth}}
	 SCF\_MAXITER\\ 

	 & Maximum number of SCF iterations per cycle \\ 
\end{tabular*}
\begin{tabular*}{\textwidth}[tb]{p{0.3\textwidth}p{0.35\textwidth}p{0.35\textwidth}}
	   & {\bf Type:} integer &  {\bf Default:} 50\\
	 & & \\
\end{tabular*}
\begin{tabular*}{\textwidth}[tb]{p{0.1\textwidth}p{0.9\textwidth}}
	 TAU\_SQUARED\\ 

	 & Do compute the DCFT energy with the $\tau^{2}$ correction to $\tau$? \\ 
\end{tabular*}
\begin{tabular*}{\textwidth}[tb]{p{0.3\textwidth}p{0.35\textwidth}p{0.35\textwidth}}
	   & {\bf Type:} boolean &  {\bf Default:} false\\
	 & & \\
\end{tabular*}
\begin{tabular*}{\textwidth}[tb]{p{0.1\textwidth}p{0.9\textwidth}}
	 TIKHONOW\_OMEGA\\ 

	 & The shift applied to the denominator \\ 
\end{tabular*}
\begin{tabular*}{\textwidth}[tb]{p{0.3\textwidth}p{0.35\textwidth}p{0.35\textwidth}}
	   & {\bf Type:} double &  {\bf Default:} 0.0\\
	 & & \\
\end{tabular*}
\begin{tabular*}{\textwidth}[tb]{p{0.1\textwidth}p{0.9\textwidth}}
	 TPDM\\ 

	 & Do compute the full two particle density matrix at the end of the computation, for properties? \\ 
\end{tabular*}
\begin{tabular*}{\textwidth}[tb]{p{0.3\textwidth}p{0.35\textwidth}p{0.35\textwidth}}
	   & {\bf Type:} boolean &  {\bf Default:} false\\
	 & & \\
\end{tabular*}

\subsection{DETCI}
\begin{tabular*}{\textwidth}[tb]{p{0.1\textwidth}p{0.9\textwidth}}
	 ACTIVE\\ 

	 & An array giving the number of active orbitals (occupied plus unoccupied) per irrep (shorthand to make MCSCF easier to specify than using RAS keywords) \\ 
\end{tabular*}
\begin{tabular*}{\textwidth}[tb]{p{0.3\textwidth}p{0.35\textwidth}p{0.35\textwidth}}
	   & {\bf Type:} array &  {\bf Default:} No Default\\
	 & & \\
\end{tabular*}
\begin{tabular*}{\textwidth}[tb]{p{0.1\textwidth}p{0.9\textwidth}}
	 AVG\_STATES\\ 

	 & Array giving the root numbers of the states to average in a state-averaged procedure such as SA-CASSCF. Root numbering starts from 1. \\ 
\end{tabular*}
\begin{tabular*}{\textwidth}[tb]{p{0.3\textwidth}p{0.35\textwidth}p{0.35\textwidth}}
	   & {\bf Type:} array &  {\bf Default:} No Default\\
	 & & \\
\end{tabular*}
\begin{tabular*}{\textwidth}[tb]{p{0.1\textwidth}p{0.9\textwidth}}
	 AVG\_WEIGHTS\\ 

	 & Array giving the weights for each state in a state-averaged procedure \\ 
\end{tabular*}
\begin{tabular*}{\textwidth}[tb]{p{0.3\textwidth}p{0.35\textwidth}p{0.35\textwidth}}
	   & {\bf Type:} array &  {\bf Default:} No Default\\
	 & & \\
\end{tabular*}
\begin{tabular*}{\textwidth}[tb]{p{0.1\textwidth}p{0.9\textwidth}}
	 A\_RAS3\_MAX\\ 

	 & maximum number of alpha electrons in RAS III \\ 
\end{tabular*}
\begin{tabular*}{\textwidth}[tb]{p{0.3\textwidth}p{0.35\textwidth}p{0.35\textwidth}}
	   & {\bf Type:} integer &  {\bf Default:} -1\\
	 & & \\
\end{tabular*}
\begin{tabular*}{\textwidth}[tb]{p{0.1\textwidth}p{0.9\textwidth}}
	 B\_RAS3\_MAX\\ 

	 & maximum number of beta electrons in RAS III \\ 
\end{tabular*}
\begin{tabular*}{\textwidth}[tb]{p{0.3\textwidth}p{0.35\textwidth}p{0.35\textwidth}}
	   & {\bf Type:} integer &  {\bf Default:} -1\\
	 & & \\
\end{tabular*}
\begin{tabular*}{\textwidth}[tb]{p{0.1\textwidth}p{0.9\textwidth}}
	 CC\\ 

	 & Do coupled-cluster computation? \\ 
\end{tabular*}
\begin{tabular*}{\textwidth}[tb]{p{0.3\textwidth}p{0.35\textwidth}p{0.35\textwidth}}
	   & {\bf Type:} boolean &  {\bf Default:} false\\
	 & & \\
\end{tabular*}
\begin{tabular*}{\textwidth}[tb]{p{0.1\textwidth}p{0.9\textwidth}}
	 CC\_A\_RAS3\_MAX\\ 

	 & maximum number of alpha electrons in RAS III, for CC \\ 
\end{tabular*}
\begin{tabular*}{\textwidth}[tb]{p{0.3\textwidth}p{0.35\textwidth}p{0.35\textwidth}}
	   & {\bf Type:} integer &  {\bf Default:} -1\\
	 & & \\
\end{tabular*}
\begin{tabular*}{\textwidth}[tb]{p{0.1\textwidth}p{0.9\textwidth}}
	 CC\_B\_RAS3\_MAX\\ 

	 & maximum number of beta electrons in RAS III, for CC \\ 
\end{tabular*}
\begin{tabular*}{\textwidth}[tb]{p{0.3\textwidth}p{0.35\textwidth}p{0.35\textwidth}}
	   & {\bf Type:} integer &  {\bf Default:} -1\\
	 & & \\
\end{tabular*}
\begin{tabular*}{\textwidth}[tb]{p{0.1\textwidth}p{0.9\textwidth}}
	 CC\_EX\_LEVEL\\ 

	 & The CC excitation level \\ 
\end{tabular*}
\begin{tabular*}{\textwidth}[tb]{p{0.3\textwidth}p{0.35\textwidth}p{0.35\textwidth}}
	   & {\bf Type:} integer &  {\bf Default:} 2\\
	 & & \\
\end{tabular*}
\begin{tabular*}{\textwidth}[tb]{p{0.1\textwidth}p{0.9\textwidth}}
	 CC\_RAS34\_MAX\\ 

	 & maximum number of electrons in RAS III + IV, for CC \\ 
\end{tabular*}
\begin{tabular*}{\textwidth}[tb]{p{0.3\textwidth}p{0.35\textwidth}p{0.35\textwidth}}
	   & {\bf Type:} integer &  {\bf Default:} -1\\
	 & & \\
\end{tabular*}
\begin{tabular*}{\textwidth}[tb]{p{0.1\textwidth}p{0.9\textwidth}}
	 CC\_RAS3\_MAX\\ 

	 & maximum number of electrons in RAS III, for CC \\ 
\end{tabular*}
\begin{tabular*}{\textwidth}[tb]{p{0.3\textwidth}p{0.35\textwidth}p{0.35\textwidth}}
	   & {\bf Type:} integer &  {\bf Default:} -1\\
	 & & \\
\end{tabular*}
\begin{tabular*}{\textwidth}[tb]{p{0.1\textwidth}p{0.9\textwidth}}
	 CC\_RAS4\_MAX\\ 

	 & maximum number of electrons in RAS IV, for CC \\ 
\end{tabular*}
\begin{tabular*}{\textwidth}[tb]{p{0.3\textwidth}p{0.35\textwidth}p{0.35\textwidth}}
	   & {\bf Type:} integer &  {\bf Default:} -1\\
	 & & \\
\end{tabular*}
\begin{tabular*}{\textwidth}[tb]{p{0.1\textwidth}p{0.9\textwidth}}
	 CC\_VAL\_EX\_LEVEL\\ 

	 & The CC valence excitation level \\ 
\end{tabular*}
\begin{tabular*}{\textwidth}[tb]{p{0.3\textwidth}p{0.35\textwidth}p{0.35\textwidth}}
	   & {\bf Type:} integer &  {\bf Default:} 0\\
	 & & \\
\end{tabular*}
\begin{tabular*}{\textwidth}[tb]{p{0.1\textwidth}p{0.9\textwidth}}
	 CC\_VECS\_READ\\ 

	 & Do import a CC vector from disk? \\ 
\end{tabular*}
\begin{tabular*}{\textwidth}[tb]{p{0.3\textwidth}p{0.35\textwidth}p{0.35\textwidth}}
	   & {\bf Type:} boolean &  {\bf Default:} false\\
	 & & \\
\end{tabular*}
\begin{tabular*}{\textwidth}[tb]{p{0.1\textwidth}p{0.9\textwidth}}
	 CC\_VECS\_WRITE\\ 

	 & Do export a CC vector to disk? \\ 
\end{tabular*}
\begin{tabular*}{\textwidth}[tb]{p{0.3\textwidth}p{0.35\textwidth}p{0.35\textwidth}}
	   & {\bf Type:} boolean &  {\bf Default:} false\\
	 & & \\
\end{tabular*}
\begin{tabular*}{\textwidth}[tb]{p{0.1\textwidth}p{0.9\textwidth}}
	 CIBLKS\_PRINT\\ 

	 & Do print a summary of the CI blocks? \\ 
\end{tabular*}
\begin{tabular*}{\textwidth}[tb]{p{0.3\textwidth}p{0.35\textwidth}p{0.35\textwidth}}
	   & {\bf Type:} boolean &  {\bf Default:} false\\
	 & & \\
\end{tabular*}
\begin{tabular*}{\textwidth}[tb]{p{0.1\textwidth}p{0.9\textwidth}}
	 COLLAPSE\_SIZE\\ 

	 & Gives the number of vectors to retain when the Davidson subspace is collapsed (see MAX\_NUM\_VECS below). If greater than one, the collapsed subspace retains the best estimate of the CI vector for the previous n iterations. Defaults to 1. \\ 
\end{tabular*}
\begin{tabular*}{\textwidth}[tb]{p{0.3\textwidth}p{0.35\textwidth}p{0.35\textwidth}}
	   & {\bf Type:} integer &  {\bf Default:} 1\\
	 & & \\
\end{tabular*}
\begin{tabular*}{\textwidth}[tb]{p{0.1\textwidth}p{0.9\textwidth}}
	 DETCI\_FREEZE\_CORE\\ 

	 & Do freeze core orbitals? \\ 
\end{tabular*}
\begin{tabular*}{\textwidth}[tb]{p{0.3\textwidth}p{0.35\textwidth}p{0.35\textwidth}}
	   & {\bf Type:} boolean &  {\bf Default:} true\\
	 & & \\
\end{tabular*}
\begin{tabular*}{\textwidth}[tb]{p{0.1\textwidth}p{0.9\textwidth}}
	 DIAG\_METHOD\\ 

	 & This specifies which method is to be used in diagonalizing the Hamiltonian. The valid options are: RSP, to form the entire H matrix and diagonalize using libciomr to obtain all eigenvalues (n.b. requires HUGE memory); OLSEN, to use Olsen's preconditioned inverse subspace method (1990); MITRUSHENKOV, to use a 2x2 Olsen/Davidson method; and DAVIDSON (or SEM) to use Liu's Simultaneous Expansion Method, which is identical to the Davidson method if only one root is to be found. There also exists a SEM debugging mode, SEMTEST. The SEM method is the most robust, but it also requires 2(N*M)+1 CI vectors on disk, where N is the maximum number of iterations and M is the number of roots. \\ 
\end{tabular*}
\begin{tabular*}{\textwidth}[tb]{p{0.3\textwidth}p{0.35\textwidth}p{0.35\textwidth}}
	   & {\bf Type:} string &  {\bf Default:} SEM\\
	 & & \\
\end{tabular*}
\begin{tabular*}{\textwidth}[tb]{p{0.1\textwidth}p{0.9\textwidth}}
	 DIIS\\ 

	 & Do use DIIS extrapolation to accelerate CC convergence? \\ 
\end{tabular*}
\begin{tabular*}{\textwidth}[tb]{p{0.3\textwidth}p{0.35\textwidth}p{0.35\textwidth}}
	   & {\bf Type:} boolean &  {\bf Default:} true\\
	 & & \\
\end{tabular*}
\begin{tabular*}{\textwidth}[tb]{p{0.1\textwidth}p{0.9\textwidth}}
	 DIIS\_FREQ\\ 

	 & How often to do a DIIS extrapolation. 1 means do DIIS every iteration, 2 is every other iteration, etc. \\ 
\end{tabular*}
\begin{tabular*}{\textwidth}[tb]{p{0.3\textwidth}p{0.35\textwidth}p{0.35\textwidth}}
	   & {\bf Type:} integer &  {\bf Default:} 1\\
	 & & \\
\end{tabular*}
\begin{tabular*}{\textwidth}[tb]{p{0.1\textwidth}p{0.9\textwidth}}
	 DIIS\_MAX\_VECS\\ 

	 & Maximum number of error vectors stored for DIIS extrapolation \\ 
\end{tabular*}
\begin{tabular*}{\textwidth}[tb]{p{0.3\textwidth}p{0.35\textwidth}p{0.35\textwidth}}
	   & {\bf Type:} integer &  {\bf Default:} 5\\
	 & & \\
\end{tabular*}
\begin{tabular*}{\textwidth}[tb]{p{0.1\textwidth}p{0.9\textwidth}}
	 DIIS\_MIN\_VECS\\ 

	 & Minimum number of error vectors stored for DIIS extrapolation \\ 
\end{tabular*}
\begin{tabular*}{\textwidth}[tb]{p{0.3\textwidth}p{0.35\textwidth}p{0.35\textwidth}}
	   & {\bf Type:} integer &  {\bf Default:} 2\\
	 & & \\
\end{tabular*}
\begin{tabular*}{\textwidth}[tb]{p{0.1\textwidth}p{0.9\textwidth}}
	 DIIS\_START\_ITER\\ 

	 & Iteration at which to start using DIIS \\ 
\end{tabular*}
\begin{tabular*}{\textwidth}[tb]{p{0.3\textwidth}p{0.35\textwidth}p{0.35\textwidth}}
	   & {\bf Type:} integer &  {\bf Default:} 1\\
	 & & \\
\end{tabular*}
\begin{tabular*}{\textwidth}[tb]{p{0.1\textwidth}p{0.9\textwidth}}
	 DIPMOM\\ 

	 & Do compute the dipole moment? \\ 
\end{tabular*}
\begin{tabular*}{\textwidth}[tb]{p{0.3\textwidth}p{0.35\textwidth}p{0.35\textwidth}}
	   & {\bf Type:} boolean &  {\bf Default:} false\\
	 & & \\
\end{tabular*}
\begin{tabular*}{\textwidth}[tb]{p{0.1\textwidth}p{0.9\textwidth}}
	 EX\_LEVEL\\ 

	 & The CI excitation level \\ 
\end{tabular*}
\begin{tabular*}{\textwidth}[tb]{p{0.3\textwidth}p{0.35\textwidth}p{0.35\textwidth}}
	   & {\bf Type:} integer &  {\bf Default:} 2\\
	 & & \\
\end{tabular*}
\begin{tabular*}{\textwidth}[tb]{p{0.1\textwidth}p{0.9\textwidth}}
	 E\_CONVERGENCE\\ 

	 & Convergence criterion for energy. See the note at the beginning of Section \ref{keywords}. \\ 
\end{tabular*}
\begin{tabular*}{\textwidth}[tb]{p{0.3\textwidth}p{0.35\textwidth}p{0.35\textwidth}}
	   & {\bf Type:} double &  {\bf Default:} 1e-6\\
	 & & \\
\end{tabular*}
\begin{tabular*}{\textwidth}[tb]{p{0.1\textwidth}p{0.9\textwidth}}
	 FCI\\ 

	 & Do a full CI (FCI)? If TRUE, overrides the value of EX\_LEVEL \\ 
\end{tabular*}
\begin{tabular*}{\textwidth}[tb]{p{0.3\textwidth}p{0.35\textwidth}p{0.35\textwidth}}
	   & {\bf Type:} boolean &  {\bf Default:} false\\
	 & & \\
\end{tabular*}
\begin{tabular*}{\textwidth}[tb]{p{0.1\textwidth}p{0.9\textwidth}}
	 FOLLOW\_ROOT\\ 

	 & The root to write out the two-particle density matrix for (the one-particle density matrices are written for all roots). Useful for a state-specific CASSCF or CI optimization on an excited state. \\ 
\end{tabular*}
\begin{tabular*}{\textwidth}[tb]{p{0.3\textwidth}p{0.35\textwidth}p{0.35\textwidth}}
	   & {\bf Type:} integer &  {\bf Default:} 1\\
	 & & \\
\end{tabular*}
\begin{tabular*}{\textwidth}[tb]{p{0.1\textwidth}p{0.9\textwidth}}
	 ICORE\\ 

	 & Specifies how to handle buffering of CI vectors. A value of 0 makes the program perform I/O one RAS subblock at a time; 1 uses entire CI vectors at a time; and 2 uses one irrep block at a time. Values of 0 or 2 cause some inefficiency in the I/O (requiring multiple reads of the C vector when constructing H in the iterative subspace if DIAG\_METHOD = SEM), but require less core memory. \\ 
\end{tabular*}
\begin{tabular*}{\textwidth}[tb]{p{0.3\textwidth}p{0.35\textwidth}p{0.35\textwidth}}
	   & {\bf Type:} integer &  {\bf Default:} 1\\
	 & & \\
\end{tabular*}
\begin{tabular*}{\textwidth}[tb]{p{0.1\textwidth}p{0.9\textwidth}}
	 ISTOP\\ 

	 & Do stop DETCI after string information is formed and before integrals are read? \\ 
\end{tabular*}
\begin{tabular*}{\textwidth}[tb]{p{0.3\textwidth}p{0.35\textwidth}p{0.35\textwidth}}
	   & {\bf Type:} boolean &  {\bf Default:} false\\
	 & & \\
\end{tabular*}
\begin{tabular*}{\textwidth}[tb]{p{0.1\textwidth}p{0.9\textwidth}}
	 LSE\\ 

	 & Do use least-squares extrapolation in iterative solution of CI vector? \\ 
\end{tabular*}
\begin{tabular*}{\textwidth}[tb]{p{0.3\textwidth}p{0.35\textwidth}p{0.35\textwidth}}
	   & {\bf Type:} boolean &  {\bf Default:} false\\
	 & & \\
\end{tabular*}
\begin{tabular*}{\textwidth}[tb]{p{0.1\textwidth}p{0.9\textwidth}}
	 LSE\_COLLAPSE\\ 

	 & Number of iterations between least-squares extrapolations \\ 
\end{tabular*}
\begin{tabular*}{\textwidth}[tb]{p{0.3\textwidth}p{0.35\textwidth}p{0.35\textwidth}}
	   & {\bf Type:} integer &  {\bf Default:} 3\\
	 & & \\
\end{tabular*}
\begin{tabular*}{\textwidth}[tb]{p{0.1\textwidth}p{0.9\textwidth}}
	 LSE\_TOLERANCE\\ 

	 & Minimum converged energy for least-squares extrapolation to be performed \\ 
\end{tabular*}
\begin{tabular*}{\textwidth}[tb]{p{0.3\textwidth}p{0.35\textwidth}p{0.35\textwidth}}
	   & {\bf Type:} double &  {\bf Default:} 3\\
	 & & \\
\end{tabular*}
\begin{tabular*}{\textwidth}[tb]{p{0.1\textwidth}p{0.9\textwidth}}
	 MAXITER\\ 

	 & Maximum number of iterations to diagonalize the Hamiltonian \\ 
\end{tabular*}
\begin{tabular*}{\textwidth}[tb]{p{0.3\textwidth}p{0.35\textwidth}p{0.35\textwidth}}
	   & {\bf Type:} integer &  {\bf Default:} 12\\
	 & & \\
\end{tabular*}
\begin{tabular*}{\textwidth}[tb]{p{0.1\textwidth}p{0.9\textwidth}}
	 MAX\_NUM\_VECS\\ 

	 & Gives the maximum number of Davidson subspace vectors which can be held on disk for the CI coefficient and sigma vectors. (There is one H(diag) vector and the number of D vectors is equal to the number of roots). When the number of vectors on disk reaches the value of MAX\_NUM\_VECS, the Davidson subspace will be collapsed to COLLAPSE\_SIZE vectors for each root. This is very helpful for saving disk space. Defaults to MAXITER * NUM\_ROOTS + NUM\_INIT\_VECS. \\ 
\end{tabular*}
\begin{tabular*}{\textwidth}[tb]{p{0.3\textwidth}p{0.35\textwidth}p{0.35\textwidth}}
	   & {\bf Type:} integer &  {\bf Default:} 0\\
	 & & \\
\end{tabular*}
\begin{tabular*}{\textwidth}[tb]{p{0.1\textwidth}p{0.9\textwidth}}
	 MPN\\ 

	 & Do compute the MPn series out to kth order where k is determined by MAX\_NUM\_VECS? For open-shell systems (REF=ROHF, WFN = ZAPTN), DETCI will compute the ZAPTn series. GUESS\_VECTOR must be set to UNIT, HD\_OTF must be set to TRUE, and HD\_AVG must be set to orb\_ener; these should happen by default for MPN=TRUE. \\ 
\end{tabular*}
\begin{tabular*}{\textwidth}[tb]{p{0.3\textwidth}p{0.35\textwidth}p{0.35\textwidth}}
	   & {\bf Type:} boolean &  {\bf Default:} false\\
	 & & \\
\end{tabular*}
\begin{tabular*}{\textwidth}[tb]{p{0.1\textwidth}p{0.9\textwidth}}
	 MS0\\ 

	 & Do use the $M_s = 0$ component of the state? Defaults to TRUE if closed-shell and FALSE otherwise. Related to the S option. \\ 
\end{tabular*}
\begin{tabular*}{\textwidth}[tb]{p{0.3\textwidth}p{0.35\textwidth}p{0.35\textwidth}}
	   & {\bf Type:} boolean &  {\bf Default:} false\\
	 & & \\
\end{tabular*}
\begin{tabular*}{\textwidth}[tb]{p{0.1\textwidth}p{0.9\textwidth}}
	 NAT\_ORBS\_WRITE\\ 

	 & Do write the natural orbitals? \\ 
\end{tabular*}
\begin{tabular*}{\textwidth}[tb]{p{0.3\textwidth}p{0.35\textwidth}p{0.35\textwidth}}
	   & {\bf Type:} boolean &  {\bf Default:} false\\
	 & & \\
\end{tabular*}
\begin{tabular*}{\textwidth}[tb]{p{0.1\textwidth}p{0.9\textwidth}}
	 NAT\_ORBS\_WRITE\_ROOT\\ 

	 & Sets the root number for which CI natural orbitals are written to PSIF\_CHKPT. The default value is 1 (lowest root). \\ 
\end{tabular*}
\begin{tabular*}{\textwidth}[tb]{p{0.3\textwidth}p{0.35\textwidth}p{0.35\textwidth}}
	   & {\bf Type:} integer &  {\bf Default:} 1\\
	 & & \\
\end{tabular*}
\begin{tabular*}{\textwidth}[tb]{p{0.1\textwidth}p{0.9\textwidth}}
	 NUM\_AMPS\_PRINT\\ 

	 & Number of important CC amplitudes per excitation level to print. CC analog to NUM\_DETS\_PRINT \\ 
\end{tabular*}
\begin{tabular*}{\textwidth}[tb]{p{0.3\textwidth}p{0.35\textwidth}p{0.35\textwidth}}
	   & {\bf Type:} integer &  {\bf Default:} 10\\
	 & & \\
\end{tabular*}
\begin{tabular*}{\textwidth}[tb]{p{0.1\textwidth}p{0.9\textwidth}}
	 NUM\_DETS\_PRINT\\ 

	 & Number of important determinants to print \\ 
\end{tabular*}
\begin{tabular*}{\textwidth}[tb]{p{0.3\textwidth}p{0.35\textwidth}p{0.35\textwidth}}
	   & {\bf Type:} integer &  {\bf Default:} 20\\
	 & & \\
\end{tabular*}
\begin{tabular*}{\textwidth}[tb]{p{0.1\textwidth}p{0.9\textwidth}}
	 NUM\_ROOTS\\ 

	 & number of CI roots to find \\ 
\end{tabular*}
\begin{tabular*}{\textwidth}[tb]{p{0.3\textwidth}p{0.35\textwidth}p{0.35\textwidth}}
	   & {\bf Type:} integer &  {\bf Default:} 1\\
	 & & \\
\end{tabular*}
\begin{tabular*}{\textwidth}[tb]{p{0.1\textwidth}p{0.9\textwidth}}
	 NUM\_THREADS\\ 

	 & Number of threads \\ 
\end{tabular*}
\begin{tabular*}{\textwidth}[tb]{p{0.3\textwidth}p{0.35\textwidth}p{0.35\textwidth}}
	   & {\bf Type:} integer &  {\bf Default:} 1\\
	 & & \\
\end{tabular*}
\begin{tabular*}{\textwidth}[tb]{p{0.1\textwidth}p{0.9\textwidth}}
	 NUM\_VECS\_WRITE\\ 

	 & Number of vectors to export \\ 
\end{tabular*}
\begin{tabular*}{\textwidth}[tb]{p{0.3\textwidth}p{0.35\textwidth}p{0.35\textwidth}}
	   & {\bf Type:} integer &  {\bf Default:} 1\\
	 & & \\
\end{tabular*}
\begin{tabular*}{\textwidth}[tb]{p{0.1\textwidth}p{0.9\textwidth}}
	 OPDM\\ 

	 & Do compute one-particle density matrix if not otherwise required? \\ 
\end{tabular*}
\begin{tabular*}{\textwidth}[tb]{p{0.3\textwidth}p{0.35\textwidth}p{0.35\textwidth}}
	   & {\bf Type:} boolean &  {\bf Default:} false\\
	 & & \\
\end{tabular*}
\begin{tabular*}{\textwidth}[tb]{p{0.1\textwidth}p{0.9\textwidth}}
	 OPDM\_AVG\\ 

	 & Do average the OPDM over several roots in order to obtain a state-average one-particle density matrix? This density matrix can be diagonalized to obtain the CI natural orbitals. \\ 
\end{tabular*}
\begin{tabular*}{\textwidth}[tb]{p{0.3\textwidth}p{0.35\textwidth}p{0.35\textwidth}}
	   & {\bf Type:} boolean &  {\bf Default:} false\\
	 & & \\
\end{tabular*}
\begin{tabular*}{\textwidth}[tb]{p{0.1\textwidth}p{0.9\textwidth}}
	 OPDM\_PRINT\\ 

	 & Do print the one-particle density matrix for each root? \\ 
\end{tabular*}
\begin{tabular*}{\textwidth}[tb]{p{0.3\textwidth}p{0.35\textwidth}p{0.35\textwidth}}
	   & {\bf Type:} boolean &  {\bf Default:} false\\
	 & & \\
\end{tabular*}
\begin{tabular*}{\textwidth}[tb]{p{0.1\textwidth}p{0.9\textwidth}}
	 PRECONDITIONER\\ 

	 & This specifies the type of preconditioner to use in the selected diagonalization method. The valid options are: DAVIDSON which approximates the Hamiltonian matrix by the diagonal elements; H0BLOCK\_INV which uses an exact Hamiltonian of H0\_BLOCKSIZE and explicitly inverts it; GEN\_DAVIDSON which does a spectral decomposition of H0BLOCK; ITER\_INV using an iterative approach to obtain the correction vector of H0BLOCK. The H0BLOCK\_INV, GEN\_DAVIDSON, and ITER\_INV approaches are all formally equivalent but the ITER\_INV is less computationally expensive. Default is DAVIDSON. \\ 
\end{tabular*}
\begin{tabular*}{\textwidth}[tb]{p{0.3\textwidth}p{0.35\textwidth}p{0.35\textwidth}}
	   & {\bf Type:} string &  {\bf Default:} DAVIDSON\\
	 & & \\
\end{tabular*}
\begin{tabular*}{\textwidth}[tb]{p{0.1\textwidth}p{0.9\textwidth}}
	 RAS34\_MAX\\ 

	 & maximum number of electrons in RAS III + IV \\ 
\end{tabular*}
\begin{tabular*}{\textwidth}[tb]{p{0.3\textwidth}p{0.35\textwidth}p{0.35\textwidth}}
	   & {\bf Type:} integer &  {\bf Default:} -1\\
	 & & \\
\end{tabular*}
\begin{tabular*}{\textwidth}[tb]{p{0.1\textwidth}p{0.9\textwidth}}
	 RAS3\_MAX\\ 

	 & maximum number of electrons in RAS III \\ 
\end{tabular*}
\begin{tabular*}{\textwidth}[tb]{p{0.3\textwidth}p{0.35\textwidth}p{0.35\textwidth}}
	   & {\bf Type:} integer &  {\bf Default:} -1\\
	 & & \\
\end{tabular*}
\begin{tabular*}{\textwidth}[tb]{p{0.1\textwidth}p{0.9\textwidth}}
	 RAS4\_MAX\\ 

	 & maximum number of electrons in RAS IV \\ 
\end{tabular*}
\begin{tabular*}{\textwidth}[tb]{p{0.3\textwidth}p{0.35\textwidth}p{0.35\textwidth}}
	   & {\bf Type:} integer &  {\bf Default:} -1\\
	 & & \\
\end{tabular*}
\begin{tabular*}{\textwidth}[tb]{p{0.1\textwidth}p{0.9\textwidth}}
	 REFERENCE\\ 

	 & Reference wavefunction type \\ 

	  & {\bf Possible Values:} RHF, ROHF \\ 
\end{tabular*}
\begin{tabular*}{\textwidth}[tb]{p{0.3\textwidth}p{0.35\textwidth}p{0.35\textwidth}}
	   & {\bf Type:} string &  {\bf Default:} RHF\\
	 & & \\
\end{tabular*}
\begin{tabular*}{\textwidth}[tb]{p{0.1\textwidth}p{0.9\textwidth}}
	 RESTART\\ 

	 & Do result a DETCI iteration that terminated prematurely? It assumes that the CI and sigma vectors are on disk; the number of vectors specified by RESTART\_VECS is collapsed down to one vector per root. \\ 
\end{tabular*}
\begin{tabular*}{\textwidth}[tb]{p{0.3\textwidth}p{0.35\textwidth}p{0.35\textwidth}}
	   & {\bf Type:} boolean &  {\bf Default:} false\\
	 & & \\
\end{tabular*}
\begin{tabular*}{\textwidth}[tb]{p{0.1\textwidth}p{0.9\textwidth}}
	 RESTRICTED\_DOCC\\ 

	 & An array giving the number of restricted doubly-occupied orbitals per irrep (not excited in CI wavefunctions, but orbitals can be optimized in MCSCF) \\ 
\end{tabular*}
\begin{tabular*}{\textwidth}[tb]{p{0.3\textwidth}p{0.35\textwidth}p{0.35\textwidth}}
	   & {\bf Type:} array &  {\bf Default:} No Default\\
	 & & \\
\end{tabular*}
\begin{tabular*}{\textwidth}[tb]{p{0.1\textwidth}p{0.9\textwidth}}
	 RESTRICTED\_UOCC\\ 

	 & An array giving the number of restricted unoccupied orbitals per irrep (not occupied in CI wavefunctions, but orbitals can be optimized in MCSCF) \\ 
\end{tabular*}
\begin{tabular*}{\textwidth}[tb]{p{0.3\textwidth}p{0.35\textwidth}p{0.35\textwidth}}
	   & {\bf Type:} array &  {\bf Default:} No Default\\
	 & & \\
\end{tabular*}
\begin{tabular*}{\textwidth}[tb]{p{0.1\textwidth}p{0.9\textwidth}}
	 R\_CONVERGENCE\\ 

	 & Convergence criterion for CI residual vector in the Davidson algorithm (RMS error). The default is 1e-4 for energies and 1e-7 for gradients. See the note at the beginning of Section \ref{keywords}. \\ 
\end{tabular*}
\begin{tabular*}{\textwidth}[tb]{p{0.3\textwidth}p{0.35\textwidth}p{0.35\textwidth}}
	   & {\bf Type:} double &  {\bf Default:} 1e-4\\
	 & & \\
\end{tabular*}
\begin{tabular*}{\textwidth}[tb]{p{0.1\textwidth}p{0.9\textwidth}}
	 S\\ 

	 & The value of the spin quantum number S is given by this option. The default is determined by the value of the multiplicity. This is used for two things: (1) determining the phase of the redundant half of the CI vector when the Ms=0 component is used (i.e., Ms0 = TRUE), and (2) making sure the guess vector has the desired value of $<S^2>$ (if CALC\_SSQ is TRUE and ICORE=1). \\ 
\end{tabular*}
\begin{tabular*}{\textwidth}[tb]{p{0.3\textwidth}p{0.35\textwidth}p{0.35\textwidth}}
	   & {\bf Type:} double &  {\bf Default:} 0.0\\
	 & & \\
\end{tabular*}
\begin{tabular*}{\textwidth}[tb]{p{0.1\textwidth}p{0.9\textwidth}}
	 S\_SQUARED\\ 

	 & Do calculate the value of $<S^2>$ for each root? \\ 
\end{tabular*}
\begin{tabular*}{\textwidth}[tb]{p{0.3\textwidth}p{0.35\textwidth}p{0.35\textwidth}}
	   & {\bf Type:} boolean &  {\bf Default:} false\\
	 & & \\
\end{tabular*}
\begin{tabular*}{\textwidth}[tb]{p{0.1\textwidth}p{0.9\textwidth}}
	 TDM\\ 

	 & Do compute the transition density? Note: only transition densities between roots of the same symmetry will be evaluated. DETCI does not compute states of different irreps within the same computation; to do this, lower the symmetry of the computation. \\ 
\end{tabular*}
\begin{tabular*}{\textwidth}[tb]{p{0.3\textwidth}p{0.35\textwidth}p{0.35\textwidth}}
	   & {\bf Type:} boolean &  {\bf Default:} false\\
	 & & \\
\end{tabular*}
\begin{tabular*}{\textwidth}[tb]{p{0.1\textwidth}p{0.9\textwidth}}
	 TDM\_PRINT\\ 

	 & Do print the transition density? \\ 
\end{tabular*}
\begin{tabular*}{\textwidth}[tb]{p{0.3\textwidth}p{0.35\textwidth}p{0.35\textwidth}}
	   & {\bf Type:} boolean &  {\bf Default:} false\\
	 & & \\
\end{tabular*}
\begin{tabular*}{\textwidth}[tb]{p{0.1\textwidth}p{0.9\textwidth}}
	 TDM\_WRITE\\ 

	 & Do write the transition density? \\ 
\end{tabular*}
\begin{tabular*}{\textwidth}[tb]{p{0.3\textwidth}p{0.35\textwidth}p{0.35\textwidth}}
	   & {\bf Type:} boolean &  {\bf Default:} false\\
	 & & \\
\end{tabular*}
\begin{tabular*}{\textwidth}[tb]{p{0.1\textwidth}p{0.9\textwidth}}
	 TPDM\\ 

	 & Do compute two-particle density matrix if not otherwise required? \\ 
\end{tabular*}
\begin{tabular*}{\textwidth}[tb]{p{0.3\textwidth}p{0.35\textwidth}p{0.35\textwidth}}
	   & {\bf Type:} boolean &  {\bf Default:} false\\
	 & & \\
\end{tabular*}
\begin{tabular*}{\textwidth}[tb]{p{0.1\textwidth}p{0.9\textwidth}}
	 TPDM\_PRINT\\ 

	 & Do print the two-particle density matrix? (Warning: large tensor) \\ 
\end{tabular*}
\begin{tabular*}{\textwidth}[tb]{p{0.3\textwidth}p{0.35\textwidth}p{0.35\textwidth}}
	   & {\bf Type:} boolean &  {\bf Default:} false\\
	 & & \\
\end{tabular*}
\begin{tabular*}{\textwidth}[tb]{p{0.1\textwidth}p{0.9\textwidth}}
	 UPDATE\\ 

	 & DAVIDSON employs the standard DAVIDSON update or correction vector formula, while OLSEN uses the OLSEN correction vector. Default is DAVIDSON. \\ 

	  & {\bf Possible Values:} DAVIDSON, OLSEN \\ 
\end{tabular*}
\begin{tabular*}{\textwidth}[tb]{p{0.3\textwidth}p{0.35\textwidth}p{0.35\textwidth}}
	   & {\bf Type:} string &  {\bf Default:} DAVIDSON\\
	 & & \\
\end{tabular*}
\begin{tabular*}{\textwidth}[tb]{p{0.1\textwidth}p{0.9\textwidth}}
	 VAL\_EX\_LEVEL\\ 

	 & In a RAS CI, this is the additional excitation level for allowing electrons out of RAS I into RAS II. The maximum number of holes in RAS I is therefore EX\_LEVEL + VAL\_EX\_LEVEL. \\ 
\end{tabular*}
\begin{tabular*}{\textwidth}[tb]{p{0.3\textwidth}p{0.35\textwidth}p{0.35\textwidth}}
	   & {\bf Type:} integer &  {\bf Default:} 0\\
	 & & \\
\end{tabular*}
\begin{tabular*}{\textwidth}[tb]{p{0.1\textwidth}p{0.9\textwidth}}
	 VECS\_WRITE\\ 

	 & Do store converged vector(s) at the end of the run? The vector(s) is(are) stored in a transparent format such that other programs can use it easily. The format is specified in src/lib/libqt/slaterdset.h. \\ 
\end{tabular*}
\begin{tabular*}{\textwidth}[tb]{p{0.3\textwidth}p{0.35\textwidth}p{0.35\textwidth}}
	   & {\bf Type:} boolean &  {\bf Default:} false\\
	 & & \\
\end{tabular*}

\subsection{DFCC}

{\normalsize Performs density-fitted coupled cluster computations.}\\
\begin{tabular*}{\textwidth}[tb]{c}
	  \\ 
\end{tabular*}
\begin{tabular*}{\textwidth}[tb]{p{0.1\textwidth}p{0.9\textwidth}}
	 BASIS\\ 

	 & Primary basis set \\ 
\end{tabular*}
\begin{tabular*}{\textwidth}[tb]{p{0.3\textwidth}p{0.35\textwidth}p{0.35\textwidth}}
	   & {\bf Type:} string &  {\bf Default:} NONE\\
	 & & \\
\end{tabular*}
\begin{tabular*}{\textwidth}[tb]{p{0.1\textwidth}p{0.9\textwidth}}
	 DEALIAS\_BASIS\_CC\\ 

	 & Dealias basis for PS integrals \\ 
\end{tabular*}
\begin{tabular*}{\textwidth}[tb]{p{0.3\textwidth}p{0.35\textwidth}p{0.35\textwidth}}
	   & {\bf Type:} string &  {\bf Default:} No Default\\
	 & & \\
\end{tabular*}
\begin{tabular*}{\textwidth}[tb]{p{0.1\textwidth}p{0.9\textwidth}}
	 DEALIAS\_BETA\\ 

	 & Dealias basis beta parameter \\ 
\end{tabular*}
\begin{tabular*}{\textwidth}[tb]{p{0.3\textwidth}p{0.35\textwidth}p{0.35\textwidth}}
	   & {\bf Type:} double &  {\bf Default:} 3.5\\
	 & & \\
\end{tabular*}
\begin{tabular*}{\textwidth}[tb]{p{0.1\textwidth}p{0.9\textwidth}}
	 DEALIAS\_DELTA\\ 

	 & Dealias basis delta parameter \\ 
\end{tabular*}
\begin{tabular*}{\textwidth}[tb]{p{0.3\textwidth}p{0.35\textwidth}p{0.35\textwidth}}
	   & {\bf Type:} double &  {\bf Default:} 2.0\\
	 & & \\
\end{tabular*}
\begin{tabular*}{\textwidth}[tb]{p{0.1\textwidth}p{0.9\textwidth}}
	 DEALIAS\_N\_CAP\\ 

	 & Dealias basis N cap parameter \\ 
\end{tabular*}
\begin{tabular*}{\textwidth}[tb]{p{0.3\textwidth}p{0.35\textwidth}p{0.35\textwidth}}
	   & {\bf Type:} integer &  {\bf Default:} 1\\
	 & & \\
\end{tabular*}
\begin{tabular*}{\textwidth}[tb]{p{0.1\textwidth}p{0.9\textwidth}}
	 DEALIAS\_N\_CORE\\ 

	 & Dealias basis N core parameter \\ 
\end{tabular*}
\begin{tabular*}{\textwidth}[tb]{p{0.3\textwidth}p{0.35\textwidth}p{0.35\textwidth}}
	   & {\bf Type:} integer &  {\bf Default:} 1\\
	 & & \\
\end{tabular*}
\begin{tabular*}{\textwidth}[tb]{p{0.1\textwidth}p{0.9\textwidth}}
	 DEALIAS\_N\_DIFFUSE\\ 

	 & Dealias basis N diffuse parameter \\ 
\end{tabular*}
\begin{tabular*}{\textwidth}[tb]{p{0.3\textwidth}p{0.35\textwidth}p{0.35\textwidth}}
	   & {\bf Type:} integer &  {\bf Default:} 1\\
	 & & \\
\end{tabular*}
\begin{tabular*}{\textwidth}[tb]{p{0.1\textwidth}p{0.9\textwidth}}
	 DEALIAS\_N\_INTERCALATER\\ 

	 & Dealias basis N intercalater parameter \\ 
\end{tabular*}
\begin{tabular*}{\textwidth}[tb]{p{0.3\textwidth}p{0.35\textwidth}p{0.35\textwidth}}
	   & {\bf Type:} integer &  {\bf Default:} 1\\
	 & & \\
\end{tabular*}
\begin{tabular*}{\textwidth}[tb]{p{0.1\textwidth}p{0.9\textwidth}}
	 DEALIAS\_N\_L\\ 

	 & Dealias basis highest delta l parameter \\ 
\end{tabular*}
\begin{tabular*}{\textwidth}[tb]{p{0.3\textwidth}p{0.35\textwidth}p{0.35\textwidth}}
	   & {\bf Type:} integer &  {\bf Default:} 1\\
	 & & \\
\end{tabular*}
\begin{tabular*}{\textwidth}[tb]{p{0.1\textwidth}p{0.9\textwidth}}
	 DENOMINATOR\_ALGORITHM\\ 

	 & Denominator algorithm for PT methods \\ 

	  & {\bf Possible Values:} LAPLACE, CHOLESKY \\ 
\end{tabular*}
\begin{tabular*}{\textwidth}[tb]{p{0.3\textwidth}p{0.35\textwidth}p{0.35\textwidth}}
	   & {\bf Type:} string &  {\bf Default:} LAPLACE\\
	 & & \\
\end{tabular*}
\begin{tabular*}{\textwidth}[tb]{p{0.1\textwidth}p{0.9\textwidth}}
	 DENOMINATOR\_DELTA\\ 

	 & Maximum denominator error allowed (Max error norm in Delta tensor) \\ 
\end{tabular*}
\begin{tabular*}{\textwidth}[tb]{p{0.3\textwidth}p{0.35\textwidth}p{0.35\textwidth}}
	   & {\bf Type:} double &  {\bf Default:} 1.0e-6\\
	 & & \\
\end{tabular*}
\begin{tabular*}{\textwidth}[tb]{p{0.1\textwidth}p{0.9\textwidth}}
	 DF\_BASIS\_CC\\ 

	 & Auxiliary basis set for density fitting MO integrals. Defaults to BASIS-RI. \\ 
\end{tabular*}
\begin{tabular*}{\textwidth}[tb]{p{0.3\textwidth}p{0.35\textwidth}p{0.35\textwidth}}
	   & {\bf Type:} string &  {\bf Default:} NONE\\
	 & & \\
\end{tabular*}
\begin{tabular*}{\textwidth}[tb]{p{0.1\textwidth}p{0.9\textwidth}}
	 DIIS\\ 

	 & Do use DIIS extrapolation to accelerate convergence? \\ 
\end{tabular*}
\begin{tabular*}{\textwidth}[tb]{p{0.3\textwidth}p{0.35\textwidth}p{0.35\textwidth}}
	   & {\bf Type:} boolean &  {\bf Default:} true\\
	 & & \\
\end{tabular*}
\begin{tabular*}{\textwidth}[tb]{p{0.1\textwidth}p{0.9\textwidth}}
	 DIIS\_MAX\_VECS\\ 

	 & Maximum number of error vectors stored for DIIS extrapolation \\ 
\end{tabular*}
\begin{tabular*}{\textwidth}[tb]{p{0.3\textwidth}p{0.35\textwidth}p{0.35\textwidth}}
	   & {\bf Type:} integer &  {\bf Default:} 6\\
	 & & \\
\end{tabular*}
\begin{tabular*}{\textwidth}[tb]{p{0.1\textwidth}p{0.9\textwidth}}
	 DIIS\_MIN\_VECS\\ 

	 & Minimum number of error vectors stored for DIIS extrapolation \\ 
\end{tabular*}
\begin{tabular*}{\textwidth}[tb]{p{0.3\textwidth}p{0.35\textwidth}p{0.35\textwidth}}
	   & {\bf Type:} integer &  {\bf Default:} 2\\
	 & & \\
\end{tabular*}
\begin{tabular*}{\textwidth}[tb]{p{0.1\textwidth}p{0.9\textwidth}}
	 E\_CONVERGENCE\\ 

	 & Convergence criterion for CC energy. See the note at the beginning of Section \ref{keywords}. \\ 
\end{tabular*}
\begin{tabular*}{\textwidth}[tb]{p{0.3\textwidth}p{0.35\textwidth}p{0.35\textwidth}}
	   & {\bf Type:} double &  {\bf Default:} 1e-8\\
	 & & \\
\end{tabular*}
\begin{tabular*}{\textwidth}[tb]{p{0.1\textwidth}p{0.9\textwidth}}
	 FITTING\_COND\\ 

	 & Desired Fitting condition (inverse of max condition number) \\ 
\end{tabular*}
\begin{tabular*}{\textwidth}[tb]{p{0.3\textwidth}p{0.35\textwidth}p{0.35\textwidth}}
	   & {\bf Type:} double &  {\bf Default:} 1.0e-10\\
	 & & \\
\end{tabular*}
\begin{tabular*}{\textwidth}[tb]{p{0.1\textwidth}p{0.9\textwidth}}
	 FITTING\_TYPE\\ 

	 & Fitting metric algorithm \\ 

	  & {\bf Possible Values:} EIG, CHOLESKY, QR \\ 
\end{tabular*}
\begin{tabular*}{\textwidth}[tb]{p{0.3\textwidth}p{0.35\textwidth}p{0.35\textwidth}}
	   & {\bf Type:} string &  {\bf Default:} EIG\\
	 & & \\
\end{tabular*}
\begin{tabular*}{\textwidth}[tb]{p{0.1\textwidth}p{0.9\textwidth}}
	 INTS\_TOLERANCE\\ 

	 & Minimum absolute value below which integrals are neglected. See the note at the beginning of Section \ref{keywords}. \\ 
\end{tabular*}
\begin{tabular*}{\textwidth}[tb]{p{0.3\textwidth}p{0.35\textwidth}p{0.35\textwidth}}
	   & {\bf Type:} double &  {\bf Default:} 0.0\\
	 & & \\
\end{tabular*}
\begin{tabular*}{\textwidth}[tb]{p{0.1\textwidth}p{0.9\textwidth}}
	 MAXITER\\ 

	 & Maximum number iterations \\ 
\end{tabular*}
\begin{tabular*}{\textwidth}[tb]{p{0.3\textwidth}p{0.35\textwidth}p{0.35\textwidth}}
	   & {\bf Type:} integer &  {\bf Default:} 40\\
	 & & \\
\end{tabular*}
\begin{tabular*}{\textwidth}[tb]{p{0.1\textwidth}p{0.9\textwidth}}
	 MP2\_ALGORITHM\\ 

	 & MP2 Algorithm: \begin{tabular}{ccc} Algorithm Keyword & MP2J & MP2K \\ \hline MP2 & MP2 & MP2 \\ DF & DF & DF \\ PS & PS & PS \\ PS1 & DF & PS/DF \\ PS2 & DF & PS/PS \\ PS3 & PS & PS/DF \\ PS4 & PS & PS/PS \\ TEST\_DENOM & Test & Test \\ TEST\_PS & Test & Test \\ TEST\_PS\_OMEGA & Test & Test \\ TEST\_DPS\_OMEGA & Test & Test \\ TEST\_DF & Test & Test \\ \end{tabular} \\ 

	  & {\bf Possible Values:} MP2, DF, PS, PS1, PS2, PS3, PS4, TEST\_DENOM, TEST\_PS, TEST\_PS\_OMEGA, TEST\_DPS\_OMEGA, TEST\_DF \\ 
\end{tabular*}
\begin{tabular*}{\textwidth}[tb]{p{0.3\textwidth}p{0.35\textwidth}p{0.35\textwidth}}
	   & {\bf Type:} string &  {\bf Default:} DF\\
	 & & \\
\end{tabular*}
\begin{tabular*}{\textwidth}[tb]{p{0.1\textwidth}p{0.9\textwidth}}
	 MP2\_OS\_SCALE\\ 

	 & OS Scale \\ 
\end{tabular*}
\begin{tabular*}{\textwidth}[tb]{p{0.3\textwidth}p{0.35\textwidth}p{0.35\textwidth}}
	   & {\bf Type:} double &  {\bf Default:} 6.0/5.0\\
	 & & \\
\end{tabular*}
\begin{tabular*}{\textwidth}[tb]{p{0.1\textwidth}p{0.9\textwidth}}
	 MP2\_SS\_SCALE\\ 

	 & SS Scale \\ 
\end{tabular*}
\begin{tabular*}{\textwidth}[tb]{p{0.3\textwidth}p{0.35\textwidth}p{0.35\textwidth}}
	   & {\bf Type:} double &  {\bf Default:} 1.0/3.0\\
	 & & \\
\end{tabular*}
\begin{tabular*}{\textwidth}[tb]{p{0.1\textwidth}p{0.9\textwidth}}
	 PS\_ALPHA\\ 

	 & Pseudospectral partition alpha \\ 
\end{tabular*}
\begin{tabular*}{\textwidth}[tb]{p{0.3\textwidth}p{0.35\textwidth}p{0.35\textwidth}}
	   & {\bf Type:} double &  {\bf Default:} 1.0\\
	 & & \\
\end{tabular*}
\begin{tabular*}{\textwidth}[tb]{p{0.1\textwidth}p{0.9\textwidth}}
	 PS\_BASIS\_TOLERANCE\\ 

	 & The DFT basis cutoff. See the note at the beginning of Section \ref{keywords}. \\ 
\end{tabular*}
\begin{tabular*}{\textwidth}[tb]{p{0.3\textwidth}p{0.35\textwidth}p{0.35\textwidth}}
	   & {\bf Type:} double &  {\bf Default:} 0.0\\
	 & & \\
\end{tabular*}
\begin{tabular*}{\textwidth}[tb]{p{0.1\textwidth}p{0.9\textwidth}}
	 PS\_BS\_RADIUS\_ALPHA\\ 

	 & Factor for effective BS radius in radial grid \\ 
\end{tabular*}
\begin{tabular*}{\textwidth}[tb]{p{0.3\textwidth}p{0.35\textwidth}p{0.35\textwidth}}
	   & {\bf Type:} double &  {\bf Default:} 1.0\\
	 & & \\
\end{tabular*}
\begin{tabular*}{\textwidth}[tb]{p{0.1\textwidth}p{0.9\textwidth}}
	 PS\_FITTING\_ALGORITHM\\ 

	 & Fitting algorithm to use for pseudospectral \\ 

	  & {\bf Possible Values:} DEALIASED, RENORMALIZED, QUADRATURE \\ 
\end{tabular*}
\begin{tabular*}{\textwidth}[tb]{p{0.3\textwidth}p{0.35\textwidth}p{0.35\textwidth}}
	   & {\bf Type:} string &  {\bf Default:} CONDITIONED\\
	 & & \\
\end{tabular*}
\begin{tabular*}{\textwidth}[tb]{p{0.1\textwidth}p{0.9\textwidth}}
	 PS\_GRID\_FILE\\ 

	 & Filename to read grid from \\ 
\end{tabular*}
\begin{tabular*}{\textwidth}[tb]{p{0.3\textwidth}p{0.35\textwidth}p{0.35\textwidth}}
	   & {\bf Type:} string &  {\bf Default:} No Default\\
	 & & \\
\end{tabular*}
\begin{tabular*}{\textwidth}[tb]{p{0.1\textwidth}p{0.9\textwidth}}
	 PS\_GRID\_PATH\\ 

	 & File path to read grids from \\ 
\end{tabular*}
\begin{tabular*}{\textwidth}[tb]{p{0.3\textwidth}p{0.35\textwidth}p{0.35\textwidth}}
	   & {\bf Type:} string &  {\bf Default:} No Default\\
	 & & \\
\end{tabular*}
\begin{tabular*}{\textwidth}[tb]{p{0.1\textwidth}p{0.9\textwidth}}
	 PS\_MAX\_POINTS\\ 

	 & The number of grid points per evaluation block \\ 
\end{tabular*}
\begin{tabular*}{\textwidth}[tb]{p{0.3\textwidth}p{0.35\textwidth}p{0.35\textwidth}}
	   & {\bf Type:} integer &  {\bf Default:} 5000\\
	 & & \\
\end{tabular*}
\begin{tabular*}{\textwidth}[tb]{p{0.1\textwidth}p{0.9\textwidth}}
	 PS\_MIN\_POINTS\\ 

	 & The number of grid points per evaluation block \\ 
\end{tabular*}
\begin{tabular*}{\textwidth}[tb]{p{0.3\textwidth}p{0.35\textwidth}p{0.35\textwidth}}
	   & {\bf Type:} integer &  {\bf Default:} 0\\
	 & & \\
\end{tabular*}
\begin{tabular*}{\textwidth}[tb]{p{0.1\textwidth}p{0.9\textwidth}}
	 PS\_MIN\_S\_DEALIAS\\ 

	 & Minumum eigenvalue for dealias basis \\ 
\end{tabular*}
\begin{tabular*}{\textwidth}[tb]{p{0.3\textwidth}p{0.35\textwidth}p{0.35\textwidth}}
	   & {\bf Type:} double &  {\bf Default:} 1.0e-7\\
	 & & \\
\end{tabular*}
\begin{tabular*}{\textwidth}[tb]{p{0.1\textwidth}p{0.9\textwidth}}
	 PS\_MIN\_S\_PRIMARY\\ 

	 & Minumum eigenvalue for primary basis \\ 
\end{tabular*}
\begin{tabular*}{\textwidth}[tb]{p{0.3\textwidth}p{0.35\textwidth}p{0.35\textwidth}}
	   & {\bf Type:} double &  {\bf Default:} 1.0e-7\\
	 & & \\
\end{tabular*}
\begin{tabular*}{\textwidth}[tb]{p{0.1\textwidth}p{0.9\textwidth}}
	 PS\_NUCLEAR\_SCHEME\\ 

	 & Nuclear Scheme \\ 

	  & {\bf Possible Values:} TREUTLER, BECKE, NAIVE, STRATMANN \\ 
\end{tabular*}
\begin{tabular*}{\textwidth}[tb]{p{0.3\textwidth}p{0.35\textwidth}p{0.35\textwidth}}
	   & {\bf Type:} string &  {\bf Default:} TREUTLER\\
	 & & \\
\end{tabular*}
\begin{tabular*}{\textwidth}[tb]{p{0.1\textwidth}p{0.9\textwidth}}
	 PS\_NUM\_RADIAL\\ 

	 & Number of radial points \\ 
\end{tabular*}
\begin{tabular*}{\textwidth}[tb]{p{0.3\textwidth}p{0.35\textwidth}p{0.35\textwidth}}
	   & {\bf Type:} integer &  {\bf Default:} 5\\
	 & & \\
\end{tabular*}
\begin{tabular*}{\textwidth}[tb]{p{0.1\textwidth}p{0.9\textwidth}}
	 PS\_OMEGA\\ 

	 & Pseudospectral range-separation parameter \\ 
\end{tabular*}
\begin{tabular*}{\textwidth}[tb]{p{0.3\textwidth}p{0.35\textwidth}p{0.35\textwidth}}
	   & {\bf Type:} double &  {\bf Default:} 1.0\\
	 & & \\
\end{tabular*}
\begin{tabular*}{\textwidth}[tb]{p{0.1\textwidth}p{0.9\textwidth}}
	 PS\_ORDER\_SPHERICAL\\ 

	 & Maximum order of spherical grids \\ 
\end{tabular*}
\begin{tabular*}{\textwidth}[tb]{p{0.3\textwidth}p{0.35\textwidth}p{0.35\textwidth}}
	   & {\bf Type:} integer &  {\bf Default:} 7\\
	 & & \\
\end{tabular*}
\begin{tabular*}{\textwidth}[tb]{p{0.1\textwidth}p{0.9\textwidth}}
	 PS\_PRUNING\_ALPHA\\ 

	 & Spread alpha for logarithmic pruning \\ 
\end{tabular*}
\begin{tabular*}{\textwidth}[tb]{p{0.3\textwidth}p{0.35\textwidth}p{0.35\textwidth}}
	   & {\bf Type:} double &  {\bf Default:} 1.0\\
	 & & \\
\end{tabular*}
\begin{tabular*}{\textwidth}[tb]{p{0.1\textwidth}p{0.9\textwidth}}
	 PS\_PRUNING\_SCHEME\\ 

	 & Pruning Scheme \\ 

	  & {\bf Possible Values:} FLAT, P\_GAUSSIAN, D\_GAUSSIAN, P\_SLATER, D\_SLATER, LOG\_GAUSSIAN, LOG\_SLATER \\ 
\end{tabular*}
\begin{tabular*}{\textwidth}[tb]{p{0.3\textwidth}p{0.35\textwidth}p{0.35\textwidth}}
	   & {\bf Type:} string &  {\bf Default:} FLAT\\
	 & & \\
\end{tabular*}
\begin{tabular*}{\textwidth}[tb]{p{0.1\textwidth}p{0.9\textwidth}}
	 PS\_RADIAL\_SCHEME\\ 

	 & Radial Scheme \\ 

	  & {\bf Possible Values:} TREUTLER, BECKE, MULTIEXP, EM, MURA \\ 
\end{tabular*}
\begin{tabular*}{\textwidth}[tb]{p{0.3\textwidth}p{0.35\textwidth}p{0.35\textwidth}}
	   & {\bf Type:} string &  {\bf Default:} TREUTLER\\
	 & & \\
\end{tabular*}
\begin{tabular*}{\textwidth}[tb]{p{0.1\textwidth}p{0.9\textwidth}}
	 PS\_SPHERICAL\_SCHEME\\ 

	 & Spherical Scheme \\ 

	  & {\bf Possible Values:} LEBEDEV \\ 
\end{tabular*}
\begin{tabular*}{\textwidth}[tb]{p{0.3\textwidth}p{0.35\textwidth}p{0.35\textwidth}}
	   & {\bf Type:} string &  {\bf Default:} LEBEDEV\\
	 & & \\
\end{tabular*}
\begin{tabular*}{\textwidth}[tb]{p{0.1\textwidth}p{0.9\textwidth}}
	 PS\_USE\_OMEGA\\ 

	 & Do use range-separation procedure in PS? \\ 
\end{tabular*}
\begin{tabular*}{\textwidth}[tb]{p{0.3\textwidth}p{0.35\textwidth}p{0.35\textwidth}}
	   & {\bf Type:} boolean &  {\bf Default:} true\\
	 & & \\
\end{tabular*}
\begin{tabular*}{\textwidth}[tb]{p{0.1\textwidth}p{0.9\textwidth}}
	 RPA\_ALGORITHM\\ 

	 & RPA algorithm: \begin{tabular}{cc} DF & $\mathcal{O}(N^5)$ \\ CD & $\mathcal{O}(N^4)$ \\ \end{tabular} \\ 

	  & {\bf Possible Values:} CD, DF \\ 
\end{tabular*}
\begin{tabular*}{\textwidth}[tb]{p{0.3\textwidth}p{0.35\textwidth}p{0.35\textwidth}}
	   & {\bf Type:} string &  {\bf Default:} CD\\
	 & & \\
\end{tabular*}
\begin{tabular*}{\textwidth}[tb]{p{0.1\textwidth}p{0.9\textwidth}}
	 RPA\_ALPHA\\ 

	 & RPA alpha parameter \\ 
\end{tabular*}
\begin{tabular*}{\textwidth}[tb]{p{0.3\textwidth}p{0.35\textwidth}p{0.35\textwidth}}
	   & {\bf Type:} double &  {\bf Default:} 1.0\\
	 & & \\
\end{tabular*}
\begin{tabular*}{\textwidth}[tb]{p{0.1\textwidth}p{0.9\textwidth}}
	 RPA\_DELTA\\ 

	 & RPA Cholesky delta \\ 
\end{tabular*}
\begin{tabular*}{\textwidth}[tb]{p{0.3\textwidth}p{0.35\textwidth}p{0.35\textwidth}}
	   & {\bf Type:} double &  {\bf Default:} 1.0e-6\\
	 & & \\
\end{tabular*}
\begin{tabular*}{\textwidth}[tb]{p{0.1\textwidth}p{0.9\textwidth}}
	 RPA\_PLUS\_EPSILON\\ 

	 & Continue RPA numerical SPD Tolerance \\ 
\end{tabular*}
\begin{tabular*}{\textwidth}[tb]{p{0.3\textwidth}p{0.35\textwidth}p{0.35\textwidth}}
	   & {\bf Type:} double &  {\bf Default:} 1.0e-12\\
	 & & \\
\end{tabular*}
\begin{tabular*}{\textwidth}[tb]{p{0.1\textwidth}p{0.9\textwidth}}
	 RPA\_RISKY\\ 

	 & Do continue RPA even if T's are not numerically SPD? \\ 
\end{tabular*}
\begin{tabular*}{\textwidth}[tb]{p{0.3\textwidth}p{0.35\textwidth}p{0.35\textwidth}}
	   & {\bf Type:} boolean &  {\bf Default:} false\\
	 & & \\
\end{tabular*}
\begin{tabular*}{\textwidth}[tb]{p{0.1\textwidth}p{0.9\textwidth}}
	 R\_CONVERGENCE\\ 

	 & Convergence criterion for cluster amplitudes (RMS change). See the note at the beginning of Section \ref{keywords}. \\ 
\end{tabular*}
\begin{tabular*}{\textwidth}[tb]{p{0.3\textwidth}p{0.35\textwidth}p{0.35\textwidth}}
	   & {\bf Type:} double &  {\bf Default:} 1e-8\\
	 & & \\
\end{tabular*}
\begin{tabular*}{\textwidth}[tb]{p{0.1\textwidth}p{0.9\textwidth}}
	 WAVEFUNCTION\\ 

	 & Type of wavefunction \\ 

	  & {\bf Possible Values:} MP2, MP3, CCD, DRPA \\ 
\end{tabular*}
\begin{tabular*}{\textwidth}[tb]{p{0.3\textwidth}p{0.35\textwidth}p{0.35\textwidth}}
	   & {\bf Type:} string &  {\bf Default:} MP2\\
	 & & \\
\end{tabular*}

\subsection{DFMP2}

{\normalsize Performs density-fitted MP2 computations for RHF/UHF/ROHF reference wavefunctions.}\\
\begin{tabular*}{\textwidth}[tb]{c}
	  \\ 
\end{tabular*}
\begin{tabular*}{\textwidth}[tb]{p{0.1\textwidth}p{0.9\textwidth}}
	 BASIS\\ 

	 & Primary basis set \\ 
\end{tabular*}
\begin{tabular*}{\textwidth}[tb]{p{0.3\textwidth}p{0.35\textwidth}p{0.35\textwidth}}
	   & {\bf Type:} string &  {\bf Default:} NONE\\
	 & & \\
\end{tabular*}
\begin{tabular*}{\textwidth}[tb]{p{0.1\textwidth}p{0.9\textwidth}}
	 DFMP2\_MEM\_FACTOR\\ 

	 & \% of memory for DF-MP2 three-index buffers \\ 
\end{tabular*}
\begin{tabular*}{\textwidth}[tb]{p{0.3\textwidth}p{0.35\textwidth}p{0.35\textwidth}}
	   & {\bf Type:} double &  {\bf Default:} 0.9\\
	 & & \\
\end{tabular*}
\begin{tabular*}{\textwidth}[tb]{p{0.1\textwidth}p{0.9\textwidth}}
	 DF\_BASIS\_MP2\\ 

	 & Auxiliary basis set for MP2 density fitting computations. Defaults to BASIS-RI. \\ 
\end{tabular*}
\begin{tabular*}{\textwidth}[tb]{p{0.3\textwidth}p{0.35\textwidth}p{0.35\textwidth}}
	   & {\bf Type:} string &  {\bf Default:} No Default\\
	 & & \\
\end{tabular*}
\begin{tabular*}{\textwidth}[tb]{p{0.1\textwidth}p{0.9\textwidth}}
	 DF\_INTS\_NUM\_THREADS\\ 

	 & Number of threads to compute integrals with. 0 is wild card \\ 
\end{tabular*}
\begin{tabular*}{\textwidth}[tb]{p{0.3\textwidth}p{0.35\textwidth}p{0.35\textwidth}}
	   & {\bf Type:} integer &  {\bf Default:} 0\\
	 & & \\
\end{tabular*}
\begin{tabular*}{\textwidth}[tb]{p{0.1\textwidth}p{0.9\textwidth}}
	 INTS\_TOLERANCE\\ 

	 & Minimum absolute value below which integrals are neglected. See the note at the beginning of Section \ref{keywords}. \\ 
\end{tabular*}
\begin{tabular*}{\textwidth}[tb]{p{0.3\textwidth}p{0.35\textwidth}p{0.35\textwidth}}
	   & {\bf Type:} double &  {\bf Default:} 0.0\\
	 & & \\
\end{tabular*}
\begin{tabular*}{\textwidth}[tb]{p{0.1\textwidth}p{0.9\textwidth}}
	 MP2\_OS\_SCALE\\ 

	 & OS Scale \\ 
\end{tabular*}
\begin{tabular*}{\textwidth}[tb]{p{0.3\textwidth}p{0.35\textwidth}p{0.35\textwidth}}
	   & {\bf Type:} double &  {\bf Default:} 6.0/5.0\\
	 & & \\
\end{tabular*}
\begin{tabular*}{\textwidth}[tb]{p{0.1\textwidth}p{0.9\textwidth}}
	 MP2\_SS\_SCALE\\ 

	 & SS Scale \\ 
\end{tabular*}
\begin{tabular*}{\textwidth}[tb]{p{0.3\textwidth}p{0.35\textwidth}p{0.35\textwidth}}
	   & {\bf Type:} double &  {\bf Default:} 1.0/3.0\\
	 & & \\
\end{tabular*}

\subsection{FINDIF}

{\normalsize Performs finite difference computations of energy derivative, with respect to nuclear displacements for geometry optimizations and vibrational frequency analyses, where the required analytical derivatives are not available.}\\
\begin{tabular*}{\textwidth}[tb]{c}
	  \\ 
\end{tabular*}
\begin{tabular*}{\textwidth}[tb]{p{0.1\textwidth}p{0.9\textwidth}}
	 DISP\_SIZE\\ 

	 & Displacement size in au for finite-differences. \\ 
\end{tabular*}
\begin{tabular*}{\textwidth}[tb]{p{0.3\textwidth}p{0.35\textwidth}p{0.35\textwidth}}
	   & {\bf Type:} double &  {\bf Default:} 0.005\\
	 & & \\
\end{tabular*}
\begin{tabular*}{\textwidth}[tb]{p{0.1\textwidth}p{0.9\textwidth}}
	 POINTS\\ 

	 & Number of points for finite-differences (3 or 5) \\ 
\end{tabular*}
\begin{tabular*}{\textwidth}[tb]{p{0.3\textwidth}p{0.35\textwidth}p{0.35\textwidth}}
	   & {\bf Type:} integer &  {\bf Default:} 3\\
	 & & \\
\end{tabular*}

\subsection{LMP2}

{\normalsize Performs local MP2 computations for RHF reference functions}\\
\begin{tabular*}{\textwidth}[tb]{c}
	  \\ 
\end{tabular*}
\begin{tabular*}{\textwidth}[tb]{p{0.1\textwidth}p{0.9\textwidth}}
	 DF\_BASIS\_MP2\\ 

	 & Auxiliary basis set for MP2 density fitting calculations \\ 
\end{tabular*}
\begin{tabular*}{\textwidth}[tb]{p{0.3\textwidth}p{0.35\textwidth}p{0.35\textwidth}}
	   & {\bf Type:} string &  {\bf Default:} No Default\\
	 & & \\
\end{tabular*}
\begin{tabular*}{\textwidth}[tb]{p{0.1\textwidth}p{0.9\textwidth}}
	 DF\_LMP2\\ 

	 & Do use density fitting? Turned on with specification of fitting basis. \\ 
\end{tabular*}
\begin{tabular*}{\textwidth}[tb]{p{0.3\textwidth}p{0.35\textwidth}p{0.35\textwidth}}
	   & {\bf Type:} boolean &  {\bf Default:} true\\
	 & & \\
\end{tabular*}
\begin{tabular*}{\textwidth}[tb]{p{0.1\textwidth}p{0.9\textwidth}}
	 DIIS\\ 

	 & Do use DIIS extrapolation to accelerate convergence? \\ 
\end{tabular*}
\begin{tabular*}{\textwidth}[tb]{p{0.3\textwidth}p{0.35\textwidth}p{0.35\textwidth}}
	   & {\bf Type:} boolean &  {\bf Default:} true\\
	 & & \\
\end{tabular*}
\begin{tabular*}{\textwidth}[tb]{p{0.1\textwidth}p{0.9\textwidth}}
	 DIIS\_MAX\_VECS\\ 

	 & Maximum number of error vectors stored for DIIS extrapolation \\ 
\end{tabular*}
\begin{tabular*}{\textwidth}[tb]{p{0.3\textwidth}p{0.35\textwidth}p{0.35\textwidth}}
	   & {\bf Type:} integer &  {\bf Default:} 5\\
	 & & \\
\end{tabular*}
\begin{tabular*}{\textwidth}[tb]{p{0.1\textwidth}p{0.9\textwidth}}
	 DIIS\_START\_ITER\\ 

	 & Iteration at which to start DIIS extrapolation \\ 
\end{tabular*}
\begin{tabular*}{\textwidth}[tb]{p{0.3\textwidth}p{0.35\textwidth}p{0.35\textwidth}}
	   & {\bf Type:} integer &  {\bf Default:} 3\\
	 & & \\
\end{tabular*}
\begin{tabular*}{\textwidth}[tb]{p{0.1\textwidth}p{0.9\textwidth}}
	 DISTANT\_PAIR\_CUTOFF\\ 

	 & Distant pair cutoff \\ 
\end{tabular*}
\begin{tabular*}{\textwidth}[tb]{p{0.3\textwidth}p{0.35\textwidth}p{0.35\textwidth}}
	   & {\bf Type:} double &  {\bf Default:} 8.0\\
	 & & \\
\end{tabular*}
\begin{tabular*}{\textwidth}[tb]{p{0.1\textwidth}p{0.9\textwidth}}
	 DOMAIN\_PRINT\_EXIT\\ 

	 & Do exit after printing the domains? \\ 
\end{tabular*}
\begin{tabular*}{\textwidth}[tb]{p{0.3\textwidth}p{0.35\textwidth}p{0.35\textwidth}}
	   & {\bf Type:} boolean &  {\bf Default:} false\\
	 & & \\
\end{tabular*}
\begin{tabular*}{\textwidth}[tb]{p{0.1\textwidth}p{0.9\textwidth}}
	 E\_CONVERGENCE\\ 

	 & Convergence criterion for energy (change). See the note at the beginning of Section \ref{keywords}. \\ 
\end{tabular*}
\begin{tabular*}{\textwidth}[tb]{p{0.3\textwidth}p{0.35\textwidth}p{0.35\textwidth}}
	   & {\bf Type:} double &  {\bf Default:} 1e-7\\
	 & & \\
\end{tabular*}
\begin{tabular*}{\textwidth}[tb]{p{0.1\textwidth}p{0.9\textwidth}}
	 FOCK\_TOLERANCE\\ 

	 & Minimum absolute value below which parts of the Fock matrix are skipped. See the note at the beginning of Section \ref{keywords}. \\ 
\end{tabular*}
\begin{tabular*}{\textwidth}[tb]{p{0.3\textwidth}p{0.35\textwidth}p{0.35\textwidth}}
	   & {\bf Type:} double &  {\bf Default:} 1e-2\\
	 & & \\
\end{tabular*}
\begin{tabular*}{\textwidth}[tb]{p{0.1\textwidth}p{0.9\textwidth}}
	 INTS\_TOLERANCE\\ 

	 & Minimum absolute value below which integrals are neglected. See the note at the beginning of Section \ref{keywords}. \\ 
\end{tabular*}
\begin{tabular*}{\textwidth}[tb]{p{0.3\textwidth}p{0.35\textwidth}p{0.35\textwidth}}
	   & {\bf Type:} double &  {\bf Default:} 1e-7\\
	 & & \\
\end{tabular*}
\begin{tabular*}{\textwidth}[tb]{p{0.1\textwidth}p{0.9\textwidth}}
	 LOCAL\_CUTOFF\\ 

	 & Localization cutoff \\ 
\end{tabular*}
\begin{tabular*}{\textwidth}[tb]{p{0.3\textwidth}p{0.35\textwidth}p{0.35\textwidth}}
	   & {\bf Type:} double &  {\bf Default:} 0.02\\
	 & & \\
\end{tabular*}
\begin{tabular*}{\textwidth}[tb]{p{0.1\textwidth}p{0.9\textwidth}}
	 MAXITER\\ 

	 & Maximum number of iterations \\ 
\end{tabular*}
\begin{tabular*}{\textwidth}[tb]{p{0.3\textwidth}p{0.35\textwidth}p{0.35\textwidth}}
	   & {\bf Type:} integer &  {\bf Default:} 50\\
	 & & \\
\end{tabular*}
\begin{tabular*}{\textwidth}[tb]{p{0.1\textwidth}p{0.9\textwidth}}
	 MEMORY\\ 

	 &  \\ 
\end{tabular*}
\begin{tabular*}{\textwidth}[tb]{p{0.3\textwidth}p{0.35\textwidth}p{0.35\textwidth}}
	   & {\bf Type:} integer &  {\bf Default:} 2000\\
	 & & \\
\end{tabular*}
\begin{tabular*}{\textwidth}[tb]{p{0.1\textwidth}p{0.9\textwidth}}
	 MP2\_OS\_SCALE\\ 

	 &  \\ 
\end{tabular*}
\begin{tabular*}{\textwidth}[tb]{p{0.3\textwidth}p{0.35\textwidth}p{0.35\textwidth}}
	   & {\bf Type:} double &  {\bf Default:} 6.0/5.0\\
	 & & \\
\end{tabular*}
\begin{tabular*}{\textwidth}[tb]{p{0.1\textwidth}p{0.9\textwidth}}
	 MP2\_SS\_SCALE\\ 

	 &  \\ 
\end{tabular*}
\begin{tabular*}{\textwidth}[tb]{p{0.3\textwidth}p{0.35\textwidth}p{0.35\textwidth}}
	   & {\bf Type:} double &  {\bf Default:} 1.0/3.0\\
	 & & \\
\end{tabular*}
\begin{tabular*}{\textwidth}[tb]{p{0.1\textwidth}p{0.9\textwidth}}
	 NEGLECT\_DISTANT\_PAIR\\ 

	 & Do neglect distant pairs? \\ 
\end{tabular*}
\begin{tabular*}{\textwidth}[tb]{p{0.3\textwidth}p{0.35\textwidth}p{0.35\textwidth}}
	   & {\bf Type:} boolean &  {\bf Default:} true\\
	 & & \\
\end{tabular*}
\begin{tabular*}{\textwidth}[tb]{p{0.1\textwidth}p{0.9\textwidth}}
	 REFERENCE\\ 

	 & Reference wavefunction type \\ 

	  & {\bf Possible Values:} RHF \\ 
\end{tabular*}
\begin{tabular*}{\textwidth}[tb]{p{0.3\textwidth}p{0.35\textwidth}p{0.35\textwidth}}
	   & {\bf Type:} string &  {\bf Default:} RHF\\
	 & & \\
\end{tabular*}
\begin{tabular*}{\textwidth}[tb]{p{0.1\textwidth}p{0.9\textwidth}}
	 R\_CONVERGENCE\\ 

	 & Convergence criterion for T2 amplitudes (RMS change). See the note at the beginning of Section \ref{keywords}. \\ 
\end{tabular*}
\begin{tabular*}{\textwidth}[tb]{p{0.3\textwidth}p{0.35\textwidth}p{0.35\textwidth}}
	   & {\bf Type:} double &  {\bf Default:} 1e-5\\
	 & & \\
\end{tabular*}
\begin{tabular*}{\textwidth}[tb]{p{0.1\textwidth}p{0.9\textwidth}}
	 SCREEN\_INTS\\ 

	 & Do screen integrals? \\ 
\end{tabular*}
\begin{tabular*}{\textwidth}[tb]{p{0.3\textwidth}p{0.35\textwidth}p{0.35\textwidth}}
	   & {\bf Type:} boolean &  {\bf Default:} false\\
	 & & \\
\end{tabular*}
\begin{tabular*}{\textwidth}[tb]{p{0.1\textwidth}p{0.9\textwidth}}
	 SCS\\ 

	 & Do ? \\ 
\end{tabular*}
\begin{tabular*}{\textwidth}[tb]{p{0.3\textwidth}p{0.35\textwidth}p{0.35\textwidth}}
	   & {\bf Type:} boolean &  {\bf Default:} false\\
	 & & \\
\end{tabular*}
\begin{tabular*}{\textwidth}[tb]{p{0.1\textwidth}p{0.9\textwidth}}
	 SCS\_N\\ 

	 & Do ? \\ 
\end{tabular*}
\begin{tabular*}{\textwidth}[tb]{p{0.3\textwidth}p{0.35\textwidth}p{0.35\textwidth}}
	   & {\bf Type:} boolean &  {\bf Default:} false\\
	 & & \\
\end{tabular*}

\subsection{MCSCF}

{\normalsize Performs RHF/UHF/ROHF/TCSCF, and more general MCSCF computations, and is called as the starting point for multireference coupled cluster computations.}\\
\begin{tabular*}{\textwidth}[tb]{c}
	  \\ 
\end{tabular*}
\begin{tabular*}{\textwidth}[tb]{p{0.1\textwidth}p{0.9\textwidth}}
	 ACTV\\ 

	 & The number of active orbitals, per irrep (alternative name for ACTIVE) \\ 
\end{tabular*}
\begin{tabular*}{\textwidth}[tb]{p{0.3\textwidth}p{0.35\textwidth}p{0.35\textwidth}}
	   & {\bf Type:} array &  {\bf Default:} No Default\\
	 & & \\
\end{tabular*}
\begin{tabular*}{\textwidth}[tb]{p{0.1\textwidth}p{0.9\textwidth}}
	 CANONICALIZE\_ACTIVE\_FAVG\\ 

	 & Do canonicalize the active orbitals such that the average Fock matrix is diagonal? \\ 
\end{tabular*}
\begin{tabular*}{\textwidth}[tb]{p{0.3\textwidth}p{0.35\textwidth}p{0.35\textwidth}}
	   & {\bf Type:} boolean &  {\bf Default:} false\\
	 & & \\
\end{tabular*}
\begin{tabular*}{\textwidth}[tb]{p{0.1\textwidth}p{0.9\textwidth}}
	 CANONICALIZE\_INACTIVE\_FAVG\\ 

	 & Do canonicalize the inactive (DOCC and Virtual) orbitals such that the average Fock matrix is diagonal? \\ 
\end{tabular*}
\begin{tabular*}{\textwidth}[tb]{p{0.3\textwidth}p{0.35\textwidth}p{0.35\textwidth}}
	   & {\bf Type:} boolean &  {\bf Default:} false\\
	 & & \\
\end{tabular*}
\begin{tabular*}{\textwidth}[tb]{p{0.1\textwidth}p{0.9\textwidth}}
	 CI\_DIIS\\ 

	 & Do use DIIS extrapolation to accelerate convergence of the CI coefficients? \\ 
\end{tabular*}
\begin{tabular*}{\textwidth}[tb]{p{0.3\textwidth}p{0.35\textwidth}p{0.35\textwidth}}
	   & {\bf Type:} boolean &  {\bf Default:} false\\
	 & & \\
\end{tabular*}
\begin{tabular*}{\textwidth}[tb]{p{0.1\textwidth}p{0.9\textwidth}}
	 DIIS\\ 

	 & Do use DIIS extrapolation to accelerate convergence of the SCF energy (MO coefficients only)? \\ 
\end{tabular*}
\begin{tabular*}{\textwidth}[tb]{p{0.3\textwidth}p{0.35\textwidth}p{0.35\textwidth}}
	   & {\bf Type:} boolean &  {\bf Default:} true\\
	 & & \\
\end{tabular*}
\begin{tabular*}{\textwidth}[tb]{p{0.1\textwidth}p{0.9\textwidth}}
	 DIIS\_MAX\_VECS\\ 

	 & Maximum number of error vectors stored for DIIS extrapolation \\ 
\end{tabular*}
\begin{tabular*}{\textwidth}[tb]{p{0.3\textwidth}p{0.35\textwidth}p{0.35\textwidth}}
	   & {\bf Type:} integer &  {\bf Default:} 7\\
	 & & \\
\end{tabular*}
\begin{tabular*}{\textwidth}[tb]{p{0.1\textwidth}p{0.9\textwidth}}
	 DOCC\\ 

	 & The number of doubly occupied orbitals, per irrep \\ 
\end{tabular*}
\begin{tabular*}{\textwidth}[tb]{p{0.3\textwidth}p{0.35\textwidth}p{0.35\textwidth}}
	   & {\bf Type:} array &  {\bf Default:} No Default\\
	 & & \\
\end{tabular*}
\begin{tabular*}{\textwidth}[tb]{p{0.1\textwidth}p{0.9\textwidth}}
	 D\_CONVERGENCE\\ 

	 & Convergence criterion for density. See the note at the beginning of Section \ref{keywords}. \\ 
\end{tabular*}
\begin{tabular*}{\textwidth}[tb]{p{0.3\textwidth}p{0.35\textwidth}p{0.35\textwidth}}
	   & {\bf Type:} double &  {\bf Default:} 1e-12\\
	 & & \\
\end{tabular*}
\begin{tabular*}{\textwidth}[tb]{p{0.1\textwidth}p{0.9\textwidth}}
	 E\_CONVERGENCE\\ 

	 & Convergence criterion for energy. See the note at the beginning of Section \ref{keywords}. \\ 
\end{tabular*}
\begin{tabular*}{\textwidth}[tb]{p{0.3\textwidth}p{0.35\textwidth}p{0.35\textwidth}}
	   & {\bf Type:} double &  {\bf Default:} 1e-12\\
	 & & \\
\end{tabular*}
\begin{tabular*}{\textwidth}[tb]{p{0.1\textwidth}p{0.9\textwidth}}
	 FAVG\\ 

	 & Do use the average Fock matrix during the SCF optimization? \\ 
\end{tabular*}
\begin{tabular*}{\textwidth}[tb]{p{0.3\textwidth}p{0.35\textwidth}p{0.35\textwidth}}
	   & {\bf Type:} boolean &  {\bf Default:} false\\
	 & & \\
\end{tabular*}
\begin{tabular*}{\textwidth}[tb]{p{0.1\textwidth}p{0.9\textwidth}}
	 FAVG\_START\\ 

	 & Iteration at which to begin using the averaged Fock matrix \\ 
\end{tabular*}
\begin{tabular*}{\textwidth}[tb]{p{0.3\textwidth}p{0.35\textwidth}p{0.35\textwidth}}
	   & {\bf Type:} integer &  {\bf Default:} 5\\
	 & & \\
\end{tabular*}
\begin{tabular*}{\textwidth}[tb]{p{0.1\textwidth}p{0.9\textwidth}}
	 FOLLOW\_ROOT\\ 

	 & Which solution of the SCF equations to find, where 1 is the SCF ground state \\ 
\end{tabular*}
\begin{tabular*}{\textwidth}[tb]{p{0.3\textwidth}p{0.35\textwidth}p{0.35\textwidth}}
	   & {\bf Type:} integer &  {\bf Default:} 1\\
	 & & \\
\end{tabular*}
\begin{tabular*}{\textwidth}[tb]{p{0.1\textwidth}p{0.9\textwidth}}
	 FORCE\_TWOCON\\ 

	 & Do attempt to force a two configruation solution by starting with CI coefficents of $\pm \sqrt{\frac{1}{2}}$ \\ 
\end{tabular*}
\begin{tabular*}{\textwidth}[tb]{p{0.3\textwidth}p{0.35\textwidth}p{0.35\textwidth}}
	   & {\bf Type:} boolean &  {\bf Default:} false\\
	 & & \\
\end{tabular*}
\begin{tabular*}{\textwidth}[tb]{p{0.1\textwidth}p{0.9\textwidth}}
	 INTERNAL\_ROTATIONS\\ 

	 & Do ? \\ 
\end{tabular*}
\begin{tabular*}{\textwidth}[tb]{p{0.3\textwidth}p{0.35\textwidth}p{0.35\textwidth}}
	   & {\bf Type:} boolean &  {\bf Default:} true\\
	 & & \\
\end{tabular*}
\begin{tabular*}{\textwidth}[tb]{p{0.1\textwidth}p{0.9\textwidth}}
	 LEVEL\_SHIFT\\ 

	 & Level shift to aid convergence \\ 
\end{tabular*}
\begin{tabular*}{\textwidth}[tb]{p{0.3\textwidth}p{0.35\textwidth}p{0.35\textwidth}}
	   & {\bf Type:} double &  {\bf Default:} 0.0\\
	 & & \\
\end{tabular*}
\begin{tabular*}{\textwidth}[tb]{p{0.1\textwidth}p{0.9\textwidth}}
	 MAXITER\\ 

	 & Maximum number of iterations \\ 
\end{tabular*}
\begin{tabular*}{\textwidth}[tb]{p{0.3\textwidth}p{0.35\textwidth}p{0.35\textwidth}}
	   & {\bf Type:} integer &  {\bf Default:} 100\\
	 & & \\
\end{tabular*}
\begin{tabular*}{\textwidth}[tb]{p{0.1\textwidth}p{0.9\textwidth}}
	 MO\_READ\\ 

	 & Do read in from file the MOs from a previous computation? \\ 
\end{tabular*}
\begin{tabular*}{\textwidth}[tb]{p{0.3\textwidth}p{0.35\textwidth}p{0.35\textwidth}}
	   & {\bf Type:} boolean &  {\bf Default:} true\\
	 & & \\
\end{tabular*}
\begin{tabular*}{\textwidth}[tb]{p{0.1\textwidth}p{0.9\textwidth}}
	 REFERENCE\\ 

	 & Reference wavefunction type \\ 

	  & {\bf Possible Values:} RHF, ROHF, UHF, TWOCON, MCSCF, GENERAL \\ 
\end{tabular*}
\begin{tabular*}{\textwidth}[tb]{p{0.3\textwidth}p{0.35\textwidth}p{0.35\textwidth}}
	   & {\bf Type:} string &  {\bf Default:} RHF\\
	 & & \\
\end{tabular*}
\begin{tabular*}{\textwidth}[tb]{p{0.1\textwidth}p{0.9\textwidth}}
	 SOCC\\ 

	 & The number of singly occupied orbitals, per irrep \\ 
\end{tabular*}
\begin{tabular*}{\textwidth}[tb]{p{0.3\textwidth}p{0.35\textwidth}p{0.35\textwidth}}
	   & {\bf Type:} array &  {\bf Default:} No Default\\
	 & & \\
\end{tabular*}
\begin{tabular*}{\textwidth}[tb]{p{0.1\textwidth}p{0.9\textwidth}}
	 TURN\_ON\_ACTV\\ 

	 &  \\ 
\end{tabular*}
\begin{tabular*}{\textwidth}[tb]{p{0.3\textwidth}p{0.35\textwidth}p{0.35\textwidth}}
	   & {\bf Type:} integer &  {\bf Default:} 0\\
	 & & \\
\end{tabular*}
\begin{tabular*}{\textwidth}[tb]{p{0.1\textwidth}p{0.9\textwidth}}
	 WFN\_SYM\\ 

	 & The symmetry of the SCF wavefunction. \\ 

	  & {\bf Possible Values:} A, AG, AU, AP, APP, A1, A2, B, BG, BU, B1, B2, B3, B1G, B2G, B3G, B1U, B2U, B3U, 0, 1, 2, 3, 4, 5, 6, 7, 8 \\ 
\end{tabular*}
\begin{tabular*}{\textwidth}[tb]{p{0.3\textwidth}p{0.35\textwidth}p{0.35\textwidth}}
	   & {\bf Type:} string &  {\bf Default:} 1\\
	 & & \\
\end{tabular*}

\subsection{MINTS}

{\normalsize Called at the beginning of SCF computations, whenever disk-based molecular integrals are required}\\
\begin{tabular*}{\textwidth}[tb]{c}
	  \\ 
\end{tabular*}
\begin{tabular*}{\textwidth}[tb]{p{0.1\textwidth}p{0.9\textwidth}}
	 BASIS\\ 

	 & Primary basis set \\ 
\end{tabular*}
\begin{tabular*}{\textwidth}[tb]{p{0.3\textwidth}p{0.35\textwidth}p{0.35\textwidth}}
	   & {\bf Type:} string &  {\bf Default:} No Default\\
	 & & \\
\end{tabular*}

\subsection{MP2}

{\normalsize Performs second order Moller-Plesset perturbation theory (MP2) computations. This code can compute RHF/ROHF/UHF energies, and RHF gradient/property computations. However, given the small errors introduced, we recommend using the new density fitted MP2 codes instead, which are much more efficient.}\\
\begin{tabular*}{\textwidth}[tb]{c}
	  \\ 
\end{tabular*}
\begin{tabular*}{\textwidth}[tb]{p{0.1\textwidth}p{0.9\textwidth}}
	 CACHELEVEL\\ 

	 & The amount of cacheing of data to perform \\ 
\end{tabular*}
\begin{tabular*}{\textwidth}[tb]{p{0.3\textwidth}p{0.35\textwidth}p{0.35\textwidth}}
	   & {\bf Type:} integer &  {\bf Default:} 2\\
	 & & \\
\end{tabular*}
\begin{tabular*}{\textwidth}[tb]{p{0.1\textwidth}p{0.9\textwidth}}
	 CACHETYPE\\ 

	 & The criterion used to retain/release cached data \\ 

	  & {\bf Possible Values:} LRU, LOW \\ 
\end{tabular*}
\begin{tabular*}{\textwidth}[tb]{p{0.3\textwidth}p{0.35\textwidth}p{0.35\textwidth}}
	   & {\bf Type:} string &  {\bf Default:} LRU\\
	 & & \\
\end{tabular*}
\begin{tabular*}{\textwidth}[tb]{p{0.1\textwidth}p{0.9\textwidth}}
	 MP2\_OS\_SCALE\\ 

	 & The scale factor used for opposite-spin pairs in SCS computations \\ 
\end{tabular*}
\begin{tabular*}{\textwidth}[tb]{p{0.3\textwidth}p{0.35\textwidth}p{0.35\textwidth}}
	   & {\bf Type:} double &  {\bf Default:} 6.0/5.0\\
	 & & \\
\end{tabular*}
\begin{tabular*}{\textwidth}[tb]{p{0.1\textwidth}p{0.9\textwidth}}
	 MP2\_SS\_SCALE\\ 

	 & The scale factor used for same-spin pairs in SCS computations \\ 
\end{tabular*}
\begin{tabular*}{\textwidth}[tb]{p{0.3\textwidth}p{0.35\textwidth}p{0.35\textwidth}}
	   & {\bf Type:} double &  {\bf Default:} 1.0/3.0\\
	 & & \\
\end{tabular*}
\begin{tabular*}{\textwidth}[tb]{p{0.1\textwidth}p{0.9\textwidth}}
	 OPDM\\ 

	 & Do compute the one particle density matrix, for properties? \\ 
\end{tabular*}
\begin{tabular*}{\textwidth}[tb]{p{0.3\textwidth}p{0.35\textwidth}p{0.35\textwidth}}
	   & {\bf Type:} boolean &  {\bf Default:} false\\
	 & & \\
\end{tabular*}
\begin{tabular*}{\textwidth}[tb]{p{0.1\textwidth}p{0.9\textwidth}}
	 OPDM\_RELAX\\ 

	 & Do add relaxation terms to the one particle density matrix, for properties? \\ 
\end{tabular*}
\begin{tabular*}{\textwidth}[tb]{p{0.3\textwidth}p{0.35\textwidth}p{0.35\textwidth}}
	   & {\bf Type:} boolean &  {\bf Default:} false\\
	 & & \\
\end{tabular*}
\begin{tabular*}{\textwidth}[tb]{p{0.1\textwidth}p{0.9\textwidth}}
	 REFERENCE\\ 

	 & Reference wavefunction type \\ 

	  & {\bf Possible Values:} RHF, UHF, ROHF \\ 
\end{tabular*}
\begin{tabular*}{\textwidth}[tb]{p{0.3\textwidth}p{0.35\textwidth}p{0.35\textwidth}}
	   & {\bf Type:} string &  {\bf Default:} RHF\\
	 & & \\
\end{tabular*}
\begin{tabular*}{\textwidth}[tb]{p{0.1\textwidth}p{0.9\textwidth}}
	 SCS\\ 

	 & Do perform a spin component scaled MP2 computation? \\ 
\end{tabular*}
\begin{tabular*}{\textwidth}[tb]{p{0.3\textwidth}p{0.35\textwidth}p{0.35\textwidth}}
	   & {\bf Type:} boolean &  {\bf Default:} false\\
	 & & \\
\end{tabular*}
\begin{tabular*}{\textwidth}[tb]{p{0.1\textwidth}p{0.9\textwidth}}
	 SCS\_N\\ 

	 & Do perform a spin component scaled (N) MP2 computation? \\ 
\end{tabular*}
\begin{tabular*}{\textwidth}[tb]{p{0.3\textwidth}p{0.35\textwidth}p{0.35\textwidth}}
	   & {\bf Type:} boolean &  {\bf Default:} false\\
	 & & \\
\end{tabular*}

\subsection{MRCC}

{\normalsize Interface to MRCC program written by Mih\'{a}ly K\'{a}llay.}\\
\begin{tabular*}{\textwidth}[tb]{c}
	  \\ 
\end{tabular*}
\begin{tabular*}{\textwidth}[tb]{p{0.1\textwidth}p{0.9\textwidth}}
	 E\_CONVERGENCE\\ 

	 & See the note at the beginning of Section \ref{keywords}. This becomes {\tt tol} (option \#16) in fort.56. \\ 
\end{tabular*}
\begin{tabular*}{\textwidth}[tb]{p{0.3\textwidth}p{0.35\textwidth}p{0.35\textwidth}}
	   & {\bf Type:} double &  {\bf Default:} 1e-8\\
	 & & \\
\end{tabular*}
\begin{tabular*}{\textwidth}[tb]{p{0.1\textwidth}p{0.9\textwidth}}
	 INTS\_TOLERANCE\\ 

	 & Minimum absolute value below which integrals are neglected. See the note at the beginning of Section \ref{keywords}. \\ 
\end{tabular*}
\begin{tabular*}{\textwidth}[tb]{p{0.3\textwidth}p{0.35\textwidth}p{0.35\textwidth}}
	   & {\bf Type:} double &  {\bf Default:} 1.0e-12\\
	 & & \\
\end{tabular*}
\begin{tabular*}{\textwidth}[tb]{p{0.1\textwidth}p{0.9\textwidth}}
	 MRCC\_LEVEL\\ 

	 & Maximum excitation level. This is used ONLY if it is explicity set by the user. Single-reference case: all excitations up to this level are included, e.g., 2 for CCSD, 3 for CCSDT, 4 for CCSDTQ, etc. This becomes {\tt ex.lev} (option \#1) in fort.56. \\ 
\end{tabular*}
\begin{tabular*}{\textwidth}[tb]{p{0.3\textwidth}p{0.35\textwidth}p{0.35\textwidth}}
	   & {\bf Type:} integer &  {\bf Default:} 2\\
	 & & \\
\end{tabular*}
\begin{tabular*}{\textwidth}[tb]{p{0.1\textwidth}p{0.9\textwidth}}
	 MRCC\_NUM\_SINGLET\_ROOTS\\ 

	 & Number of singlet roots. (Strictly speaking number of of roots with M\_s=0 and S is even.) Use this option only with closed shell reference determinant, it must be zero otherwise. This becomes {\tt nsing} (option \#2) in fort.56. \\ 
\end{tabular*}
\begin{tabular*}{\textwidth}[tb]{p{0.3\textwidth}p{0.35\textwidth}p{0.35\textwidth}}
	   & {\bf Type:} integer &  {\bf Default:} 1\\
	 & & \\
\end{tabular*}
\begin{tabular*}{\textwidth}[tb]{p{0.1\textwidth}p{0.9\textwidth}}
	 MRCC\_NUM\_TRIPLET\_ROOTS\\ 

	 & Number of triplet roots. (Strictly speaking number of of roots with M\_s=0 and S is odd.) See notes at option MRCC\_NUM\_SINGLET\_ROOTS. This becomes {\tt ntrip} (option \#3) in fort.56. \\ 
\end{tabular*}
\begin{tabular*}{\textwidth}[tb]{p{0.3\textwidth}p{0.35\textwidth}p{0.35\textwidth}}
	   & {\bf Type:} integer &  {\bf Default:} 0\\
	 & & \\
\end{tabular*}

\subsection{OMP2}

{\normalsize Performs quadratically convergence orbital-optimized MP2 computations.}\\
\begin{tabular*}{\textwidth}[tb]{c}
	  \\ 
\end{tabular*}
\begin{tabular*}{\textwidth}[tb]{p{0.1\textwidth}p{0.9\textwidth}}
	 CACHELEVEL\\ 

	 &  \\ 
\end{tabular*}
\begin{tabular*}{\textwidth}[tb]{p{0.3\textwidth}p{0.35\textwidth}p{0.35\textwidth}}
	   & {\bf Type:} integer &  {\bf Default:} 2\\
	 & & \\
\end{tabular*}
\begin{tabular*}{\textwidth}[tb]{p{0.1\textwidth}p{0.9\textwidth}}
	 CC\_MAXITER\\ 

	 &  \\ 
\end{tabular*}
\begin{tabular*}{\textwidth}[tb]{p{0.3\textwidth}p{0.35\textwidth}p{0.35\textwidth}}
	   & {\bf Type:} integer &  {\bf Default:} 50\\
	 & & \\
\end{tabular*}
\begin{tabular*}{\textwidth}[tb]{p{0.1\textwidth}p{0.9\textwidth}}
	 CUTOFF\\ 

	 &  \\ 
\end{tabular*}
\begin{tabular*}{\textwidth}[tb]{p{0.3\textwidth}p{0.35\textwidth}p{0.35\textwidth}}
	   & {\bf Type:} integer &  {\bf Default:} 14\\
	 & & \\
\end{tabular*}
\begin{tabular*}{\textwidth}[tb]{p{0.1\textwidth}p{0.9\textwidth}}
	 DIIS\_MAX\_VECS\\ 

	 & Number of vectors used in DIIS \\ 
\end{tabular*}
\begin{tabular*}{\textwidth}[tb]{p{0.3\textwidth}p{0.35\textwidth}p{0.35\textwidth}}
	   & {\bf Type:} integer &  {\bf Default:} 4\\
	 & & \\
\end{tabular*}
\begin{tabular*}{\textwidth}[tb]{p{0.1\textwidth}p{0.9\textwidth}}
	 DO\_SCS\\ 

	 & Do ? \\ 
\end{tabular*}
\begin{tabular*}{\textwidth}[tb]{p{0.3\textwidth}p{0.35\textwidth}p{0.35\textwidth}}
	   & {\bf Type:} boolean &  {\bf Default:} false\\
	 & & \\
\end{tabular*}
\begin{tabular*}{\textwidth}[tb]{p{0.1\textwidth}p{0.9\textwidth}}
	 DO\_SOS\\ 

	 & Do ? \\ 
\end{tabular*}
\begin{tabular*}{\textwidth}[tb]{p{0.3\textwidth}p{0.35\textwidth}p{0.35\textwidth}}
	   & {\bf Type:} boolean &  {\bf Default:} false\\
	 & & \\
\end{tabular*}
\begin{tabular*}{\textwidth}[tb]{p{0.1\textwidth}p{0.9\textwidth}}
	 E\_CONVERGENCE\\ 

	 & See the note at the beginning of Section \ref{keywords}. \\ 
\end{tabular*}
\begin{tabular*}{\textwidth}[tb]{p{0.3\textwidth}p{0.35\textwidth}p{0.35\textwidth}}
	   & {\bf Type:} double &  {\bf Default:} 1e-8\\
	 & & \\
\end{tabular*}
\begin{tabular*}{\textwidth}[tb]{p{0.1\textwidth}p{0.9\textwidth}}
	 HESS\_TYPE\\ 

	 &  \\ 

	  & {\bf Possible Values:} NONE \\ 
\end{tabular*}
\begin{tabular*}{\textwidth}[tb]{p{0.3\textwidth}p{0.35\textwidth}p{0.35\textwidth}}
	   & {\bf Type:} string &  {\bf Default:} NONE\\
	 & & \\
\end{tabular*}
\begin{tabular*}{\textwidth}[tb]{p{0.1\textwidth}p{0.9\textwidth}}
	 LEVEL\_SHIFT\\ 

	 &  \\ 
\end{tabular*}
\begin{tabular*}{\textwidth}[tb]{p{0.3\textwidth}p{0.35\textwidth}p{0.35\textwidth}}
	   & {\bf Type:} double &  {\bf Default:} 0.02\\
	 & & \\
\end{tabular*}
\begin{tabular*}{\textwidth}[tb]{p{0.1\textwidth}p{0.9\textwidth}}
	 MAX\_MOGRAD\_CONVERGENCE\\ 

	 & See the note at the beginning of Section \ref{keywords}. \\ 
\end{tabular*}
\begin{tabular*}{\textwidth}[tb]{p{0.3\textwidth}p{0.35\textwidth}p{0.35\textwidth}}
	   & {\bf Type:} double &  {\bf Default:} 1e-4\\
	 & & \\
\end{tabular*}
\begin{tabular*}{\textwidth}[tb]{p{0.1\textwidth}p{0.9\textwidth}}
	 MO\_MAXITER\\ 

	 &  \\ 
\end{tabular*}
\begin{tabular*}{\textwidth}[tb]{p{0.3\textwidth}p{0.35\textwidth}p{0.35\textwidth}}
	   & {\bf Type:} integer &  {\bf Default:} 50\\
	 & & \\
\end{tabular*}
\begin{tabular*}{\textwidth}[tb]{p{0.1\textwidth}p{0.9\textwidth}}
	 MO\_READ\\ 

	 & Do read coefficient matrices from psi files? \\ 
\end{tabular*}
\begin{tabular*}{\textwidth}[tb]{p{0.3\textwidth}p{0.35\textwidth}p{0.35\textwidth}}
	   & {\bf Type:} boolean &  {\bf Default:} false\\
	 & & \\
\end{tabular*}
\begin{tabular*}{\textwidth}[tb]{p{0.1\textwidth}p{0.9\textwidth}}
	 MO\_STEP\_MAX\\ 

	 &  \\ 
\end{tabular*}
\begin{tabular*}{\textwidth}[tb]{p{0.3\textwidth}p{0.35\textwidth}p{0.35\textwidth}}
	   & {\bf Type:} double &  {\bf Default:} 0.5\\
	 & & \\
\end{tabular*}
\begin{tabular*}{\textwidth}[tb]{p{0.1\textwidth}p{0.9\textwidth}}
	 MO\_WRITE\\ 

	 & Do write coefficient matrices to psi files? \\ 
\end{tabular*}
\begin{tabular*}{\textwidth}[tb]{p{0.3\textwidth}p{0.35\textwidth}p{0.35\textwidth}}
	   & {\bf Type:} boolean &  {\bf Default:} false\\
	 & & \\
\end{tabular*}
\begin{tabular*}{\textwidth}[tb]{p{0.1\textwidth}p{0.9\textwidth}}
	 MP2\_OS\_SCALE\\ 

	 &  \\ 
\end{tabular*}
\begin{tabular*}{\textwidth}[tb]{p{0.3\textwidth}p{0.35\textwidth}p{0.35\textwidth}}
	   & {\bf Type:} double &  {\bf Default:} 6.0/5.0\\
	 & & \\
\end{tabular*}
\begin{tabular*}{\textwidth}[tb]{p{0.1\textwidth}p{0.9\textwidth}}
	 MP2\_SS\_SCALE\\ 

	 &  \\ 
\end{tabular*}
\begin{tabular*}{\textwidth}[tb]{p{0.3\textwidth}p{0.35\textwidth}p{0.35\textwidth}}
	   & {\bf Type:} double &  {\bf Default:} 1.0/3.0\\
	 & & \\
\end{tabular*}
\begin{tabular*}{\textwidth}[tb]{p{0.1\textwidth}p{0.9\textwidth}}
	 NAT\_ORBS\\ 

	 & Do ? \\ 
\end{tabular*}
\begin{tabular*}{\textwidth}[tb]{p{0.3\textwidth}p{0.35\textwidth}p{0.35\textwidth}}
	   & {\bf Type:} boolean &  {\bf Default:} false\\
	 & & \\
\end{tabular*}
\begin{tabular*}{\textwidth}[tb]{p{0.1\textwidth}p{0.9\textwidth}}
	 OMP2\_ORBS\_PRINT\\ 

	 & Do ? \\ 
\end{tabular*}
\begin{tabular*}{\textwidth}[tb]{p{0.3\textwidth}p{0.35\textwidth}p{0.35\textwidth}}
	   & {\bf Type:} boolean &  {\bf Default:} false\\
	 & & \\
\end{tabular*}
\begin{tabular*}{\textwidth}[tb]{p{0.1\textwidth}p{0.9\textwidth}}
	 OPT\_METHOD\\ 

	 &  \\ 

	  & {\bf Possible Values:} SD, DIIS \\ 
\end{tabular*}
\begin{tabular*}{\textwidth}[tb]{p{0.3\textwidth}p{0.35\textwidth}p{0.35\textwidth}}
	   & {\bf Type:} string &  {\bf Default:} DIIS\\
	 & & \\
\end{tabular*}
\begin{tabular*}{\textwidth}[tb]{p{0.1\textwidth}p{0.9\textwidth}}
	 ORTH\_TYPE\\ 

	 &  \\ 

	  & {\bf Possible Values:} GS, MGS \\ 
\end{tabular*}
\begin{tabular*}{\textwidth}[tb]{p{0.3\textwidth}p{0.35\textwidth}p{0.35\textwidth}}
	   & {\bf Type:} string &  {\bf Default:} MGS\\
	 & & \\
\end{tabular*}
\begin{tabular*}{\textwidth}[tb]{p{0.1\textwidth}p{0.9\textwidth}}
	 RMS\_MOGRAD\_CONVERGENCE\\ 

	 & See the note at the beginning of Section \ref{keywords}. \\ 
\end{tabular*}
\begin{tabular*}{\textwidth}[tb]{p{0.3\textwidth}p{0.35\textwidth}p{0.35\textwidth}}
	   & {\bf Type:} double &  {\bf Default:} 1e-5\\
	 & & \\
\end{tabular*}
\begin{tabular*}{\textwidth}[tb]{p{0.1\textwidth}p{0.9\textwidth}}
	 R\_CONVERGENCE\\ 

	 & See the note at the beginning of Section \ref{keywords}. \\ 
\end{tabular*}
\begin{tabular*}{\textwidth}[tb]{p{0.3\textwidth}p{0.35\textwidth}p{0.35\textwidth}}
	   & {\bf Type:} double &  {\bf Default:} 1e-5\\
	 & & \\
\end{tabular*}
\begin{tabular*}{\textwidth}[tb]{p{0.1\textwidth}p{0.9\textwidth}}
	 SOS\_SCALE\\ 

	 &  \\ 
\end{tabular*}
\begin{tabular*}{\textwidth}[tb]{p{0.3\textwidth}p{0.35\textwidth}p{0.35\textwidth}}
	   & {\bf Type:} double &  {\bf Default:} 1.3\\
	 & & \\
\end{tabular*}
\begin{tabular*}{\textwidth}[tb]{p{0.1\textwidth}p{0.9\textwidth}}
	 SOS\_SCALE2\\ 

	 &  \\ 
\end{tabular*}
\begin{tabular*}{\textwidth}[tb]{p{0.3\textwidth}p{0.35\textwidth}p{0.35\textwidth}}
	   & {\bf Type:} double &  {\bf Default:} 1.2\\
	 & & \\
\end{tabular*}

\subsection{OPTKING}

{\normalsize Performs geometry optimizations and vibrational frequency analyses.}\\
\begin{tabular*}{\textwidth}[tb]{c}
	  \\ 
\end{tabular*}
\begin{tabular*}{\textwidth}[tb]{p{0.1\textwidth}p{0.9\textwidth}}
	 ADD\_AUXILIARY\_BONDS\\ 

	 & Do add bond coordinates at nearby atoms for non-bonded systems? \\ 
\end{tabular*}
\begin{tabular*}{\textwidth}[tb]{p{0.3\textwidth}p{0.35\textwidth}p{0.35\textwidth}}
	   & {\bf Type:} boolean &  {\bf Default:} false\\
	 & & \\
\end{tabular*}
\begin{tabular*}{\textwidth}[tb]{p{0.1\textwidth}p{0.9\textwidth}}
	 CART\_HESS\_READ\\ 

	 & Do read Cartesian Hessian? \\ 
\end{tabular*}
\begin{tabular*}{\textwidth}[tb]{p{0.3\textwidth}p{0.35\textwidth}p{0.35\textwidth}}
	   & {\bf Type:} boolean &  {\bf Default:} false\\
	 & & \\
\end{tabular*}
\begin{tabular*}{\textwidth}[tb]{p{0.1\textwidth}p{0.9\textwidth}}
	 CONSECUTIVE\_BACKSTEPS\\ 

	 & Set number of consecutive backward steps allowed in optimization \\ 
\end{tabular*}
\begin{tabular*}{\textwidth}[tb]{p{0.3\textwidth}p{0.35\textwidth}p{0.35\textwidth}}
	   & {\bf Type:} integer &  {\bf Default:} 1\\
	 & & \\
\end{tabular*}
\begin{tabular*}{\textwidth}[tb]{p{0.1\textwidth}p{0.9\textwidth}}
	 COVALENT\_CONNECT\\ 

	 & When determining connectivity, a bond is assigned if interatomic distance is less than (this number) * sum of covalent radii {double} \\ 
\end{tabular*}
\begin{tabular*}{\textwidth}[tb]{p{0.3\textwidth}p{0.35\textwidth}p{0.35\textwidth}}
	   & {\bf Type:} double &  {\bf Default:} 1.3\\
	 & & \\
\end{tabular*}
\begin{tabular*}{\textwidth}[tb]{p{0.1\textwidth}p{0.9\textwidth}}
	 FINAL\_GEOM\_WRITE\\ 

	 & Do save and print the geometry from the last projected step at the end of a geometry optimization? Otherwise (and by default), save and print the previous geometry at which was computed the gradient that satisfied the convergence criteria. \\ 
\end{tabular*}
\begin{tabular*}{\textwidth}[tb]{p{0.3\textwidth}p{0.35\textwidth}p{0.35\textwidth}}
	   & {\bf Type:} boolean &  {\bf Default:} false\\
	 & & \\
\end{tabular*}
\begin{tabular*}{\textwidth}[tb]{p{0.1\textwidth}p{0.9\textwidth}}
	 FRAG\_MODE\\ 

	 & For multi-fragment molecules, treat as single bonded molecule or via interfragment coordinates. A primary difference is that in MULTI mode, the interfragment coordinates are not redundant. \\ 

	  & {\bf Possible Values:} SINGLE, MULTI \\ 
\end{tabular*}
\begin{tabular*}{\textwidth}[tb]{p{0.3\textwidth}p{0.35\textwidth}p{0.35\textwidth}}
	   & {\bf Type:} string &  {\bf Default:} SINGLE\\
	 & & \\
\end{tabular*}
\begin{tabular*}{\textwidth}[tb]{p{0.1\textwidth}p{0.9\textwidth}}
	 FREEZE\_INTRAFRAG\\ 

	 & Do freeze all fragments rigid? \\ 
\end{tabular*}
\begin{tabular*}{\textwidth}[tb]{p{0.3\textwidth}p{0.35\textwidth}p{0.35\textwidth}}
	   & {\bf Type:} boolean &  {\bf Default:} false\\
	 & & \\
\end{tabular*}
\begin{tabular*}{\textwidth}[tb]{p{0.1\textwidth}p{0.9\textwidth}}
	 G\_CONVERGENCE\\ 

	 & Set of optimization criteria. Specification of MAX\_ or RMS\_ G\_CONVERGENCE options// will append or overwrite the criteria set here. \\ 

	  & {\bf Possible Values:} GAU, GAU\_LOOSE, GAU\_TIGHT, GAU\_VERYTIGHT, QCHEM, NWCHEM, NWCHEM\_LOOSE, NWCHEM\_TIGHT, MOLPRO, PSI3, CFOUR \\ 
\end{tabular*}
\begin{tabular*}{\textwidth}[tb]{p{0.3\textwidth}p{0.35\textwidth}p{0.35\textwidth}}
	   & {\bf Type:} string &  {\bf Default:} QCHEM\\
	 & & \\
\end{tabular*}
\begin{tabular*}{\textwidth}[tb]{p{0.1\textwidth}p{0.9\textwidth}}
	 HESS\_UPDATE\\ 

	 & Hessian update scheme \\ 

	  & {\bf Possible Values:} NONE, BFGS, MS, POWELL, BOFILL \\ 
\end{tabular*}
\begin{tabular*}{\textwidth}[tb]{p{0.3\textwidth}p{0.35\textwidth}p{0.35\textwidth}}
	   & {\bf Type:} string &  {\bf Default:} BFGS\\
	 & & \\
\end{tabular*}
\begin{tabular*}{\textwidth}[tb]{p{0.1\textwidth}p{0.9\textwidth}}
	 HESS\_UPDATE\_LIMIT\\ 

	 & Do limit the magnitude of changes caused by the Hessian update? \\ 
\end{tabular*}
\begin{tabular*}{\textwidth}[tb]{p{0.3\textwidth}p{0.35\textwidth}p{0.35\textwidth}}
	   & {\bf Type:} boolean &  {\bf Default:} true\\
	 & & \\
\end{tabular*}
\begin{tabular*}{\textwidth}[tb]{p{0.1\textwidth}p{0.9\textwidth}}
	 HESS\_UPDATE\_LIMIT\_MAX\\ 

	 & If HESS\_UPDATE\_LIMIT is true, changes to the Hessian from the update are limited to the larger of (HESS\_UPDATE\_LIMIT\_SCALE)*(the previous value) and HESS\_UPDATE\_LIMIT\_MAX (in au). \\ 
\end{tabular*}
\begin{tabular*}{\textwidth}[tb]{p{0.3\textwidth}p{0.35\textwidth}p{0.35\textwidth}}
	   & {\bf Type:} double &  {\bf Default:} 1.00\\
	 & & \\
\end{tabular*}
\begin{tabular*}{\textwidth}[tb]{p{0.1\textwidth}p{0.9\textwidth}}
	 HESS\_UPDATE\_LIMIT\_SCALE\\ 

	 & If the above is true, changes to the Hessian from the update are limited to the larger of (HESS\_UPDATE\_LIMIT\_SCALE)*(the previous value) and HESS\_UPDATE\_LIMIT\_MAX (in au). \\ 
\end{tabular*}
\begin{tabular*}{\textwidth}[tb]{p{0.3\textwidth}p{0.35\textwidth}p{0.35\textwidth}}
	   & {\bf Type:} double &  {\bf Default:} 0.50\\
	 & & \\
\end{tabular*}
\begin{tabular*}{\textwidth}[tb]{p{0.1\textwidth}p{0.9\textwidth}}
	 HESS\_UPDATE\_USE\_LAST\\ 

	 & Number of previous steps to use in Hessian update, 0 uses all \\ 
\end{tabular*}
\begin{tabular*}{\textwidth}[tb]{p{0.3\textwidth}p{0.35\textwidth}p{0.35\textwidth}}
	   & {\bf Type:} integer &  {\bf Default:} 6\\
	 & & \\
\end{tabular*}
\begin{tabular*}{\textwidth}[tb]{p{0.1\textwidth}p{0.9\textwidth}}
	 H\_BOND\_CONNECT\\ 

	 & For now, this is a general maximum distance for the definition of H-bonds \\ 
\end{tabular*}
\begin{tabular*}{\textwidth}[tb]{p{0.3\textwidth}p{0.35\textwidth}p{0.35\textwidth}}
	   & {\bf Type:} double &  {\bf Default:} 4.3\\
	 & & \\
\end{tabular*}
\begin{tabular*}{\textwidth}[tb]{p{0.1\textwidth}p{0.9\textwidth}}
	 INTCOS\_GENERATE\_EXIT\\ 

	 & Do only generate the internal coordinates and then stop? \\ 
\end{tabular*}
\begin{tabular*}{\textwidth}[tb]{p{0.3\textwidth}p{0.35\textwidth}p{0.35\textwidth}}
	   & {\bf Type:} boolean &  {\bf Default:} false\\
	 & & \\
\end{tabular*}
\begin{tabular*}{\textwidth}[tb]{p{0.1\textwidth}p{0.9\textwidth}}
	 INTERFRAG\_DIST\_INV\\ 

	 & Do use $\frac{1}{R_{AB}}$ for the stretching coordinate between fragments? Otherwise, use $R_{AB}$. \\ 
\end{tabular*}
\begin{tabular*}{\textwidth}[tb]{p{0.3\textwidth}p{0.35\textwidth}p{0.35\textwidth}}
	   & {\bf Type:} boolean &  {\bf Default:} false\\
	 & & \\
\end{tabular*}
\begin{tabular*}{\textwidth}[tb]{p{0.1\textwidth}p{0.9\textwidth}}
	 INTERFRAG\_HESS\\ 

	 & Whether to use the default of FISCHER\_LIKE force constants for the initial guess {DEFAULT, FISCHER\_LIKE} \\ 

	  & {\bf Possible Values:} DEFAULT, FISCHER\_LIKE \\ 
\end{tabular*}
\begin{tabular*}{\textwidth}[tb]{p{0.3\textwidth}p{0.35\textwidth}p{0.35\textwidth}}
	   & {\bf Type:} string &  {\bf Default:} DEFAULT\\
	 & & \\
\end{tabular*}
\begin{tabular*}{\textwidth}[tb]{p{0.1\textwidth}p{0.9\textwidth}}
	 INTERFRAG\_MODE\\ 

	 & When interfragment coordinates are present, use as reference points either principal axes or fixed linear combinations of atoms. \\ 

	  & {\bf Possible Values:} FIXED, INTERFRAGMENT \\ 
\end{tabular*}
\begin{tabular*}{\textwidth}[tb]{p{0.3\textwidth}p{0.35\textwidth}p{0.35\textwidth}}
	   & {\bf Type:} string &  {\bf Default:} FIXED\\
	 & & \\
\end{tabular*}
\begin{tabular*}{\textwidth}[tb]{p{0.1\textwidth}p{0.9\textwidth}}
	 INTRAFRAG\_HESS\\ 

	 & What model Hessian to use to guess intrafragment force constants {SCHLEGEL, FISCHER} \\ 

	  & {\bf Possible Values:} FISCHER, SCHLEGEL, LINDH, SIMPLE \\ 
\end{tabular*}
\begin{tabular*}{\textwidth}[tb]{p{0.3\textwidth}p{0.35\textwidth}p{0.35\textwidth}}
	   & {\bf Type:} string &  {\bf Default:} FISCHER\\
	 & & \\
\end{tabular*}
\begin{tabular*}{\textwidth}[tb]{p{0.1\textwidth}p{0.9\textwidth}}
	 INTRAFRAG\_STEP\_LIMIT\\ 

	 & Initial maximum step size in bohr or radian along an internal coordinate \\ 
\end{tabular*}
\begin{tabular*}{\textwidth}[tb]{p{0.3\textwidth}p{0.35\textwidth}p{0.35\textwidth}}
	   & {\bf Type:} double &  {\bf Default:} 0.4\\
	 & & \\
\end{tabular*}
\begin{tabular*}{\textwidth}[tb]{p{0.1\textwidth}p{0.9\textwidth}}
	 INTRAFRAG\_STEP\_LIMIT\_MAX\\ 

	 & Upper bound for dynamic trust radius [au] \\ 
\end{tabular*}
\begin{tabular*}{\textwidth}[tb]{p{0.3\textwidth}p{0.35\textwidth}p{0.35\textwidth}}
	   & {\bf Type:} double &  {\bf Default:} 1.0\\
	 & & \\
\end{tabular*}
\begin{tabular*}{\textwidth}[tb]{p{0.1\textwidth}p{0.9\textwidth}}
	 INTRAFRAG\_STEP\_LIMIT\_MIN\\ 

	 & Lower bound for dynamic trust radius [au] \\ 
\end{tabular*}
\begin{tabular*}{\textwidth}[tb]{p{0.3\textwidth}p{0.35\textwidth}p{0.35\textwidth}}
	   & {\bf Type:} double &  {\bf Default:} 0.001\\
	 & & \\
\end{tabular*}
\begin{tabular*}{\textwidth}[tb]{p{0.1\textwidth}p{0.9\textwidth}}
	 IRC\_DIRECTION\\ 

	 & IRC mapping direction \\ 

	  & {\bf Possible Values:} FORWARD, BACKWARD \\ 
\end{tabular*}
\begin{tabular*}{\textwidth}[tb]{p{0.3\textwidth}p{0.35\textwidth}p{0.35\textwidth}}
	   & {\bf Type:} string &  {\bf Default:} FORWARD\\
	 & & \\
\end{tabular*}
\begin{tabular*}{\textwidth}[tb]{p{0.1\textwidth}p{0.9\textwidth}}
	 IRC\_STEP\_SIZE\\ 

	 & IRC step size in bohr(amu)$^{1/2}$ \\ 
\end{tabular*}
\begin{tabular*}{\textwidth}[tb]{p{0.3\textwidth}p{0.35\textwidth}p{0.35\textwidth}}
	   & {\bf Type:} double &  {\bf Default:} 0.2\\
	 & & \\
\end{tabular*}
\begin{tabular*}{\textwidth}[tb]{p{0.1\textwidth}p{0.9\textwidth}}
	 MAX\_DISP\_G\_CONVERGENCE\\ 

	 & QCHEM optimization criteria: maximum displacement. See the note at the beginning of Section \ref{keywords}. \\ 
\end{tabular*}
\begin{tabular*}{\textwidth}[tb]{p{0.3\textwidth}p{0.35\textwidth}p{0.35\textwidth}}
	   & {\bf Type:} double &  {\bf Default:} 1.2e-3\\
	 & & \\
\end{tabular*}
\begin{tabular*}{\textwidth}[tb]{p{0.1\textwidth}p{0.9\textwidth}}
	 MAX\_ENERGY\_G\_CONVERGENCE\\ 

	 & QCHEM optimization criteria: maximum energy change. See the note at the beginning of Section \ref{keywords}. \\ 
\end{tabular*}
\begin{tabular*}{\textwidth}[tb]{p{0.3\textwidth}p{0.35\textwidth}p{0.35\textwidth}}
	   & {\bf Type:} double &  {\bf Default:} 1.0e-6\\
	 & & \\
\end{tabular*}
\begin{tabular*}{\textwidth}[tb]{p{0.1\textwidth}p{0.9\textwidth}}
	 MAX\_FORCE\_G\_CONVERGENCE\\ 

	 & QCHEM optimization criteria: maximum force. See the note at the beginning of Section \ref{keywords}. \\ 
\end{tabular*}
\begin{tabular*}{\textwidth}[tb]{p{0.3\textwidth}p{0.35\textwidth}p{0.35\textwidth}}
	   & {\bf Type:} double &  {\bf Default:} 3.0e-4\\
	 & & \\
\end{tabular*}
\begin{tabular*}{\textwidth}[tb]{p{0.1\textwidth}p{0.9\textwidth}}
	 OPT\_TYPE\\ 

	 & Specifies minimum search, transition-state search, or IRC following \\ 

	  & {\bf Possible Values:} MIN, TS, IRC \\ 
\end{tabular*}
\begin{tabular*}{\textwidth}[tb]{p{0.3\textwidth}p{0.35\textwidth}p{0.35\textwidth}}
	   & {\bf Type:} string &  {\bf Default:} MIN\\
	 & & \\
\end{tabular*}
\begin{tabular*}{\textwidth}[tb]{p{0.1\textwidth}p{0.9\textwidth}}
	 RFO\_FOLLOW\_ROOT\\ 

	 & Do follow the initial RFO vector after the first step? \\ 
\end{tabular*}
\begin{tabular*}{\textwidth}[tb]{p{0.3\textwidth}p{0.35\textwidth}p{0.35\textwidth}}
	   & {\bf Type:} boolean &  {\bf Default:} false\\
	 & & \\
\end{tabular*}
\begin{tabular*}{\textwidth}[tb]{p{0.1\textwidth}p{0.9\textwidth}}
	 RFO\_ROOT\\ 

	 & Root for RFO to follow, 0 being lowest (for a minimum) \\ 
\end{tabular*}
\begin{tabular*}{\textwidth}[tb]{p{0.3\textwidth}p{0.35\textwidth}p{0.35\textwidth}}
	   & {\bf Type:} integer &  {\bf Default:} 0\\
	 & & \\
\end{tabular*}
\begin{tabular*}{\textwidth}[tb]{p{0.1\textwidth}p{0.9\textwidth}}
	 RMS\_DISP\_G\_CONVERGENCE\\ 

	 & QCHEM optimization criteria: rms displacement. See the note at the beginning of Section \ref{keywords}. \\ 
\end{tabular*}
\begin{tabular*}{\textwidth}[tb]{p{0.3\textwidth}p{0.35\textwidth}p{0.35\textwidth}}
	   & {\bf Type:} double &  {\bf Default:} 1.2e-3\\
	 & & \\
\end{tabular*}
\begin{tabular*}{\textwidth}[tb]{p{0.1\textwidth}p{0.9\textwidth}}
	 RMS\_FORCE\_G\_CONVERGENCE\\ 

	 & QCHEM optimization criteria: rms force. See the note at the beginning of Section \ref{keywords}. \\ 
\end{tabular*}
\begin{tabular*}{\textwidth}[tb]{p{0.3\textwidth}p{0.35\textwidth}p{0.35\textwidth}}
	   & {\bf Type:} double &  {\bf Default:} 3.0e-4\\
	 & & \\
\end{tabular*}
\begin{tabular*}{\textwidth}[tb]{p{0.1\textwidth}p{0.9\textwidth}}
	 STEP\_TYPE\\ 

	 & Geometry optimization step type, either Newton-Raphson or Rational Function Optimization \\ 

	  & {\bf Possible Values:} RFO, NR, SD \\ 
\end{tabular*}
\begin{tabular*}{\textwidth}[tb]{p{0.3\textwidth}p{0.35\textwidth}p{0.35\textwidth}}
	   & {\bf Type:} string &  {\bf Default:} RFO\\
	 & & \\
\end{tabular*}
\begin{tabular*}{\textwidth}[tb]{p{0.1\textwidth}p{0.9\textwidth}}
	 TEST\_B\\ 

	 & Do test B matrix? \\ 
\end{tabular*}
\begin{tabular*}{\textwidth}[tb]{p{0.3\textwidth}p{0.35\textwidth}p{0.35\textwidth}}
	   & {\bf Type:} boolean &  {\bf Default:} false\\
	 & & \\
\end{tabular*}
\begin{tabular*}{\textwidth}[tb]{p{0.1\textwidth}p{0.9\textwidth}}
	 TEST\_DERIVATIVE\_B\\ 

	 & Do test derivative B matrix? \\ 
\end{tabular*}
\begin{tabular*}{\textwidth}[tb]{p{0.3\textwidth}p{0.35\textwidth}p{0.35\textwidth}}
	   & {\bf Type:} boolean &  {\bf Default:} false\\
	 & & \\
\end{tabular*}

\subsection{PSIMRCC}

{\normalsize Performs multireference coupled cluster computations. This theory should be used only by advanced users with a good working knowledge of multireference techniques.}\\
\begin{tabular*}{\textwidth}[tb]{c}
	  \\ 
\end{tabular*}
\begin{tabular*}{\textwidth}[tb]{p{0.1\textwidth}p{0.9\textwidth}}
	 ACTIVE\\ 

	 & The number of active orbitals per irrep \\ 
\end{tabular*}
\begin{tabular*}{\textwidth}[tb]{p{0.3\textwidth}p{0.35\textwidth}p{0.35\textwidth}}
	   & {\bf Type:} array &  {\bf Default:} No Default\\
	 & & \\
\end{tabular*}
\begin{tabular*}{\textwidth}[tb]{p{0.1\textwidth}p{0.9\textwidth}}
	 CORR\_ANSATZ\\ 

	 & The ansatz to use for MRCC computations \\ 

	  & {\bf Possible Values:} SR, MK, BW, APBW \\ 
\end{tabular*}
\begin{tabular*}{\textwidth}[tb]{p{0.3\textwidth}p{0.35\textwidth}p{0.35\textwidth}}
	   & {\bf Type:} string &  {\bf Default:} MK\\
	 & & \\
\end{tabular*}
\begin{tabular*}{\textwidth}[tb]{p{0.1\textwidth}p{0.9\textwidth}}
	 CORR\_CCSD\_T\\ 

	 & The type of CCSD(T) computation to perform \\ 

	  & {\bf Possible Values:} STANDARD, PITTNER \\ 
\end{tabular*}
\begin{tabular*}{\textwidth}[tb]{p{0.3\textwidth}p{0.35\textwidth}p{0.35\textwidth}}
	   & {\bf Type:} string &  {\bf Default:} STANDARD\\
	 & & \\
\end{tabular*}
\begin{tabular*}{\textwidth}[tb]{p{0.1\textwidth}p{0.9\textwidth}}
	 CORR\_CHARGE\\ 

	 & The molecular charge of the target state \\ 
\end{tabular*}
\begin{tabular*}{\textwidth}[tb]{p{0.3\textwidth}p{0.35\textwidth}p{0.35\textwidth}}
	   & {\bf Type:} integer &  {\bf Default:} 0\\
	 & & \\
\end{tabular*}
\begin{tabular*}{\textwidth}[tb]{p{0.1\textwidth}p{0.9\textwidth}}
	 CORR\_MULTP\\ 

	 & The multiplicity, $M_S(M_S+1)$, of the target state. Must be specified if different from the reference $M_s$. \\ 
\end{tabular*}
\begin{tabular*}{\textwidth}[tb]{p{0.3\textwidth}p{0.35\textwidth}p{0.35\textwidth}}
	   & {\bf Type:} integer &  {\bf Default:} 1\\
	 & & \\
\end{tabular*}
\begin{tabular*}{\textwidth}[tb]{p{0.1\textwidth}p{0.9\textwidth}}
	 CORR\_REFERENCE\\ 

	 & Reference wavefunction type used in MRCC computations \\ 

	  & {\bf Possible Values:} RHF, ROHF, TCSCF, MCSCF, GENERAL \\ 
\end{tabular*}
\begin{tabular*}{\textwidth}[tb]{p{0.3\textwidth}p{0.35\textwidth}p{0.35\textwidth}}
	   & {\bf Type:} string &  {\bf Default:} GENERAL\\
	 & & \\
\end{tabular*}
\begin{tabular*}{\textwidth}[tb]{p{0.1\textwidth}p{0.9\textwidth}}
	 CORR\_WFN\\ 

	 & The type of correlated wavefunction \\ 

	  & {\bf Possible Values:} PT2, CCSD, MP2-CCSD, CCSD\_T \\ 
\end{tabular*}
\begin{tabular*}{\textwidth}[tb]{p{0.3\textwidth}p{0.35\textwidth}p{0.35\textwidth}}
	   & {\bf Type:} string &  {\bf Default:} CCSD\\
	 & & \\
\end{tabular*}
\begin{tabular*}{\textwidth}[tb]{p{0.1\textwidth}p{0.9\textwidth}}
	 COUPLING\\ 

	 & The order of coupling terms to include in MRCCSDT computations \\ 

	  & {\bf Possible Values:} NONE, LINEAR, QUADRATIC, CUBIC \\ 
\end{tabular*}
\begin{tabular*}{\textwidth}[tb]{p{0.3\textwidth}p{0.35\textwidth}p{0.35\textwidth}}
	   & {\bf Type:} string &  {\bf Default:} CUBIC\\
	 & & \\
\end{tabular*}
\begin{tabular*}{\textwidth}[tb]{p{0.1\textwidth}p{0.9\textwidth}}
	 COUPLING\_TERMS\\ 

	 & Do include the terms that couple the reference determinants? \\ 
\end{tabular*}
\begin{tabular*}{\textwidth}[tb]{p{0.3\textwidth}p{0.35\textwidth}p{0.35\textwidth}}
	   & {\bf Type:} boolean &  {\bf Default:} true\\
	 & & \\
\end{tabular*}
\begin{tabular*}{\textwidth}[tb]{p{0.1\textwidth}p{0.9\textwidth}}
	 DAMPING\_PERCENTAGE\\ 

	 & The amount (percentage) of damping to apply to the amplitude updates. 0 will result in a full update, 100 will completely stall the update. A value around 20 (which corresponds to 20\% of the amplitudes from the previous iteration being mixed into the current iteration) can help in cases where oscillatory convergence is observed. \\ 
\end{tabular*}
\begin{tabular*}{\textwidth}[tb]{p{0.3\textwidth}p{0.35\textwidth}p{0.35\textwidth}}
	   & {\bf Type:} double &  {\bf Default:} 0.0\\
	 & & \\
\end{tabular*}
\begin{tabular*}{\textwidth}[tb]{p{0.1\textwidth}p{0.9\textwidth}}
	 DIAGONALIZE\_HEFF\\ 

	 & Do diagonalize the effective Hamiltonian? \\ 
\end{tabular*}
\begin{tabular*}{\textwidth}[tb]{p{0.3\textwidth}p{0.35\textwidth}p{0.35\textwidth}}
	   & {\bf Type:} boolean &  {\bf Default:} false\\
	 & & \\
\end{tabular*}
\begin{tabular*}{\textwidth}[tb]{p{0.1\textwidth}p{0.9\textwidth}}
	 DIAGONAL\_CCSD\_T\\ 

	 & Do include the diagonal corrections in (T) computations? \\ 
\end{tabular*}
\begin{tabular*}{\textwidth}[tb]{p{0.3\textwidth}p{0.35\textwidth}p{0.35\textwidth}}
	   & {\bf Type:} boolean &  {\bf Default:} true\\
	 & & \\
\end{tabular*}
\begin{tabular*}{\textwidth}[tb]{p{0.1\textwidth}p{0.9\textwidth}}
	 DIIS\_MAX\_VECS\\ 

	 & Maximum number of error vectors stored for DIIS extrapolation \\ 
\end{tabular*}
\begin{tabular*}{\textwidth}[tb]{p{0.3\textwidth}p{0.35\textwidth}p{0.35\textwidth}}
	   & {\bf Type:} integer &  {\bf Default:} 7\\
	 & & \\
\end{tabular*}
\begin{tabular*}{\textwidth}[tb]{p{0.1\textwidth}p{0.9\textwidth}}
	 DIIS\_START\\ 

	 & The number of DIIS vectors needed before extrapolation is performed \\ 
\end{tabular*}
\begin{tabular*}{\textwidth}[tb]{p{0.3\textwidth}p{0.35\textwidth}p{0.35\textwidth}}
	   & {\bf Type:} integer &  {\bf Default:} 2\\
	 & & \\
\end{tabular*}
\begin{tabular*}{\textwidth}[tb]{p{0.1\textwidth}p{0.9\textwidth}}
	 E\_CONVERGENCE\\ 

	 & Convergence criterion for energy. See the note at the beginning of Section \ref{keywords}. \\ 
\end{tabular*}
\begin{tabular*}{\textwidth}[tb]{p{0.3\textwidth}p{0.35\textwidth}p{0.35\textwidth}}
	   & {\bf Type:} double &  {\bf Default:} 1e-9\\
	 & & \\
\end{tabular*}
\begin{tabular*}{\textwidth}[tb]{p{0.1\textwidth}p{0.9\textwidth}}
	 FAVG\_CCSD\_T\\ 

	 & Do use the averaged Fock matrix over all references in (T) computations? \\ 
\end{tabular*}
\begin{tabular*}{\textwidth}[tb]{p{0.3\textwidth}p{0.35\textwidth}p{0.35\textwidth}}
	   & {\bf Type:} boolean &  {\bf Default:} false\\
	 & & \\
\end{tabular*}
\begin{tabular*}{\textwidth}[tb]{p{0.1\textwidth}p{0.9\textwidth}}
	 FOLLOW\_ROOT\\ 

	 & Which root of the effective hamiltonian is the target state? \\ 
\end{tabular*}
\begin{tabular*}{\textwidth}[tb]{p{0.3\textwidth}p{0.35\textwidth}p{0.35\textwidth}}
	   & {\bf Type:} integer &  {\bf Default:} 1\\
	 & & \\
\end{tabular*}
\begin{tabular*}{\textwidth}[tb]{p{0.1\textwidth}p{0.9\textwidth}}
	 FROZEN\_DOCC\\ 

	 & The number of frozen occupied orbitals per irrep \\ 
\end{tabular*}
\begin{tabular*}{\textwidth}[tb]{p{0.3\textwidth}p{0.35\textwidth}p{0.35\textwidth}}
	   & {\bf Type:} array &  {\bf Default:} No Default\\
	 & & \\
\end{tabular*}
\begin{tabular*}{\textwidth}[tb]{p{0.1\textwidth}p{0.9\textwidth}}
	 FROZEN\_UOCC\\ 

	 & The number of frozen virtual orbitals per irrep \\ 
\end{tabular*}
\begin{tabular*}{\textwidth}[tb]{p{0.3\textwidth}p{0.35\textwidth}p{0.35\textwidth}}
	   & {\bf Type:} array &  {\bf Default:} No Default\\
	 & & \\
\end{tabular*}
\begin{tabular*}{\textwidth}[tb]{p{0.1\textwidth}p{0.9\textwidth}}
	 HEFF4\\ 

	 & Do include the fourth-order contributions to the effective Hamiltonian? \\ 
\end{tabular*}
\begin{tabular*}{\textwidth}[tb]{p{0.3\textwidth}p{0.35\textwidth}p{0.35\textwidth}}
	   & {\bf Type:} boolean &  {\bf Default:} true\\
	 & & \\
\end{tabular*}
\begin{tabular*}{\textwidth}[tb]{p{0.1\textwidth}p{0.9\textwidth}}
	 HEFF\_PRINT\\ 

	 & Do print the effective Hamiltonian? \\ 
\end{tabular*}
\begin{tabular*}{\textwidth}[tb]{p{0.3\textwidth}p{0.35\textwidth}p{0.35\textwidth}}
	   & {\bf Type:} boolean &  {\bf Default:} false\\
	 & & \\
\end{tabular*}
\begin{tabular*}{\textwidth}[tb]{p{0.1\textwidth}p{0.9\textwidth}}
	 LOCK\_SINGLET\\ 

	 & Do lock onto a singlet root? \\ 
\end{tabular*}
\begin{tabular*}{\textwidth}[tb]{p{0.3\textwidth}p{0.35\textwidth}p{0.35\textwidth}}
	   & {\bf Type:} boolean &  {\bf Default:} false\\
	 & & \\
\end{tabular*}
\begin{tabular*}{\textwidth}[tb]{p{0.1\textwidth}p{0.9\textwidth}}
	 MAXITER\\ 

	 & Maximum number of iterations to determine the amplitudes \\ 
\end{tabular*}
\begin{tabular*}{\textwidth}[tb]{p{0.3\textwidth}p{0.35\textwidth}p{0.35\textwidth}}
	   & {\bf Type:} integer &  {\bf Default:} 100\\
	 & & \\
\end{tabular*}
\begin{tabular*}{\textwidth}[tb]{p{0.1\textwidth}p{0.9\textwidth}}
	 MP2\_CCSD\_METHOD\\ 

	 & How to perform MP2\_CCSD computations \\ 

	  & {\bf Possible Values:} I, IA, II \\ 
\end{tabular*}
\begin{tabular*}{\textwidth}[tb]{p{0.3\textwidth}p{0.35\textwidth}p{0.35\textwidth}}
	   & {\bf Type:} string &  {\bf Default:} II\\
	 & & \\
\end{tabular*}
\begin{tabular*}{\textwidth}[tb]{p{0.1\textwidth}p{0.9\textwidth}}
	 MP2\_GUESS\\ 

	 & Do start from a MP2 guess? \\ 
\end{tabular*}
\begin{tabular*}{\textwidth}[tb]{p{0.3\textwidth}p{0.35\textwidth}p{0.35\textwidth}}
	   & {\bf Type:} boolean &  {\bf Default:} true\\
	 & & \\
\end{tabular*}
\begin{tabular*}{\textwidth}[tb]{p{0.1\textwidth}p{0.9\textwidth}}
	 NO\_SINGLES\\ 

	 & Do ? \\ 
\end{tabular*}
\begin{tabular*}{\textwidth}[tb]{p{0.3\textwidth}p{0.35\textwidth}p{0.35\textwidth}}
	   & {\bf Type:} boolean &  {\bf Default:} false\\
	 & & \\
\end{tabular*}
\begin{tabular*}{\textwidth}[tb]{p{0.1\textwidth}p{0.9\textwidth}}
	 NUM\_THREADS\\ 

	 & Number of threads \\ 
\end{tabular*}
\begin{tabular*}{\textwidth}[tb]{p{0.3\textwidth}p{0.35\textwidth}p{0.35\textwidth}}
	   & {\bf Type:} integer &  {\bf Default:} 1\\
	 & & \\
\end{tabular*}
\begin{tabular*}{\textwidth}[tb]{p{0.1\textwidth}p{0.9\textwidth}}
	 OFFDIAGONAL\_CCSD\_T\\ 

	 & Do include the off-diagonal corrections in (T) computations? \\ 
\end{tabular*}
\begin{tabular*}{\textwidth}[tb]{p{0.3\textwidth}p{0.35\textwidth}p{0.35\textwidth}}
	   & {\bf Type:} boolean &  {\bf Default:} true\\
	 & & \\
\end{tabular*}
\begin{tabular*}{\textwidth}[tb]{p{0.1\textwidth}p{0.9\textwidth}}
	 PT\_ENERGY\\ 

	 & The type of perturbation theory computation to perform \\ 

	  & {\bf Possible Values:} SECOND\_ORDER, SCS\_SECOND\_ORDER, PSEUDO\_SECOND\_ORDER, SCS\_PSEUDO\_SECOND\_ORDER \\ 
\end{tabular*}
\begin{tabular*}{\textwidth}[tb]{p{0.3\textwidth}p{0.35\textwidth}p{0.35\textwidth}}
	   & {\bf Type:} string &  {\bf Default:} SECOND\_ORDER\\
	 & & \\
\end{tabular*}
\begin{tabular*}{\textwidth}[tb]{p{0.1\textwidth}p{0.9\textwidth}}
	 RESTRICTED\_DOCC\\ 

	 & The number of doubly occupied orbitals per irrep \\ 
\end{tabular*}
\begin{tabular*}{\textwidth}[tb]{p{0.3\textwidth}p{0.35\textwidth}p{0.35\textwidth}}
	   & {\bf Type:} array &  {\bf Default:} No Default\\
	 & & \\
\end{tabular*}
\begin{tabular*}{\textwidth}[tb]{p{0.1\textwidth}p{0.9\textwidth}}
	 R\_CONVERGENCE\\ 

	 & Convergence criterion for amplitudes (residuals). See the note at the beginning of Section \ref{keywords}. \\ 
\end{tabular*}
\begin{tabular*}{\textwidth}[tb]{p{0.3\textwidth}p{0.35\textwidth}p{0.35\textwidth}}
	   & {\bf Type:} double &  {\bf Default:} 1e-9\\
	 & & \\
\end{tabular*}
\begin{tabular*}{\textwidth}[tb]{p{0.1\textwidth}p{0.9\textwidth}}
	 SMALL\_CUTOFF\\ 

	 &  \\ 
\end{tabular*}
\begin{tabular*}{\textwidth}[tb]{p{0.3\textwidth}p{0.35\textwidth}p{0.35\textwidth}}
	   & {\bf Type:} integer &  {\bf Default:} 0\\
	 & & \\
\end{tabular*}
\begin{tabular*}{\textwidth}[tb]{p{0.1\textwidth}p{0.9\textwidth}}
	 TIKHONOW\_MAX\\ 

	 & The cycle after which Tikhonow regularization is stopped. Set to zero to allow regularization in all iterations \\ 
\end{tabular*}
\begin{tabular*}{\textwidth}[tb]{p{0.3\textwidth}p{0.35\textwidth}p{0.35\textwidth}}
	   & {\bf Type:} integer &  {\bf Default:} 5\\
	 & & \\
\end{tabular*}
\begin{tabular*}{\textwidth}[tb]{p{0.1\textwidth}p{0.9\textwidth}}
	 TIKHONOW\_OMEGA\\ 

	 & The shift to apply to the denominators, {\it c.f.} Taube and Bartlett, JCP, 130, 144112 (2009) \\ 
\end{tabular*}
\begin{tabular*}{\textwidth}[tb]{p{0.3\textwidth}p{0.35\textwidth}p{0.35\textwidth}}
	   & {\bf Type:} double &  {\bf Default:} 0.0\\
	 & & \\
\end{tabular*}
\begin{tabular*}{\textwidth}[tb]{p{0.1\textwidth}p{0.9\textwidth}}
	 TRIPLES\_ALGORITHM\\ 

	 & The type of algorithm to use for (T) computations \\ 

	  & {\bf Possible Values:} SPIN\_ADAPTED, RESTRICTED, UNRESTRICTED \\ 
\end{tabular*}
\begin{tabular*}{\textwidth}[tb]{p{0.3\textwidth}p{0.35\textwidth}p{0.35\textwidth}}
	   & {\bf Type:} string &  {\bf Default:} RESTRICTED\\
	 & & \\
\end{tabular*}
\begin{tabular*}{\textwidth}[tb]{p{0.1\textwidth}p{0.9\textwidth}}
	 TRIPLES\_DIIS\\ 

	 & Do use DIIS extrapolation to accelerate convergence for iterative triples excitations? \\ 
\end{tabular*}
\begin{tabular*}{\textwidth}[tb]{p{0.3\textwidth}p{0.35\textwidth}p{0.35\textwidth}}
	   & {\bf Type:} boolean &  {\bf Default:} false\\
	 & & \\
\end{tabular*}
\begin{tabular*}{\textwidth}[tb]{p{0.1\textwidth}p{0.9\textwidth}}
	 USE\_SPIN\_SYM\\ 

	 & Do use symmetry to map equivalent determinants onto each other, for efficiency? \\ 
\end{tabular*}
\begin{tabular*}{\textwidth}[tb]{p{0.3\textwidth}p{0.35\textwidth}p{0.35\textwidth}}
	   & {\bf Type:} boolean &  {\bf Default:} true\\
	 & & \\
\end{tabular*}
\begin{tabular*}{\textwidth}[tb]{p{0.1\textwidth}p{0.9\textwidth}}
	 WFN\_SYM\\ 

	 & The symmetry of the target wavefunction, specified either by Sch\"onflies symbol, or irrep number (in Cotton ordering) \\ 

	  & {\bf Possible Values:} A, AG, AU, AP, APP, A1, A2, B, BG, BU, B1, B2, B3, B1G, B2G, B3G, B1U, B2U, B3U, 0, 1, 2, 3, 4, 5, 6, 7, 8 \\ 
\end{tabular*}
\begin{tabular*}{\textwidth}[tb]{p{0.3\textwidth}p{0.35\textwidth}p{0.35\textwidth}}
	   & {\bf Type:} string &  {\bf Default:} 1\\
	 & & \\
\end{tabular*}
\begin{tabular*}{\textwidth}[tb]{p{0.1\textwidth}p{0.9\textwidth}}
	 ZERO\_INTERNAL\_AMPS\\ 

	 & Do zero the internal amplitudes, i.e., those that map reference determinants onto each other? \\ 
\end{tabular*}
\begin{tabular*}{\textwidth}[tb]{p{0.3\textwidth}p{0.35\textwidth}p{0.35\textwidth}}
	   & {\bf Type:} boolean &  {\bf Default:} true\\
	 & & \\
\end{tabular*}

\subsection{RESPONSE}

{\normalsize Performs SCF linear response computations.}\\
\begin{tabular*}{\textwidth}[tb]{c}
	  \\ 
\end{tabular*}
\begin{tabular*}{\textwidth}[tb]{p{0.1\textwidth}p{0.9\textwidth}}
	 OMEGA\\ 

	 &  \\ 
\end{tabular*}
\begin{tabular*}{\textwidth}[tb]{p{0.3\textwidth}p{0.35\textwidth}p{0.35\textwidth}}
	   & {\bf Type:} array &  {\bf Default:} No Default\\
	 & & \\
\end{tabular*}
\begin{tabular*}{\textwidth}[tb]{p{0.1\textwidth}p{0.9\textwidth}}
	 PROPERTY\\ 

	 &  \\ 
\end{tabular*}
\begin{tabular*}{\textwidth}[tb]{p{0.3\textwidth}p{0.35\textwidth}p{0.35\textwidth}}
	   & {\bf Type:} string &  {\bf Default:} POLARIZABILITY\\
	 & & \\
\end{tabular*}
\begin{tabular*}{\textwidth}[tb]{p{0.1\textwidth}p{0.9\textwidth}}
	 REFERENCE\\ 

	 & Reference wavefunction type \\ 
\end{tabular*}
\begin{tabular*}{\textwidth}[tb]{p{0.3\textwidth}p{0.35\textwidth}p{0.35\textwidth}}
	   & {\bf Type:} string &  {\bf Default:} RHF\\
	 & & \\
\end{tabular*}

\subsection{SAPT}

{\normalsize Performs symmetry adapted perturbation theory (SAPT) analysis to quantitatively analyze noncovalent interactions.}\\
\begin{tabular*}{\textwidth}[tb]{c}
	  \\ 
\end{tabular*}
\begin{tabular*}{\textwidth}[tb]{p{0.1\textwidth}p{0.9\textwidth}}
	 AIO\_CPHF\\ 

	 & Do use asynchronous I/O in the CPHF solver? \\ 
\end{tabular*}
\begin{tabular*}{\textwidth}[tb]{p{0.3\textwidth}p{0.35\textwidth}p{0.35\textwidth}}
	   & {\bf Type:} boolean &  {\bf Default:} false\\
	 & & \\
\end{tabular*}
\begin{tabular*}{\textwidth}[tb]{p{0.1\textwidth}p{0.9\textwidth}}
	 AIO\_DF\_INTS\\ 

	 & Do use asynchronous I/O in the DF integral formation? \\ 
\end{tabular*}
\begin{tabular*}{\textwidth}[tb]{p{0.3\textwidth}p{0.35\textwidth}p{0.35\textwidth}}
	   & {\bf Type:} boolean &  {\bf Default:} false\\
	 & & \\
\end{tabular*}
\begin{tabular*}{\textwidth}[tb]{p{0.1\textwidth}p{0.9\textwidth}}
	 DEBUG\\ 

	 & The ubiquitous debug flag \\ 
\end{tabular*}
\begin{tabular*}{\textwidth}[tb]{p{0.3\textwidth}p{0.35\textwidth}p{0.35\textwidth}}
	   & {\bf Type:} integer &  {\bf Default:} 0\\
	 & & \\
\end{tabular*}
\begin{tabular*}{\textwidth}[tb]{p{0.1\textwidth}p{0.9\textwidth}}
	 DENOMINATOR\_ALGORITHM\\ 

	 & Denominator algorithm for PT methods \\ 

	  & {\bf Possible Values:} LAPLACE, CHOLESKY \\ 
\end{tabular*}
\begin{tabular*}{\textwidth}[tb]{p{0.3\textwidth}p{0.35\textwidth}p{0.35\textwidth}}
	   & {\bf Type:} string &  {\bf Default:} LAPLACE\\
	 & & \\
\end{tabular*}
\begin{tabular*}{\textwidth}[tb]{p{0.1\textwidth}p{0.9\textwidth}}
	 DENOMINATOR\_DELTA\\ 

	 & Maximum denominator error allowed (Max error norm in Delta tensor) \\ 
\end{tabular*}
\begin{tabular*}{\textwidth}[tb]{p{0.3\textwidth}p{0.35\textwidth}p{0.35\textwidth}}
	   & {\bf Type:} double &  {\bf Default:} 1.0e-6\\
	 & & \\
\end{tabular*}
\begin{tabular*}{\textwidth}[tb]{p{0.1\textwidth}p{0.9\textwidth}}
	 DF\_BASIS\_ELST\\ 

	 & Auxiliary basis set for SAPT Elst10 and Exch10 density fitting computations. Defaults to BASIS-RI. \\ 
\end{tabular*}
\begin{tabular*}{\textwidth}[tb]{p{0.3\textwidth}p{0.35\textwidth}p{0.35\textwidth}}
	   & {\bf Type:} string &  {\bf Default:} No Default\\
	 & & \\
\end{tabular*}
\begin{tabular*}{\textwidth}[tb]{p{0.1\textwidth}p{0.9\textwidth}}
	 DF\_BASIS\_SAPT\\ 

	 & Auxiliary basis set for SAPT density fitting computations. Defaults to BASIS-RI. \\ 
\end{tabular*}
\begin{tabular*}{\textwidth}[tb]{p{0.3\textwidth}p{0.35\textwidth}p{0.35\textwidth}}
	   & {\bf Type:} string &  {\bf Default:} No Default\\
	 & & \\
\end{tabular*}
\begin{tabular*}{\textwidth}[tb]{p{0.1\textwidth}p{0.9\textwidth}}
	 DO\_THIRD\_ORDER\\ 

	 & Do compute third-order corrections? \\ 
\end{tabular*}
\begin{tabular*}{\textwidth}[tb]{p{0.3\textwidth}p{0.35\textwidth}p{0.35\textwidth}}
	   & {\bf Type:} boolean &  {\bf Default:} false\\
	 & & \\
\end{tabular*}
\begin{tabular*}{\textwidth}[tb]{p{0.1\textwidth}p{0.9\textwidth}}
	 D\_CONVERGENCE\\ 

	 & Convergence criterion for density in the SAPT Ind20 term. See the note at the beginning of Section \ref{keywords}. \\ 
\end{tabular*}
\begin{tabular*}{\textwidth}[tb]{p{0.3\textwidth}p{0.35\textwidth}p{0.35\textwidth}}
	   & {\bf Type:} double &  {\bf Default:} 1e-8\\
	 & & \\
\end{tabular*}
\begin{tabular*}{\textwidth}[tb]{p{0.1\textwidth}p{0.9\textwidth}}
	 E\_CONVERGENCE\\ 

	 & Convergence criterion for energy (change) in the SAPT Ind20 term. See the note at the beginning of Section \ref{keywords}. \\ 
\end{tabular*}
\begin{tabular*}{\textwidth}[tb]{p{0.3\textwidth}p{0.35\textwidth}p{0.35\textwidth}}
	   & {\bf Type:} double &  {\bf Default:} 1e-10\\
	 & & \\
\end{tabular*}
\begin{tabular*}{\textwidth}[tb]{p{0.1\textwidth}p{0.9\textwidth}}
	 INTS\_TOLERANCE\\ 

	 & Minimum absolute value below which integrals are neglected. See the note at the beginning of Section \ref{keywords}. \\ 
\end{tabular*}
\begin{tabular*}{\textwidth}[tb]{p{0.3\textwidth}p{0.35\textwidth}p{0.35\textwidth}}
	   & {\bf Type:} double &  {\bf Default:} 1.0e-12\\
	 & & \\
\end{tabular*}
\begin{tabular*}{\textwidth}[tb]{p{0.1\textwidth}p{0.9\textwidth}}
	 MAXITER\\ 

	 & Maxmum number of CPHF iterations \\ 
\end{tabular*}
\begin{tabular*}{\textwidth}[tb]{p{0.3\textwidth}p{0.35\textwidth}p{0.35\textwidth}}
	   & {\bf Type:} integer &  {\bf Default:} 50\\
	 & & \\
\end{tabular*}
\begin{tabular*}{\textwidth}[tb]{p{0.1\textwidth}p{0.9\textwidth}}
	 NAT\_ORBS\\ 

	 & Do compute natural orbitals? \\ 
\end{tabular*}
\begin{tabular*}{\textwidth}[tb]{p{0.3\textwidth}p{0.35\textwidth}p{0.35\textwidth}}
	   & {\bf Type:} boolean &  {\bf Default:} false\\
	 & & \\
\end{tabular*}
\begin{tabular*}{\textwidth}[tb]{p{0.1\textwidth}p{0.9\textwidth}}
	 NAT\_ORBS\_T2\\ 

	 & Do use natural orbitals for T2's? \\ 
\end{tabular*}
\begin{tabular*}{\textwidth}[tb]{p{0.3\textwidth}p{0.35\textwidth}p{0.35\textwidth}}
	   & {\bf Type:} boolean &  {\bf Default:} false\\
	 & & \\
\end{tabular*}
\begin{tabular*}{\textwidth}[tb]{p{0.1\textwidth}p{0.9\textwidth}}
	 NO\_RESPONSE\\ 

	 & Don't solve the CPHF equations? \\ 
\end{tabular*}
\begin{tabular*}{\textwidth}[tb]{p{0.3\textwidth}p{0.35\textwidth}p{0.35\textwidth}}
	   & {\bf Type:} boolean &  {\bf Default:} false\\
	 & & \\
\end{tabular*}
\begin{tabular*}{\textwidth}[tb]{p{0.1\textwidth}p{0.9\textwidth}}
	 OCC\_TOLERANCE\\ 

	 & Minimum occupation below which natural orbitals are neglected. See the note at the beginning of Section \ref{keywords}. \\ 
\end{tabular*}
\begin{tabular*}{\textwidth}[tb]{p{0.3\textwidth}p{0.35\textwidth}p{0.35\textwidth}}
	   & {\bf Type:} double &  {\bf Default:} 1.0e-6\\
	 & & \\
\end{tabular*}
\begin{tabular*}{\textwidth}[tb]{p{0.1\textwidth}p{0.9\textwidth}}
	 PRINT\\ 

	 & The amount of information to print to the output file \\ 
\end{tabular*}
\begin{tabular*}{\textwidth}[tb]{p{0.3\textwidth}p{0.35\textwidth}p{0.35\textwidth}}
	   & {\bf Type:} integer &  {\bf Default:} 1\\
	 & & \\
\end{tabular*}
\begin{tabular*}{\textwidth}[tb]{p{0.1\textwidth}p{0.9\textwidth}}
	 SAPT\_LEVEL\\ 

	 & The level of theory for SAPT \\ 

	  & {\bf Possible Values:} SAPT0, SAPT2, SAPT2+, SAPT2+3 \\ 
\end{tabular*}
\begin{tabular*}{\textwidth}[tb]{p{0.3\textwidth}p{0.35\textwidth}p{0.35\textwidth}}
	   & {\bf Type:} string &  {\bf Default:} SAPT0\\
	 & & \\
\end{tabular*}
\begin{tabular*}{\textwidth}[tb]{p{0.1\textwidth}p{0.9\textwidth}}
	 SAPT\_MEM\_CHECK\\ 

	 & Do force SAPT2 and higher to die if it thinks there isn't enough memory? \\ 
\end{tabular*}
\begin{tabular*}{\textwidth}[tb]{p{0.3\textwidth}p{0.35\textwidth}p{0.35\textwidth}}
	   & {\bf Type:} boolean &  {\bf Default:} true\\
	 & & \\
\end{tabular*}
\begin{tabular*}{\textwidth}[tb]{p{0.1\textwidth}p{0.9\textwidth}}
	 SAPT\_MEM\_SAFETY\\ 

	 & Memory safety \\ 
\end{tabular*}
\begin{tabular*}{\textwidth}[tb]{p{0.3\textwidth}p{0.35\textwidth}p{0.35\textwidth}}
	   & {\bf Type:} double &  {\bf Default:} 0.9\\
	 & & \\
\end{tabular*}
\begin{tabular*}{\textwidth}[tb]{p{0.1\textwidth}p{0.9\textwidth}}
	 SAPT\_OS\_SCALE\\ 

	 & The scale factor used for opposite-spin pairs in SCS computations \\ 
\end{tabular*}
\begin{tabular*}{\textwidth}[tb]{p{0.3\textwidth}p{0.35\textwidth}p{0.35\textwidth}}
	   & {\bf Type:} double &  {\bf Default:} 6.0/5.0\\
	 & & \\
\end{tabular*}
\begin{tabular*}{\textwidth}[tb]{p{0.1\textwidth}p{0.9\textwidth}}
	 SAPT\_SS\_SCALE\\ 

	 & The scale factor used for same-spin pairs in SCS computations \\ 
\end{tabular*}
\begin{tabular*}{\textwidth}[tb]{p{0.3\textwidth}p{0.35\textwidth}p{0.35\textwidth}}
	   & {\bf Type:} double &  {\bf Default:} 1.0/3.0\\
	 & & \\
\end{tabular*}

\subsection{SCF}

{\normalsize Performs self consistent field (Hartree-Fock and Density Functional Theory) computations. These are the starting points for most computations, so this code is called in most cases.}\\
\begin{tabular*}{\textwidth}[tb]{c}
	  \\ 
\end{tabular*}
\subsubsection{Convergence Control/Stabilization }
\begin{tabular*}{\textwidth}[tb]{p{0.1\textwidth}p{0.9\textwidth}}
	 DAMPING\_CONVERGENCE\\ 

	 & The density convergence threshold after which damping is no longer performed, if it is enabled. It is recommended to leave damping on until convergence, which is the default. See the note at the beginning of Section \ref{keywords}. \\ 
\end{tabular*}
\begin{tabular*}{\textwidth}[tb]{p{0.3\textwidth}p{0.35\textwidth}p{0.35\textwidth}}
	   & {\bf Type:} double &  {\bf Default:} 1.0e-18\\
	 & & \\
\end{tabular*}
\begin{tabular*}{\textwidth}[tb]{p{0.1\textwidth}p{0.9\textwidth}}
	 DAMPING\_PERCENTAGE\\ 

	 & The amount (percentage) of damping to apply to the early density updates. 0 will result in a full update, 100 will completely stall the update. A value around 20 (which corresponds to 20\% of the previous iteration's density being mixed into the current density) could help to solve problems with oscillatory convergence. \\ 
\end{tabular*}
\begin{tabular*}{\textwidth}[tb]{p{0.3\textwidth}p{0.35\textwidth}p{0.35\textwidth}}
	   & {\bf Type:} double &  {\bf Default:} 100.0\\
	 & & \\
\end{tabular*}
\begin{tabular*}{\textwidth}[tb]{p{0.1\textwidth}p{0.9\textwidth}}
	 DIIS\\ 

	 & Do use DIIS extrapolation to accelerate convergence? \\ 
\end{tabular*}
\begin{tabular*}{\textwidth}[tb]{p{0.3\textwidth}p{0.35\textwidth}p{0.35\textwidth}}
	   & {\bf Type:} boolean &  {\bf Default:} true\\
	 & & \\
\end{tabular*}
\begin{tabular*}{\textwidth}[tb]{p{0.1\textwidth}p{0.9\textwidth}}
	 DIIS\_MAX\_VECS\\ 

	 & Maximum number of error vectors stored for DIIS extrapolation \\ 
\end{tabular*}
\begin{tabular*}{\textwidth}[tb]{p{0.3\textwidth}p{0.35\textwidth}p{0.35\textwidth}}
	   & {\bf Type:} integer &  {\bf Default:} 10\\
	 & & \\
\end{tabular*}
\begin{tabular*}{\textwidth}[tb]{p{0.1\textwidth}p{0.9\textwidth}}
	 DIIS\_MIN\_VECS\\ 

	 & Minimum number of error vectors stored for DIIS extrapolation \\ 
\end{tabular*}
\begin{tabular*}{\textwidth}[tb]{p{0.3\textwidth}p{0.35\textwidth}p{0.35\textwidth}}
	   & {\bf Type:} integer &  {\bf Default:} 2\\
	 & & \\
\end{tabular*}
\begin{tabular*}{\textwidth}[tb]{p{0.1\textwidth}p{0.9\textwidth}}
	 DIIS\_START\\ 

	 & The minimum iteration to start storing DIIS vectors \\ 
\end{tabular*}
\begin{tabular*}{\textwidth}[tb]{p{0.3\textwidth}p{0.35\textwidth}p{0.35\textwidth}}
	   & {\bf Type:} integer &  {\bf Default:} 1\\
	 & & \\
\end{tabular*}
\begin{tabular*}{\textwidth}[tb]{p{0.1\textwidth}p{0.9\textwidth}}
	 D\_CONVERGENCE\\ 

	 & Convergence criterion for SCF density. See the note at the beginning of Section \ref{keywords}. \\ 
\end{tabular*}
\begin{tabular*}{\textwidth}[tb]{p{0.3\textwidth}p{0.35\textwidth}p{0.35\textwidth}}
	   & {\bf Type:} double &  {\bf Default:} 1e-8\\
	 & & \\
\end{tabular*}
\begin{tabular*}{\textwidth}[tb]{p{0.1\textwidth}p{0.9\textwidth}}
	 E\_CONVERGENCE\\ 

	 & Convergence criterion for SCF energy. See the note at the beginning of Section \ref{keywords}. \\ 
\end{tabular*}
\begin{tabular*}{\textwidth}[tb]{p{0.3\textwidth}p{0.35\textwidth}p{0.35\textwidth}}
	   & {\bf Type:} double &  {\bf Default:} 1e-8\\
	 & & \\
\end{tabular*}
\begin{tabular*}{\textwidth}[tb]{p{0.1\textwidth}p{0.9\textwidth}}
	 MAXITER\\ 

	 & Maximum number of iterations \\ 
\end{tabular*}
\begin{tabular*}{\textwidth}[tb]{p{0.3\textwidth}p{0.35\textwidth}p{0.35\textwidth}}
	   & {\bf Type:} integer &  {\bf Default:} 100\\
	 & & \\
\end{tabular*}
\begin{tabular*}{\textwidth}[tb]{p{0.1\textwidth}p{0.9\textwidth}}
	 MOM\_OCC\\ 

	 & The absolute indices of orbitals to excite from in MOM (+/- for alpha/beta) \\ 
\end{tabular*}
\begin{tabular*}{\textwidth}[tb]{p{0.3\textwidth}p{0.35\textwidth}p{0.35\textwidth}}
	   & {\bf Type:} array &  {\bf Default:} No Default\\
	 & & \\
\end{tabular*}
\begin{tabular*}{\textwidth}[tb]{p{0.1\textwidth}p{0.9\textwidth}}
	 MOM\_START\\ 

	 & The iteration to start MOM on (or 0 for no MOM) \\ 
\end{tabular*}
\begin{tabular*}{\textwidth}[tb]{p{0.3\textwidth}p{0.35\textwidth}p{0.35\textwidth}}
	   & {\bf Type:} integer &  {\bf Default:} 0\\
	 & & \\
\end{tabular*}
\begin{tabular*}{\textwidth}[tb]{p{0.1\textwidth}p{0.9\textwidth}}
	 MOM\_VIR\\ 

	 & The absolute indices of orbitals to excite to in MOM (+/- for alpha/beta) \\ 
\end{tabular*}
\begin{tabular*}{\textwidth}[tb]{p{0.3\textwidth}p{0.35\textwidth}p{0.35\textwidth}}
	   & {\bf Type:} array &  {\bf Default:} No Default\\
	 & & \\
\end{tabular*}
\subsubsection{DFSCF Algorithm }
\begin{tabular*}{\textwidth}[tb]{p{0.1\textwidth}p{0.9\textwidth}}
	 DF\_INTS\_NUM\_THREADS\\ 

	 & Number of threads for integrals (may be turned down if memory is an issue). 0 is blank \\ 
\end{tabular*}
\begin{tabular*}{\textwidth}[tb]{p{0.3\textwidth}p{0.35\textwidth}p{0.35\textwidth}}
	   & {\bf Type:} integer &  {\bf Default:} 0\\
	 & & \\
\end{tabular*}
\subsubsection{Environmental Effects }
\begin{tabular*}{\textwidth}[tb]{p{0.1\textwidth}p{0.9\textwidth}}
	 EXTERN\\ 

	 & An ExternalPotential (built by Python or NULL/None) \\ 
\end{tabular*}
\begin{tabular*}{\textwidth}[tb]{p{0.3\textwidth}p{0.35\textwidth}p{0.35\textwidth}}
	   & {\bf Type:} python &  {\bf Default:} No Default\\
	 & & \\
\end{tabular*}
\begin{tabular*}{\textwidth}[tb]{p{0.1\textwidth}p{0.9\textwidth}}
	 PERTURB\_H\\ 

	 & Perturb the Hamiltonian? \\ 
\end{tabular*}
\begin{tabular*}{\textwidth}[tb]{p{0.3\textwidth}p{0.35\textwidth}p{0.35\textwidth}}
	   & {\bf Type:} boolean &  {\bf Default:} false\\
	 & & \\
\end{tabular*}
\begin{tabular*}{\textwidth}[tb]{p{0.1\textwidth}p{0.9\textwidth}}
	 PERTURB\_MAGNITUDE\\ 

	 & Size of the perturbation \\ 
\end{tabular*}
\begin{tabular*}{\textwidth}[tb]{p{0.3\textwidth}p{0.35\textwidth}p{0.35\textwidth}}
	   & {\bf Type:} double &  {\bf Default:} 0.0\\
	 & & \\
\end{tabular*}
\begin{tabular*}{\textwidth}[tb]{p{0.1\textwidth}p{0.9\textwidth}}
	 PERTURB\_WITH\\ 

	 & The operator used to perturb the Hamiltonian, if requested \\ 

	  & {\bf Possible Values:} DIPOLE\_X, DIPOLE\_Y, DIPOLE\_Z \\ 
\end{tabular*}
\begin{tabular*}{\textwidth}[tb]{p{0.3\textwidth}p{0.35\textwidth}p{0.35\textwidth}}
	   & {\bf Type:} string &  {\bf Default:} DIPOLE\_X\\
	 & & \\
\end{tabular*}
\begin{tabular*}{\textwidth}[tb]{p{0.1\textwidth}p{0.9\textwidth}}
	 PROCESS\_GRID\\ 

	 & SUBESCTION Parallel Runtime \\ 
\end{tabular*}
\begin{tabular*}{\textwidth}[tb]{p{0.3\textwidth}p{0.35\textwidth}p{0.35\textwidth}}
	   & {\bf Type:} array &  {\bf Default:} No Default\\
	 & & \\
\end{tabular*}
\subsubsection{Fractional Occupation UHF/UKS }
\begin{tabular*}{\textwidth}[tb]{p{0.1\textwidth}p{0.9\textwidth}}
	 FRAC\_DIIS\\ 

	 & Do use DIIS extrapolation to accelerate convergence in frac? \\ 
\end{tabular*}
\begin{tabular*}{\textwidth}[tb]{p{0.3\textwidth}p{0.35\textwidth}p{0.35\textwidth}}
	   & {\bf Type:} boolean &  {\bf Default:} true\\
	 & & \\
\end{tabular*}
\begin{tabular*}{\textwidth}[tb]{p{0.1\textwidth}p{0.9\textwidth}}
	 FRAC\_OCC\\ 

	 & The absolute indices of occupied orbitals to fractionally occupy (+/- for alpha/beta) \\ 
\end{tabular*}
\begin{tabular*}{\textwidth}[tb]{p{0.3\textwidth}p{0.35\textwidth}p{0.35\textwidth}}
	   & {\bf Type:} array &  {\bf Default:} No Default\\
	 & & \\
\end{tabular*}
\begin{tabular*}{\textwidth}[tb]{p{0.1\textwidth}p{0.9\textwidth}}
	 FRAC\_START\\ 

	 & The iteration to start fractionally occupying orbitals (or 0 for no fractional occupation) \\ 
\end{tabular*}
\begin{tabular*}{\textwidth}[tb]{p{0.3\textwidth}p{0.35\textwidth}p{0.35\textwidth}}
	   & {\bf Type:} integer &  {\bf Default:} 0\\
	 & & \\
\end{tabular*}
\begin{tabular*}{\textwidth}[tb]{p{0.1\textwidth}p{0.9\textwidth}}
	 FRAC\_VAL\\ 

	 & The occupations of the orbital indices specified above (0.0 >= occ >= 1.0) \\ 
\end{tabular*}
\begin{tabular*}{\textwidth}[tb]{p{0.3\textwidth}p{0.35\textwidth}p{0.35\textwidth}}
	   & {\bf Type:} array &  {\bf Default:} No Default\\
	 & & \\
\end{tabular*}
\subsubsection{General Wavefunction Info }
\begin{tabular*}{\textwidth}[tb]{p{0.1\textwidth}p{0.9\textwidth}}
	 BASIS\\ 

	 & Primary basis set \\ 
\end{tabular*}
\begin{tabular*}{\textwidth}[tb]{p{0.3\textwidth}p{0.35\textwidth}p{0.35\textwidth}}
	   & {\bf Type:} string &  {\bf Default:} No Default\\
	 & & \\
\end{tabular*}
\begin{tabular*}{\textwidth}[tb]{p{0.1\textwidth}p{0.9\textwidth}}
	 DF\_BASIS\_SCF\\ 

	 & Auxiliary basis set for SCF density fitting computations. Defaults to BASIS-JKFIT. \\ 
\end{tabular*}
\begin{tabular*}{\textwidth}[tb]{p{0.3\textwidth}p{0.35\textwidth}p{0.35\textwidth}}
	   & {\bf Type:} string &  {\bf Default:} No Default\\
	 & & \\
\end{tabular*}
\begin{tabular*}{\textwidth}[tb]{p{0.1\textwidth}p{0.9\textwidth}}
	 GUESS\\ 

	 & The type of guess orbitals \\ 

	  & {\bf Possible Values:} CORE, GWH, SAD, READ \\ 
\end{tabular*}
\begin{tabular*}{\textwidth}[tb]{p{0.3\textwidth}p{0.35\textwidth}p{0.35\textwidth}}
	   & {\bf Type:} string &  {\bf Default:} CORE\\
	 & & \\
\end{tabular*}
\begin{tabular*}{\textwidth}[tb]{p{0.1\textwidth}p{0.9\textwidth}}
	 INTS\_TOLERANCE\\ 

	 & Minimum absolute value below which TEI are neglected. See the note at the beginning of Section \ref{keywords}. \\ 
\end{tabular*}
\begin{tabular*}{\textwidth}[tb]{p{0.3\textwidth}p{0.35\textwidth}p{0.35\textwidth}}
	   & {\bf Type:} double &  {\bf Default:} 0.0\\
	 & & \\
\end{tabular*}
\begin{tabular*}{\textwidth}[tb]{p{0.1\textwidth}p{0.9\textwidth}}
	 REFERENCE\\ 

	 & Reference wavefunction type \\ 

	  & {\bf Possible Values:} RHF, ROHF, UHF, CUHF, RKS, UKS \\ 
\end{tabular*}
\begin{tabular*}{\textwidth}[tb]{p{0.3\textwidth}p{0.35\textwidth}p{0.35\textwidth}}
	   & {\bf Type:} string &  {\bf Default:} RHF\\
	 & & \\
\end{tabular*}
\begin{tabular*}{\textwidth}[tb]{p{0.1\textwidth}p{0.9\textwidth}}
	 SCF\_TYPE\\ 

	 & What algorithm to use for the SCF computation \\ 

	  & {\bf Possible Values:} PK, OUT\_OF\_CORE, DIRECT, DF \\ 
\end{tabular*}
\begin{tabular*}{\textwidth}[tb]{p{0.3\textwidth}p{0.35\textwidth}p{0.35\textwidth}}
	   & {\bf Type:} string &  {\bf Default:} PK\\
	 & & \\
\end{tabular*}
\begin{tabular*}{\textwidth}[tb]{p{0.1\textwidth}p{0.9\textwidth}}
	 S\_MIN\_EIGENVALUE\\ 

	 & Minimum S matrix eigenvalue to be used before compensating for linear dependencies \\ 
\end{tabular*}
\begin{tabular*}{\textwidth}[tb]{p{0.3\textwidth}p{0.35\textwidth}p{0.35\textwidth}}
	   & {\bf Type:} double &  {\bf Default:} 1e-7\\
	 & & \\
\end{tabular*}
\begin{tabular*}{\textwidth}[tb]{p{0.1\textwidth}p{0.9\textwidth}}
	 S\_ORTHOGONALIZATION\\ 

	 & SO orthogonalization: symmetric or canonical? \\ 

	  & {\bf Possible Values:} SYMMETRIC, CANONICAL \\ 
\end{tabular*}
\begin{tabular*}{\textwidth}[tb]{p{0.3\textwidth}p{0.35\textwidth}p{0.35\textwidth}}
	   & {\bf Type:} string &  {\bf Default:} SYMMETRIC\\
	 & & \\
\end{tabular*}
\subsubsection{SAD Guess Algorithm }
\begin{tabular*}{\textwidth}[tb]{p{0.1\textwidth}p{0.9\textwidth}}
	 DFT\_BASIS\_TOLERANCE\\ 

	 & DFT basis cutoff. See the note at the beginning of Section \ref{keywords}. \\ 
\end{tabular*}
\begin{tabular*}{\textwidth}[tb]{p{0.3\textwidth}p{0.35\textwidth}p{0.35\textwidth}}
	   & {\bf Type:} double &  {\bf Default:} 0.0\\
	 & & \\
\end{tabular*}
\begin{tabular*}{\textwidth}[tb]{p{0.1\textwidth}p{0.9\textwidth}}
	 DFT\_BOXING\_SCHEME\\ 

	 & The boxing scheme for DFT. \\ 

	  & {\bf Possible Values:} NAIVE, OCTREE \\ 
\end{tabular*}
\begin{tabular*}{\textwidth}[tb]{p{0.3\textwidth}p{0.35\textwidth}p{0.35\textwidth}}
	   & {\bf Type:} string &  {\bf Default:} NAIVE\\
	 & & \\
\end{tabular*}
\begin{tabular*}{\textwidth}[tb]{p{0.1\textwidth}p{0.9\textwidth}}
	 DFT\_BS\_RADIUS\_ALPHA\\ 

	 & Factor for effective BS radius in radial grid. \\ 
\end{tabular*}
\begin{tabular*}{\textwidth}[tb]{p{0.3\textwidth}p{0.35\textwidth}p{0.35\textwidth}}
	   & {\bf Type:} double &  {\bf Default:} 1.0\\
	 & & \\
\end{tabular*}
\begin{tabular*}{\textwidth}[tb]{p{0.1\textwidth}p{0.9\textwidth}}
	 DFT\_FUNCTIONAL\\ 

	 & The DFT combined functional name (for now). \\ 
\end{tabular*}
\begin{tabular*}{\textwidth}[tb]{p{0.3\textwidth}p{0.35\textwidth}p{0.35\textwidth}}
	   & {\bf Type:} string &  {\bf Default:} No Default\\
	 & & \\
\end{tabular*}
\begin{tabular*}{\textwidth}[tb]{p{0.1\textwidth}p{0.9\textwidth}}
	 DFT\_GRID\_NAME\\ 

	 & SUBSECTION DFT */ /*- The DFT grid specification, such as SG1. \\ 

	  & {\bf Possible Values:} SG1 \\ 
\end{tabular*}
\begin{tabular*}{\textwidth}[tb]{p{0.3\textwidth}p{0.35\textwidth}p{0.35\textwidth}}
	   & {\bf Type:} string &  {\bf Default:} No Default\\
	 & & \\
\end{tabular*}
\begin{tabular*}{\textwidth}[tb]{p{0.1\textwidth}p{0.9\textwidth}}
	 DFT\_MAX\_POINTS\\ 

	 & The number of grid points per evaluation block. \\ 
\end{tabular*}
\begin{tabular*}{\textwidth}[tb]{p{0.3\textwidth}p{0.35\textwidth}p{0.35\textwidth}}
	   & {\bf Type:} integer &  {\bf Default:} 5000\\
	 & & \\
\end{tabular*}
\begin{tabular*}{\textwidth}[tb]{p{0.1\textwidth}p{0.9\textwidth}}
	 DFT\_MIN\_POINTS\\ 

	 & The number of grid points per evaluation block. \\ 
\end{tabular*}
\begin{tabular*}{\textwidth}[tb]{p{0.3\textwidth}p{0.35\textwidth}p{0.35\textwidth}}
	   & {\bf Type:} integer &  {\bf Default:} 0\\
	 & & \\
\end{tabular*}
\begin{tabular*}{\textwidth}[tb]{p{0.1\textwidth}p{0.9\textwidth}}
	 DFT\_NUCLEAR\_SCHEME\\ 

	 & Nuclear Scheme. \\ 

	  & {\bf Possible Values:} TREUTLER, BECKE, NAIVE, STRATMANN \\ 
\end{tabular*}
\begin{tabular*}{\textwidth}[tb]{p{0.3\textwidth}p{0.35\textwidth}p{0.35\textwidth}}
	   & {\bf Type:} string &  {\bf Default:} TREUTLER\\
	 & & \\
\end{tabular*}
\begin{tabular*}{\textwidth}[tb]{p{0.1\textwidth}p{0.9\textwidth}}
	 DFT\_NUM\_RADIAL\\ 

	 & Number of radial points. \\ 
\end{tabular*}
\begin{tabular*}{\textwidth}[tb]{p{0.3\textwidth}p{0.35\textwidth}p{0.35\textwidth}}
	   & {\bf Type:} integer &  {\bf Default:} 99\\
	 & & \\
\end{tabular*}
\begin{tabular*}{\textwidth}[tb]{p{0.1\textwidth}p{0.9\textwidth}}
	 DFT\_OMEGA\\ 

	 & The DFT Range-separation parameter (only used if changed by the user). \\ 
\end{tabular*}
\begin{tabular*}{\textwidth}[tb]{p{0.3\textwidth}p{0.35\textwidth}p{0.35\textwidth}}
	   & {\bf Type:} double &  {\bf Default:} 0.0\\
	 & & \\
\end{tabular*}
\begin{tabular*}{\textwidth}[tb]{p{0.1\textwidth}p{0.9\textwidth}}
	 DFT\_ORDER\_SPHERICAL\\ 

	 & Maximum order of spherical grids. \\ 
\end{tabular*}
\begin{tabular*}{\textwidth}[tb]{p{0.3\textwidth}p{0.35\textwidth}p{0.35\textwidth}}
	   & {\bf Type:} integer &  {\bf Default:} 15\\
	 & & \\
\end{tabular*}
\begin{tabular*}{\textwidth}[tb]{p{0.1\textwidth}p{0.9\textwidth}}
	 DFT\_PRUNING\_ALPHA\\ 

	 & Spread alpha for logarithmic pruning. \\ 
\end{tabular*}
\begin{tabular*}{\textwidth}[tb]{p{0.3\textwidth}p{0.35\textwidth}p{0.35\textwidth}}
	   & {\bf Type:} double &  {\bf Default:} 1.0\\
	 & & \\
\end{tabular*}
\begin{tabular*}{\textwidth}[tb]{p{0.1\textwidth}p{0.9\textwidth}}
	 DFT\_PRUNING\_SCHEME\\ 

	 & Pruning Scheme. \\ 

	  & {\bf Possible Values:} FLAT, P\_GAUSSIAN, D\_GAUSSIAN, P\_SLATER, D\_SLATER, LOG\_GAUSSIAN, LOG\_SLATER \\ 
\end{tabular*}
\begin{tabular*}{\textwidth}[tb]{p{0.3\textwidth}p{0.35\textwidth}p{0.35\textwidth}}
	   & {\bf Type:} string &  {\bf Default:} FLAT\\
	 & & \\
\end{tabular*}
\begin{tabular*}{\textwidth}[tb]{p{0.1\textwidth}p{0.9\textwidth}}
	 DFT\_RADIAL\_SCHEME\\ 

	 & Radial Scheme. \\ 

	  & {\bf Possible Values:} TREUTLER, BECKE, MULTIEXP, EM, MURA \\ 
\end{tabular*}
\begin{tabular*}{\textwidth}[tb]{p{0.3\textwidth}p{0.35\textwidth}p{0.35\textwidth}}
	   & {\bf Type:} string &  {\bf Default:} TREUTLER\\
	 & & \\
\end{tabular*}
\begin{tabular*}{\textwidth}[tb]{p{0.1\textwidth}p{0.9\textwidth}}
	 DFT\_SPHERICAL\_SCHEME\\ 

	 & Spherical Scheme. \\ 

	  & {\bf Possible Values:} LEBEDEV \\ 
\end{tabular*}
\begin{tabular*}{\textwidth}[tb]{p{0.3\textwidth}p{0.35\textwidth}p{0.35\textwidth}}
	   & {\bf Type:} string &  {\bf Default:} LEBEDEV\\
	 & & \\
\end{tabular*}

\subsection{STABLE}

{\normalsize Performs wavefunction stability analysis, and is only called when specifically requested by the user}\\
\begin{tabular*}{\textwidth}[tb]{c}
	  \\ 
\end{tabular*}
\begin{tabular*}{\textwidth}[tb]{p{0.1\textwidth}p{0.9\textwidth}}
	 CACHELEVEL\\ 

	 &  \\ 
\end{tabular*}
\begin{tabular*}{\textwidth}[tb]{p{0.3\textwidth}p{0.35\textwidth}p{0.35\textwidth}}
	   & {\bf Type:} integer &  {\bf Default:} 2\\
	 & & \\
\end{tabular*}
\begin{tabular*}{\textwidth}[tb]{p{0.1\textwidth}p{0.9\textwidth}}
	 FOLLOW\\ 

	 & Do follow the most negative eigenvalue of the Hessian towards a lower energy HF solution? Follow a UHF->UHF instability of same symmetry? \\ 
\end{tabular*}
\begin{tabular*}{\textwidth}[tb]{p{0.3\textwidth}p{0.35\textwidth}p{0.35\textwidth}}
	   & {\bf Type:} boolean &  {\bf Default:} false\\
	 & & \\
\end{tabular*}
\begin{tabular*}{\textwidth}[tb]{p{0.1\textwidth}p{0.9\textwidth}}
	 NUM\_VECS\_PRINT\\ 

	 & Number of lowest MO Hessian eigenvalues to print \\ 
\end{tabular*}
\begin{tabular*}{\textwidth}[tb]{p{0.3\textwidth}p{0.35\textwidth}p{0.35\textwidth}}
	   & {\bf Type:} integer &  {\bf Default:} 0\\
	 & & \\
\end{tabular*}
\begin{tabular*}{\textwidth}[tb]{p{0.1\textwidth}p{0.9\textwidth}}
	 REFERENCE\\ 

	 & Reference wavefunction type \\ 
\end{tabular*}
\begin{tabular*}{\textwidth}[tb]{p{0.3\textwidth}p{0.35\textwidth}p{0.35\textwidth}}
	   & {\bf Type:} string &  {\bf Default:} RHF\\
	 & & \\
\end{tabular*}
\begin{tabular*}{\textwidth}[tb]{p{0.1\textwidth}p{0.9\textwidth}}
	 ROTATION\_SCHEME\\ 

	 & Method for following eigenvectors, either 0 by angles or 1 by antisymmetric matrix. \\ 
\end{tabular*}
\begin{tabular*}{\textwidth}[tb]{p{0.3\textwidth}p{0.35\textwidth}p{0.35\textwidth}}
	   & {\bf Type:} integer &  {\bf Default:} 0\\
	 & & \\
\end{tabular*}
\begin{tabular*}{\textwidth}[tb]{p{0.1\textwidth}p{0.9\textwidth}}
	 SCALE\\ 

	 & Scale factor (between 0 and 1) for orbital rotation step \\ 
\end{tabular*}
\begin{tabular*}{\textwidth}[tb]{p{0.3\textwidth}p{0.35\textwidth}p{0.35\textwidth}}
	   & {\bf Type:} double &  {\bf Default:} 0.5\\
	 & & \\
\end{tabular*}

\subsection{TRANSQT}

{\normalsize The predecessor to Transqt2. Currently used by the configuration interaction codes, but is being phased out.}\\
\begin{tabular*}{\textwidth}[tb]{c}
	  \\ 
\end{tabular*}
\begin{tabular*}{\textwidth}[tb]{p{0.1\textwidth}p{0.9\textwidth}}
	 AA\_M\_FILE\\ 

	 &  \\ 
\end{tabular*}
\begin{tabular*}{\textwidth}[tb]{p{0.3\textwidth}p{0.35\textwidth}p{0.35\textwidth}}
	   & {\bf Type:} integer &  {\bf Default:} PSIF\_MO\_AA\_TEI\\
	 & & \\
\end{tabular*}
\begin{tabular*}{\textwidth}[tb]{p{0.1\textwidth}p{0.9\textwidth}}
	 AB\_M\_FILE\\ 

	 &  \\ 
\end{tabular*}
\begin{tabular*}{\textwidth}[tb]{p{0.3\textwidth}p{0.35\textwidth}p{0.35\textwidth}}
	   & {\bf Type:} integer &  {\bf Default:} PSIF\_MO\_AB\_TEI\\
	 & & \\
\end{tabular*}
\begin{tabular*}{\textwidth}[tb]{p{0.1\textwidth}p{0.9\textwidth}}
	 AO\_BASIS\\ 

	 & The algorithm to use for the $\left<VV||VV\right>$ terms \\ 

	  & {\bf Possible Values:} NONE, DISK, DIRECT \\ 
\end{tabular*}
\begin{tabular*}{\textwidth}[tb]{p{0.3\textwidth}p{0.35\textwidth}p{0.35\textwidth}}
	   & {\bf Type:} string &  {\bf Default:} NONE\\
	 & & \\
\end{tabular*}
\begin{tabular*}{\textwidth}[tb]{p{0.1\textwidth}p{0.9\textwidth}}
	 BB\_M\_FILE\\ 

	 &  \\ 
\end{tabular*}
\begin{tabular*}{\textwidth}[tb]{p{0.3\textwidth}p{0.35\textwidth}p{0.35\textwidth}}
	   & {\bf Type:} integer &  {\bf Default:} PSIF\_MO\_BB\_TEI\\
	 & & \\
\end{tabular*}
\begin{tabular*}{\textwidth}[tb]{p{0.1\textwidth}p{0.9\textwidth}}
	 CHECK\_C\_ORTHONORM\\ 

	 & Do ? \\ 
\end{tabular*}
\begin{tabular*}{\textwidth}[tb]{p{0.3\textwidth}p{0.35\textwidth}p{0.35\textwidth}}
	   & {\bf Type:} boolean &  {\bf Default:} false\\
	 & & \\
\end{tabular*}
\begin{tabular*}{\textwidth}[tb]{p{0.1\textwidth}p{0.9\textwidth}}
	 DELETE\_AO\\ 

	 & Don't ? \\ 
\end{tabular*}
\begin{tabular*}{\textwidth}[tb]{p{0.3\textwidth}p{0.35\textwidth}p{0.35\textwidth}}
	   & {\bf Type:} boolean &  {\bf Default:} true\\
	 & & \\
\end{tabular*}
\begin{tabular*}{\textwidth}[tb]{p{0.1\textwidth}p{0.9\textwidth}}
	 DELETE\_RESTR\_DOCC\\ 

	 & Don't ? \\ 
\end{tabular*}
\begin{tabular*}{\textwidth}[tb]{p{0.3\textwidth}p{0.35\textwidth}p{0.35\textwidth}}
	   & {\bf Type:} boolean &  {\bf Default:} true\\
	 & & \\
\end{tabular*}
\begin{tabular*}{\textwidth}[tb]{p{0.1\textwidth}p{0.9\textwidth}}
	 DELETE\_TPDM\\ 

	 & Don't ? \\ 
\end{tabular*}
\begin{tabular*}{\textwidth}[tb]{p{0.3\textwidth}p{0.35\textwidth}p{0.35\textwidth}}
	   & {\bf Type:} boolean &  {\bf Default:} true\\
	 & & \\
\end{tabular*}
\begin{tabular*}{\textwidth}[tb]{p{0.1\textwidth}p{0.9\textwidth}}
	 DO\_ALL\_TEI\\ 

	 & Do ? \\ 
\end{tabular*}
\begin{tabular*}{\textwidth}[tb]{p{0.3\textwidth}p{0.35\textwidth}p{0.35\textwidth}}
	   & {\bf Type:} boolean &  {\bf Default:} false\\
	 & & \\
\end{tabular*}
\begin{tabular*}{\textwidth}[tb]{p{0.1\textwidth}p{0.9\textwidth}}
	 FIRST\_TMP\_FILE\\ 

	 &  \\ 
\end{tabular*}
\begin{tabular*}{\textwidth}[tb]{p{0.3\textwidth}p{0.35\textwidth}p{0.35\textwidth}}
	   & {\bf Type:} integer &  {\bf Default:} 150\\
	 & & \\
\end{tabular*}
\begin{tabular*}{\textwidth}[tb]{p{0.1\textwidth}p{0.9\textwidth}}
	 FZC\_A\_FILE\\ 

	 &  \\ 
\end{tabular*}
\begin{tabular*}{\textwidth}[tb]{p{0.3\textwidth}p{0.35\textwidth}p{0.35\textwidth}}
	   & {\bf Type:} integer &  {\bf Default:} PSIF\_OEI\\
	 & & \\
\end{tabular*}
\begin{tabular*}{\textwidth}[tb]{p{0.1\textwidth}p{0.9\textwidth}}
	 FZC\_B\_FILE\\ 

	 &  \\ 
\end{tabular*}
\begin{tabular*}{\textwidth}[tb]{p{0.3\textwidth}p{0.35\textwidth}p{0.35\textwidth}}
	   & {\bf Type:} integer &  {\bf Default:} PSIF\_OEI\\
	 & & \\
\end{tabular*}
\begin{tabular*}{\textwidth}[tb]{p{0.1\textwidth}p{0.9\textwidth}}
	 FZC\_FILE\\ 

	 &  \\ 
\end{tabular*}
\begin{tabular*}{\textwidth}[tb]{p{0.3\textwidth}p{0.35\textwidth}p{0.35\textwidth}}
	   & {\bf Type:} integer &  {\bf Default:} PSIF\_OEI\\
	 & & \\
\end{tabular*}
\begin{tabular*}{\textwidth}[tb]{p{0.1\textwidth}p{0.9\textwidth}}
	 INTS\_TOLERANCE\\ 

	 & Minimum absolute value below which integrals are neglected. See the note at the beginning of Section \ref{keywords}. \\ 
\end{tabular*}
\begin{tabular*}{\textwidth}[tb]{p{0.3\textwidth}p{0.35\textwidth}p{0.35\textwidth}}
	   & {\bf Type:} double &  {\bf Default:} 1e-14\\
	 & & \\
\end{tabular*}
\begin{tabular*}{\textwidth}[tb]{p{0.1\textwidth}p{0.9\textwidth}}
	 IVO\\ 

	 & Do ? \\ 
\end{tabular*}
\begin{tabular*}{\textwidth}[tb]{p{0.3\textwidth}p{0.35\textwidth}p{0.35\textwidth}}
	   & {\bf Type:} boolean &  {\bf Default:} false\\
	 & & \\
\end{tabular*}
\begin{tabular*}{\textwidth}[tb]{p{0.1\textwidth}p{0.9\textwidth}}
	 J\_FILE\\ 

	 &  \\ 
\end{tabular*}
\begin{tabular*}{\textwidth}[tb]{p{0.3\textwidth}p{0.35\textwidth}p{0.35\textwidth}}
	   & {\bf Type:} integer &  {\bf Default:} 91\\
	 & & \\
\end{tabular*}
\begin{tabular*}{\textwidth}[tb]{p{0.1\textwidth}p{0.9\textwidth}}
	 KEEP\_J\\ 

	 & Do keep half-transformed integrals? \\ 
\end{tabular*}
\begin{tabular*}{\textwidth}[tb]{p{0.3\textwidth}p{0.35\textwidth}p{0.35\textwidth}}
	   & {\bf Type:} boolean &  {\bf Default:} false\\
	 & & \\
\end{tabular*}
\begin{tabular*}{\textwidth}[tb]{p{0.1\textwidth}p{0.9\textwidth}}
	 KEEP\_PRESORT\\ 

	 & Do ? \\ 
\end{tabular*}
\begin{tabular*}{\textwidth}[tb]{p{0.3\textwidth}p{0.35\textwidth}p{0.35\textwidth}}
	   & {\bf Type:} boolean &  {\bf Default:} false\\
	 & & \\
\end{tabular*}
\begin{tabular*}{\textwidth}[tb]{p{0.1\textwidth}p{0.9\textwidth}}
	 LAGRAN\_DOUBLE\\ 

	 & Do ? \\ 
\end{tabular*}
\begin{tabular*}{\textwidth}[tb]{p{0.3\textwidth}p{0.35\textwidth}p{0.35\textwidth}}
	   & {\bf Type:} boolean &  {\bf Default:} false\\
	 & & \\
\end{tabular*}
\begin{tabular*}{\textwidth}[tb]{p{0.1\textwidth}p{0.9\textwidth}}
	 LAGRAN\_HALVE\\ 

	 & Do ? \\ 
\end{tabular*}
\begin{tabular*}{\textwidth}[tb]{p{0.3\textwidth}p{0.35\textwidth}p{0.35\textwidth}}
	   & {\bf Type:} boolean &  {\bf Default:} false\\
	 & & \\
\end{tabular*}
\begin{tabular*}{\textwidth}[tb]{p{0.1\textwidth}p{0.9\textwidth}}
	 LAG\_IN\_FILE\\ 

	 &  \\ 
\end{tabular*}
\begin{tabular*}{\textwidth}[tb]{p{0.3\textwidth}p{0.35\textwidth}p{0.35\textwidth}}
	   & {\bf Type:} integer &  {\bf Default:} PSIF\_MO\_LAG\\
	 & & \\
\end{tabular*}
\begin{tabular*}{\textwidth}[tb]{p{0.1\textwidth}p{0.9\textwidth}}
	 MAX\_BUCKETS\\ 

	 &  \\ 
\end{tabular*}
\begin{tabular*}{\textwidth}[tb]{p{0.3\textwidth}p{0.35\textwidth}p{0.35\textwidth}}
	   & {\bf Type:} integer &  {\bf Default:} 499\\
	 & & \\
\end{tabular*}
\begin{tabular*}{\textwidth}[tb]{p{0.1\textwidth}p{0.9\textwidth}}
	 MODE\\ 

	 &  \\ 

	  & {\bf Possible Values:} TO\_MO, TO\_AO \\ 
\end{tabular*}
\begin{tabular*}{\textwidth}[tb]{p{0.3\textwidth}p{0.35\textwidth}p{0.35\textwidth}}
	   & {\bf Type:} string &  {\bf Default:} TO\_MO\\
	 & & \\
\end{tabular*}
\begin{tabular*}{\textwidth}[tb]{p{0.1\textwidth}p{0.9\textwidth}}
	 MOORDER\\ 

	 &  \\ 
\end{tabular*}
\begin{tabular*}{\textwidth}[tb]{p{0.3\textwidth}p{0.35\textwidth}p{0.35\textwidth}}
	   & {\bf Type:} array &  {\bf Default:} No Default\\
	 & & \\
\end{tabular*}
\begin{tabular*}{\textwidth}[tb]{p{0.1\textwidth}p{0.9\textwidth}}
	 MP2R12A\\ 

	 &  \\ 

	  & {\bf Possible Values:} MP2R12AERI, MP2R12AR12, MP2R12AR12T1 \\ 
\end{tabular*}
\begin{tabular*}{\textwidth}[tb]{p{0.3\textwidth}p{0.35\textwidth}p{0.35\textwidth}}
	   & {\bf Type:} string &  {\bf Default:} MP2R12AERI\\
	 & & \\
\end{tabular*}
\begin{tabular*}{\textwidth}[tb]{p{0.1\textwidth}p{0.9\textwidth}}
	 M\_FILE\\ 

	 &  \\ 
\end{tabular*}
\begin{tabular*}{\textwidth}[tb]{p{0.3\textwidth}p{0.35\textwidth}p{0.35\textwidth}}
	   & {\bf Type:} integer &  {\bf Default:} 0\\
	 & & \\
\end{tabular*}
\begin{tabular*}{\textwidth}[tb]{p{0.1\textwidth}p{0.9\textwidth}}
	 OEI\_A\_FILE\\ 

	 &  \\ 
\end{tabular*}
\begin{tabular*}{\textwidth}[tb]{p{0.3\textwidth}p{0.35\textwidth}p{0.35\textwidth}}
	   & {\bf Type:} integer &  {\bf Default:} PSIF\_OEI\\
	 & & \\
\end{tabular*}
\begin{tabular*}{\textwidth}[tb]{p{0.1\textwidth}p{0.9\textwidth}}
	 OEI\_B\_FILE\\ 

	 &  \\ 
\end{tabular*}
\begin{tabular*}{\textwidth}[tb]{p{0.3\textwidth}p{0.35\textwidth}p{0.35\textwidth}}
	   & {\bf Type:} integer &  {\bf Default:} PSIF\_OEI\\
	 & & \\
\end{tabular*}
\begin{tabular*}{\textwidth}[tb]{p{0.1\textwidth}p{0.9\textwidth}}
	 OEI\_FILE\\ 

	 &  \\ 
\end{tabular*}
\begin{tabular*}{\textwidth}[tb]{p{0.3\textwidth}p{0.35\textwidth}p{0.35\textwidth}}
	   & {\bf Type:} integer &  {\bf Default:} PSIF\_OEI\\
	 & & \\
\end{tabular*}
\begin{tabular*}{\textwidth}[tb]{p{0.1\textwidth}p{0.9\textwidth}}
	 OPDM\_IN\_FILE\\ 

	 &  \\ 
\end{tabular*}
\begin{tabular*}{\textwidth}[tb]{p{0.3\textwidth}p{0.35\textwidth}p{0.35\textwidth}}
	   & {\bf Type:} integer &  {\bf Default:} PSIF\_MO\_OPDM\\
	 & & \\
\end{tabular*}
\begin{tabular*}{\textwidth}[tb]{p{0.1\textwidth}p{0.9\textwidth}}
	 OPDM\_OUT\_FILE\\ 

	 &  \\ 
\end{tabular*}
\begin{tabular*}{\textwidth}[tb]{p{0.3\textwidth}p{0.35\textwidth}p{0.35\textwidth}}
	   & {\bf Type:} integer &  {\bf Default:} PSIF\_AO\_OPDM\\
	 & & \\
\end{tabular*}
\begin{tabular*}{\textwidth}[tb]{p{0.1\textwidth}p{0.9\textwidth}}
	 PITZER\\ 

	 & Do ? \\ 
\end{tabular*}
\begin{tabular*}{\textwidth}[tb]{p{0.3\textwidth}p{0.35\textwidth}p{0.35\textwidth}}
	   & {\bf Type:} boolean &  {\bf Default:} false\\
	 & & \\
\end{tabular*}
\begin{tabular*}{\textwidth}[tb]{p{0.1\textwidth}p{0.9\textwidth}}
	 PRESORT\_FILE\\ 

	 &  \\ 
\end{tabular*}
\begin{tabular*}{\textwidth}[tb]{p{0.3\textwidth}p{0.35\textwidth}p{0.35\textwidth}}
	   & {\bf Type:} integer &  {\bf Default:} PSIF\_SO\_PRESORT\\
	 & & \\
\end{tabular*}
\begin{tabular*}{\textwidth}[tb]{p{0.1\textwidth}p{0.9\textwidth}}
	 PRINT\_LVL\\ 

	 &  \\ 
\end{tabular*}
\begin{tabular*}{\textwidth}[tb]{p{0.3\textwidth}p{0.35\textwidth}p{0.35\textwidth}}
	   & {\bf Type:} integer &  {\bf Default:} 1\\
	 & & \\
\end{tabular*}
\begin{tabular*}{\textwidth}[tb]{p{0.1\textwidth}p{0.9\textwidth}}
	 PRINT\_MOS\\ 

	 & Do ? \\ 
\end{tabular*}
\begin{tabular*}{\textwidth}[tb]{p{0.3\textwidth}p{0.35\textwidth}p{0.35\textwidth}}
	   & {\bf Type:} boolean &  {\bf Default:} false\\
	 & & \\
\end{tabular*}
\begin{tabular*}{\textwidth}[tb]{p{0.1\textwidth}p{0.9\textwidth}}
	 PRINT\_OE\_INTEGRALS\\ 

	 & Do ? \\ 
\end{tabular*}
\begin{tabular*}{\textwidth}[tb]{p{0.3\textwidth}p{0.35\textwidth}p{0.35\textwidth}}
	   & {\bf Type:} boolean &  {\bf Default:} false\\
	 & & \\
\end{tabular*}
\begin{tabular*}{\textwidth}[tb]{p{0.1\textwidth}p{0.9\textwidth}}
	 PRINT\_REORDER\\ 

	 & Do ? \\ 
\end{tabular*}
\begin{tabular*}{\textwidth}[tb]{p{0.3\textwidth}p{0.35\textwidth}p{0.35\textwidth}}
	   & {\bf Type:} boolean &  {\bf Default:} false\\
	 & & \\
\end{tabular*}
\begin{tabular*}{\textwidth}[tb]{p{0.1\textwidth}p{0.9\textwidth}}
	 PRINT\_SORTED\_OE\_INTS\\ 

	 & Do ? \\ 
\end{tabular*}
\begin{tabular*}{\textwidth}[tb]{p{0.3\textwidth}p{0.35\textwidth}p{0.35\textwidth}}
	   & {\bf Type:} boolean &  {\bf Default:} false\\
	 & & \\
\end{tabular*}
\begin{tabular*}{\textwidth}[tb]{p{0.1\textwidth}p{0.9\textwidth}}
	 PRINT\_SORTED\_TE\_INTS\\ 

	 & Do ? \\ 
\end{tabular*}
\begin{tabular*}{\textwidth}[tb]{p{0.3\textwidth}p{0.35\textwidth}p{0.35\textwidth}}
	   & {\bf Type:} boolean &  {\bf Default:} false\\
	 & & \\
\end{tabular*}
\begin{tabular*}{\textwidth}[tb]{p{0.1\textwidth}p{0.9\textwidth}}
	 PRINT\_TE\_INTEGRALS\\ 

	 & Do ? \\ 
\end{tabular*}
\begin{tabular*}{\textwidth}[tb]{p{0.3\textwidth}p{0.35\textwidth}p{0.35\textwidth}}
	   & {\bf Type:} boolean &  {\bf Default:} false\\
	 & & \\
\end{tabular*}
\begin{tabular*}{\textwidth}[tb]{p{0.1\textwidth}p{0.9\textwidth}}
	 PSIMRCC\\ 

	 & Do ? \\ 
\end{tabular*}
\begin{tabular*}{\textwidth}[tb]{p{0.3\textwidth}p{0.35\textwidth}p{0.35\textwidth}}
	   & {\bf Type:} boolean &  {\bf Default:} false\\
	 & & \\
\end{tabular*}
\begin{tabular*}{\textwidth}[tb]{p{0.1\textwidth}p{0.9\textwidth}}
	 QRHF\\ 

	 & Do ? \\ 
\end{tabular*}
\begin{tabular*}{\textwidth}[tb]{p{0.3\textwidth}p{0.35\textwidth}p{0.35\textwidth}}
	   & {\bf Type:} boolean &  {\bf Default:} false\\
	 & & \\
\end{tabular*}
\begin{tabular*}{\textwidth}[tb]{p{0.1\textwidth}p{0.9\textwidth}}
	 REFERENCE\\ 

	 & Reference wavefunction type \\ 
\end{tabular*}
\begin{tabular*}{\textwidth}[tb]{p{0.3\textwidth}p{0.35\textwidth}p{0.35\textwidth}}
	   & {\bf Type:} string &  {\bf Default:} RHF\\
	 & & \\
\end{tabular*}
\begin{tabular*}{\textwidth}[tb]{p{0.1\textwidth}p{0.9\textwidth}}
	 REORDER\\ 

	 & Do ? \\ 
\end{tabular*}
\begin{tabular*}{\textwidth}[tb]{p{0.3\textwidth}p{0.35\textwidth}p{0.35\textwidth}}
	   & {\bf Type:} boolean &  {\bf Default:} false\\
	 & & \\
\end{tabular*}
\begin{tabular*}{\textwidth}[tb]{p{0.1\textwidth}p{0.9\textwidth}}
	 RESTRICTED\_DOCC\\ 

	 & An array giving the number of restricted doubly-occupied orbitals per irrep (not excited in CI wavefunctions, but orbitals can be optimized in MCSCF) \\ 
\end{tabular*}
\begin{tabular*}{\textwidth}[tb]{p{0.3\textwidth}p{0.35\textwidth}p{0.35\textwidth}}
	   & {\bf Type:} array &  {\bf Default:} No Default\\
	 & & \\
\end{tabular*}
\begin{tabular*}{\textwidth}[tb]{p{0.1\textwidth}p{0.9\textwidth}}
	 RESTRICTED\_UOCC\\ 

	 & An array giving the number of restricted unoccupied orbitals per irrep (not occupied in CI wavefunctions, but orbitals can be optimized in MCSCF) \\ 
\end{tabular*}
\begin{tabular*}{\textwidth}[tb]{p{0.3\textwidth}p{0.35\textwidth}p{0.35\textwidth}}
	   & {\bf Type:} array &  {\bf Default:} No Default\\
	 & & \\
\end{tabular*}
\begin{tabular*}{\textwidth}[tb]{p{0.1\textwidth}p{0.9\textwidth}}
	 SORTED\_TEI\_FILE\\ 

	 &  \\ 
\end{tabular*}
\begin{tabular*}{\textwidth}[tb]{p{0.3\textwidth}p{0.35\textwidth}p{0.35\textwidth}}
	   & {\bf Type:} integer &  {\bf Default:} PSIF\_MO\_TEI\\
	 & & \\
\end{tabular*}
\begin{tabular*}{\textwidth}[tb]{p{0.1\textwidth}p{0.9\textwidth}}
	 SO\_S\_FILE\\ 

	 &  \\ 
\end{tabular*}
\begin{tabular*}{\textwidth}[tb]{p{0.3\textwidth}p{0.35\textwidth}p{0.35\textwidth}}
	   & {\bf Type:} integer &  {\bf Default:} PSIF\_OEI\\
	 & & \\
\end{tabular*}
\begin{tabular*}{\textwidth}[tb]{p{0.1\textwidth}p{0.9\textwidth}}
	 SO\_TEI\_FILE\\ 

	 &  \\ 
\end{tabular*}
\begin{tabular*}{\textwidth}[tb]{p{0.3\textwidth}p{0.35\textwidth}p{0.35\textwidth}}
	   & {\bf Type:} integer &  {\bf Default:} PSIF\_SO\_TEI\\
	 & & \\
\end{tabular*}
\begin{tabular*}{\textwidth}[tb]{p{0.1\textwidth}p{0.9\textwidth}}
	 SO\_T\_FILE\\ 

	 &  \\ 
\end{tabular*}
\begin{tabular*}{\textwidth}[tb]{p{0.3\textwidth}p{0.35\textwidth}p{0.35\textwidth}}
	   & {\bf Type:} integer &  {\bf Default:} PSIF\_OEI\\
	 & & \\
\end{tabular*}
\begin{tabular*}{\textwidth}[tb]{p{0.1\textwidth}p{0.9\textwidth}}
	 SO\_V\_FILE\\ 

	 &  \\ 
\end{tabular*}
\begin{tabular*}{\textwidth}[tb]{p{0.3\textwidth}p{0.35\textwidth}p{0.35\textwidth}}
	   & {\bf Type:} integer &  {\bf Default:} PSIF\_OEI\\
	 & & \\
\end{tabular*}
\begin{tabular*}{\textwidth}[tb]{p{0.1\textwidth}p{0.9\textwidth}}
	 TPDM\_ADD\_REF\\ 

	 & Do ? \\ 
\end{tabular*}
\begin{tabular*}{\textwidth}[tb]{p{0.3\textwidth}p{0.35\textwidth}p{0.35\textwidth}}
	   & {\bf Type:} boolean &  {\bf Default:} false\\
	 & & \\
\end{tabular*}
\begin{tabular*}{\textwidth}[tb]{p{0.1\textwidth}p{0.9\textwidth}}
	 TPDM\_FILE\\ 

	 &  \\ 
\end{tabular*}
\begin{tabular*}{\textwidth}[tb]{p{0.3\textwidth}p{0.35\textwidth}p{0.35\textwidth}}
	   & {\bf Type:} integer &  {\bf Default:} PSIF\_MO\_TPDM\\
	 & & \\
\end{tabular*}

\subsection{TRANSQT2}

{\normalsize Performs transformations of integrals into the molecular orbital (MO) basis. This module is currently used by the (non-density fitted) MP2 and coupled cluster codes, but is being phased out.}\\
\begin{tabular*}{\textwidth}[tb]{c}
	  \\ 
\end{tabular*}
\begin{tabular*}{\textwidth}[tb]{p{0.1\textwidth}p{0.9\textwidth}}
	 AO\_BASIS\\ 

	 & The algorithm to use for the $\left<VV||VV\right>$ terms \\ 

	  & {\bf Possible Values:} NONE, DISK, DIRECT \\ 
\end{tabular*}
\begin{tabular*}{\textwidth}[tb]{p{0.3\textwidth}p{0.35\textwidth}p{0.35\textwidth}}
	   & {\bf Type:} string &  {\bf Default:} NONE\\
	 & & \\
\end{tabular*}
\begin{tabular*}{\textwidth}[tb]{p{0.1\textwidth}p{0.9\textwidth}}
	 CACHELEVEL\\ 

	 &  \\ 
\end{tabular*}
\begin{tabular*}{\textwidth}[tb]{p{0.3\textwidth}p{0.35\textwidth}p{0.35\textwidth}}
	   & {\bf Type:} integer &  {\bf Default:} 2\\
	 & & \\
\end{tabular*}
\begin{tabular*}{\textwidth}[tb]{p{0.1\textwidth}p{0.9\textwidth}}
	 DELETE\_TEI\\ 

	 & Boolean to delete the SO-basis two-electron integral file after the transformation \\ 
\end{tabular*}
\begin{tabular*}{\textwidth}[tb]{p{0.3\textwidth}p{0.35\textwidth}p{0.35\textwidth}}
	   & {\bf Type:} boolean &  {\bf Default:} true\\
	 & & \\
\end{tabular*}
\begin{tabular*}{\textwidth}[tb]{p{0.1\textwidth}p{0.9\textwidth}}
	 INTS\_TOLERANCE\\ 

	 & Minimum absolute value below which integrals are neglected. See the note at the beginning of Section \ref{keywords}. \\ 
\end{tabular*}
\begin{tabular*}{\textwidth}[tb]{p{0.3\textwidth}p{0.35\textwidth}p{0.35\textwidth}}
	   & {\bf Type:} double &  {\bf Default:} 1e-14\\
	 & & \\
\end{tabular*}
\begin{tabular*}{\textwidth}[tb]{p{0.1\textwidth}p{0.9\textwidth}}
	 PRINT\_TEI\\ 

	 & Do ? \\ 
\end{tabular*}
\begin{tabular*}{\textwidth}[tb]{p{0.3\textwidth}p{0.35\textwidth}p{0.35\textwidth}}
	   & {\bf Type:} boolean &  {\bf Default:} false\\
	 & & \\
\end{tabular*}
\begin{tabular*}{\textwidth}[tb]{p{0.1\textwidth}p{0.9\textwidth}}
	 REFERENCE\\ 

	 & Reference wavefunction type \\ 
\end{tabular*}
\begin{tabular*}{\textwidth}[tb]{p{0.3\textwidth}p{0.35\textwidth}p{0.35\textwidth}}
	   & {\bf Type:} string &  {\bf Default:} RHF\\
	 & & \\
\end{tabular*}
\begin{tabular*}{\textwidth}[tb]{p{0.1\textwidth}p{0.9\textwidth}}
	 SEMICANONICAL\\ 

	 & Convert ROHF MOs to semicanonical MOs \\ 
\end{tabular*}
\begin{tabular*}{\textwidth}[tb]{p{0.3\textwidth}p{0.35\textwidth}p{0.35\textwidth}}
	   & {\bf Type:} boolean &  {\bf Default:} true\\
	 & & \\
\end{tabular*}
}
