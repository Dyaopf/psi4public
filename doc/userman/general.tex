\subsection{General Options} \label{general}
\renewcommand{\optionname}[2]{\texttt{\nameref{op-#2-#1}}}
                                                                                
Several options are relevant to many types of computations.  Those are
described briefly here.

\vspace*{0.2in}
\noindent
\begin{tabular*}{\textwidth}[tb]{p{0.3\textwidth}p{0.7\textwidth}}
         REFERENCE & The reference wavefunction used in the computation \\

          & {\bf Possible Values:} RHF, ROHF, UHF, CUHF, RKS, UKS \\
\end{tabular*}
\begin{tabular*}{\textwidth}[tb]{p{0.3\textwidth}p{0.35\textwidth}p{0.35\textwidth}}
           & {\bf Type:} string &  {\bf Default:} RHF\\
         & & \\
\end{tabular*}
The ``reference wavefunction'' specified by the \optionname{REFERENCE}{SCF} keyword is the
type of Hartree--Fock or Kohn--Sham wavefunction to be evaluated during
the computation.  The reference wavefunction may be the goal of the
computation (for \texttt{energy(\qq scf\qq)}), or it may serve as a reference (zeroth-order
solution) for various post-Hartree--Fock computations like CCSD(T) or MP2.

\PSIfour\ can use several types of reference wavefunctions.  For
closed-shell molecules, the appropriate choice is almost always restricted
Hartree--Fock (RHF).  For open-shell molecules, one can choose either
restricted orbitals (restricted open-shell Hartree--Fock (ROHF)), or
unrestricted orbitals (unrestricted Hartree--Fock (UHF)).  We have also
implemented the so-called constrained unrestricted Hartree--Fock (CUHF)
method, which is an alternative approach to ROHF (and should give identical
results) that may be easier to converge \cite{Tsuchimochi:2010:141102}.  Not
all post-Hartree--Fock modules in \PSIfour\ can use all possible reference
types.

For Kohn--Sham density functional theory computations, the possible
references are restricted Kohn--Sham (RKS, for closed-shell molecules)
and unrestricted Kohn--Sham (UKS, for open-shell molecules).
A restricted open-shell Kohn--Sham procedure is not recommended
\cite{Pople:1995:303} and not implemented.

\vspace*{0.2in}
\noindent
\begin{tabular*}{\textwidth}[tb]{p{0.3\textwidth}p{0.7\textwidth}}
         DOCC & An array containing the number of doubly-occupied orbitals
per irrep (in Cotton order) \\
\end{tabular*}
\begin{tabular*}{\textwidth}[tb]{p{0.3\textwidth}p{0.35\textwidth}p{0.35\textwidth}}
           & {\bf Type:} array &  {\bf Default:} No Default\\
         & & \\
\end{tabular*}
\begin{tabular*}{\textwidth}[tb]{p{0.3\textwidth}p{0.7\textwidth}}
         SOCC & An array containing the number of singly-occupied orbitals
per irrep (in Cotton order) \\
\end{tabular*}
\begin{tabular*}{\textwidth}[tb]{p{0.3\textwidth}p{0.35\textwidth}p{0.35\textwidth}}
           & {\bf Type:} array &  {\bf Default:} No Default\\
         & & \\
\end{tabular*}

DOCC and SOCC specify the electron configuration of interest by indicating
how many doubly and singly occupied orbitals there are for each irreducible
representation.  These are arrays of integers, and the ordering is that
given by Cotton.  SOCC is only relevant for open-shell molecules.  If DOCC
and SOCC are not given, the program will attempt to guess the orbital
occupations per irreducible representation automatically. For high-symmetry
molecules, quantum chemistry programs like \PSIfour\ can frequently guess
wrong, so having the ability to manually specify the correct configuration
is useful.  The automatic guess is less likely to be wrong if it is run in
smaller basis sets (like STO-3G); doing so can be a practical way to get
the right occupations, which can then be specified manually for
larger-basis computations.


