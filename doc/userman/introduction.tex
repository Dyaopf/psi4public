\section{Introduction} \label{introduction}

\subsection{Overview} 

\PSIfour\ provides a wide variety of quantum chemical methods using
state-of-the-art numerical methods and algorithms.  Several parts of
the code feature shared-memory parallelization to run efficiently on
multi-core machines.  An advanced parser written in Python allows the user
input to have a very simple style for routine computations, but it can also
automate very complex tasks with ease. 

In this section, we provide an overview of
some of the features of \PSIfour\ along with the prerequisite steps for
running calculations.  Section \ref{tutorial} provides a brief tutorial to
help new users get started.  Section \ref{input} offers further details
into the structure of \PSIfour\ input files and a discussion of some of
the most important options.  Later sections deal with the different types
of computations which can be done using \PSIfour\ (e.g., Hartree-Fock,
MP2, coupled-cluster) and general procedures such as geometry optimization
and vibrational frequency analysis.  The appendix includes a complete
description of all possible input keywords for each module,
as well as tables of available basis sets and a listing of the sample input
files available under {\tt psi4/samples}.
For the latest \PSIfour\ documentation, check \htmladdnormallink{{\tt
www.psicode.org}} {http://www.psicode.org/}.

\subsection{Citing \PSIfour}

The following citation should be used in any publication utilizing the
\PSIfour\ program package:

\begin{quotation}
\noindent
J. M. Turney, A. C. Simmonett, R. M. Parrish, E. G. Hohenstein,
F. Evangelista, J. T. Fermann, B. J. Mintz, L. A. Burns, J. J. Wilke,
M. L. Abrams, N. J. Russ, M. L. Leininger, C. L. Janssen, E. T. Seidl,
W. D. Allen, H. F. Schaefer, R. A. King, E. F. Valeev, C. D. Sherrill,
and T. Daniel Crawford,
{\em WIREs: Comput. Mol. Sci.}, in press (doi: 10.1002/wcms.93).

\end{quotation}

Depending on the particular modules used, the user may also wish to
cite some of the following references for theoretical, algorithmic,
or implementation contributions specific to \PSIfour\ (in addition to
appropriate references for the underlying theory):

{\bf SAPT0} 
\begin{enumerate}
\item ``Large-scale Symmetry-adapted Perturbation Theory Computations via
  Density Fitting and Laplace Transformation Techniques: Investigating the
  Fundamental Forces of {DNA}-Intercalator Interactions,'' E. G. Hohenstein,
  R. M. Parrish, C. D. Sherrill, J. M. Turney, and H. F. Schaefer, {\em J.
  Chem. Phys.} {\bf 135}, 174017 (2011).
\item ``Density Fitting and Cholesky Decomposition Approximations
  in Symmetry-Adapted Perturbation Theory: Implementation and Application
  to Probe the Nature of $\pi$-$\pi$ Interactions in Linear Acenes,''
  E. G. Hohenstein and C. D. Sherrill, {\em J. Chem. Phys.} {\bf 132},
  184111 (2010).
\end{enumerate}

{\bf SAPT2} 
\begin{enumerate}
\item ``Density Fitting of Intramonomer Correlation Effects in
  Symmetry-Adapted Perturbation Theory,''
  E. G. Hohenstein and C. D. Sherrill, {\em J. Chem. Phys.} {\bf 133},
  014101 (2010).
\end{enumerate}

{\bf SAPT2+, SAPT2+(3), SAPT2+3} 
\begin{enumerate}
\item ``Wavefunction Methods for Noncovalent Interactions,'' E. G.
  Hohenstein and C. D. Sherrill, {\em WIREs: Comput. Mol. Sci.}, in press.
\end{enumerate}

{\bf Using Natural Orbitals in SAPT2+, SAPT2+(3), SAPT2+3, etc.} 
\begin{enumerate}
\item ``Efficient Evaluation of Triple Excitations in Symmetry-Adapted
  Perturbation Theory via MP2 Natural Orbitals,'' E. G. Hohenstein
  and C. D. Sherrill, {\em J. Chem. Phys.} {\bf 133}, 104107 (2010).
\end{enumerate}

{\bf DCFT} 
\begin{enumerate}
\item ``Density Cumulant Functional Theory: First Implementation and
  Benchmark Results for the {DCFT-06} Model,'' A. C. Simmonett,
  J. J. Wilke, H. F. Schaefer, and W. Kutzelnigg, {\em J. Chem. Phys.}
  {\bf 133}, 174122 (2010).
\end{enumerate}

{\bf ADC(2)} 
\begin{enumerate}
\item ``Spin-adaptation of Open-shell Electron Propagator Theory,''
  M. Saitow, T. Ida, and M. Mizuno, {\em Chem. Phys. Lett.} {\bf 486},
  171-177 (2010).
\end{enumerate}

{\bf PR-CIS(D) and PR-ADC(2)} 
\begin{enumerate}
\item ``Excited State Calculation for Free-Base and Metalloporphyrins with
  the Partially Renormalized Polarization Propagator Approach,''
  M. Saitow and Y. Mochizuki, {\em Chem. Phys. Lett.}, in press.
\end{enumerate}

{\bf CI} 
\begin{enumerate}
\item ``The Configuration Interaction Method: Advances in Highly 
   Correlated Approaches,'' C. D. Sherrill and H. F. Schaefer, in
   {\em Advances in Quantum Chemistry}, vol. 34, P.-O. L{\"o}wdin, Ed.
   (Academic Press, New York, 1999), pp. 143-269.
\end{enumerate}

{\bf Mk-MRCCSD} 
\begin{enumerate}
\item ``Coupling Term Derivation and General Implementation of
  State-Specific Multireference Coupled-Cluster Theories,''
  F. A. Evangelista, W. D. Allen, and H. F. Schaefer, 
  {\em J. Chem. Phys.} {\bf 127}, 024102 (2007).
\end{enumerate}

{\bf Mk-MRCCSDT(-n)} 
\begin{enumerate}
\item ``Triple Excitations in State-Specific Multireference Coupled
  Cluster Theory: Application of Mk-MRCCSDT and Mk-MRCCSDT-n Methods to
  Model Systems,'' F. A. Evangelista, A. C. Simmonett, W. D. Allen,
  H. F. Schaefer, and J. Gauss, {\em J. Chem. Phys.} {\bf 128}, 124104
  (2008).
\end{enumerate}

\subsection{Obtaining and Installing \PSIfour}
\label{installation}

The latest version of the \PSIfour\ program package may be obtained at
\htmladdnormallink{{\tt www.psicode.org}}{http://www.psicode.org}.  The
source code is available as a gzipped tar archive (named, for example, {\tt
psi4.X.tar.gz}), and binaries may be available for certain architectures.
For detailed installation and testing instructions, please refer to the
installation instructions at the \PSIfour\ website above, or the file {\tt
psi4/INSTALL} distributed with the package.

\subsubsection{Scratch File Configuration}
One very important part of user configuration at the end of the
installation process is to tell \PSIfour\ where to write its temporary
(``scratch'') files.  Electronic structure packages like \PSIfour\ can
create rather large temporary disk files.  It is very important to 
ensure that PSI4 is writing its temporary files to a disk drive
phsyically attached to the computer running the computation.  If it
is not, it will significantly slow down the program and the network.
By default, PSI4 will write temporary files to \file{/tmp}, but this
directory is often not large enough for typical computations.  Therefore,
you need to (a) make sure there is a sufficiently large directory on a
locally-attached disk drive (100GB--1TB or more, depending on the size of
the molecules to be studied), and (b) tell \PSIfour\ the path to this
directory.  The \PSIfour\ installation instructions explain how to set up a
resource file, {\tt .psi4rc}, for each user providing this information.


\subsection{Supported Architectures}
The majority of \PSIfour\ was developed on Mac and Linux machines.  In
principle, it should work on any Unix system; however, we have not tested
extensively on sytems other than Mac and Linux.  There is not a Windows
version of \PSIfour.

\PSIfour\ has been successfully compiled using Intel, GCC, and Clang
compilers.  For the Intel compilers, use versions 11 or
12.1 (we have had trouble with version 12.0).  


\subsection{Capabilities}

\PSIfour\ can perform {\em ab initio} computations employing basis
sets of contrated Gaussian-type functions of virtually arbitrary
orbital quantum number.  Many parts of \PSIfour\ can recognize and
exploit the largest Abelian subgroup of the molecular point group.
Table \ref{table:methods} displays the range of theoretical methods
available in \PSIfour.

\begin{table}
\caption{Summary of theoretical methods available in \PSIfour.} \label{table:methods}
\parsep 10pt
\begin{center}
\begin{tabular}{lcc} \hline\hline
Method                & Energy & Gradient \\ \hline
RHF/ROHF/UHF SCF     & Y & N \\
RHF/ROHF/UHF DF-SCF   & Y & N \\
%HF DBOC               & Y & N \\
%TCSCF                 & Y & Y \\
%CASSCF                & Y & Y \\
%RASSCF                & Y & Y \\
CIS/RPA/TDHF          & Y & N \\
UHF DCFT              & Y & N \\
RHF SAPT              & Y & N \\
RHF MP2               & Y & Y \\
UHF/ROHF MP2          & Y & N \\
RHF DF-MP2            & Y & N \\
RHF ADC(2)            & Y & N \\
%RHF MP2-R12           & Y & N \\
RHF/ROHF CI(n)        & Y & Y \\
RHF/ROHF RAS-CI       & Y & Y \\
RHF/ROHF MP(n)        & Y & N \\
RHF/ROHF ZAPT(n)      & Y & N \\
%RAS-CI DBOC           & Y & N \\
RHF/UHF/ROHF CCSD     & Y & Y \\
RHF/UHF/ROHF CCSD(T)  & Y & Y$^*$ \\
RHF/UHF/ROHF EOM-CCSD & Y & Y \\
\hline\hline
\end{tabular}
\end{center}
\footnotesize{$^*$ CCSD(T) gradients implemented only via an experimental
code.  A more efficient and robust implementation will appear in the next
release.}
\end{table}
Geometry optimization (currently restricted to true minima on the potential
energy surface) can be performed using either analytic gradients
or energy points.  Likewise, vibrational frequencies can be 
computed using analytic second derivatives, by finite
differences of analytic gradients, or finite differences of energies.
\PSIfour\ can also compute an extensive list of one-electron properties.

\subsection{Technical Support} The \PSIfour\ package is
distributed for free and without any guarantee of reliability,
accuracy, suitability for any particular purpose.  No obligation
to provide technical support is expressed or implied.  As time
allows, the developers will attempt to answer inquiries directed to
\htmladdnormallink{{\tt crawdad@vt.edu}}{mailto:crawdad@vt.edu}.
For bug reports, specific and detailed information, with example
inputs, would be appreciated.  Questions or comments regarding
this user's manual may be sent to \htmladdnormallink{{\tt
sherrill@gatech.edu}}{mailto:sherrill@gatech.edu}.



