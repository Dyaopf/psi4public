\section{Evaluation of one-electron properties} \label{sec:oeprop}

\PSIthree\ is capable of computing a number of
one-electron properties
Table \ref{table:oepropsummary} summarizes these capabilities. This section
describes details of how to have \PSIthree\ compute desired one-electron properties
\begin{table}[h]
\begin{center}
\begin{tabular}{lcl}
\hline
\hline
\multicolumn{1}{c}{Feature} & On by default? & \multicolumn{1}{c}{Notes}  \\
\hline
Electric dipole moment & Y & \\
Electric quadrupole moment & N & Set \keyword{MPMAX} to 2 or 3.\\
Electric octupole moment & N & Set \keyword{MPMAX} to 3.\\
Electrostatic potential & Y & At the nuclei; on 2-D grid set {\tt GRID=2}. \\
Electric field & Y & At the nuclei. \\
Electric field gradient & Y & At the nuclei. \\
Hyperfine coupling constant & N & Set {\tt SPIN\_PROP=true}. \\
Relativistic (MVD) corrections & N & Set \keyword{MPMAX} to 2. \\
Electron density & Y & At the nuclei; on 2-D grid set {\tt GRID=2}; \\
& & on 3-D grid set {\tt GRID=6}. \\
Spin density & N & Set {\tt SPIN\_PROP=true}; at the nuclei; \\
& & on 3-D grid set {\tt GRID=6}. \\
Electron density gradient & N & on 2-D grid set {\tt GRID=3}. \\
Spin density gradient & N & on 2-D grid set {\tt GRID=3} and {\tt SPIN\_PROP=true}. \\
Electron density Laplacian & N & on 2-D grid set {\tt GRID=4}. \\
Spin density Laplacian & N & on 2-D grid set {\tt GRID=4} and {\tt SPIN\_PROP=true}. \\
Molecular Orbitals (MO) & N & on 3-D grid set {\tt GRID=5}. \\
Natural Orbitals (NO) & N & Set \keyword{WRTNOS} to true; written to \chkptfile. \\
MO/NO spatial extents & N & Set \keyword{MPMAX} to 2 or 3; \\
& & MOs are used if {\tt WFN=SCF}, otherwise NOs. \\
\hline
\hline
\end{tabular}
\end{center}
\caption{Current one-electron property capabilities of \PSIthree.}
\label{table:oepropsummary}
\end{table}

\subsection{Basic Keywords}

To compute one-electron properties at a fixed geometry,
the following keywords are common:
\begin{description}
\item[JOBTYPE = string]\mbox{}\\
This keyword should be set to {\tt oeprop} for \PSIthree\
to compute electron properties. There is no default.
For CI wavefunctions, limited properties such as dipole and 
transition moments may be evaluated directly
in {\tt detci} without having to specify {\tt JOBTYPE = oeprop}.
\item[WFN = string]\mbox{}\\
Acceptable values are {\tt scf} for HF, {\tt mp2} for MP2,
{\tt detci} for CI, {\tt detcas} for CASSCF, and {\tt ccsd}
for CCSD. There is no default.
\item[REFERENCE = string]\mbox{}\\
Acceptable value are {\tt rhf} and {\tt rohf}. There is no default.
\item[FREEZE\_CORE = boolean]\mbox{}\\
Specifies whether core orbitals (which are determined automatically) are to
be excluded from the correlated calculations.  Default is {\tt false}.
\item[PRINT = integer]\mbox{}\\
The desired print level for detailed output. Defaults to 1.
\item[MPMAX = integer]\mbox{}\\
This integer specifies the highest-order electric multipole moment
to be computed. Valid values are 1 (dipole), 2 (up to quadrupole), or
3 (up to octupole). Default is 1.
\item[MP\_REF = integer]\mbox{}\\
This integer specifies the reference point for the evaluation of
electric multipole moments. Valid values are 1 (center of mass),
2 (origin), 3 (center of electronic change) and
4 (center of the nuclear charge). For charge-neutral systems the
choice of \keyword{MP\_REF} is irrelevant. Default is 1.
\item[GRID = integer]\mbox{}\\
This integer specifies the type of one-electron property and the
type of grid on which to evaluate it. The valid choices are
\begin{itemize}
\item {\tt 0} -- compute nothing
\item {\tt 1} -- electrostatic potential on a 2-D grid
\item {\tt 2} -- electron density on a 2-D grid (spin density, if {\tt SPIN\_PROP=true})
\item {\tt 3} -- projection of electron density gradient on a 2-D grid (spin density
gradient, if {\tt SPIN\_PROP=true})
\item {\tt 4} -- Laplacian of electron density on a 2-D grid (Laplacian of
spin density, if {\tt SPIN\_PROP=true})
\item {\tt 5} -- values of molecular orbitals on a 3-D grid
\item {\tt 6} -- electron density on a 3-D grid (spin density if {\tt SPIN\_PROP=true})
\end{itemize}
Default is {\tt 0}.

\item[NIX = integer]\mbox{}\\
The number of grid points along the x direction. This parameter has be greater than 1.
Default is 20.

\item[NIY = integer]\mbox{}\\
The number of grid points along the y direction. This parameter has be greater than 1.
Default is 20.

\item[NIZ = integer]\mbox{}\\
The number of grid points along the z direction (if a 3-D grid is chosen).
This parameter has be greater than 1. Default is 20.

\item[GRID\_FORMAT = string]\mbox{}\\
This keyword specifies in which format to produce grid data. The only valid choice
for 2-D grids is {\tt plotmtv} (format of plotting software program {\tt PlotMTV}).
For 3-D grids, valid choices are {\tt gausscube} ({\tt Gaussian 94} cube format)
and {\tt megapovplus} (format of 3D rendering software {\tt MegaPOV+}).
The defaults are {\tt plotmtv} and {\tt gausscube} for 2-d and 3-d grids,
respectively.

\item[MO\_TO\_PLOT = vector]\mbox{}\\
Specifies indices of the molecular orbitals to be computed on the 3-d grid. Indices can be specified
as:
\begin{itemize}
\item
unsigned integer - index in Pitzer ordering (ordered accoring to irreps, not eigenvalues).
Ranges from 1 to the number of MOs.
\item
signed integer - index with respect to Fermi level. +1 means LUMO, +2 means
second lowest virtual orbital, -1 means HOMO, etc.
\end{itemize}
All indices have to be either unsigned or signed, you can't mix and match,
or you will get unpredictable results.
Default is to compute HOMO and LUMO.

\item[SPIN\_PROP = boolean]\mbox{}\\
Whether to compute spin-dependent properties.
Default is {\tt false}.

\item[WRTNOS = boolean]\mbox{}\\
If set to {\tt true},
natural orbitals will be written to the checkpoint file.
Default is {\tt false}.

\end{description}


\subsection{Evaluation of properties on rectalinear grids}
\PSIthree\ can evaluate a number of one-electron properties
on {\em rectalinear} 2-D and 3-D grids. In most cases,
3-D grids are utilized. In such cases you only need
to specify the appropriate value for \keyword{GRID}
and \PSIthree\ will automatically construct a rectalinear
3-D grid that covers the entire molecular system.
However, there's no default way to construct a useful 2-D
grid in general. Even in the 3-D case you may want to
``zoom in'' on a particular part of the molecule.
Hence one needs to be able to specify general 2-D
and 3-D grids. In the absence of graphical user interface,
\PSIthree\ has a very flexible system for specifying
arbitrary rectalinear grids.

The following keywords may be used in construction of
the grid:

\begin{description}

\item[GRID\_ORIGIN = real\_vector]\mbox{}\\
A vector of 3 real numbers, this keyword specifies
the origin of the grid coordinate system.
A rectangular grid box which envelops the entire molecule
will be computed automatically if \keyword{GRID\_ORIGIN} is missing, however,
there is no default for 2-D grids.

\item[GRID\_UNIT\_X = real\_vector]\mbox{}\\
A vector of 3 real numbers, this keyword specifies
the direction of the first (x) side of the grid.
It doesn't have have to be of unit length.
There is no default for 2-D grids.

\item[GRID\_UNIT\_Y = real\_vector]\mbox{}\\
A vector of 3 real numbers, this keyword specifies
the direction of the second (y) side.
It doesn't have to be of unit length
or even orthogonal to \keyword{GRID\_UNIT\_X}.
There is no default for 2-D grids.

\item[GRID\_UNIT\_XY0 = real\_vector]\mbox{}\\
A vector of 2 real numbers, this keyword specifies
the coordinates of the lower left corner of a 2-D grid in 
the 2-D coordinate system defined by \keyword{GRID\_ORIGIN}, \keyword{GRID\_UNIT\_X},
and \keyword{GRID\_UNIT\_Y}. This keyword is only used
to specify a 2-D grid.
There is no default.

\item[GRID\_UNIT\_XY1 = real\_vector]\mbox{}\\
A vector of 2 real numbers, this keyword specifies
the coordinates of the upper right corner of a 2-D grid in 
the 2-D coordinate system defined by \keyword{GRID\_ORIGIN}, \keyword{GRID\_UNIT\_X},
and \keyword{GRID\_UNIT\_Y}. This keyword is only used
to specify a 2-D grid.
There is no default.

\item[GRID\_UNIT\_XYZ0 = real\_vector]\mbox{}\\
A vector of 3 real numbers, this keyword specifies
the coordinates of the far lower left corner of a 3-D grid in 
the 3-D coordinate system defined by \keyword{GRID\_ORIGIN}, \keyword{GRID\_UNIT\_X},
and \keyword{GRID\_UNIT\_Y}. This keyword is only used
to specify a 3-D grid.
There is no default.

\item[GRID\_UNIT\_XYZ1 = real\_vector]\mbox{}\\
A vector of 3 real numbers, this keyword specifies
the coordinates of the near upper right corner of a 3-D grid in 
the 3-D coordinate system defined by \keyword{GRID\_ORIGIN}, \keyword{GRID\_UNIT\_X},
and \keyword{GRID\_UNIT\_Y}. This keyword is only used
to specify a 3-D grid.
There is no default.

\end{description}

In addition, the following keywords are useful for evaluation of certain
properties on 2-D grids:

\begin{description}

\item[GRID\_ZMIN = real]\mbox{}\\
This keyword specifies the lower limit on displayed
z-values for contour plots of electron density and 
its Laplacian. Only useful when {\tt GRID=2} or {\tt GRID=4}.
Default is 0.0

\item[GRID\_ZMAX = real]\mbox{}\\
This keyword specifies the upper limit on displayed
z-values for contour plots of electron density and 
its Laplacian. Only useful when {\tt GRID=2} or {\tt GRID=4}.
Default is 3.0

\item[EDGRAD\_LOGSCALE = integer]\mbox{}\\
This keyword controls the logarithmic scaling of the produced electron density gradient 
plot. Turns the scaling off if set to zero, otherwise the higher value - 
the stronger the gradient field will be scaled.
Recommended value (default) is 5. This keyword is only useful when
{\tt GRID=3}.

\end{description}

\subsection{Grid specification mini-tutorial}

Let's look at how to set up input for spin density evaluation on a
two-dimensional grid.  The relevant input section of \PSIthree\ might look like
this:
\begin{verbatim}

  jobtype = oeprop

  grid = 2
  spin_prop = true
  grid_origin = (0.0 -5.0 -5.0)
  grid_unit_x = (0.0 1.0 0.0)
  grid_unit_y = (0.0 0.0 1.0)
  grid_xy0 = (0.0 0.0)
  grid_xy1 = (10.0 10.0)
  nix = 30
  niy = 30

\end{verbatim}
\keyword{grid} specifies the type of a property and the type of a grid
\PSIoeprop\ needs to compute.
Since \keyword{spin\_prop}\ is set and {\tt grid=2}, the spin density will be
evaluated on a grid.

Grid specification is a little bit tricky but very
flexible. \keyword{grid\_origin}\ specifies the origin of the
rectangular coordinate system associated with the grid in the
reference frame. \keyword{grid\_unit\_x}\ specifies a reference frame
vector which designates the direction of the x-axis of the grid
coordinate system.  \keyword{grid\_unit\_y}\ is analogously a
reference frame vector which, along with the \keyword{grid\_unit\_x},
completely specifies the grid coordinate system.
\keyword{grid\_unit\_x}\ and \keyword{grid\_unit\_y}\ do not have to
be normalized, neither they need to be orthogonal to either other -
orthogonalization is done automatically to ensure that unit vectors of
the grid coordinate system are normalized in the reference frame too.
\keyword{grid\_xy0}\ is a vector in the grid coordinate system that
specifies a vertex of the grid rectangle with the most negative
coordinates. Similarly, \keyword{grid\_xy1}\ specifies a vertex of the
the grid rectangle diagonally opposite to \keyword{grid\_xy0}.
Finally, \keyword{nix}\ and \keyword{niy}\ specify the number of
intervals into which the $x$ and $y$ sides of the grid rectangle are
subdivided.  To summarize, the above input specifies a rectangular (in
fact, square) 30 by 30 grid of dimensions 10.0 by 10.0 lying in the
$yz$ plane and centered at origin of molecular frame.

Running \PSIthree\ on such input will create a file called
\file{sdens.dat} (for file names refer to man page on \PSIoeprop),
which can be fed directly to {\tt PlotMTV} to plot the 2-D data.

Specification of a three-dimensional grid for plotting MOs
({\tt grid = 5}) or densities ({\tt grid = 6}) is just slightly
more complicated.
For example, let's look at producing data for plotting a HOMO and
a LUMO. The indices of the
MOs which needs to be plotted will be specified by keyword
\keyword{mo\_to\_plot}.
The reference frame is
specified by keywords \keyword{grid\_origin}, \keyword{grid\_unit\_x}\
and \keyword{grid\_origin\_y}\ (the third axis of the grid coordinate
system is specified by by the vector product of
\keyword{grid\_unit\_x}\ and \keyword{grid\_unit\_y}).  Since in this
case we are dealing with the three-dimensional grid coordinate system,
one needs to specify two diagonally opposite vertices of the grid box
via \keyword{grid\_xyz0}\ and \keyword{grid\_xyz1}.  The number of
intervals along $z$ is specified via \keyword{niz}.  The relevant section of
input file may look like this:
\begin{verbatim}

  jobtype = oeprop

  grid = 5
  mo_to_plot = (-1 +1)
  grid_origin = (-5.0 -5.0 -5.0)
  grid_unit_x = (1.0 0.0 0.0)
  grid_unit_y = (0.0 1.0 0.0)
  grid_xyz0 = (0.0 0.0 0.0)
  grid_xyz1 = (10.0 10.0 10.0)
  nix = 30
  niy = 30
  niz = 30

\end{verbatim}

Running \PSIthree\ on input like this will produce a {\tt Gaussian Cube}
file called {\tt mo.cube}, which can be used to render images of HOMO
and LUMO using
an external visualization software.

\subsection{Plotting grid data}
2-D grids should be plotted by an interactive visualization
code {\tt PlotMTV}. {\tt PlotMTV} is a freeware code
developed by Kenny Toh. It can be downloaded off many web sites in
source or binary form.

3-D grids can be produced in two formats: {\tt megapovplus}  and {\tt gausscube}
(see \keyword{GRID\_FORMAT}).
First is used to render high-quality images with a program {\tt MegaPov} (version
0.5).  {\tt MegaPov} is an unofficial patch for a ray-tracing code
{\tt POV-Ray}. Information on {\tt MegaPov} can be found at
\htmladdnormallink{{http://nathan.kopp.com/patched.htm}}{http://nathan.kopp.com/patched.htm}.
{\tt Gaussian Cube} files can be processed by a number of programs. We cannot recommend
any particular program for that purpose here.

\subsection{Visualizing Molecular Obitals with gOpenMol}
The {\tt Gaussian Cube} files generated by oeprop can be converted and viewed with gOpenMol. 
gOpenMol offers good looking plots in a graphical user interface. Information on 
downloading gOpenMol and samples of gOpenMol output may be found at 
\url{http://www.csc.fi/gopenmol/}.

Installation instructions are included with the gOpenMol download. Once installed, the first 
step to viewing molecular orbitals is to convert the \keyword{mo.cube} into a format that 
gOpenMol recognizes. Under the Run menu, select \keyword{gCube2plt/g94cub2pl (cube) $\dots$}, 
this will bring up a window with the heading \keyword{Run gCube2plt/g94cub2pl}. In the 
input file name field, select the \keyword{mo.cube} file you want to convert. Likewise, in 
the output file name field type the name of the output file you want. Click the Apply button 
to perform the conversion. This procedure will create a \keyword{.plt} and a 
\keyword{.crd} file. Once converted, click Dismiss to close the window. The 
{\tt Gaussian Cube} file is now converted and in a form that gOpenMol can recognize.

In order to view the molecular orbital, the first step is to import the coordinate file 
(\keyword{.crd}). This is done under the File menu$\rightarrow$Import$\rightarrow$Coords$\dots$. 
Again, a window will pop up. In the Import file name field chose the \keyword{.crd} you just 
created from the conversion procedure. Click apply, then Dismiss to close the window. Now
we have to import the \keyword{.plt} file to view the molecular orbital. Under the the 
Plot menu selct Contour$\dots$, this will bring up a window. In the File name field, 
either type the full path of the file name or use browse to select the \keyword{.plt} 
file you just created in the conversion, then click Import. In the Define contour levels 
we have to define the contour cutoffs for the positive and negative parts of the wave 
function seperately. I recommend trying 0.1 in the first box and -0.1 in the second. Click 
Apply to view the molecular orbital. You can change the colors of the positive and negative 
sections independently by clicking on the Colour button next to the respective cutoffs. Also, 
in the Details$\dots$ section, you can fine tune the properties of the molecular orbital, 
such as, the opacity, solid vs. mesh, smoothness, and cullface state. You can play around 
with various settings to get the surface to look exactly how you want it to. There is more 
information in the Help$\rightarrow$Tutorials menu on this subject as well as many other abilities 
of gOpenMol.
