\section{A \PSIfour\ Tutorial} \label{tutorial}

\subsection{Before Getting Started: A Warning about Scratch Files}
Generally, electronic structure programs like \PSIfour\ make
significant use of disk drives.  Therefore, it is very important
to ensure that PSI4 is writing its temporary files to a disk drive
phsyically attached to the computer running the computation.  If it
is not, it will significantly slow down the program and the network.
By default, PSI4 will write temporary files to \file{/tmp}, but you
will want to set up a default scratch path (as described in sections
\ref{scratchfiles} and \ref{psirc}) because the \file{/tmp} directory
is usually not large enough except for small test cases.  In any 
event, you want to be very careful that you are not writing scratch
files to an NFS-mounted directory that is physically attached to a 
fileserver elsewhere on the network.

\subsection{Basic Input File Structure} 

PSI4 reads input from a text file, which can be prepared in any standard
text editor.  The default input file name is \file{input.dat} and the
default output file name is \file{output.dat}.  So that you can give your
files meaningful names, these defaults can be changed by specifying
the input file name and output file name on the the command line.
The syntax is:

{\tt psi4 input-name output-name}

\subsubsection{IPV1-Style Input}
PSI4 is a modular single executable, with each module performing specific tasks
and computations.  Which modules are run for a particular computation
depends on the type of computation and the particular keywords specified
in the input file.  All keywords in PSI4 use the structure {\tt keyword =
value}, where values may be strings, booleans, integers, or real numbers.
If the value is a string which contains a special character (such as a
space or a dash) you must enclose the string in double quotation marks.
You can give keywords in the input file for specific modules; however,
in the first few examples, we will place all our keywords in one section
of our input file called {\tt psi}.  Generally, every module you run
during your computation will read the keywords in {\tt psi}, so you
can place all your keywords in this section if you choose to do so.

\subsection{Running a basic SCF calculation}
In our first example, we will consider a Hartree-Fock SCF computation
for the water molecule using a cc-pVDZ basis set.  We will specify the
geometry of our water molecule using a standard z-matrix.

\begin{verbatim}
psi:(
 label = "cc-pVDZ SCF H2O"
 jobtype = sp
 wfn = scf
 reference = rhf
 basis = "cc-pVDZ"
 zmat = (
   o
   h 1 0.957
   h 1 0.957 2 104.5
  )
 )

\end{verbatim}

In each computation, you can specify the type of wavefunction (keyword
{\tt wfn}), the reference wavefunction for post-Hartree-Fock computations
(keyword {\tt reference}), and the type of computation you want to
perform (keyword {\tt jobtype}).  In the example above, we used a
restricted Hartree-Fock (RHF) reference in an SCF computation of a
single-point energy.  To change the level of electron correlation, one
would specify a different wavefunction type using the keyword {\tt wfn}.
In the example above, to perform an MP2 computation, simply set {\tt
wfn = mp2}.

\subsection{Geometry Optimization and Vibrational Frequency Analysis}
The above example was a simple single-point energy computation.
To perform a different type of computation, change the keyword {\tt
jobtype}.  In the example below, we will set up
a CCSD geometry optimization.  To illustrate a more flexible z-matrix
input, we will now define variables for the bond length and bond angle
(in the {\tt zvars} section).

\begin{verbatim}
% 6-31G** H2O Test optimization calculation

psi: (
  label = "6-31G** SCF H2O"
  jobtype = opt
  wfn = ccsd
  reference = rhf
  dertype = first
  basis = "6-31G**"
  zmat = (
    o
    h 1 roh
    h 1 roh 2 ahoh
  )
  zvars = (
    roh     0.96031231
    ahoh  104.09437511
  )
)
\end{verbatim}

Once you have optimized the geometry of a molecule, you might wish to
perform a frequency analysis to determine the nature of the stationary
point.  To do this, change the value of {\tt jobtype} to {\tt freq}.
For an SCF frequeny calculation, you would also set {\tt dertype =
second} to compute the second derivatives analytically.  Unfortunately,
analytical second derivitives are not available in \PSIfour\ for
wavefunctions beyond SCF, so instead use the highest order analytical
derivitives that are available for the type of wavefunction you
have chosen.  This information is given in Table \ref{table:methods}.
For our CCSD example, the highest-order derivitives available are first,
so {\tt dertype = first}.

\begin{verbatim}
% 6-31G** H2O Test computation of frequencies

psi: (
  label = "6-31G** SCF H2O"
  jobtype = freq
  wfn = ccsd
  reference = rhf
  dertype = first
  basis = "6-31G**"
  zmat = (
    o
    h 1 roh
    h 1 roh 2 ahoh
  )
  zvars = (
    roh     0.96031231
    ahoh  104.09437511
  )
)
\end{verbatim}

\subsection{More Advanced Input Options}
If you wish to add comments to your input file, you can start any line
with \% and the line will be a comment line.  This can make the input
file easier to understand because you can provide explainations about
each keyword.  Another way to make the input file more organized is
to seperate it into sections that correspond to particular modules
the calculation will use.  This can be particularly helpful for more
complicated computations which can utilize many of keywords.  In the example
below, a CCSD(T) computation for the BH molecule is performed using a
cc-pVDZ basis set.  The keywords are divided into sections and several
new keywords are introduced, including ones to specify symmetry and
orbital occupations.  Orbitial occupations are specified by
a list of integers enclosed in parentheses.  These integers give the
number of orbitials which belong to each irreducible representation in
the point group.  The ordering of the irreps are those given by Cotton
in {\em Chemical Applications of Group Theory}.  In this example,
comment lines will be included to explain the new keywords used.

\begin{verbatim}
psi: (
  wfn = ccsd_t
  reference = rhf
)

default: (
  label = "BH cc-pVDZ CCSD(T)"

% Allocating memory for the calculation
  memory = (600.0 MB)

% charge and multiplicity (2S+1) default to values of 0 and 1, respectively
  charge = 0
  multp = 1

% The program will generally guess the symmetry of the molecule, but
% it can be overridden.  Here we specify C2V because only D2H and its
% subgroups can be used by the program.
  symmetry = c2v

% Number of doubly-occupied orbitals per irrep can be specified manually
% if desired
  docc = (3 0 0 0)

% Freeze the 1A1 orbital (Boron 1s-like) in the CCSD(T) computation
  frozen_docc = (1 0 0 0)
)

% The input section contains information about the molecule and the basis
% set.  The geometry here is specified by cartesian coordinates.
input: (
  basis = "cc-pVDZ"
  units = angstroms
  geometry = (
    ( b      0.0000        0.0000        0.0000)
    ( h      0.0000        0.0000        0.8000)
      )
  origin = (0.0 0.0 0.0)
)
% The modular input structure lets you specify convergence criteria for
% each part of the computation separately
scf: (
  maxiter = 100
  convergence = 11
)
\end{verbatim}

The final example of this tutorial demonstrates an example of a
complete-active-space self-consistent-field (CASSCF)
computation.  CAS computations require specification of several additional
keywords because you must specify which orbitals you wish to be in the
active space.  The notation and ordering for specifying CAS orbitals is the
same as for occupied orbitals.

\begin{verbatim}

% 6-31G** H2O Test CASSCF Energy Point

psi: (
  label = "6-31G** CASSCF H2O"
  jobtype = sp
  wfn = casscf
  reference = rhf
% The restricted_docc orbitals are those which are optimized, but are not
% in the active space.
  restricted_docc = (1 0 0 0)

% The active space orbitals; here, the valence orbitals are chosen
  active          = (3 0 1 2)

  basis = "6-31G**"
  zmat = (
    o
    h 1 1.00
    h 1 1.00 2 103.1
  )
)
\end{verbatim}

