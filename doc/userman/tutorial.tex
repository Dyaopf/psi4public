\section{A \PSIfour\ Tutorial} \label{sec:tutorial}

\subsection{Basic Input File Structure}

PSI4 reads input from a text file, which can be prepared in any standard
text editor.  The default input file name is \file{input.dat} and the
default output file name is \file{output.dat}.  So that you can give your
files meaningful names, these defaults can be changed by specifying
the input file name and output file name on the the command line.
The syntax is:

{\tt psi4 input-name output-name}

\subsection{Running a Basic Hartree--Fock Calculation}
In our first example, we will consider a Hartree--Fock SCF computation
for the water molecule using a cc-pVDZ basis set.  We will specify the
geometry of our water molecule using a standard Z-matrix.

\begin{Snippet}

# Any line starting with the # character is a comment line
# Here's a sample HF/cc-pVDZ H2O computation

molecule h2o {
  O
  H 1 0.96
  H 1 0.96 2 104.5
}

set basis cc-pVDZ
energy('scf')
\end{Snippet}

For your convenience, this example can be found in {\tt
psi4/samples/tu1-h2o-energy/input.dat}.  You can run it if you wish.
Once \PSIfour\ is in your path (see the User Configuration section of
the installation instructions), you can run this computation by typing
\begin{verbatim}
psi4 input.dat output.dat
\end{verbatim}
If everything goes well, the computation should complete and should report
a final restricted Hartree--Fock energy in a section like this:
\begin{verbatim}
  Energy converged.

  @RHF Final Energy:   -76.02665366589162
\end{verbatim}
By default, the energy should be converged to about 1.0\e{-8}, so agreement
is only expected for about the first 8 digits after the decimal.  If the
computation does not complete, there is probably a problem with the
compilation or installation of the program (see the installation
instructions in Sec. \ref{sec:installation}).

This very simple input is sufficient to run the requested information.
Notice that we didn't tell the program some otherwise useful information
like the charge on the molecule (0, it's neutral), the spin multiplicity
(1 for a closed-shell molecule with all electrons paired), or the reference
wavefunction to use (restricted Hartree--Fock, or RHF, is usually
appropriate for a closed-shell molecule).  The program correctly guessed
all of these options for us.  We can change the default behavior through
additional keywords.

Let's consider what we would do for an open-shell molecule, where
not all electrons are paired.  For example, let's run a computation
on methylene (CH$_2$), whose ground electronic state has two unpaired
electrons (triplet electronic state, or a spin multiplicity $2S+1 = 3$).
In this case, the default spin multiplicity (1) is not correct, so we
need to tell the program the true value (3).  Like many programs, \PSIfour\
can get the charge and multiplicity as the first two integers in the
Z-matrix.  Note the line with {\tt 0 3} at the beginning of the molecule
specification below.  In this example we will also specify the bond length
and bond angle as variables ({\tt R} and {\tt A}), whose values are given
at the end of the Z-matrix specification.

\begin{Snippet}
# Here's a sample UHF/6-31G** CH2 computation

molecule ch2 {
  0 3
  C
  H 1 R
  H 1 R 2 A

  R = 1.075
  A = 133.93
}

set basis 6-31G**
set reference uhf

energy ('scf')
\end{Snippet}

This sample input can be found in {\tt psi4/samples/tu2-ch2-energy}, and as
before it can be run through the command {\tt psi4 input.dat output.dat}
(actually, because {\tt psi4} by default looks for an input file named {\tt
input.dat} and writes by default to a file called {\tt output.dat}, in this
case one could also just type {\tt psi4}).  If it works, it should print
the final energy as
\begin{verbatim}
  @UHF Final Energy:   -38.92534160932308
\end{verbatim}
Notice we added a new keyword, {\tt set reference uhf}, to the input.  For
open-shell molecules, we have a choice of unrestricted orbitals
(unrestricted Hartree--Fock, or UHF), or restricted orbitals (restricted
open-shell Hartree--Fock, or ROHF).  Usually, UHF is a little easier to
converge (although it may be more susceptible to spin contamination than
ROHF).

\subsection{Geometry Optimization and Vibrational Frequency Analysis}
The above examples were simple single-point energy computations
(as specified by the {\tt energy()} function).  Of course there are other
kinds of computations to perform, such as geometry optimizations and
vibrational frequency computations.  These can be specified by replacing
{\tt energy()} with {\tt optimize()} or {\tt frequencies()}, respectively.

Here's an example of optimizing the H$_2$O molecule using Hartree--Fock with
a cc-pVDZ basis set (located in {\tt psi4/samples/tu3-h2o-opt}).
\begin{Snippet}
# Optimize H2O HF/cc-pVDZ

molecule h2o {
  O
  H 1 0.96
  H 1 0.96 2 104.5
}

set basis cc-pVDZ
optimize('scf')
\end{Snippet}

This should perform a series of gradient computations.  The gradient points
which way is downhill in energy, and the optimizer then modifies the
geometry to follow the gradient.  After about 4 cycles, the geometry should
converge with a message like {\tt Optimization is complete!}.  As indicated
in the following table (printed by the optimizer at the end of the
computation), the energy decreases with each step,
and the maximum force on each atom also decreases with each step (in
principle these numbers could increase in some iterations, but here they do
not).

\begin{Snippet}
----------------------------------------------------------------------
Step         Energy             Delta(E)      MAX force   MAX Delta(q)
----------------------------------------------------------------------
  1    -76.026653665892    -76.026653665892    1.52e-02   1.52e-02
  2    -76.026907793199     -0.000254127307    9.55e-03   9.55e-03
  3    -76.027052927171     -0.000145133972    4.47e-04   4.47e-04
  4    -76.027053472137     -0.000000544965    1.16e-04   1.16e-04
----------------------------------------------------------------------
\end{Snippet}

To get harmonic vibrational frequencies, {\em first we must set up an input
using the OPTIMIZED GEOMETRY}.  We can easily get the optimized geometry
from the previous computation.  Looking at the output from running the
previous example, we see that the OH bond length is about 0.9463 {\AA}ngstroms,
and the bond angle is about 104.575 degrees.  It's good to give this many
digits (or more) to make sure there's not significant roundoff error in the
geometry when running a frequency computation.  So, our frequency
computation input (which can be found as test case {\tt
psi4/samples/tu4-h2o-freq}) is:
\begin{Snippet}
# Frequencies for H2O HF/cc-pVDZ at optimized geometry

molecule h2o {
  O
  H 1 0.9463
  H 1 0.9463 2 104.575
}

set basis cc-pVDZ
frequencies('scf')
\end{Snippet}
Alternatively, it's also possible for \PSIfour\ to use Cartesian coordinate
input.  Here, the Cartesian coordinates of the optimized geometry can be
extracted from the {\em bottom} of the optimization output.  The input
would then look like
this:
\begin{Snippet}
molecule h2o {
  O     0.0000000000  -0.0000000000  -0.1224239500
  H     0.0000000000  -1.4147069876   0.9714784639
  H    -0.0000000000   1.4147069876   0.9714784639
}

set basis cc-pVDZ
frequencies('scf')
\end{Snippet}
If either of the inputs above are run, the program should do some
computations and then finally report the following harmonic vibrational
frequencies (roundoff errors of around 0.1 cm$^{-1}$ may exist):
\begin{Snippet}
          Irrep      Harmonic Frequency
                          (cm-1)
        -----------------------------------------------
             A1         1776.1735
             A1         4113.8031
             B2         4211.7879
        -----------------------------------------------
\end{Snippet}
Notice that the symmetry type of the normal modes is specified (A1, A1,
B2).  The program also prints out the normal modes in terms of Cartesian
coordinates of each atom.  For example, the normal mode at 1776 cm$^{-1}$
is
\begin{Snippet}
   Frequency:       1776.17
   Force constant:   0.1194
             X       Y       Z
  O        0.000   0.000  -0.067
  H        0.000   0.416   0.536
  H        0.000  -0.416   0.536
\end{Snippet}
where the table shows the displacements in the X, Y, and Z dimensions for
each atom along the normal mode coordinate.  (This information could be used
to animate the vibrational frequency using visualization software.)


\subsection{Analysis of Intermolecular Interactions}
Now let's consider something a little more interesting.  \PSIfour\
contains code to analyze the nature of intermolecular interactions
between two molecules, via symmetry-adapted perturbation theory
(SAPT) \cite{Jeziorski:1994:1887}.  This kind of analysis gives a lot
of insight into the nature of intermolecular interactions, and \PSIfour\
makes these computations easier than ever.

For a SAPT computation, the input needs to provide information on two
distinct molecules.  This is very easy, we just give a Z-matrix or set of
Cartesian coordinates for each molecule, and separate the two with two
dashes, like this:
\begin{Snippet}
# Example SAPT computation for ethene*ethine (i.e., ethylene*acetylene),
# test case 16 from the S22 database

molecule dimer {
0 1
C   0.000000  -0.667578  -2.124659
C   0.000000   0.667578  -2.124659
H   0.923621  -1.232253  -2.126185
H  -0.923621  -1.232253  -2.126185
H  -0.923621   1.232253  -2.126185
H   0.923621   1.232253  -2.126185
--
0 1
C   0.000000   0.000000   2.900503
C   0.000000   0.000000   1.693240
H   0.000000   0.000000   0.627352
H   0.000000   0.000000   3.963929
units angstrom

no_reorient
symmetry c1
}
\end{Snippet}

Notice we have a couple of new keywords in the molecule specification.
{\tt no\_reorient} tells the program not to reorient the molecule
to a different coordinate system (this is important for SAPT to make
sure the same coordinate frame is used for the computations of the two
monomers and for the dimer).  The next keyword tells \PSIfour\ to run
in C1 point-group symmetry (i.e., without using symmetry), even if a
higher symmetry is detected.  SAPT computations need to be run with
symmetry turned off in this way.

Here's the second half of the input:
\begin{Snippet}
set globals {
    basis jun-cc-pVDZ
    scf_type DF
    freeze_core True
}

energy('sapt0')
\end{Snippet}

Before, we have been setting keywords individually with commands like
{\tt set basis cc-pVDZ}.  Because we have a few more options now, it's
convenient to place them together into the {\tt set globals} or {\tt set}
block, bounded by {\tt \{\dots\}}.  This
will set all of these options as ``global'' options (meaning that they are
visible to all parts of the program).  Most common \PSIfour\ options can be
set in a {\tt globals} section like this.  If an option needs to be visible
only to one part of the program (e.g., we only want to increase the
energy convergence in the SCF code, but not the rest of the
code), it can be placed in a section of input visible to that part of the
program (e.g., {\tt set scf e\_convergence 1.0E-8}).

Back to our SAPT example, we have found that for basic-level SAPT
computations (i.e., SAPT0), a good error cancellation is found
\cite{Hohenstein:2012:WIREs} with the jun-cc-pVDZ basis (this is the
usual aug-cc-pVDZ basis, but without diffuse functions on hydrogen and
without diffuse $d$ functions on heavy atoms) \cite{Papajak:2011:10}. So,
we'll use that as our standard basis set.  The SAPT code is designed to
use density fitting techniques, because they introduce minimal errors
whil providing much faster computations \cite{Hohenstein:2010:184111,
Hohenstein:2010:014101}. Since we're using density fitting for the SAPT,
we might as well also use it for the Hartree--Fock computations that are
performed as part of the SAPT.  We can specify that with the option
and value pair \optionname{scf\_type} \optionval{df}.

Density fitting procedures require the use of auxiliary basis sets that
pair with the primary basis set.  Fortunately, \PSIfour\ is usually smart
enough to figure out what auxiliary basis sets are needed for a given
computation.  In this case, jun-cc-pVDZ is a standard enough basis set
(just a simple truncation of the very popular aug-cc-pVDZ basis set)
that \PSIfour\ correctly guesses that we want the jun-cc-pVDZ-JKFIT
auxiliary basis for the Hartree--Fock, and the jun-cc-pVDZ-RI basis set
for the SAPT procedure.

To speed up the computation a little, we also tell the SAPT procedure to
freeze the core electrons with the \optionname{freeze\_core} option.  The SAPT
procedure is invoked with the simple call, {\tt energy(\qq sapt0\qq)}.  This
call knows to automatically run two monomer computations and a dimer
computation and then use these results to perform the SAPT analysis.  The
various energy components are printed at the end of the output, in addition
to the total SAPT0 interaction energy.  An explanation of the various
energy components can be found in the review by Jeziorski, Moszynski, and
Szalewicz \cite{Jeziorski:1994:1887}, and this is discussed in more detail
in the SAPT section later in this manual.  For now, we'll note that most of
the SAPT energy components are negative; this means those are attractive
contributions (the zero of energy in a SAPT computation is defined as
non-interacting monomers).  The exchange contributions are positive
(repulsive).   In this example, the most attractive contribution between
ethylene and acetylene is an electrostatic term of $-2.12$ kcal mol$^{-1}$
({\tt Elst10,r} where the 1 indicates the first-order
perturbation theory result with respect to the intermolecular interaction,
and the 0 indicates zeroth-order with respect to intramolecular electron
correlation).  The next most attractive contribution is the {\tt Disp20}
term (2nd order intermolecular dispersion, which looks like an MP2 in which
one excitation is placed on each monomer), contributing an attraction of
$-1.21$ kcal mol$^{-1}$.  It is not surprising that the electrostatic
contribution is dominant, because the geometry chosen for this example has the
acetylene perpendicular to the ethylene, with the acetylene hydrogen
pointing directly at the double bond in ethylene; this will be attractive
because the H atoms in acetylene bear a partial positive charge, while the
electron-rich double bond in ethylene bears a partial negative charge.  At
the same time, the dispersion interaction should be smaller because the
perpendicular geometry does not allow much overlap between the monomers.
Hence, the SAPT analysis helps clarify (and quantify) our physical
understanding about the interaction between the two monomers.

\subsection{Potential Surface Scans and Counterpoise Correction Made Easy
with Psithon}

Finally, let's consider an example that shows how the Python driver
in \PSIfour\ simplifies some routine tasks.  \PSIfour\ can interpret
valid Python code in addition to the computational chemistry directives
we've seen in the previous examples; we call this mixture Psithon.
The Python computer language is very easy to pick up, and even users
previously unfamiliar with Python can use it to simplify tasks by
modifying some of the example input files supplied with \PSIfour\
in the {\tt psi4/samples} directory.

Suppose you want to do a limited potential energy surface scan, such as
computing the interaction energy between two neon atoms at various
interatomic distances.  One simple but unappealing way to do this is to
create separate input files for each distance to be studied.  But most of
these input files are identical, except that the interatomic distance is
different.  Psithon lets you specify all this in a single input file,
looping over the different distances with an array like this: {\tt
Rvals=[2.5, 3.0, 4.0]}.

Let's also suppose you want to do counterpoise (CP) corrected energies.
Counterpoise correction involves computing the dimer energy and then
subtracting out the energies of the two monomers, each evaluated in the
dimer basis.  Again, each of these computations could be run in a separate
input file, but because counterpoise correction is a fairly standard
procedure for intermolecular interactions, \PSIfour\ knows about it and has
a built-in routine to perform counterpoise correction.  It only needs to
know what method you want to do the counterpoise correction on (here, let's
consider CCSD(T)), and it needs to know what's monomer A and what's monomer
B.  This last issue of specifying the monomers separately was already dealt
with in the previous SAPT example, where we saw that two dashes in the {\tt
molecule} block can be used to separate monomers.

So, we're going to do counterpoise-corrected CCSD(T) energies for Ne$_2$ at
a series of different interatomic distances.  And let's print out a table
of the interatomic distances we've considered, and the CP-corrected CCSD(T)
interaction energies (in kcal mol$^{-1}$) at each geometry.  Doing all this
in a single input is surprisingly easy in \PSIfour.  Here's the input
(available as {\tt psi4/samples/tu6-cp-ne2}).

\begin{Snippet}
#! Example potential energy surface scan and CP-correction for Ne2

molecule dimer {
  Ne
--
  Ne 1 R
}

Rvals=[2.5, 3.0, 4.0]

set basis aug-cc-pVDZ
set freeze_core True

# Initialize a blank dictionary of counterpoise corrected energies
# (Need this for the syntax below to work)
ecp = {}

for R in Rvals:
  dimer.R = R
  ecp[R] = cp('ccsd(t)')

print "\n"
print "CP-corrected CCSD(T)/aug-cc-pVDZ interaction energies\n\n"
print "        R (Ang)         E_int (kcal/mol)             \n"
print "-----------------------------------------------------\n"
for R in Rvals:
  e = ecp[R] * psi_hartree2kcalmol
  print "        %3.1f            %10.6f\n" % (R, e)
\end{Snippet}

First, you can see the {\tt molecule} block has a couple dashes to
separate the monomers from each other.  Also note we've used a Z-matrix to
specify the geometry, and we've used a variable ({\tt R}) as the
interatomic distance.  We have {\em not} specified the value of {\tt R} in
the {\tt molecule} block like we normally would.  That's because we're
going to vary it during the scan across the potential energy surface.
Below the {\tt molecule} block, you can see the {\tt Rvals} array
specified.  This is a Python array holding the interatomic distances we
want to consider.  In Python, arrays are surrounded by square brackets, and
elements are separated by commas.

The next lines, {\tt set basis aug-cc-pVDZ} and {\tt set freeze\_core
True}, are familiar from previous test cases.  Next comes a slightly
unusual-looking line, {\tt ecp = \{\}}.  This is Python's way of initializing
a ``dictionary.''  We're going to use this dictionary to store the
counterpoise-corrected energies as they become available.  A dictionary is
like an array, but we can index it using strings or floating-point numbers
instead of integers if we want.  Here, we will index it using
floating-point numbers, namely, the {\tt R} values.  This winds up being
slightly more elegant than a regular array in later parts of the input
file.

The next section, beginning with {\tt for R in Rvals:}, loops over the
interatomic distances, {\tt R}, in our aray {\tt Rvals}.  In Python,
loops need to be indented, and each line in the loop has to be indented
by the same amount.  The first line in the loop, {\tt dimer.R = R},
sets the Z-matrix variable {\tt R} of the molecule called {\tt dimer}
to the {\tt R} value extracted from the {\tt Rvals} array.  The next line,
{\tt ecp[R] = cp(\qq ccsd(t)\qq)}, computes the counterpoise-corrected
CCSD(T) energy and places it in the {\tt ecp} dictionary with {\tt R} as
the index.  Note we didn't need to specify ghost atoms, and we didn't need
to call the monomer and dimer computations separately.  The built-in
Psithon function {\tt cp()} does it all for us, automatically.

And that's it!  The only remaining part of the example input is a little table
of the different R values and the CP-corrected CCSD(T) energies, converted from
atomic units (hartree) to kcal mol$^{-1}$ by multiplying by the
automatically-defined conversion factor {\tt psi\_hartree2kcalmol}, which is
defined in section \ref{sec:psirc}.  Notice the loop over {\tt R} to create
the table looks just like the loop over {\tt R} to run the different
computations, and the CP-corrected energies {\tt ecp[R]} are accessed the same
way they were stored.  The {\tt print} line at the end just specifies some
formatting for the printed table (first element is a floating point number 3
spaces wide with one digit after the decimal, and the second element is a
floating point number 10 spaces wide with 6 digits after the decimal); the
format strings are the same as in the C programming language.  For tables more
complicated than the simple one used here, Psithon has built-in support for
tables (see the next section).

The following section goes over Psithon in much more detail, but
hopefully this example already makes it clear that many complex tasks
can be done very easily in \PSIfour.

