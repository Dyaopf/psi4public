\section{A \PSIfour\ Tutorial} \label{tutorial}

\subsection{Basic Input File Structure} 

PSI4 reads input from a text file, which can be prepared in any standard
text editor.  The default input file name is \file{input.dat} and the
default output file name is \file{output.dat}.  So that you can give your
files meaningful names, these defaults can be changed by specifying
the input file name and output file name on the the command line.
The syntax is:

{\tt psi4 input-name output-name}

\subsection{Running a basic SCF calculation}
In our first example, we will consider a Hartree-Fock SCF computation
for the water molecule using a cc-pVDZ basis set.  We will specify the
geometry of our water molecule using a standard z-matrix.

\begin{Snippet}

# Any line starting with the # character is a comment line
# Here's a sample HF/cc-pVDZ H2O computation

molecule h2o {
  O 
  H 1 0.96
  H 1 0.96 2 104.5
}

set basis cc-pVDZ
energy('scf')
\end{Snippet}

For your convenience, this sample input can be found in {\tt
psi4/samples/tu1-h2o-energy/input.dat}.  You can run it if you wish.
Once \PSIfour\ is in your path (see the User Configuration section of the
installation manual), you can run this computation by typing
\begin{verbatim}
psi4 input.dat output.dat
\end{verbatim}
If everything goes well, the computation should complete and should report
a final restricted Hartree-Fock energy in a section like this:
\begin{verbatim}
  Energy converged.

  @RHF Final Energy:   -76.02665366589162
\end{verbatim}
By default, the energy should be converged to about 1.0E-8, so agreement
is only expected for about the first 8 digits after the decimal.  If the
computation does not complete, there is probably a problem with the
compilation or installation of the program (see the installation manual).

This very simple input is sufficient to run the requested information.
Notice that we didn't tell the program some otherwise useful information
like the charge on the molecule (0, it's neutral), the spin multiplicity
(0 for a closed-shell molecule with all electrons paired), or the reference
wavefunction to use (restricted Hartree-Fock, or RHF, is usually
appropriate for a closed-shell molecule).  The program correctly guessed
all of these options for us.  We can change the default behavior through
additional keywords.  

Let's consider what we would do for an open-shell molecule, where
not all electrons are paired.  For example, let's run a computation
on methylene (CH$_2$), whose ground electronic state has two unpaired
electrons (triplet electronic state, or a spin multiplicity $2S+1 = 3$).
In this case, the default spin multiplicity (0) is not correct, so we
need to tell the program the true value (3).  Like many programs, \PSIfour\
can get the charge and multiplicity as the first two integers in the
Z-matrix.  Note the line with ``0 3'' at the beginning of the molecule
specification below.  In this example we will also specify the bond length
and bond angle as variables ({\tt R} and {\tt A}), whose values are given
after a blank line at the end of the Z-matrix specification.

\begin{Snippet}
# Here's a sample UHF/6-31G** CH2 computation

molecule ch2 {
  0 3
  C
  H 1 R
  H 1 R 2 A

  R = 1.075
  A = 133.93
}

set basis 6-31G**
set reference uhf

energy ('scf')
\end{Snippet}

This sample input can be found in {\tt psi4/samples/tu2-ch2-energy}, and as
before it can be run through the command {\tt psi4 input.dat output.dat}
(actually, because {\tt psi4} by default looks for an input file named {\tt
input.dat} and writes by default to a file called {\tt output.dat}, in this
case one could also just type {\tt psi4}).  If it works, it should print
the final energy as
\begin{verbatim}
  @UHF Final Energy:   -38.92534160932308
\end{verbatim}
Notice we added a new keyword, {\tt set reference uhf}, to the input.  For
open-shell molecules, we have a choice of unrestricted orbitals
(unrestricted Hartree-Fock, or UHF), or restricted orbitals (restricted
open-shell Hartree-Fock, or ROHF).  Usually, UHF is a little easier to
converge (although it may be more susceptible to spin contamination than
ROHF).

\subsection{Geometry Optimization and Vibrational Frequency Analysis}
The above example was a simple single-point energy computation.
To perform a different type of computation, change the keyword {\tt
jobtype}.  In the example below, we will set up
a CCSD geometry optimization.  To illustrate a more flexible z-matrix
input, we will now define variables for the bond length and bond angle
(in the {\tt zvars} section).

\begin{verbatim}
% 6-31G** H2O Test optimization calculation

psi: (
  label = "6-31G** SCF H2O"
  jobtype = opt
  wfn = ccsd
  reference = rhf
  dertype = first
  basis = "6-31G**"
  zmat = (
    o
    h 1 roh
    h 1 roh 2 ahoh
  )
  zvars = (
    roh     0.96031231
    ahoh  104.09437511
  )
)
\end{verbatim}

Once you have optimized the geometry of a molecule, you might wish to
perform a frequency analysis to determine the nature of the stationary
point.  To do this, change the value of {\tt jobtype} to {\tt freq}.
For an SCF frequeny calculation, you would also set {\tt dertype =
second} to compute the second derivatives analytically.  Unfortunately,
analytical second derivitives are not available in \PSIfour\ for
wavefunctions beyond SCF, so instead use the highest order analytical
derivitives that are available for the type of wavefunction you
have chosen.  This information is given in Table \ref{table:methods}.
For our CCSD example, the highest-order derivitives available are first,
so {\tt dertype = first}.

\begin{verbatim}
% 6-31G** H2O Test computation of frequencies

psi: (
  label = "6-31G** SCF H2O"
  jobtype = freq
  wfn = ccsd
  reference = rhf
  dertype = first
  basis = "6-31G**"
  zmat = (
    o
    h 1 roh
    h 1 roh 2 ahoh
  )
  zvars = (
    roh     0.96031231
    ahoh  104.09437511
  )
)
\end{verbatim}

\subsection{More Advanced Input Options}
If you wish to add comments to your input file, you can start any line
with \% and the line will be a comment line.  This can make the input
file easier to understand because you can provide explainations about
each keyword.  Another way to make the input file more organized is
to seperate it into sections that correspond to particular modules
the calculation will use.  This can be particularly helpful for more
complicated computations which can utilize many of keywords.  In the example
below, a CCSD(T) computation for the BH molecule is performed using a
cc-pVDZ basis set.  The keywords are divided into sections and several
new keywords are introduced, including ones to specify symmetry and
orbital occupations.  Orbitial occupations are specified by
a list of integers enclosed in parentheses.  These integers give the
number of orbitials which belong to each irreducible representation in
the point group.  The ordering of the irreps are those given by Cotton
in {\em Chemical Applications of Group Theory}.  In this example,
comment lines will be included to explain the new keywords used.

\begin{verbatim}
psi: (
  wfn = ccsd_t
  reference = rhf
)

default: (
  label = "BH cc-pVDZ CCSD(T)"

% Allocating memory for the calculation
  memory = (600.0 MB)

% charge and multiplicity (2S+1) default to values of 0 and 1, respectively
  charge = 0
  multp = 1

% The program will generally guess the symmetry of the molecule, but
% it can be overridden.  Here we specify C2V because only D2H and its
% subgroups can be used by the program.
  symmetry = c2v

% Number of doubly-occupied orbitals per irrep can be specified manually
% if desired
  docc = (3 0 0 0)

% Freeze the 1A1 orbital (Boron 1s-like) in the CCSD(T) computation
  frozen_docc = (1 0 0 0)
)

% The input section contains information about the molecule and the basis
% set.  The geometry here is specified by cartesian coordinates.
input: (
  basis = "cc-pVDZ"
  units = angstroms
  geometry = (
    ( b      0.0000        0.0000        0.0000)
    ( h      0.0000        0.0000        0.8000)
      )
  origin = (0.0 0.0 0.0)
)
% The modular input structure lets you specify convergence criteria for
% each part of the computation separately
scf: (
  maxiter = 100
  convergence = 11
)
\end{verbatim}

The final example of this tutorial demonstrates an example of a
complete-active-space self-consistent-field (CASSCF)
computation.  CAS computations require specification of several additional
keywords because you must specify which orbitals you wish to be in the
active space.  The notation and ordering for specifying CAS orbitals is the
same as for occupied orbitals.

\begin{verbatim}

% 6-31G** H2O Test CASSCF Energy Point

psi: (
  label = "6-31G** CASSCF H2O"
  jobtype = sp
  wfn = casscf
  reference = rhf
% The restricted_docc orbitals are those which are optimized, but are not
% in the active space.
  restricted_docc = (1 0 0 0)

% The active space orbitals; here, the valence orbitals are chosen
  active          = (3 0 1 2)

  basis = "6-31G**"
  zmat = (
    o
    h 1 1.00
    h 1 1.00 2 103.1
  )
)
\end{verbatim}

