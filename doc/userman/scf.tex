\subsection{Self-Consistent-Field} \label{scf}

\subsubsection{Introduction}

Self-Consistent-Field (SCF) theory forms the cornerstone of ab initio quantum
chemistry. SCF is specialized to either conventional Hartree-Fock (HF) molecular
orbital theory or the alternative generalized Kohn-Sham Density Functional
Theory (KS-DFT). 

In Hartree-Fock, the goal is to produce an optimized set of occupied molecular
orbitals which describe the action of the electrons according to a one-particle
Hamiltonian encapsulating electron-nuclear attraction, electronic kinetic
energy, and mean-field electron-electron repulsion, all under the requirement of
wavefunction antisymmetry. Molecular properties obtained by Hartree-Fock theory
are generally at least qualitatively correct, although they can be
quantitatively poor in many instances. If, as is usually the case, higher
accuracy is desired, the occupied and virtual molecular orbitals are used as a
starting point for the myriad of post-Hartree-Fock methods, all of which attempt
to recover the exact correlations due to electron-electron repulsion.  

DFT is an alternative approach which attempts to directly compute all
observables as functionals of the electronic density. The method is tractable
due to the two Hohenberg-Kohn Theorems. The first Theorem states that two
electronic systems with the same ground state density are both bound by the same
potential, to within a constant offset. This implies both that the ground-state
density is sufficient to describe all ground state observables of the system and
that the functional corresponding to electronic kinetic energy and
electron-electron repulsion is universal. The second Theorem states that the
exact energy functional is variational, obtaining its minimal value when the
input density corresponds to the ground state density. This allows for a
self-consistent computational procedure in which the density is varied until the
energy functional is minimized. In practice, the exact energy functional is
unknown (and would be as complex as Full Configuration Interaction if it were
known), however many good approximations exists. A major problem with
density-based DFT is that approximations to the kinetic energy functional are
notoriously poor. This is usually overcome by the Kohn-Sham approach, which
obtains most of the kinetic energy from a set of non-interacting orbitals
constructed from the density. Within this approach, the kinetic energy,
nuclear attraction energy, and mean-field Coulomb repulsion are treated in the
same way as in Hartree-Fock, and the remaining exchange, correlation, and
residual kinetic energy terms are treated via an approximate functional,
typically built from some semi-local model of the exchange and Coulomb holes. In
the generalized Kohn-Sham approach, the orbitals used to construct the
non-interacting kinetic energy are also used to produce an orbital-dependent
contribution to the exchange energy, either by adding a static fraction of the
non-interacting exact exchange (the adiabatic connection or ``hybrid''
approach), or by partitioning the exchange hole into short range components to
treated by the local exchange functional, and long range components to be
treated by the long-range noninteracting exact exchange (the range-separated
approach).

\subsubsection{Spin/Electronic Configuration Treatment}

\PSIfour solves the SCF equations in a basis of Gaussian functions using an
iterative procedure. These equations attempt to determine the occupied molecular
orbitals, and thereby ground state density, which minimize the Hartree-Fock or
KS-DFT energy within the user-specified spin specialization and electronic
configuration. Within Hartree-Fock, the allowed spin-specializations are
Restricted-Hartree-Fock (RHF), Unrestricted-Hartree-Fock (UHF),
Restricted-Open-Shell-Hartree-Fock (ROHF), and
Constrained-Unrestricted-Hartree-Fock (CUHF). Within KS-DFT, the allowed
spin-specializations are Restricted-Kohn-Sham (RKS) and Unrestricted-Kohn-Sham
(UKS). Restricted-Open-Shell-Kohn-Sham (ROKS) and
Constrained-Unrestricted-Kohn-Sham (CUKS) are not implemented, as they always
predict a ground state with positive definite spin density, which is not
physical. Spin specialization is specified by the \texttt{REFERENCE} option,
which defaults to \texttt{RHF}.  Electronic configuration is specified by the
charge/multiplicity field in the molecular geometry input. 
If spatial
symmetry is used, the assignment of occupied orbitals to irreps may be specified
by the \texttt{DOCC} and \texttt{SOCC} integer arrays. If these arrays are not
specified by the user, the program will assign occupation to irreps according to
the lowest orbital energies at each SCF iteration.  

\PSIfour implements Hartree-Fock and KS-DFT within the \texttt{SCF} module.
Minimal input for a restricted closed-shell Hartree-Fock solution involves
specifying a molecular geometry field, a Gaussian basis set via the
option\texttt{BASIS}, and calling the \texttt{energy} procedure. An example is:
\begin{Snippet}
molecule {
0 1              # Charge/Multiplicity line (optional)
O                # Required molecular geometry (H2O in Z-matrix)
H 1 1.0
H 1 1.0 2 104.5
}

set scf {        
basis sto-3g     # Required basis specification
reference rhf    # (optional)
}

energy(`scf')    # Calls the energy procedure with a specific type of SCF
\end{Snippet}

\subsubsection{Orthogonalization}

One of the first steps in the SCF procedure is the determination of an
orthogonal basis (known as the OSO basis) from the atomic orbital basis (known
as the AO basis). The Molecular Orbital basis (MO basis) is then built as a
particular unitary transformation of the OSO basis. In \PSIfour, the
determination of the OSO basis is accomplished via either symmetric or
canonical orthogonalization. Symmetric orthogonalization uses the symmetric
inverse square root of the overlap matrix for the orthogonalization matrix. Use
of symmetric orthogonalization always yields the same number of OSO functions
(and thereby MOs) as AO functions. However, this may lead to numerical problems
if the overlap matrix has small eigenvalues, which may occur for large systems
or for systems where diffuse basis sets are used. This problem may be avoided by
using canonical orthogonalization, in which an asymmetric inverse square root of
the overlap matrix is formed, with numerical stability enhanced by the
elimination of eigenvectors corresponding to very small eigenvalues. As a few
combinations of AO basis functions may be discarded, the number of
canonical-orthogonalized OSOs and MOs may be slightly smaller than the number of
AOs. In \PSIfour, symmetric orthogonalization is used by default, unless the
smallest overlap eigenvalue falls below the user-supplied double option
\texttt{S\_MIN\_EIGENVALUE}, which defaults to $10^{-7}$. If the smallest
eigenvalue is below this cutoff, canonical orthogonalization is forced, and all
eigenvectors corresponding to eigenvalues below the cutoff are eliminated.
Use of canonical orthogonalization can be forced by setting the
\texttt{S\_ORTHOGONALIZATION} option to \texttt{CANONICAL}. Note that in
practice, the MOs and OSOs are built separately within each irrep from the
symmetry-adapted combinations of AOs known as Unique Symmetry Orbitals (USOs).
For canonical orthogonalization, this implies that the number of MOs and OSOs
per irrep may be slightly smaller than the number of USOs per irrep.

A contrived example demonstrating OSOs/MOs vs. AOs with symmetry is shown below:
\begin{Snippet}
Input:

molecule h2o {
0 1
O
H 1 1.0
H 1 1.0 2 104.5
symmetry c2 # Two irreps is easier to comprehend
}

set {
s_min_eigenvalue 0.0001 # Set an unreasonably tight 
                        # tolerance to force canonical
basis aug-cc-pv5z       # This diffuse basis will have 
                        # small-ish eigenvalues for even H2O
}

energy(`scf')

Output:

  ... Initialization ...

  ==> Pre-Iterations <==

  Minimum eigenvalue in the overlap matrix is 1.6888059293E-05.
  Using Canonical Orthogonalization with cutoff of 1.0000000000E-04.
  Overall, 3 of 287 possible MOs eliminated. 

  ... Initial Orbital Guess Information ...

   -------------------------------------------------------
    Irrep   Nso     Nmo     Nalpha   Nbeta   Ndocc  Nsocc
   -------------------------------------------------------
     A        145     144       3       3       3       0
     B        142     140       2       2       2       0
   -------------------------------------------------------
    Total     287     284       5       5       5       0
   -------------------------------------------------------

\end{Snippet}   
In this example, there are 287 AO basis functions after spherical harmonics are
applied. These are used to produce 287 symmetry adapted USOs, 145 of which are
assigned to irrep A, and 142 of which are assigned to irrep B. Within irrep A,
144 OSOs fall above the eigenvalue cutoff, and within irrep B 140 OSOs fall
above the eigenvalue cutoff. In total, 284 molecular orbitals are chosen from
287 AOs/USOs. The table also shows the initial assignment of electrons to
irreps.  

\subsubsection{Initial Guess/Convergence Stabilization}

In each step of the SCF procedure, a new Fock or Kohn-Sham potential is built
according to the previous density, following which the potential is diagonalized
to produce new molecular orbitals, from which a new density is computed. This
procedure is continued until either convergence is reached or a preset maximum
number of iterations is exceeded. Convergence is determined by both change in
energy and root-mean-square change in density matrix values, which must be below
the user-specified $\mathtt{E\_CONVERGE}$ and
$\mathtt{D\_CONVERGE}$, respectively. The maximum number of iterations is
specified by the \texttt{MAXITER} option. It should be noted that SCF is a
chaotic process, and, as such, often requires careful selection of initial
orbitals and damping during iterations to ensure convergence. This is
particularly likely for large systems, metallic systems, multireference systems,
open-shell systems, anions, and systems with diffuse basis sets. 

For initial orbital selection, several options are available. The default guess
is a core Hamiltonian which ignores even mean-field electronic repulsion. A much
more accurate guess is the Superposition-of-Atomic-Densities (SAD) guess, which
builds the initial molecular density as a sum of atomic UHF densities. This
guess is currently only available in RHF and RKS spin specializations. This
guess also produces fractionally occupied closed-shell orbitals, as required by
the density-fitted exchange routines (discussed below). For open-shell systems,
the Generalized Wolfsberg-Helmholtz (GWH) guess is available, and may be
superior to a core guess. SAD and GWH may be invoked by the \texttt{GUESS}
option. A more involved alternative to these approaches is to converge the SCF
in a small basis and then project the resultant occupied molecular orbitals up
into a larger basis. This can be done automatically by placing a
\texttt{cast\_up = `SMALL\_BASIS\_NAME'} modifier in the \texttt{energy}
procedure call. We recommend the 3-21G basis for the small basis due to its
efficient mix of flexibility and compactness. An example of performing an RHF
solution of water by SAD guessing in a 3-21G basis and then casting up to
cc-pvtz is shown below: 
\begin{Snippet}
molecule h2o {
0 1
O
H 1 1.0
H 1 1.0 2 104.5
}

set {
basis cc-pvtz 
guess sad
}

energy(`scf', cast_up = `3-21G')
\end{Snippet}

With regard to convergence stabilization, Pulay's
Direct Inversion of the Iterative Subspace (DIIS) extrapolation and Gill's
Maximum Overlap Method (MOM) are both implemented. DIIS uses previous iterates
of the Fock Matrix together with an error criterion based on the orbital
gradient to produce an informed estimate of the next Fock Matrix. DIIS is almost
always necessary to converge the SCF procedure, and is therefore turned on by
default. In rare cases, the DIIS algorithm may need to be modified or turned off
altogether, which may be accomplished via the options detailed below. MOM was
developed to combat a particular class of convergence failure: occupation
flipping. In some cases, midway though the SCF procedure, a partially converged
orbital which should be occupied in the fully-optimized SCF solution has a
slightly higher orbital eigenvalue than some other orbital which should be
destined to be a virtual orbital. This results in the virtual orbital being
spuriously occupied for one or more iterations. Sometimes this resolves itself
without help, other times the occupation flips back and forth between two, four,
or more orbitals. This is typically visible in the output as a non-converging
SCF which eventually settles down to steady oscillation between two (or more)
different total energies. This behavior can be ameliorated by choosing occupied
orbitals by ``shape'' instead of by orbital eigenvalue, i.e., by choosing the
set of new orbitals which looks most like some previously known ``good'' set.
The ``good'' set is typically the occupied orbitals from an one of the
oscillating iterations with the lowest total energy. For an oscillating system
where the lowest total energy occurs on iterations $N,N+2,\ldots$, invoking
\texttt{MOM\_START N} can often rescue the convergence of the SCF. MOM can be
used in concert with DIIS, though care should be taken to not turn MOM on until
the oscillatory behavior begins. 

%
%It is important to point out that the
%SCF approach does not rigorously guarantee that the final orbitals
%actually correspond to a minimum in orbital space; at convergence,
%the only guarantee is that the gradient of the energy with respect 
%to orbital rotations is zero: this could be a global minimum, a local
%minimum, or a saddle point in orbital rotation space.  While this
%is not usually an issue (typically the lowest minimum consistent with 
%the electron configuration is found), it can be a problem sometimes for 
%radicals, diradicals, bond breaking, or unusual bonding situations.  The
%\PSIstable\ module can be used to test for the stability of Hartree-Fock
%wave functions (note: this module from \PSIthree\ has not yet been adapted
%to work in \PSIfour).

The most commonly used keywords are found below.  More specialized keywords
are described in the Appendix.

\noindent
\begin{tabular*}{\textwidth}[tb]{p{0.3\textwidth}p{0.7\textwidth}}
         SCF\_TYPE & What algorithm to use for the SCF computation \\
          & {\bf Possible Values:} PK, OUT\_OF\_CORE, DIRECT, DF,
PSEUDOSPECTRAL, POISSON, L\_DF, CD, 1C\_CD \\
\end{tabular*}
\begin{tabular*}{\textwidth}[tb]{p{0.3\textwidth}p{0.35\textwidth}p{0.35\textwidth}}
           & {\bf Type:} string &  {\bf Default:} PK\\
         & & \\
\end{tabular*}
\begin{tabular*}{\textwidth}[tb]{p{0.3\textwidth}p{0.7\textwidth}}
         MAXITER & The maximum number of iterations \\
\end{tabular*}
\begin{tabular*}{\textwidth}[tb]{p{0.3\textwidth}p{0.35\textwidth}p{0.35\textwidth}}
           & {\bf Type:} integer &  {\bf Default:} 100\\
         & & \\
\end{tabular*}
\begin{tabular*}{\textwidth}[tb]{p{0.3\textwidth}p{0.7\textwidth}}
         D\_CONVERGE & The density convergence criterion \\
\end{tabular*}
\begin{tabular*}{\textwidth}[tb]{p{0.3\textwidth}p{0.35\textwidth}p{0.35\textwidth}}
           & {\bf Type:} integer &  {\bf Default:} 8\\
         & & \\
\end{tabular*}
\begin{tabular*}{\textwidth}[tb]{p{0.3\textwidth}p{0.7\textwidth}}
         REFERENCE & The reference wavefunction used in the computation \\

          & {\bf Possible Values:} RHF, ROHF, UHF, CUHF, RKS, UKS \\
\end{tabular*}
\begin{tabular*}{\textwidth}[tb]{p{0.3\textwidth}p{0.35\textwidth}p{0.35\textwidth}}
           & {\bf Type:} string &  {\bf Default:} RHF\\
         & & \\
\end{tabular*}
\begin{tabular*}{\textwidth}[tb]{p{0.3\textwidth}p{0.7\textwidth}}
         GUESS & The type of guess orbitals \\

          & {\bf Possible Values:} CORE, GWH, SAD, READ \\
\end{tabular*}
\begin{tabular*}{\textwidth}[tb]{p{0.3\textwidth}p{0.35\textwidth}p{0.35\textwidth}}
           & {\bf Type:} string &  {\bf Default:} CORE\\
         & & \\
\end{tabular*}


