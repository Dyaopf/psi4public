\subsection{Self-Consistent Field (SCF)} \label{scf}

\subsubsection{Introduction}

Self-Consistent-Field (SCF) theory forms the cornerstone of \textit{ab initio} quantum
chemistry. Here SCF refers both to conventional Hartree--Fock (HF) molecular
orbital theory and also to generalized Kohn--Sham Density Functional
Theory (KS-DFT). \PSIfour contains a mature code for several common HF
references and a pilot (very shortly to be optimized) implementation of most
flavors of KS-DFT.

An illustrative example of using the SCF module is as follows:
\begin{Snippet}
molecule {
0 3
O
O 1 1.21
}

set {
basis cc-pvdz
guess sad
reference uhf
scf_type pk
}

energy('scf')
\end{Snippet}

This will run a UHF computation for triplet molecular oxygen (the ground state)
using a PK algorithm for the Electron Repulsion Integrals (ERI) and starting
from a Superposition of Atomic Densities (SAD) guess. After printing all manner
of titles, geometries, sizings, and algorithm choices, the SCF finally reaches
the iterations:
\begin{Snippet}
                        Total Energy        Delta E     Density RMS

   @UHF iter   0:  -149.76856421865352   -4.69109e+01   0.00000e+00
   @UHF iter   1:  -149.59793338958522    1.70631e-01   5.72371e-02
   @UHF iter   2:  -149.62408782458331   -2.61544e-02   8.04195e-03 DIIS
   @UHF iter   3:  -149.62679515182390   -2.70733e-03   2.51542e-03 DIIS
   @UHF iter   4:  -149.62726459105770   -4.69439e-04   1.06897e-03 DIIS
   @UHF iter   5:  -149.62730549814114   -4.09071e-05   2.70311e-04 DIIS
   @UHF iter   6:  -149.62730736371790   -1.86558e-06   5.94924e-05 DIIS
   @UHF iter   7:  -149.62730740227752   -3.85596e-08   9.93250e-06 DIIS
   @UHF iter   8:  -149.62730740325136   -9.73841e-10   1.88088e-06 DIIS
   @UHF iter   9:  -149.62730740326214   -1.07718e-11   1.80706e-07 DIIS
   @UHF iter  10:  -149.62730740326231   -1.70530e-13   2.19128e-08 DIIS
\end{Snippet}
The algorithm takes 10 true iterations to converge the energy and density to the
default of 1.0\e{-8}, plus the trivial iteration due to the SAD guess.
The energy on the zero-th iteration is not variational due to the improper
idempotence properties of the SAD guess, but the first true iteration is within
2.0\e{-4} relative error of the final answer, highlighting the
efficiency of the SAD guess. The energy and density then converge smoothly,
assisted by Pulay's Direct Inversion of the Iterative Subspace (DIIS), which is
activated by default. DIIS from a high-quality guess is usually sufficient to
converge the nonlinear SCF equations, however enhanced control of DIIS
parameters and additional convergence algorithms are available and detailed
below. 

After the iterations are completed, a number of one-electron properties are
printed, and some bookkeeping is performed to set up possible correlated
computations. Additional one-electron properties are available by increasing the
\optionname{PRINT} option. Also printed are the occupied and virtual orbital
energies, which are useful in elucidating the stability and reactivity of the
system. Finally, for UHF/UKS references, the $S^2$ diagnostic is printed, as the
UHF/UKS ansatz does not constrain the wavefunction to be an eigenfunction of the
$\hat S^2$ operator. In this case, the output shows that the current UHF
wavefunction actually has an expectation value of $\langle S^2\rangle = 2.03319$, which does
not equal the expected eigenvalue of $2$.  If this metric rises significantly
above a few hundredths, it may be necessary to switch to an ROHF reference,
which is guaranteed to return a spin eigenfunction. 
  
\subsubsection{Background}

In Hartree--Fock, the goal is to produce an optimized set of occupied molecular
orbitals which describe the action of the electrons according to a one-particle
Hamiltonian encapsulating electron-nuclear attraction, electronic kinetic
energy, and mean-field electron-electron repulsion, all under the requirement of
wavefunction antisymmetry. Molecular properties obtained by Hartree--Fock theory
are generally at least qualitatively correct, although they can be
quantitatively poor in many instances. If, as is usually the case, higher
accuracy is desired, the occupied and virtual molecular orbitals are used as a
starting point for the myriad of post-Hartree--Fock methods, all of which attempt
to recover the exact correlations due to electron-electron repulsion.  

DFT is an alternative approach which attempts to directly compute all
observables as functionals of the electronic density. The method is tractable
due to the two Hohenberg-Kohn Theorems. The first Theorem states that two
electronic systems with the same ground state density are both bound by the same
potential, to within a constant offset. This implies both that the ground-state
density is sufficient to describe all ground state observables of the system and
that the functional corresponding to electronic kinetic energy and
electron-electron repulsion is universal. The second Theorem states that the
exact energy functional is variational, obtaining its minimal value when the
input density corresponds to the ground state density. This allows for a
self-consistent computational procedure in which the density is varied until the
energy functional is minimized. In practice, the exact energy functional is
unknown (and would be as complex as Full Configuration Interaction if it were
known), however many good approximations exist. A major problem with
density-based DFT is that approximations to the kinetic energy functional are
notoriously poor. This is usually overcome by the Kohn--Sham approach, which
obtains most of the kinetic energy from a set of non-interacting orbitals
constructed from the density. Within this approach, the kinetic energy,
nuclear attraction energy, and mean-field Coulomb repulsion are treated in the
same way as in Hartree--Fock, and the remaining exchange, correlation, and
residual kinetic energy terms are treated via an approximate functional,
typically built from some semi-local model of the exchange and Coulomb holes. In
the generalized Kohn--Sham approach, the orbitals used to construct the
non-interacting kinetic energy are also used to produce an orbital-dependent
contribution to the exchange energy, either by adding a static fraction of the
non-interacting exact exchange (the adiabatic connection or ``hybrid''
approach), or by partitioning the exchange hole into short range components be to
treated by the local exchange functional, and long range components to be
treated by the long-range noninteracting exact exchange (the range-separated
approach).

\subsubsection{Spin/Electronic Configuration Treatment}

\PSIfour solves the SCF equations in a basis of Gaussian functions using an
iterative procedure. These equations attempt to determine the occupied molecular
orbitals, and thereby ground state density, which minimize the Hartree--Fock or
KS-DFT energy within the user-specified spin specialization and electronic
configuration. Within Hartree--Fock, the allowed spin-specializations are
Restricted-Hartree--Fock (RHF), Unrestricted-Hartree--Fock (UHF),
Restricted-Open-Shell-Hartree--Fock (ROHF), and
Constrained-Unrestricted-Hartree--Fock (CUHF). Within KS-DFT, the allowed
spin-specializations are Restricted-Kohn--Sham (RKS) and Unrestricted-Kohn--Sham
(UKS). Restricted-Open-Shell-Kohn--Sham (ROKS) and
Constrained-Unrestricted-Kohn--Sham (CUKS) are not implemented, as they always
predict a ground state with positive definite spin density, which is not
physical. Spin specialization is specified by the \optionname{REFERENCE} option,
which defaults to \optionval{RHF}.  Electronic configuration is specified by the
charge/multiplicity field in the molecular geometry input.  If spatial symmetry
is used, the assignment of occupied orbitals to irreps may be specified by the
\optionname{DOCC} and \optionname{SOCC} integer arrays. If these arrays are not
specified by the user, the program will assign occupation to irreps according to
the lowest orbital energies at each SCF iteration (Aufbau ordering).  

\subsubsection{Orthogonalization}

One of the first steps in the SCF procedure is the determination of an
orthogonal basis (known as the OSO basis) from the atomic orbital basis (known
as the AO basis). The Molecular Orbital basis (MO basis) is then built as a
particular unitary transformation of the OSO basis. In \PSIfour, the
determination of the OSO basis is accomplished via either symmetric or
canonical orthogonalization. Symmetric orthogonalization uses the symmetric
inverse square root of the overlap matrix for the orthogonalization matrix. Use
of symmetric orthogonalization always yields the same number of OSO functions
(and thereby MOs) as AO functions. However, this may lead to numerical problems
if the overlap matrix has small eigenvalues, which may occur for large systems
or for systems where diffuse basis sets are used. This problem may be avoided by
using canonical orthogonalization, in which an asymmetric inverse square root of
the overlap matrix is formed, with numerical stability enhanced by the
elimination of eigenvectors corresponding to very small eigenvalues. As a few
combinations of AO basis functions may be discarded, the number of
canonical-orthogonalized OSOs and MOs may be slightly smaller than the number of
AOs. In \PSIfour, symmetric orthogonalization is used by default, unless the
smallest overlap eigenvalue falls below the user-supplied double option
\optionname{S\_TOLERANCE}, which defaults to 1.0\e{-7}. If the smallest
eigenvalue is below this cutoff, canonical orthogonalization is forced, and all
eigenvectors corresponding to eigenvalues below the cutoff are eliminated.
Use of canonical orthogonalization can be forced by setting the
\optionname{S\_ORTHOGONALIZATION} option to \optionval{CANONICAL}. Note that in
practice, the MOs and OSOs are built separately within each irrep from the
symmetry-adapted combinations of AOs known as Unique Symmetry Orbitals (USOs).
For canonical orthogonalization, this implies that the number of MOs and OSOs
per irrep may be slightly smaller than the number of USOs per irrep.

A contrived example demonstrating OSOs/MOs vs. AOs with symmetry is shown below:
\begin{Snippet}
Input:

molecule h2o {
0 1
O
H 1 1.0
H 1 1.0 2 104.5
symmetry c2 # Two irreps is easier to comprehend
}

set {
s_tolerance 0.0001      # Set an unreasonably tight 
                        # tolerance to force canonical
basis aug-cc-pv5z       # This diffuse basis will have 
                        # small-ish eigenvalues for even H2O
}

energy('scf')

Output:

  ... Initialization ...

  ==> Pre-Iterations <==

  Minimum eigenvalue in the overlap matrix is 1.6888059293E-05.
  Using Canonical Orthogonalization with cutoff of 1.0000000000E-04.
  Overall, 3 of 287 possible MOs eliminated. 

  ... Initial Orbital Guess Information ...

   -------------------------------------------------------
    Irrep   Nso     Nmo     Nalpha   Nbeta   Ndocc  Nsocc
   -------------------------------------------------------
     A        145     144       3       3       3       0
     B        142     140       2       2       2       0
   -------------------------------------------------------
    Total     287     284       5       5       5       0
   -------------------------------------------------------

\end{Snippet}   
In this example, there are 287 AO basis functions after spherical harmonics are
applied. These are used to produce 287 symmetry adapted USOs, 145 of which are
assigned to irrep A, and 142 of which are assigned to irrep B. Within irrep A,
144 OSOs fall above the eigenvalue cutoff, and within irrep B 140 OSOs fall
above the eigenvalue cutoff. In total, 284 molecular orbitals are chosen from
287 AOs/USOs. The table also shows the initial assignment of electrons to
irreps.  

\subsubsection{Initial Guess/Convergence Stabilization}

In each step of the SCF procedure, a new Fock or Kohn--Sham potential is built
according to the previous density, following which the potential is diagonalized
to produce new molecular orbitals, from which a new density is computed. This
procedure is continued until either convergence is reached or a preset maximum
number of iterations is exceeded. Convergence is determined by both change in
energy and root-mean-square change in density matrix values, which must be below
the user-specified $\optionname{E\_CONVERGENCE}$ and
$\optionname{D\_CONVERGENCE}$, respectively. The maximum number of iterations is
specified by the \optionname{MAXITER} option. It should be noted that SCF is a
chaotic process, and, as such, often requires careful selection of initial
orbitals and damping during iterations to ensure convergence. This is
particularly likely for large systems, metallic systems, multireference systems,
open-shell systems, anions, and systems with diffuse basis sets. 

For initial orbital selection, several options are available. The default guess
is a core Hamiltonian which ignores even mean-field electronic repulsion. A much
more accurate guess is the Superposition-of-Atomic-Densities (SAD) guess, which
builds the initial molecular density as a sum of atomic UHF densities. As
discussed above, the ``zero-th'' SCF iteration involving the SAD density need
not be variational, and the resultant energy is usually garbage. However, the
density obtained in the zero-th iteration is usually excellent, and the
resultant first-iteration energy is generally less than a fraction of a percent
off of the converged value. The SAD guess also produces fractionally occupied
closed-shell orbitals, as required by the density-fitted exchange routines
(discussed below). For open-shell systems, the Generalized Wolfsberg-Helmholtz
(GWH) guess is available, and may be superior to a core guess. \optionval{SAD} and \optionval{GWH} may
be invoked by the \optionname{GUESS} option. A more involved alternative to these
approaches is to converge the SCF in a small basis and then project the
resultant occupied molecular orbitals up into a larger basis. This can be done
automatically by placing a \pyoptionname{cast\_up =} \pyoptionval{\qq{\textit{SMALL\_BASIS\_NAME}}} modifier in
the \texttt{energy()} procedure call. We recommend the 3-21G basis for the small
basis due to its efficient mix of flexibility and compactness. An example of
performing an RHF solution of water by SAD guessing in a 3-21G basis and then
casting up to cc-pVTZ is shown below: 
\begin{Snippet}
molecule h2o {
0 1
O
H 1 1.0
H 1 1.0 2 104.5
}

set {
basis cc-pvtz 
guess sad
}

energy('scf', cast_up = '3-21G')
\end{Snippet}

With regard to convergence stabilization, Pulay's Direct Inversion of the
Iterative Subspace (DIIS) extrapolation and Gill's Maximum Overlap Method (MOM)
are both implemented. DIIS uses previous iterates of the Fock Matrix together
with an error criterion based on the orbital gradient to produce an informed
estimate of the next Fock Matrix. DIIS is almost always necessary to converge
the SCF procedure and is therefore turned on by default. In rare cases, the
DIIS algorithm may need to be modified or turned off altogether, which may be
accomplished via the options detailed below. MOM was developed to combat a
particular class of convergence failure: occupation flipping. In some cases,
midway though the SCF procedure, a partially converged orbital which should be
occupied in the fully-optimized SCF solution has a slightly higher orbital
eigenvalue than some other orbital which should be destined to be a virtual
orbital. This results in the virtual orbital being spuriously occupied for one
or more iterations. Sometimes this resolves itself without help, other times the
occupation flips back and forth between two, four, or more orbitals. This is
typically visible in the output as a non-converging SCF which eventually settles
down to steady oscillation between two (or more) different total energies. This
behavior can be ameliorated by choosing occupied orbitals by ``shape'' instead
of by orbital eigenvalue, i.e., by choosing the set of new orbitals which looks
most like some previously known ``good'' set.  The ``good'' set is typically the
occupied orbitals from an one of the oscillating iterations with the lowest
total energy. For an oscillating system where the lowest total energy occurs on
iterations $N,N+2,\ldots$, invoking \optionname{MOM\_START} \optionval{N} can often rescue the
convergence of the SCF. MOM can be used in concert with DIIS, though care should
be taken to not turn MOM on until the oscillatory behavior begins. In some
cases, a static mixing of Fock Matrices from adjacent iterations can quench
oscillations. This mixing, known as ``damping'' can be activated by setting the
\optionname{DAMPING\_PERCENTAGE} keyword to a nonzero percent. 

\subsubsection{SCF ERI Algorithms}

The key difficulty in the SCF procedure is treatment of the four-index ERI
contributions to the Fock Matrix. A number of algorithms are available in
\PSIfour for these terms. The algorithm is selected by the \optionname{SCF\_TYPE}
keyword, which defaults to \optionval{PK}. 

\begin{description}
\item[\optionval{PK}:] An in-core, presorted algorithm using exact ERIs. Quite fast
for a zero-error algorithm if enough memory is available. Integrals are
generated only once, symmetry is utilized to reduce number of integrals. 
\item[\optionval{OUT\_OF\_CORE}:] An out-of-core, unsorted algorithm using exact
ERIs. Overcomes the memory bottleneck of the current PK algorithm. Integrals are
generated only once, symmetry is utilized to reduce number of integrals. 
\item[\optionval{DIRECT}:] An in-core repeated integral evaluation algorithm using
exact ERIs. Symmetry is used to reduce the number of integrals, and no disk is
used. However, integral regeneration is quite costly, implying that this
algorithm should be used only if there is not enough disk space for the
\optionval{OUT\_OF\_CORE} algorithm. 
\item[\optionval{DF}:] A density-fitted algorithm designed for computations with
thousands of basis functions. This algorithm is highly optimized, and is
threaded with a mixture of parallel BLAS and OpenMP. Note that this algorithm
should use the -JKFIT series of auxiliary bases, \emph{not} the -RI or -MP2FIT
bases. The default guess for auxiliary basis set should work for all Dunning
bases, otherwise the \optionname{DF\_BASIS\_SCF} keyword can be used to manually
specify the auxiliary basis. This algorithm is preferred unless either
absolute accuracy is required [$\gtrsim$CCSD(T)] or a -JKFIT auxiliary basis is
unavailable for the primary basis/atoms involved. 
\end{description}

For some of these algorithms, Schwarz and/or density sieving can be used to
identify negligible integral contributions in extended systems. To activate
sieving, set the \optionname{INTS\_TOLERANCE} keyword to your desired cutoff
(1.0\e{-12} is recommended for most applications).

\subsubsection{KS-DFT} 

KS-DFT is coming soon. A pilot code is already implemented in \PSIfour, but the
efficiency and correctness of this code is by no means guaranteed. We expect to
have a production-level KS-DFT code for energies and gradients by mid-March
2012. Contact Rob Parrish at CCMST (robparrish@gmail.com) with feature requests
for the DFT code. 

\subsubsection{Recomendations}

The SCF code is already quite flexible and powerful, with new features being
added weekly. We have tried as much as possible to keep the number of options
to a minimum, and to allow all options to be used in the presence of all other
options. Below are some rough words of advice about using the SCF code for
practical calculations:
\begin{itemize}
\item The \optionval{SAD} guess is usually your friend, even for open-shell systems (at the
very least, it gets the right number of electrons, unlike some other programs).
For instance, we have found that a simple SAD guess is often as good as doing a
full SCF in a 3-21G basis and then performing a cast-up, at a fraction of the
cost.  However, SAD and \optionname{DOCC}/\optionname{SOCC} arrays do not play very well
at the moment. 
\item Do not specify the \optionname{DOCC} or \optionname{SOCC} arrays unless you are
really sure you have these correct, or the system is behaving unusually. 
This also helps with the above point.
\item For wall time, \optionval{DF} beats \optionval{OUT\_OF\_CORE}, and 
\optionval{PK} beats
\optionval{DIRECT}. Use \optionval{DF} unless you need absolute accuracy or do not
have a -JKFIT auxiliary set for your primary basis/atom type. Then use
\optionval{OUT\_OF\_CORE} unless you run out of disk space.
\item For very large systems, you may need to relax \optionname{E\_CONVERGENCE} and
\optionname{D\_CONVERGENCE}, as the error is extensive in the size of the system.
1.0\e{-7} is usually sufficient.  
\item Don't mess with the convergence options unless convergence is a problem.
We have optimized the parameters for efficiency over a wide array of system
types.  
\end{itemize} 
