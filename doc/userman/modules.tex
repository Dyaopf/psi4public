\subsection{Overview of modules of \PSIthree}
Below is a very brief list of some of the modules available in the
\PSIthree\ program.  For more information, consult the individual
manual pages.
\subsubsection{input}
This program prepares the checkpoint file (\chkptfile).  Errors will
result if the \PSIpsi\ program is run before \PSIinput.  This program
reads the \keyword{default}, \keyword{psi}, \keyword{input}, and (if given)
\keyword{basis} sections of the input file and places molecule,
geometry, and basis information into the checkpoint file.  The
molecule may be reoriented to a standard reference frame, and the
molecular point-group symmetry is identified.  It is possible to tell
\PSIinput\ to re-use orbitals from a previous calculation ({\tt input
--chkptmos}), although presently this only works if the point group has
not changed.

\subsubsection{cints}
This program computes one- and two-electron integrals and derivative
integrals.  It is multithreaded and has some advanced capabilities
related to integrals-direct computations.

\subsubsection{cscf}
This program carries out the Hartree--Fock procedure.  It can handle
RHF, ROHF, UHF, and TCSCF.  It can also simply re-orthogonalize
previously existing MO's.

\subsubsection{transqt}
This module transforms one- and two-electron integrals from the
symmetry-adapted atomic orbital (SO) basis to the molecular orbital
(MO) basis for use in correlated computations by program such as
\PSIdetci\ or \PSIccenergy.  It can also back-transform one- and
two-particle density matrices to the AO basis for contraction with the
derivative integrals to obtain energy gradients.

\subsubsection{ccsort}
This module sorts the one- and two-electron integrals for use by the
\PSIthree\ coupled-cluster programs.

\subsubsection{ccenergy}
This module computes the CCSD energy.

\subsubsection{cctriples}
This module computes the (T) correction to CCSD to give the CCSD(T)
energy.

\subsubsection{detci}
This module performs many different types of CI computation, including
CI's truncated according to substitution level (e.g., CISD, CISDT,
CISDTQ, etc), full CI, and any CI which can be formulated as a
restricted active space (RAS) CI, including second-order configuration
interaction (SOCI) and other types of multi-reference CI's.  It can
compute one- and two-particle density matrices and can obtain CI
natural orbitals.

\subsubsection{detcas}
This program works together with the \PSIdetci\ program to do CASSCF
calculations in a two-step procedure.  It obtains the orbital gradient
and rotates the orbitals to optimize them.

\subsubsection{detcasman}
This is a driver program that controls \PSIdetci\ and \PSIdetcas\ to
perform CASSCF computations.

\subsubsection{clag}
This forms the CI lagrangian, which is needed for CASSCF and CI gradients.
