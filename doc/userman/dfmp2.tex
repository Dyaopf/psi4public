\subsection{Density-Fitted Second-Order M\o{}ller-Plesset Perturbation Theory (DFMP2)} \label{dfmp2}

\subsubsection{Introduction}

Second-Order M\o{}ller-Plesset Perturbation Theory (MP2) occupies a unique role
in quantum chemistry due to its small-prefactor ${\cal O}(N^5)$ treatment of
dynamic electron correlation. This unusually cheap
\textit{ab initio} treatment of electron correlation may be made even more
efficient by means of the Density-Fitting (DF) approximation (also known as
Resolution-of-the-Identity or RI), wherein the quadratic $ov$ products in the
bra- and ket- of the $(ov|ov)$-type Electron Repulsion Integrals (ERIs)
appearing in MP2 are cast onto a linear-scaling auxiliary basis by least-squares
fitting.  Substitution of the DF factorization into the MP2 equations results in
a formal scaling and prefactor reduction of MP2, and further speed gains are
possible due heavy utilization of matrix-multiplication kernels and minimal
storage requirements in a DF approach. The method has been found to be quite
robust and accurate, and should be preferred unless extreme accuracy is required
or a fitting basis is not defined for the primary basis and atom type
encountered. In particular, we have found excellent efficiency and tractability
gains when using DFMP2 in concert with a DFSCF reference.  An efficient,
threaded, disk-based DFMP2 code is available in \PSIfour for all single
reference types available in the SCF module.  

An example utilization of the code is:
\begin{Snippet}
molecule h2o {
0 1
O
H 1 1.0
H 1 1.0 2 104.5
}

set basis cc-pvdz
set scf_type df
set freeze_core True

energy('dfmp2')
\end{Snippet}

The \texttt{energy} call executes the predefined DFMP2 procedure, first calling
the SCF module with a default RHF reference and DF algorithm for the
two-electron integrals. When the orbitals are converged, the DFMP2 module is
launched, which forms the density-fitted $(Q|ov)$ integrals and then builds the
full $(ov|ov)$ tensor in blocks, evaluating the contributions to the MP2 energy
as it goes. An RHF-MP2 wavefunction is selected automatically due to the RHF
reference. In this example, we freeze the core, both for efficiency, and
because split-valence bases like cc-pVDZ do not contain core correlation
functions. The result looks something like:
\begin{verbatim}
        ----------------------------------------------------------
         ====================> MP2 Energies <====================
        ----------------------------------------------------------
         Reference Energy          =     -76.0213974789664633 [H]
         Singles Energy            =      -0.0000000000000001 [H]
         Same-Spin Energy          =      -0.0512503261762665 [H]
         Opposite-Spin Energy      =      -0.1534098129352447 [H]
         Correlation Energy        =      -0.2046601391115113 [H]
         Total Energy              =     -76.2260576180779736 [H]
        ----------------------------------------------------------
         ==================> SCS-MP2 Energies <==================
        ----------------------------------------------------------
         SCS Same-Spin Scale       =       0.3333333333333333 [-]
         SCS Opposite-Spin Scale   =       1.2000000000000000 [-]
         SCS Same-Spin Energy      =      -0.0170834420587555 [H]
         SCS Opposite-Spin Energy  =      -0.1840917755222936 [H]
         SCS Correlation Energy    =      -0.2011752175810492 [H]
         SCS Total Energy          =     -76.2225726965475161 [H]
        ----------------------------------------------------------
\end{verbatim}
The theory, breakdown of results, and common keywords used in DFMP2 are presented below. 

\subsubsection{Theory}

M\o{}ller-Plesset Theory (MPn) or Many-Body Perturbation Theory (MBPT) through second order has the spin-orbital formula:
\begin{equation}
E_{\mathrm{total}}^{(2)}  = E_{\mathrm{Reference}} - \frac{f_{ia}
f_{ia}}{\epsilon_a - \epsilon_i} - \frac{1}{4} \frac{<ij||ab>^2}{\epsilon_a + \epsilon_b - \epsilon_i - \epsilon_j}
\end{equation}
Here $i$ and $j$ are occupied spin orbitals, $a$ and $b$ are virtual spin
orbitals, $f_{ia}$ are the $ov$ Fock Matrix elements, $\epsilon$ are the orbital
eigenvalues, and $<ij||ab>$ are the antisymmetrized physicist's ERIs. For
converged RHF and UHF references, the singles correction,
\begin{equation}
E_{\mathrm{MBPT}}^{(1)} = - \frac{f_{ia} f_{ia}}{\epsilon_a - \epsilon_i},
\end{equation}
is zero due to the Bruillion Condition, and the first contribution to the perturbation series is at the second order:
\begin{equation}
E_{\mathrm{MBPT}}^{(2)}  = - \frac{1}{4} \frac{<ij|ab>^2}{\epsilon_a + \epsilon_b - \epsilon_i - \epsilon_j}.
\end{equation}

In the DFMP2 module, the first-order contribution, or ``singles energy'' is
always evaluated. This term is a significant contributor to the total
second-order energy if a ROHF reference is used. In this case, we have chosen
to use the ROHF-MBPT(2) ansatz, in which the ROHF orbitals are
semicanonicalized, the resultant nonzero Fock matrix elements $f_{ia}$ are used
to form the singles amplitudes, and then the second-order amplitudes are formed
with the semicanonical spin orbitals via the same machinery as a UHF-MP2. Note
that the singles energy should be very close to zero for RHF and UHF references;
if it is not, there is a good chance your orbitals are not well converged.
Increase the SCF \texttt{E\_CONVERGENCE} and/or \texttt{D\_CONVERGENCE} keywords
and try again. 

To increase the efficiency of MP2 energy evaluation, spin integration
and simplification is carried out. This also allows for the identification of
Same-Spin (SS) and Opposite-Spin (OS) terms for use in Grimme's Spin-Component
Scaled (SCS) MP2. For RHF-MP2 (also seen as RMP2), the spin-free equations are
(note that the integrals are now chemist's integrals over spatial orbitals),
\begin{equation}
E_{\mathrm{MBPT,OS}}^{(2)} = - \frac{(ia|jb)(ia|jb)}{\epsilon_a + \epsilon_b - \epsilon_i - \epsilon_j},
\end{equation}
and, 
\begin{equation}
E_{\mathrm{MBPT,SS}}^{(2)} = - \frac{[(ia|jb)-(ib|ja)](ia|jb)}{\epsilon_a + \epsilon_b - \epsilon_i - \epsilon_j}.
\end{equation}
For UHF-MP2 (also seen as UMP2) and the second-order contribution to ROHF-MBPT(2) using semicanonical orbitals, the spin-free equations are,
\begin{equation}
E_{\mathrm{MBPT,OS}}^{(2)} = - \frac{(ia^\alpha|jb^\beta)(ia^\alpha|jb^\beta)}{\epsilon_a + \epsilon_b - \epsilon_i - \epsilon_j},
\end{equation}
and, 
\begin{equation}
E_{\mathrm{MBPT,SS}}^{(2)} = - \frac{1}{2}\frac{[(ia^\alpha|jb^\alpha)-(ib^\alpha|ja^\alpha)](ia^\alpha|jb^\alpha)}{\epsilon_a + \epsilon_b - \epsilon_i - \epsilon_j}
\end{equation}
\[
                           - \frac{1}{2}\frac{[(ia^\beta|jb^\beta)-(ib^\beta|ja^\beta)](ia^\beta|jb^\beta)}{\epsilon_a + \epsilon_b - \epsilon_i - \epsilon_j}.
\]
Note that the UHF-MP2 equations use three classes of integrals, while the
RHF-MP2 equations use only one class. Because of this, a UHF-MP2 or
ROHF-MBPT(2) energy should take roughly three times as long as an RHF-MP2
energy.

\subsubsection{Recommendations}

All-in-all, DFMP2 should be a simple module to use, with few keywords (fully
documented in the appendix). Some basic recommendations are included below:
\begin{itemize}
\item DFMP2 should be run with the $ov$-type RI or MP2FIT auxiliary basis sets,
\emph{not} the -JKFIT basis sets. The automatic basis selector should work fine for
all of the Dunning bases (provided the auxiliary basis exists for the atom in
question). If it does not, use the \texttt{DF\_BASIS\_MP2} keyword to manually
specify the basis. 
\item DFMP2 likes memory. At a minimum, $2Q^2$ doubles are required, where $Q$ is
the size of the auxiliary basis set. However, there is one disk transpose of the
$(Q|ov)$ tensor in the RHF-MP2 algorithm [two for UHF-MP2 and ROHF-MBPT(2)], so more
memory will reduce seek times. If you notice DFMP2 using more memory than
allowed, it is possible that the threaded three-index ERI computers are using
too much overhead memory. Set the \texttt{DF\_INTS\_NUM\_THREADS} to a smaller
number to prevent this in this section (does not affect threaded efficiency in
the rest of the code). 
\item DFMP2 likes disk. At a minimum, $2Qov$ doubles are required for RHF-MP2,
and $4Qov$ doubles are required for UHF-MP2. 
\item DFMP2 likes threads. Some of the formation of the $(Q|ov)$ tensor relies
on threaded BLAS (such as MKL) for efficiency. The main ${\cal O}(N^5)$ step is
done via small/medium-sized DGEMMs inside of OpenMP, so make sure to set the
\texttt{OMP\_NESTED} environment variable to \texttt{FALSE} to prevent thread
thrash.
\item Freezing core is good for both efficiency and correctness purposes.
Freezing virtuals is not recommended. Determination of frozen core/virtuals is
handled at the SCF level, though the DFMP2 module will remind you how many
frozen/active orbitals it is using. 
\item ROHF-MBPT(2) may be preferred to UHF-MP2, as the latter can suffer from
severe spin contamination in some cases.
\item MP2 is not suitable for systems with multireference character. The
orbital energies will come together and an explosion will occur. 
\end{itemize}  

