\section{Geometry Optimization} \label{sec:opt}
\renewcommand{\optionname}[2]{\texttt{\nameref{op-#2-#1}}}

\PSIfour\ carries out molecular optimizations using a module called
\PSIoptking.  The optking program takes as input nuclear gradients and,
optionally, nuclear second derivatives --- both in cartesian coordinates.
The default minimization algorithm employs an empirical model Hessian,
redundant internal coordinates, an RFO step, and the BFGS Hessian update.

The principal literature references include the introduction of redundant
internal coordinates by Peng, Ayala, Schlegel, and Frisch in
{\it J.\ Comput.\ Chem.} 17, 49 (1996). The general approach employed in this code
is similar to the ''model Hessian plus RF method'' described and tested by Bakken and
Helgaker in {\it J.\ Chem.\ Phys.} 117, 9160 (2002). (However, for separated
fragments, we have chosen not to employ by default their ''extra-redundant''
coordinates defined by their ''auxiliary interfragment'' bonds.  These can be
included via the option \optionname{ADD_AUXILIARY_BONDS}{OPTKING}).

The internal coordinates are generated automatically based on an assumed bond
connectivity.  The connectivity is determined by testing if the interatomic
distance is less than the sum of atomic radii times the value of
\optionname{COVALENT_CONNECT}{OPTKING}.  If the user finds that some
connectivity is lacking by default, then this value may be increased.
Otherwise, the internal coordinate definitions may be modified.  If one
desires to see or modify the internal coordinates being used, then one can set
\optionname{INTCOS_GENERATE_EXIT}{OPTKING} to true.  The internal coordinate
definitions are provided in the file named ''intco.dat''.  See the Examples
section for more detail.

The ongoing development of optking is providing for unique treatment of
coordinates which connect distinct molecular fragments.  Thus, several keywords
relate to ''interfragment modes'', though many of these capabilities are
still under development.  Presently, by default separate fragments are bonded by
nearest atoms and the whole system is treated as if it were part of one
molecule.  However, with the option \optioname{FRAG_MODE}{OPTKING}, fragments
may instead be related by a unique set of interfragment coordinates defined by
reference points within each fragment.  The reference points can be atomic
positions (which is the current default), linear combinations of
atomic positions, or located on the principal axes (not yet working).

\subsection{Hessian}
If cartesian second derivatives are available, optking can read them
and transform them into internal coordinates to make an initial Hessian in
internal coordinates.  Otherwise, several empirical Hessians are available,
including those of Schlegel from {\it Theor.\ Chim.\ Acta} 66, 333 (1984) and 
Fischer and Almlof from {\it J.\ Phys.\ Chem.} 96, 9770 (1992).
Either of these or a simple diagonal Hessian may be selected using the 
\optionname{INTRAFRAG_HESS} keyword.

All the common Hessian update schemes are available.  For formulas, see
Schlegel in {\it Ab Initio Methods in Quantum Chemistry} (1987) and
Bofill in {\it J.\ Comp.\ Chem.}, 15, 1-11 (1994). 

The Hessian may be computed during an optimization using the following
keyword: \\
\begin{tabular*}{\textwidth}[tb]{p{0.3\textwidth}p{0.7\textwidth}}
         \optionname{FULL-HESS-EVERY}{GLOBALS} & 
  Frequency with which to compute the full Hessian in the course
  of a geometry optimization. 0 means to compute the initial Hessian only,
  1 means recompute every step, and N means recompute every N steps. The
  default (-1) is to never compute the full Hessian.
\end{tabular*}
\begin{tabular*}{\textwidth}[tb]{p{0.3\textwidth}p{0.35\textwidth}p{0.35\textwidth}}
           & {\bf Type:} integer &  {\bf Default:} -1\\
         & & \\
\end{tabular*}

\subsection{Examples}
First, define your molecule and basis in your input.
\begin{Snippet}
molecule h2o {
  O
  H 1 1.0
  H 1 1.0 2 105.0
}
set globals basis dz
\end{Snippet}

Then the following are examples of various types of calculations that can be completed.

Optimize a geometry using default methods (RFO step):
\begin{Snippet}
 optimize('scf')
\end{Snippet}

Optimize using Newton-Raphson steps instead of RFO steps:
\begin{Snippet}
set optking step_type nr
optimize('scf')
\end{Snippet}

Optimize using energy points instead of gradients:
\begin{Snippet}
optimize('scf', dertype=0)
\end{Snippet}

Optimize, limiting the initial step size to 0.1 au:
\begin{Snippet}
set optking intrafrag_step_limit 0.1
optimize('scf')
\end{Snippet}

Optimize, limiting the step size to 0.1 au always: 
\begin{Snippet}
set optking {
  intrafrag_step_limit     0.1
  intrafrag_step_limit_min 0.1
  intrafrag_step_limit_max 0.1
}
optimize('scf')
\end{Snippet}

Optimize, calculating the Hessian at every step (note that the necessary option is a global option):
\begin{Snippet}
set globals full_hess_every 1
optimize('scf')
\end{Snippet}

Calculate a starting Hessian and optimize the ''transition state'' of
linear water (note that without a reasonable starting geometry and
Hessian, such a straightforward search often fails):
\begin{Snippet}
molecule h2o {
   O
   H 1 1.0
   H 1 1.0 2 160.0
}
set globals {
 basis dz
 full_hess_every 0
}
set optking opt_type ts
optimize('scf')
\end{Snippet}

At a transition state (planar HOOH), compute the second derivative, and
then follow the intrinsic reaction path to the minimum:
\begin{Snippet}
molecule hooh {
 symmetry c1
 H
 O 1 0.946347
 O 2 1.397780 1  107.243777
 H 3 0.946347 2  107.243777   1 0.0
}
set globals {
 basis dzp
 opt_type irc
 geom_maxiter 50
}
frequencies('scf')
optimize('scf')
\end{Snippet}

Generate the internal coordinates and then stop:
\begin{Snippet}
set globals intcos_generate_exit true
optimize('scf')
\end{Snippet}
The coordinates may then be found in the file ''intco.dat''.  In this case, the file contains:
\begin{Snippet}
F 1 3
R      1     2
R      1     3
B      2     1     3
\end{Snippet}
The first line indicates a fragment containing atoms 1-3.  The following lines define
two distance coordinates (bonds) and one bend coordinate.  This file can be modified, and if present,
is used in subsequent optimizations.  Since the multiple-fragment coordinates are still under
development, they are not documented here.  However, if desired, one can change the value
of \optionname{FRAG_MODE}{OPTKING}, generate the internal coordinates, and see how multiple
fragment systems are defined.

Coordinates may be frozen or fixed, by adding an asterisk after the letter of the coordinate.
To optimize with the bond lengths fixed at their initial values, it is currently necessary to
generate and then modify the internal coordinate definitions as follows:
\begin{Snippet}
F 1 3
R*     1     2
R*     1     3
B      2     1     3
\end{Snippet}

\subsection{Convergence Criteria}

\PSIoptking\ monitors five quantities to evaluate the progress of a geometry 
optimization. These are (with their keywords) the change in energy 
(\optionname{MAX-ENERGY-G-CONVERGENCE}{OPTKING}), the maximum element of 
the gradient (\optionname{MAX-FORCE-G-CONVERGENCE}{OPTKING}), the root-mean-square 
of the gradient (\optionname{RMS-FORCE-G-CONVERGENCE}{OPTKING}), the maximum element
of displacement (\optionname{MAX-DISP-G-CONVERGENCE}{OPTKING}), and the 
root-mean-square of displacement (\optionname{RMS-DISP-G-CONVERGENCE}{OPTKING}), 
all in internal coordinates and atomic units. Usually, these options will not 
be set directly. Primary control for geometry convergence lies with the keyword 
\optionname{G-CONVERGENCE}{OPTKING} which sets the aforementioned in accordance 
with Table \ref{table:optkingconv}.

\begin{table}[!htbp]
\begin{footnotesize}
\caption{Summary of sets of geometry optimization criteria available in \PSIfour.} \label{table:optkingconv}
\parsep 10pt
\begin{center}
\begin{tabular}{lcccccc}
\hline\hline
\optionname{G-CONVERGENCE}{OPTKING} & Max Energy & Max Force & RMS Force & Max Disp & RMS Disp & Notes \\
\hline
NWCHEM\_LOOSE    &           & 4.5\e{-3} & 3.0\e{-3} & 5.4\e{-3} & 3.6\e{-3} & $d$      \\
GAU\_LOOSE       &           & 2.5\e{-3} & 1.7\e{-3} & 1.0\e{-2} & 6.7\e{-3} & $f$      \\
TURBOMOLE        & 1.0\e{-6} & 1.0\e{-3} & 5.0\e{-4} & 1.0\e{-3} & 5.0\e{-4} & $d$      \\
GAU              &           & 4.5\e{-4} & 3.0\e{-4} & 1.8\e{-3} & 1.2\e{-3} & $c$, $f$ \\
CFOUR            &           &           & 1.0\e{-4} &           &           & $d$      \\
QCHEM            & 1.0\e{-6} & 3.0\e{-4} &           & 1.2\e{-3} &           & $a$, $e$ \\
MOLPRO           & 1.0\e{-6} & 3.0\e{-4} &           & 3.0\e{-4} &           & $b$, $e$ \\
GAU\_TIGHT       &           & 1.5\e{-5} & 1.0\e{-5} & 6.0\e{-5} & 4.0\e{-5} & $c$, $f$ \\
GAU\_VERYTIGHT   &           & 2.0\e{-6} & 1.0\e{-6} & 6.0\e{-6} & 4.0\e{-6} & $f$      \\ 
\hline\hline
\end{tabular}
\end{center}
$^a$ Default \\
$^b$ Baker convergence criteria are the same. \\
$^c$ Counterpart NWCHEM convergence criteria are the same. \\
$^d$ Convergence achieved when all active criteria are fulfilled. \\
$^e$ Convergence achieved when MAX\_FORCE and one of MAX\_ENERGY or MAX\_DISP are fulfilled. \\
$^f$ Normal convergence achieved when all four criteria are fulfilled. To help with flat potential surfaces, alternate convergence achieved when $100\times$ rms force is less than RMS\_FORCE criterion.
\end{footnotesize}
\end{table}

For ultimate control, specifying a value for any of the five monitored options activates that
criterium and overwrites/appends it to the criteria set by \optionname{G-CONVERGENCE}{OPTKING}.
Note that this revokes the special convergence arrangements detailed in notes $e$ and $f$ in 
Table \ref{table:optkingconv} and instead requires all active criteria to be fulfilled to 
achieve convergence.

The progress of a geometry optimization can be monitored by grepping \outputdat\ for the
tilde character ({\raise.17ex\hbox{$\scriptstyle\sim$}}). This produces a table like the one below that shows
for each iteration the value for each of the five quantities and whether the criterion
is active and fulfilled (\texttt{*}), active and unfulfilled (\texttt{\;\;}), or inactive (\texttt{o}).
\begin{scriptsize}
\begin{Snippet}
--------------------------------------------------------------------------------------------- ~
 Step     Total Energy     Delta E     MAX Force     RMS Force      MAX Disp      RMS Disp    ~
--------------------------------------------------------------------------------------------- ~
  Convergence Criteria    1.00e-06 *    3.00e-04 *             o    1.20e-03 *             o  ~
--------------------------------------------------------------------------------------------- ~
    1     -38.91591820   -3.89e+01      6.91e-02      5.72e-02 o    1.42e-01      1.19e-01 o  ~
    2     -38.92529543   -9.38e-03      6.21e-03      3.91e-03 o    2.00e-02      1.18e-02 o  ~
    3     -38.92540669   -1.11e-04      4.04e-03      2.46e-03 o    3.63e-02      2.12e-02 o  ~
    4     -38.92548668   -8.00e-05      2.30e-04 *    1.92e-04 o    1.99e-03      1.17e-03 o  ~
    5     -38.92548698   -2.98e-07 *    3.95e-05 *    3.35e-05 o    1.37e-04 *    1.05e-04 o  ~
\end{Snippet}
\end{scriptsize}

The full list of keywords for \PSIoptking\ is provided in Appendix \ref{kw-OPTKING}.

