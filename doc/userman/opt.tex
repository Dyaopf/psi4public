\section{Geometry Optimization} \label{sec:opt}

\PSIthree\ is capable of carrying out geometry optimizations (minimization
only, at present) for a variety of molecular structures using either analytic
and numerical energy gradients.  

When present, internal coordinates provided in the INTCO: section of the
input will be read and used by \PSIthree.  If these are missing, \PSIthree\
will automatically generate and use redundant, simple internal
coordinates for carrying out the optimization.  These simple stretch, bend,
torsion, and linear bend coordinates are determined by distance
criteria using the input geometry.

By default, optimization is performed in redundant internal coordinates
regardless of how the geometry was provided in the input.  Alternatively,
the user may specify zmat\_simples=true, in which case the simple internal
coordinates will be taken from the ZMAT given in the input file.  Also,
the user may specify optimization in non-redundant, delocalized internal coordinates
with delocalize=true.  In this case, the automatically generated simple
coordinates are delocalized and redandancies are removed.  Advanced users
may wish to specify the simple internal coordinates in the intco.dat file, and
then allow \PSIthree\ to delocalize them.

Only those coordinates or combinations of coordinates that are specified
by the "symm =" keyword in the INTCO: section are optimized.  Coordinates can
be approximately frozen by commenting them out within the "symm =" section.
Geometrical constraints may be precisely imposed by the addition of a section
with nearly the same format as in INTCO:.  For example, to fix the distance
between atoms 1 and 2, as well as the angle between atoms 2, 1 and 3
in an optimization, add the following to your input file.
\\
\noindent
fixed\_intco: ( \\
  stre = ( \\
    (1 2) \\
  ) \\
  bend = ( \\
    (2 1 3) \\
  ) \\
) \\

The constrained simple internals must be ones present (either manually or
automatically) among the simple internals in the INTCO: section.  Alternatively,
the z-matrix input format may be used to specify constrained optimizations.
If zmat\_simples=true, then variables in the z-matrix which end in
a dollar sign will be taken as simple internals to be optimized, and
all other variables will be taken as simple internals to keep frozen.

To aid optimizations, force constants may be computed using "jobtype = symm\_fc".
The determined force constants will be saved in a binary file PSIF\_OPTKING
(currently file 1).  Subsequent optimizations will read and use these force
constants.  In general, \PSIthree\ looks for force constants in the following
order: in this binary file, in the FCONST: section of the input, and in the fconst.dat
file.  If no force constants are found in any of these, then an empirical 
diagonal force constant matrix is generated.

For methods for which only energies are available, \PSIthree\ will use non-redundant,
symmetry-adapted delocalized internal coordinates to generate geometrical
displacements for computing finite-difference gradients. The simple
coordinates can be linearly combined by hand or automatically.  The goal
is to form 3N-6(5) symmetry-adapted internal coordinates.  The automated
delocalized coordinates may work for low-symmetry molecules without
linear angles, but have not been extensively tested.  For both analytic-
and finite-difference-gradient optimization methods, Hessian updates are
performed using the BFGS method.

The list below shows which coordinates are used by default for different types of jobs. \\
jobtype=freq    dertype=first  symmetry-adapted cartesians \\
jobtype=freq    dertype=none   symmetry-adapted cartesians \\
jobtype=fc      dertype=first  delocalized internals (or user-defined SALCs) \\
jobtype=symm\_fc dertype=first  delocalized internals (or user-defined SALCs) \\
jobtype=opt     dertype=first  redundant internals \\
jobtype=opt     dertype=none   delocalized internals (or user-defined SALCS) \\
\\

The following keywords are pertinent for geometry optimizations.
\begin{description}
\item[JOBTYPE = string]\mbox{}\\
This keyword must be set to {\tt OPT} for geometry optimizations and
{\tt SYMM\_FC} to compute force constants.
\item[DERTYPE = string]\mbox{}\\
This keyword must be set to {\tt NONE} if only energies are available
for the chosen method and {\tt FIRST} if analytic gradients are available.
\item[CONV = integer]\mbox{}\\
The maximum force criteria for optimization is $10^{-conv}$.
\item[BFGS = boolean]\mbox{}\\
If true (the default), a BFGS Hessian update is performed.
\item[BFGS\_USE\_LAST = integer]\mbox{}\\
This keyword is used to specify the number of gradient step for the BFGS
update of the Hessian.  The default is six.  
\item[SCALE\_CONNECTIVITY = float]\mbox{}\\
Determines how close atoms must be to be considered bonded in the automatic
generation of the bonded list.  The default is 1.3.
\item[DELOCALIZE = integer]\mbox{}\\
Whether to delocalize simple internal coordinates to attempt to produce
a symmetry-adapted, non-redundant set.
\item[MIX\_TYPES = boolean]\mbox{}\\
If set to false, different types of internal coordinates are not allowed
to mix in the formation of the delocalized coordinates.  Although this
produces cleaner coordinates, often the resulting delocalized coordinates
form a redundant set.
\item[ZMAT\_SIMPLES = boolean]\mbox{}\\
If set to true, the simple internal coordinates are taken from the zmat
entry in the input file.  The default is false.
\item[POINTS = 3 or 5]\mbox{}\\
Specifies a 3-point or a 5-point formula for optimization by energy points.
\item[EDISP = float]\mbox{}\\
The default displacment size (in au) for finite-difference computations.  The
default is 0.005.
\item[FRAGMENT\_DISTANCE\_INVERSE = boolean]\mbox{}\\
For interfragment coordinates.  If true, then 1/R(AB) is used, if false,
then R(AB) is used.  The default is true.
\item[FIX\_INTRAFRAGMENT = boolean]\mbox{}\\
If true, all intrafragment coordinates are constrained.
\item[FIX\_INTERFRAGMENT = boolean]\mbox{}\\
If true, all interfragment coordinates are constrained.
\item[DUMMY\_AXIS\_1 = 1 or 2 or 3]\mbox{}\\
Specifies the axis for the location of a dummy atom for the definition
of a linear bending coordinate.  The default is 2.
\item[DUMMY\_AXIS\_2 = 1 or 2 or 3]\mbox{}\\
Specifies the axis for the location of a dummy atom for the definition
of a linear bending coordinate.  The default is 3.
\item[TEST\_B = boolean]\mbox{}\\
If set to true, a numerical test of the B-matrix is performed.
\item[PRINT\_FCONST = boolean]\mbox{}\\
If set to true and jobtype=symm\_fc, then the force constants will
be written to the fconst.dat file.  This allows force constants to be
reused even if the binary PSIF\_OPTKING file is no longer present.
\item[Print options]\mbox{}\\
The following when set to true, print additional information to the
output file: PRINT\_SIMPLES, PRINT\_PARAMS, PRINT\_DELOCALIZE,
PRINT\_SYMMETRY, PRINT\_HESSIAN, PRINT\_CARTESIANS.
\item[DISPLACEMENTS = ( (integer float ...) ...)]\mbox{}\\
A user may specify displacments along internal coordinates using this
keyword.  For example, displacements = ( (2 0.01 3 0.01) ) will compute
a new cartesian geometry with the second and third internal coordinates
increased by 0.01.
\end{description}


\section{Vibrational Frequency Computations} \label{sec:freq}
\PSIthree\ is also capable of computing harmonic vibrational frequencies
for a number of different methods using energy points or analytic energy first or
second derivatives.  (At present, only RHF-SCF analytic second derivatives
are available.)  If analytic energy second derivatives are not available,
\PSIthree\ will generate displaced geometries along symmetry adapted cartesian
coordinates, compute the appropriate energies or first derivatives, and use
finite-difference methods to compute the Hessian.

The following keywords are pertinent for vibrational frequency analyses:
\begin{description}
\item[JOBTYPE = string]\mbox{}\\
This keyword must be set to {\tt FREQ} for frequency analyses.
\item[DERTYPE = string]\mbox{}\\
This keyword may be set to {\tt NONE} if only energies are available
for the chosen method, or {\tt FIRST} if analytic gradients are available.
\item[POINTS = 3 or 5]\mbox{}\\
Specifies whether frequencies are determined by a 3-point or a 5-point
formula of gradient differences.  If only energy points are used, more
displacements are required, but the effect of this keyword in terms of
accuracy is the same.
\end{description}

\begin{em}
Note: In some situations, vibrational frequency analysis via finite
differences may fail if the full point group symmetry is specified via
the {\tt symmetry} keyword.  This happens because the user-given
{\tt symmetry} value can become incompatible with the actual symmetry
of the molecule when energies or gradients are evaluated for
symmetry-lowering displacements.  In such situations, the user is
advised to let the program determine the symmetry automatically, rather
than specifying {\tt symmetry} manually.  Otherwise, an error such as the
following may result:
\end{em}

\begin{verbatim}
error: problem assigning number of operations per class
         *** stopping execution ***
\end{verbatim}

The manual pages for the \PSInormco\ and \PSIintder\ modules contain
information on additional tools useful in vibrational frequency analysis
and coordinate transformation.

