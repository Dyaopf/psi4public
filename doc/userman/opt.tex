\section{Geometry Optimization} \label{sec:opt}
\renewcommand{\optionname}[2]{\texttt{\nameref{op-#2-#1}}}

%\PSIthree\ is capable of carrying out geometry optimizations (minimization
%only, at present) for a variety of molecular structures using either analytic
%and numerical energy gradients.  
%
%When present, internal coordinates provided in the INTCO: section of the
%input will be read and used by \PSIthree.  If these are missing, \PSIthree\
%will automatically generate and use redundant, simple internal
%coordinates for carrying out the optimization.  These simple stretch, bend,
%torsion, and linear bend coordinates are determined by distance
%criteria using the input geometry.
%
%By default, optimization is performed in redundant internal coordinates
%regardless of how the geometry was provided in the input.  Alternatively,
%the user may specify zmat\_simples=true, in which case the simple internal
%coordinates will be taken from the ZMAT given in the input file.  Also,
%the user may specify optimization in non-redundant, delocalized internal coordinates
%with delocalize=true.  In this case, the automatically generated simple
%coordinates are delocalized and redandancies are removed.  Advanced users
%may wish to specify the simple internal coordinates in the intco.dat file, and
%then allow \PSIthree\ to delocalize them.
%
%Only those coordinates or combinations of coordinates that are specified
%by the "symm =" keyword in the INTCO: section are optimized.  Coordinates can
%be approximately frozen by commenting them out within the "symm =" section.
%Geometrical constraints may be precisely imposed by the addition of a section
%with nearly the same format as in INTCO:.  For example, to fix the distance
%between atoms 1 and 2, as well as the angle between atoms 2, 1 and 3
%in an optimization, add the following to your input file.
%\\
%\noindent
%fixed\_intco: ( \\
%  stre = ( \\
%    (1 2) \\
%  ) \\
%  bend = ( \\
%    (2 1 3) \\
%  ) \\
%) \\
%
%The constrained simple internals must be ones present (either manually or
%automatically) among the simple internals in the INTCO: section.  Alternatively,
%the z-matrix input format may be used to specify constrained optimizations.
%If zmat\_simples=true, then variables in the z-matrix which end in
%a dollar sign will be taken as simple internals to be optimized, and
%all other variables will be taken as simple internals to keep frozen.
%
%To aid optimizations, force constants may be computed using "jobtype = symm\_fc".
%The determined force constants will be saved in a binary file PSIF\_OPTKING
%(currently file 1).  Subsequent optimizations will read and use these force
%constants.  In general, \PSIthree\ looks for force constants in the following
%order: in this binary file, in the FCONST: section of the input, and in the fconst.dat
%file.  If no force constants are found in any of these, then an empirical 
%diagonal force constant matrix is generated.
%
%For methods for which only energies are available, \PSIthree\ will use non-redundant,
%symmetry-adapted delocalized internal coordinates to generate geometrical
%displacements for computing finite-difference gradients. The simple
%coordinates can be linearly combined by hand or automatically.  The goal
%is to form 3N-6(5) symmetry-adapted internal coordinates.  The automated
%delocalized coordinates may work for low-symmetry molecules without
%linear angles, but have not been extensively tested.  For both analytic-
%and finite-difference-gradient optimization methods, Hessian updates are
%performed using the BFGS method.
%
%The list below shows which coordinates are used by default for different types of jobs. \\
%jobtype=freq    dertype=first  symmetry-adapted cartesians \\
%jobtype=freq    dertype=none   symmetry-adapted cartesians \\
%jobtype=fc      dertype=first  delocalized internals (or user-defined SALCs) \\
%jobtype=symm\_fc dertype=first  delocalized internals (or user-defined SALCs) \\
%jobtype=opt     dertype=first  redundant internals \\
%jobtype=opt     dertype=none   delocalized internals (or user-defined SALCS) \\
%\\
%
%The following keywords are pertinent for geometry optimizations.
%\begin{description}
%\item[JOBTYPE = string]\mbox{}\\
%This keyword must be set to {\tt OPT} for geometry optimizations and
%{\tt SYMM\_FC} to compute force constants.
%\item[DERTYPE = string]\mbox{}\\
%This keyword must be set to {\tt NONE} if only energies are available
%for the chosen method and {\tt FIRST} if analytic gradients are available.
%\item[CONV = integer]\mbox{}\\
%The maximum force criteria for optimization is $10^{-conv}$.
%\item[BFGS = boolean]\mbox{}\\
%If true (the default), a BFGS Hessian update is performed.
%\item[BFGS\_USE\_LAST = integer]\mbox{}\\
%This keyword is used to specify the number of gradient step for the BFGS
%update of the Hessian.  The default is six.  
%\item[SCALE\_CONNECTIVITY = float]\mbox{}\\
%Determines how close atoms must be to be considered bonded in the automatic
%generation of the bonded list.  The default is 1.3.
%\item[DELOCALIZE = integer]\mbox{}\\
%Whether to delocalize simple internal coordinates to attempt to produce
%a symmetry-adapted, non-redundant set.
%\item[MIX\_TYPES = boolean]\mbox{}\\
%If set to false, different types of internal coordinates are not allowed
%to mix in the formation of the delocalized coordinates.  Although this
%produces cleaner coordinates, often the resulting delocalized coordinates
%form a redundant set.
%\item[ZMAT\_SIMPLES = boolean]\mbox{}\\
%If set to true, the simple internal coordinates are taken from the zmat
%entry in the input file.  The default is false.
%\item[POINTS = 3 or 5]\mbox{}\\
%Specifies a 3-point or a 5-point formula for optimization by energy points.
%\item[EDISP = float]\mbox{}\\
%The default displacment size (in au) for finite-difference computations.  The
%default is 0.005.
%\item[FRAGMENT\_DISTANCE\_INVERSE = boolean]\mbox{}\\
%For interfragment coordinates.  If true, then 1/R(AB) is used, if false,
%then R(AB) is used.  The default is true.
%\item[FIX\_INTRAFRAGMENT = boolean]\mbox{}\\
%If true, all intrafragment coordinates are constrained.
%\item[FIX\_INTERFRAGMENT = boolean]\mbox{}\\
%If true, all interfragment coordinates are constrained.
%\item[DUMMY\_AXIS\_1 = 1 or 2 or 3]\mbox{}\\
%Specifies the axis for the location of a dummy atom for the definition
%of a linear bending coordinate.  The default is 2.
%\item[DUMMY\_AXIS\_2 = 1 or 2 or 3]\mbox{}\\
%Specifies the axis for the location of a dummy atom for the definition
%of a linear bending coordinate.  The default is 3.
%\item[TEST\_B = boolean]\mbox{}\\
%If set to true, a numerical test of the B-matrix is performed.
%\item[PRINT\_FCONST = boolean]\mbox{}\\
%If set to true and jobtype=symm\_fc, then the force constants will
%be written to the fconst.dat file.  This allows force constants to be
%reused even if the binary PSIF\_OPTKING file is no longer present.
%\item[Print options]\mbox{}\\
%The following when set to true, print additional information to the
%output file: PRINT\_SIMPLES, PRINT\_PARAMS, PRINT\_DELOCALIZE,
%PRINT\_SYMMETRY, PRINT\_HESSIAN, PRINT\_CARTESIANS.
%\item[DISPLACEMENTS = ( (integer float ...) ...)]\mbox{}\\
%A user may specify displacments along internal coordinates using this
%keyword.  For example, displacements = ( (2 0.01 3 0.01) ) will compute
%a new cartesian geometry with the second and third internal coordinates
%increased by 0.01.
%\end{description}
%
%
%\section{Vibrational Frequency Computations} \label{sec:freq}
%\PSIthree\ is also capable of computing harmonic vibrational frequencies
%for a number of different methods using energy points or analytic energy first or
%second derivatives.  (At present, only RHF-SCF analytic second derivatives
%are available.)  If analytic energy second derivatives are not available,
%\PSIthree\ will generate displaced geometries along symmetry adapted cartesian
%coordinates, compute the appropriate energies or first derivatives, and use
%finite-difference methods to compute the Hessian.
%
%The following keywords are pertinent for vibrational frequency analyses:
%\begin{description}
%\item[JOBTYPE = string]\mbox{}\\
%This keyword must be set to {\tt FREQ} for frequency analyses.
%\item[DERTYPE = string]\mbox{}\\
%This keyword may be set to {\tt NONE} if only energies are available
%for the chosen method, or {\tt FIRST} if analytic gradients are available.
%\item[POINTS = 3 or 5]\mbox{}\\
%Specifies whether frequencies are determined by a 3-point or a 5-point
%formula of gradient differences.  If only energy points are used, more
%displacements are required, but the effect of this keyword in terms of
%accuracy is the same.
%\end{description}
%
%\begin{em}
%Note: In some situations, vibrational frequency analysis via finite
%differences may fail if the full point group symmetry is specified via
%the {\tt symmetry} keyword.  This happens because the user-given
%{\tt symmetry} value can become incompatible with the actual symmetry
%of the molecule when energies or gradients are evaluated for
%symmetry-lowering displacements.  In such situations, the user is
%advised to let the program determine the symmetry automatically, rather
%than specifying {\tt symmetry} manually.  Otherwise, an error such as the
%following may result:
%\end{em}
%
%\begin{verbatim}
%error: problem assigning number of operations per class
%         *** stopping execution ***
%\end{verbatim}
%
%The manual pages for the \PSInormco\ and \PSIintder\ modules contain
%information on additional tools useful in vibrational frequency analysis
%and coordinate transformation.

\PSIfour\ carries out optimizations using a module called \PSIoptking.
The optking program performs optimization of molecular geometries given
nuclear gradients, and optionally second derivatives in cartesian
coordinates. The default minimization algorithm employs redundant internal
coordinates with an RFO step and a BFGS Hessian update.

The principal literature references include the introduction of redundant
internal coordinates in C. Peng, P.Y. Ayala, H.B. Schlegel, and M.J.
Frisch, J. Comput. Chem., 17, 49 (1996). The general approach employed in
this code is similar to the 'model Hessian plus RF method' described and
tested in V. Bakken and T. Helgaker, J. Chem. Phys., 117, 9160 (2002).
(However, for separated fragments, we have chosen not to employ their
'extra-redundant' coordinates defined by their 'auxiliary interfragment'
bonds.)

\subsection{Hessian}
If cartesian second derivatives have been computed, optking can read them
and transform them into internal coordinates to make the initial Hessian.
Otherwise, and initial empirical Hessian is determined according to either
Schlegel, Theor. Chim. Acta, 66, 333 (1984) or Fischer and Almlof, J. Phys.
Chem., 96, 9770 (1992). For interfragment coordinates, the initial guess
can be computed using the Fischer formulas or a simple diagonal guess.
Formulas for the Hessian updates, see Schlegel Ab Initio Methods in Quantum
Chemistry (1987) and J. M. Bofill, J. Comp. Chem., Vol. 15, pages 1-11
(1994). 

The Hessian may be updated every few iterations using the following
keyword: \\
\begin{tabular*}{\textwidth}[tb]{p{0.3\textwidth}p{0.7\textwidth}}
         \optionname{FULL-HESS-EVERY}{GENERAL} & 
  Frequency with which to compute the full Hessian in the course
  of a geometry optimization. 0 means to compute the initial Hessian only,
  1 means recompute every step, and N means recompute every N steps. The
  default (-1) is to never compute the full Hessian.
\end{tabular*}
\begin{tabular*}{\textwidth}[tb]{p{0.3\textwidth}p{0.35\textwidth}p{0.35\textwidth}}
           & {\bf Type:} integer &  {\bf Default:} -1\\
         & & \\
\end{tabular*}


\subsection{Convergence Criteria}

\PSIoptking\ monitors five quantities to evaluate the progress of a geometry 
optimization. These are (with their keywords) the change in energy 
(\optionname{MAX-ENERGY-G-CONVERGENCE}{OPTKING}), the maximum element of 
the gradient (\optionname{MAX-FORCE-G-CONVERGENCE}{OPTKING}), the root-mean-square 
of the gradient (\optionname{RMS-FORCE-G-CONVERGENCE}{OPTKING}), the maximum element
of displacement (\optionname{MAX-DISP-G-CONVERGENCE}{OPTKING}), and the 
root-mean-square of displacement (\optionname{RMS-DISP-G-CONVERGENCE}{OPTKING}), 
all in internal coordinates and atomic units. Usually, these options will not 
be set directly. Primary control for geometry convergence lies with the keyword 
\optionname{G-CONVERGENCE}{OPTKING} which sets the aforementioned in accordance 
with Table \ref{table:optkingconv}.

\begin{table}[!htbp]
\begin{footnotesize}
\caption{Summary of sets of geometry optimization criteria available in \PSIfour.} \label{table:optkingconv}
\parsep 10pt
\begin{center}
\begin{tabular}{lcccccc}
\hline\hline
\optionname{G-CONVERGENCE}{OPTKING} & Max Energy & Max Force & RMS Force & Max Disp & RMS Disp & Notes \\
\hline
NWCHEM\_LOOSE    &           & 4.5\e{-3} & 3.0\e{-3} & 5.4\e{-3} & 3.6\e{-3} & $d$      \\
GAU\_LOOSE       &           & 2.5\e{-3} & 1.7\e{-3} & 1.0\e{-2} & 6.7\e{-3} & $f$      \\
TURBOMOLE        & 1.0\e{-6} & 1.0\e{-3} & 5.0\e{-4} & 1.0\e{-3} & 5.0\e{-4} & $d$      \\
GAU              &           & 4.5\e{-4} & 3.0\e{-4} & 1.8\e{-3} & 1.2\e{-3} & $c$, $f$ \\
CFOUR            &           &           & 1.0\e{-4} &           &           & $d$      \\
QCHEM            & 1.0\e{-6} & 3.0\e{-4} &           & 1.2\e{-3} &           & $a$, $e$ \\
MOLPRO           & 1.0\e{-6} & 3.0\e{-4} &           & 3.0\e{-4} &           & $b$, $e$ \\
GAU\_TIGHT       &           & 1.5\e{-5} & 1.0\e{-5} & 6.0\e{-5} & 4.0\e{-5} & $c$, $f$ \\
GAU\_VERYTIGHT   &           & 2.0\e{-6} & 1.0\e{-6} & 6.0\e{-6} & 4.0\e{-6} & $f$      \\ 
\hline\hline
\end{tabular}
\end{center}
$^a$ Default \\
$^b$ Baker convergence criteria are the same. \\
$^c$ Counterpart NWCHEM convergence criteria are the same. \\
$^d$ Convergence achieved when all active criteria are fulfilled. \\
$^e$ Convergence achieved when MAX\_FORCE and one of MAX\_ENERGY or MAX\_DISP are fulfilled. \\
$^f$ Normal convergence achieved when all four criteria are fulfilled. To help with flat potential surfaces, alternate convergence achieved when $100\times$ rms force is less than RMS\_FORCE criterion.
\end{footnotesize}
\end{table}

For ultimate control, specifying a value for any of the five monitored options activates that
criterium and overwrites/appends it to the criteria set by \optionname{G-CONVERGENCE}{OPTKING}.
Note that this revokes the special convergence arrangements detailed in notes $e$ and $f$ in 
Table \ref{table:optkingconv} and instead requires all active criteria to be fulfilled to 
achieve convergence.

The progress of a geometry optimization can be monitored by grepping \outputdat\ for the
tilde character ({\raise.17ex\hbox{$\scriptstyle\sim$}}). This produces a table like the one below that shows
for each iteration the value for each of the five quantities and whether the criterion
is active and fulfilled (\texttt{*}), active and unfulfilled (\texttt{\;\;}), or inactive (\texttt{o}).
\begin{scriptsize}
\begin{Snippet}
--------------------------------------------------------------------------------------------- ~
 Step     Total Energy     Delta E     MAX Force     RMS Force      MAX Disp      RMS Disp    ~
--------------------------------------------------------------------------------------------- ~
  Convergence Criteria    1.00e-06 *    3.00e-04 *             o    1.20e-03 *             o  ~
--------------------------------------------------------------------------------------------- ~
    1     -38.91591820   -3.89e+01      6.91e-02      5.72e-02 o    1.42e-01      1.19e-01 o  ~
    2     -38.92529543   -9.38e-03      6.21e-03      3.91e-03 o    2.00e-02      1.18e-02 o  ~
    3     -38.92540669   -1.11e-04      4.04e-03      2.46e-03 o    3.63e-02      2.12e-02 o  ~
    4     -38.92548668   -8.00e-05      2.30e-04 *    1.92e-04 o    1.99e-03      1.17e-03 o  ~
    5     -38.92548698   -2.98e-07 *    3.95e-05 *    3.35e-05 o    1.37e-04 *    1.05e-04 o  ~
\end{Snippet}
\end{scriptsize}

The full list of keywords for \PSIoptking\ is provided in Appendix \ref{kw-OPTKING}.

