\subsection{The MRCC Program of M. K\'{a}llay} \label{mrcc}
An interface exists between the MRCC program of M. K\'{a}llay and J. Gauss within \PSIfour.  The license and source code of the MRCC program must be obtained from Mih\'{a}ly K\'{a}llay http://www.mrcc.hu/. Currently, only single reference methods with RHF reference functions are supported.

\subsubsection{Installation}
Follow the instructions provided with the source to build the MRCC programs. To be used by \PSIfour, ensure that the program binary for {\tt dmrcc} can be found in your {\tt PATH}. If \PSIfour\ is unable to execute the binary, an error will be reported.

\subsubsection{Running MRCC}
MRCC can be invoked in similar fashion as other theories provided in \PSIfour. For example, if you want to obtain the CCSDT energy for water with cc-pVDZ using MRCC simply provide the following:

\begin{Snippet}
molecule h2o {
    O
    H 1 1.0
    H 1 1.0 2 104.5
}

set {
    basis cc-pVDZ
}

energy('mrccsdt')
\end{Snippet}

{\tt mrccsdt} in the call to {\tt energy()} instructs \PSIfour\ to first perform an RHF calculation and then call MRCC to compute the CCSDT energy. For a CCSDT(Q) energy, simply use {\tt mrccsdt(q)} in the call to {\tt energy}.  MRCC can be used to perform geometry optimization and frequency calculations as well.

At this time, \PSIfour\ is only about to automatically determine the level of theory to use with MRCC for standard coupled cluster (CCSD, CCSDT, CCSDTQ, etc.) and perturbative methods [CCSD(T), CCSDT(Q), etc.]. For other methods, you will be required to use the MRCC keywords described elsewhere in this manual.

Frozen-core approximation is also supported in the MRCC interface. To optimize, CH$_4$ with CCSDT freezing the 1s on carbon, run:

\begin{Snippet}
molecule H2O {
    O
    H 1 r
    H 1 r 2 104.5

    r = 1.0
}

set {
    basis cc-pVDZ
    freeze_core true
}

optimize('mrccsdt')
\end{Snippet}
