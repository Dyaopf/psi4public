\subsection{The MRCC Program of M. K\'{a}llay} \label{sec:mrcc}
\PSIfour\ contains code to interface to the MRCC program of M. K\'{a}llay
and J. Gauss.  The license and source code of the MRCC program must be
obtained from Mih\'{a}ly K\'{a}llay ({\tt http://www.mrcc.hu/}). Currently, only single reference methods with RHF reference functions are supported.
\renewcommand{\optionname}[2]{\texttt{\nameref{op-#2-#1}}}

\subsubsection{Installation}
Follow the instructions provided with the source to build the MRCC programs. To be used by \PSIfour, ensure that the program binary for {\tt dmrcc} can be found in your {\tt PATH}. If \PSIfour\ is unable to execute the binary, an error will be reported.

\subsubsection{Running MRCC}
MRCC can be invoked in similar fashion as other theories provided in \PSIfour. For example, if you want to obtain the CCSDT energy for water with cc-pVDZ using MRCC simply provide the following:
\begin{Snippet}
molecule h2o {
    O
    H 1 1.0
    H 1 1.0 2 104.5
}
set {
    basis cc-pVDZ
}
energy('mrccsdt')
\end{Snippet}

{\tt mrccsdt} in the call to {\tt energy()} instructs \PSIfour\ to first perform an RHF calculation and then call MRCC to compute the CCSDT energy. For a CCSDT(Q) energy, simply use {\tt mrccsdt(q)} in the call to {\tt energy}.  MRCC can be used to perform geometry optimization and frequency calculations as well.

At this time, \PSIfour\ is only able to automatically generate the proper
input file for MRCC for the methods listed in Table \ref{table:mrccauto}. For other methods, you will be required to use the MRCC keywords described elsewhere in this manual.

\begin{table}
\caption{Methods available for automatic interface with MRCC.} 
\label{table:mrccauto}
\begin{center}
\small
\begin{tabular}{lll} \hline\hline
CCSD        & CCSD(T)$^*$      & CCSD(T)\_L$^*$    \\
CCSDT       & CCSDT(Q)$^*$     & CCSDT(Q)\_L$^*$   \\
CCSDTQ      & CCSDTQ(P)$^*$    & CCSDTQ(P)\_L$^*$  \\
CCSDTQP     & CCSDTQP(H)$^*$   & CCSDTQP(H)\_L$^*$ \\
CCSDTQPH    &                  &                   \\
            &                  &                   \\
CCSDT-1a    & CCSDT-1b         & CCSDT-3           \\
CCSDTQ-1a   & CCSDTQ-1b        & CCSDTQ-3          \\
CCSDTQP-1a  & CCSDTQP-1b       & CCSDTQP-3         \\
CCSDTQPH-1a & CCSDTQPH-1b      & CCSDTQPH-3        \\
            &                  &                   \\
CC2         &                  &                   \\
CC3         &                  &                   \\
CC4         &                  &                   \\
CC5         &                  &                   \\
CC6         &                  &                   \\
\hline\hline
\multicolumn{3}{l}{
\footnotesize{$^*$Pertubative methods not available with ROHF reference.}
}
\end{tabular}
\end{center}
\end{table}

Frozen-core approximation is also supported in the MRCC interface. To optimize CH$_4$ with CCSDT freezing the 1s on carbon, run:

\begin{Snippet}
molecule H2O {
    O
    H 1 r
    H 1 r 2 104.5

    r = 1.0
}

set {
    basis cc-pVDZ
    freeze_core true
}

optimize('mrccsdt')
\end{Snippet}
