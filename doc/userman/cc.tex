\subsection{Coupled Cluster Methods} \label{cc}

The coupled cluster approach is one of the most accurate and reliable quantum
chemical techniques for including the effects of electron correlation.
\PSIthree\ is capable of computing energies, analytic gradients, and
linear response properties using a number of coupled cluster models.
Table \ref{table:ccsummary} summarizes these capabilities.  This section
describes how to carry out coupled cluster calculations within \PSIthree.
\begin{table}[h]
\begin{center}
\caption{Current coupled cluster capabilities of \PSIthree.}
\label{table:ccsummary}
\begin{tabular}{cccccc}
\hline
\hline
Reference & Method & Energy    & Gradient  &  Exc. Energies & LR Props \\
\hline
RHF       & CC2     & Y & N & Y & Y  \\
UHF       & CC2     & Y & N & Y & N  \\
ROHF      & CC2     & Y & N & Y & N  \\
RHF       & CCSD    & Y & Y & Y & Y  \\
RHF       & CCSD(T) & Y & N & --& -- \\
ROHF      & CCSD    & Y & Y & Y & N  \\
ROHF      & CCSD(T) & N & N & --& -- \\
UHF       & CCSD    & Y & Y & Y & N  \\
UHF       & CCSD(T) & Y & Y$^*$ & --& -- \\
Brueckner & CCD     & Y & N & N & N  \\
Brueckner & CCD(T)  & Y & N & --& -- \\
\hline
\hline
\end{tabular}
\end{center}
{\footnotesize $^*$CCSD(T) gradients implemented via an experimental code.
A more efficient and robust implementation will appear in the next
release.}
\end{table}

\subsubsection{Basic Keywords}

To compute a ground-state CCSD or CCSD(T) energy at a fixed geometry,
the following keywords are common:
\begin{description}
\item[WFN = string]\mbox{}\\
Acceptable values are {\tt ccsd}, {\tt ccsd\_t} [for CCSD(T)], {\tt bccd} 
(for Brueckner-orbital-based CCD), or {\tt bccd\_t} [for 
Brueckner-orbital-based CCSD(T)] There is no default.
\item[REFERENCE = string]\mbox{}\\
Acceptable values are {\tt reference = rhf}, {\tt rohf}, or {\tt uhf}.
There is no default.
\item[JOBTYPE = string]\mbox{}\\
Acceptable values are {\tt sp}, {\tt opt}, {\tt freq}, {\tt oeprop}, or 
{\tt response}.  There is no default. 
\item[CONVERGENCE = integer]\mbox{}\\
Sets the order of magnitude on the convergence of the CC wave function,
perturbed wave function, and/or lambda parameters.  The root-mean-square of
the difference in amplitude vectors from consecutive iterations is used to
determine the convergence.  The default is 7.
\item[MAXITER = integer]\mbox{}\\
The maximum number of iterations allowed for solving the CC amplitude or
lambda amplitude equations.  Defaults to 50.
\item[MEMORY = (real MB)]\mbox{}\\
Specified the amount of core memory to be used, in MB.  Defaults to 256.
Other units (e.g., KB or GB) are also allowed.
\item[BRUECKNER\_CONV = integer]\mbox{}\\
Specifies the order of magnitude convergence required for the Brueckner
orbitals.  The convergence is determined based on the largest T1
amplitude.
\item[AO\_BASIS = string]\mbox{}\\
Specifies the algorithm to be used in computing the contribution of the
four-virtual-index integrals ($\langle ab||cd\rangle$) to the CC amplitude
equations.  If {\tt AO\_BASIS=NONE}, the MO-basis integrals will be used;
if {\tt AO\_BASIS=DISK}, the AO-basis integrals, stored on disk, will be
used; if {\tt AO\_BASIS=DIRECT}, the AO-basis integrals will be computed on
the fly as necessary.  NB: The {\tt AO\_BASIS=DIRECT} option is not fully
implemented and should only be used by experts.  Default is {\tt NONE}.
Note: The developers recommend use of this keyword only as a last resort
because it significantly slows the calculation. The current algorithms for
handling the MO-basis four-virtual-index integrals have been significantly
improved and are preferable to the AO-based approach.
\item[FREEZE\_CORE = boolean]\mbox{}\\
Specifies whether core orbitals (which are determined automatically) are to
be excluded from the correlated calculations.  Default is {\tt FALSE}.
\item[RESTART = boolean]\mbox{}\\ Determine whether previous amplitude
vectors may be used as guesses in a given CC calculation.  Defaults to
{\tt TRUE}. For geometry optimizations, Brueckner calculations, etc. the
iterative solution of the CC amplitude equations may benefit considerably
by reusing old vectors as initial guesses.  Assuming that the MO phases
remain the same between updates, the CC codes will, by default, re-use
old vectors, unless the user sets {\tt RESTART = false}.
\item[PRINT = integer]\mbox{}\\
The desired print level for detailed output.  Setting this to 2 is a good
idea for larger calculations so that the progress of the calculation may be
easily followed.  Defaults to 0.
\item[CACHELEV = integer]\mbox{}\\
Sets the level of automated cacheing of four-index quantities in the CC
modules.   These modules are capable of keeping in core as much as
possible, various four-index quantities categorized by the number of
virtual/unoccupied-orbital indices they contain.  Setting {\tt CACHELEV=0}
will cache nothing (wise and sometimes necessary for very large CC 
calculations), {\tt CACHELEV=1} will keep quantities with up to one virtual
index in core (e.g., integrals of the form $\langle ij||ka\rangle$), {\tt
CACHELEV=2} will keep quantities with up two two virtual indices in core
(e.g., integrals of the form $\langle ij||ab \rangle$ or $\hat{T}_2$
amplitudes), {\tt CACHELEV=3} will keep three-virtual-index quantities in
core, and {\tt CACHELEV=4} will keep everything in core.  Note that the
cache behavior is tempered by the {\tt MEMORY} keyword, and items will be
deleted from the cache (in an order determined based on the {\tt CACHETYEP}
keyword) as additional memory is required in a given calculation.
\item[CACHETYPE = string]\mbox{}\\
Specifies the type of cache to be used, either {\tt LOW} or {\tt LRU}.  If
{\tt CACHETYPE=LOW}, then elements are deleted from the cache based on a
predefined order of priority.  If {\tt CACHETYPE=LRU}, then elements are
deleted from the cache based on a ``least recently used'' criterion: the
least recently used item is the first to be deleted.  The {\tt LOW}
criterion has been developed only ccenergy codes.  The default is {\tt LRU}
for all CC modules except {\tt ccenergy}.
\item[NUM\_AMPS = integer]\mbox{}\\
Specifies the number of wave function amplitudes to print at the end of the
energy calculation.  Defaults to 10.
\item[PRINT\_MP2\_AMPS = boolean]\mbox{}\\
Specifies if the initial guess (MP2) amplitudes should be printed in the
output file.  Defaults to {\tt FALSE}.
\end{description}

\subsubsection{Larger Calculations}

Here are a few recommendations for carrying out large-basis-set coupled cluster calculations with \PSIthree: 
\begin{enumerate}
\item Set the {\tt MEMORY} keyword to 90\% of the available physical
memory, at most.  There is a small amount of overhead associated with the
coupled cluster modules that is not accounted for by the internal CC memory
handling routines.  Thus, the user should {\em not} sepcify the entire
physical memory of the system, or swapping is likely.
\item Set the {\tt CACHELEV} keyword to 0.  This will turn off cacheing,
which, for very large calculations, can lead to heap fragmentation and
memory faults, even when sufficient physical memory exists.
\item Set the {\tt PRINT} keyword to 2.  This will help narrow where
memory bottlenecks or other errors exist in the event of a crash.
\end{enumerate}

\subsubsection{Excited State Coupled Cluster Calculations}

The most important keywords associated with EOM-CC calculations are:
\begin{description}
\item[STATES\_PER\_IRREP = (integer array)]\mbox{}\\
Specifies the desired number of excited states per irreducible representation 
for both EOM-CC and CC-LR calculations.  Note that the irreps in this
keyword denote the final state symmetry, not the symmetry of the transition.  
\item[PRINT\_SINGLES = boolean]\mbox{}\\
Specifies whether information regarding the iterative solution to the
single-excitation EOM-CC problem (normally used to obtain guesses for a
ful EOM-CCSD calculation) will be printed.
\item[RESIDUAL\_TOL = integer]\mbox{}\\
Specifies the order of magnitude cutoff used to determine the convergence of 
the Davidson algorithm residuals in the EOM-CC iterative procedure.
\item[EVAL\_TOL = integer]\mbox{}\\
Specifies the order of magnitude cutoff used to determine the convergence
of the final eigenvalues in the EOM-CC iterative procedure.
\item[EOM\_GUESS = (mixed array)]\mbox{}\\ Specifies a set
of single-excitation guess vectors for the EOM-CC procedure.  This is
especially useful for converging to difficult states.  The {\tt EOM\_GUESS}
keyword is an array, each element of which includes an occupied orbital
index (in coupled cluster ordering), a virtual orbital index, a weighting
factor, and a spin (0 for $\alpha$ and 1 for $\beta$).  The guess vector
will be normalized after it is read, so only the relative magnitudes of the
weight factors are important.
\item[JOBTYPE = string]\mbox{}\\
A value of {\tt oeprop} will result in the calculation of oscillator 
strengths, rotational strengths, and dipole moments for an RHF reference
and all but the rotational strengths for an ROHF or UHF reference.
\end{description}

\subsubsection{Linear Response (CCLR) Calculations}
The most important keywords associated with CC-LR calculations are:
\begin{description}
\item[JOBTYPE = string]\mbox{}\\
A value of {\tt RESPONSE} will invoke the linear-response programs.
\item[PROPERTY = string]\mbox{}\\
This leyword specifies the type or response
property desired.  Acceptable values are {\tt POLARIZABILITY} (default) for
dipole-polarizabilities and {\tt ROTATION} for specific rotations.
\item[OMEGA = real or (real UNITS)]\mbox{}
Specifies the desired frequency of the incident radiation field in CCLR
calculations.  Acceptable units are {\tt HZ}, {\tt NM}, and {\tt EV}.  If
given without units, atomic units (Hartrees) are assumed.
\item[MU\_IRREPS = (integer array)]\mbox{}\\
Specifies the irreducible representations associated with the $x$-, $y$-,
and $z$-axes.  This may be determined from the standard Cotton tables. 
Eventually this will be determined automatically by the program, so this
keyword will go away.
\end{description}
