\subsection{Coupled Cluster Methods} \label{sec:cc}
\renewcommand{\optionname}[2]{\texttt{\nameref{op-#2-#1}}}

The coupled cluster approach is one of the most accurate and reliable quantum
chemical techniques for including the effects of electron correlation.
Instead of the linear expansion of the wavefunction used by configuation
interaction, coupled cluster uses an exponential expansion,
\begin{eqnarray}
\ket{\Psi} &=& e^{\hat{T}} \ket{\Phi_0} \\ \nonumber
           &=& \left( 1 + {\hat{T}} + \frac{1}{2} {\hat{T}}^2
                  + \frac{1}{3!}{\hat{T}}^3 + \cdots \right) \ket{\Phi_0},
\label{eq:CCexpansion}
\end{eqnarray}
where the cluster operator ${\hat{T}}$ is written as a sum of operators that
generate singly-excited, doubly-excited, etc., determinants:
\begin{equation}
{\hat{T}} = {\hat{T}_1} + {\hat{T}_2} + {\hat{T}_3} + \cdots + {\hat{T}_N},
\end{equation}
with
\begin{eqnarray}
{\hat T}_1 \ket{\Phi_0} &=& \sum_{i}^{\rm occ} 
  \sum_a^{\rm vir} t_i^a \ket{\Phi_i^a}, \\
{\hat T}_2 \ket{\Phi_0} &=& \sum_{i<j}^{\rm occ} \sum_{a<b}^{\rm vir} 
  t_{ij}^{ab} \ket{\Phi_{ij}^{ab}}, 
\end{eqnarray}
etc.  The popular coupled cluster singles and doubles (CCSD) model
\cite{Purvis:1982} truncates the expansion at ${\hat{T}} = {\hat{T}_1}
+ {\hat{T}_2}$.  This model has the same number of parameters as
configuration interaction singles and doubles (CISD) but improves upon
it by approximately accounting for higher-order terms using products
of lower-order terms (e.g., the term ${\hat{T}_2}^2$ approximately
accounts for quadruple excitations).  The inclusion of such products
makes coupled-cluster methods {\em size extensive}, meaning that the
quality of the computation should not degrade for larger molecules.
The computational cost for CCSD scales as ${\cal{O}}(o^2 v^4)$, where
$o$ is the number of occupied orbitals and $v$ is the number of virtual
orbitals.

Improving upon CCSD, the CCSD(T) method \cite{Raghavachari:1989} includes
a perturbative estimate of the energy contributed by the ${\hat{T}_3}$
operator.  The computational cost of this additional term scales as
${\cal{O}}(o^3 v^4)$, making it rather expensive for molecules with more than
a dozen heavy atoms or so.  However, when this method is affordable, it
provides very high quality results in most cases.

\PSIfour\ is capable of computing energies and analytic gradients for a
number of coupled cluster models.  It can also compute linear response
properties (such as static or frequency-dependent polarizability,
or optical rotation angles) for some models.  Excited states can
also be computed by the CC2 and CC3 models, or by EOM-CCSD.  Table
\ref{table:ccsummary} summarizes these capabilities.  This section
describes how to carry out coupled cluster calculations within \PSIfour.
For higher-order coupled-cluster methods like CCSDT and CCSDTQ, \PSIfour\
can interface to K{\'a}llay's MRCC code (see Section \ref{sec:mrcc}).


\begin{table}[h]
\begin{center}
\caption{Current coupled cluster capabilities of \PSIfour.}
\label{table:ccsummary}
\begin{tabular}{cccccc}
\hline
\hline
Reference & Method & Energy    & Gradient  &  Exc. Energies & LR Props \\
\hline
RHF       & CC2     & Y & N & Y & Y  \\
UHF       & CC2     & Y & N & Y & N  \\
ROHF      & CC2     & Y & N & Y & N  \\
RHF       & CCSD    & Y & Y & Y & Y  \\
RHF       & CCSD(T) & Y & N & --& -- \\
ROHF      & CCSD    & Y & Y & Y & N  \\
ROHF      & CCSD(T) & Y & N & --& -- \\
UHF       & CCSD    & Y & Y & Y & N  \\
UHF       & CCSD(T) & Y & Y & --& -- \\
RHF       & CC3     & Y & N & Y & N  \\
UHF       & CC3     & Y & N & Y & N  \\
ROHF      & CC3     & Y & N & Y & N  \\
Brueckner & CCD     & Y & N & N & N  \\
Brueckner & CCD(T)  & Y & N & --& -- \\
\hline
\hline
\end{tabular}
\end{center}
\end{table}

The following wavefunctions are currently recognized by \PSIfour\ as arguments
to functions like {\tt energy()}: {\tt ccsd}, {\tt ccsd(t)}, {\tt cc2}, {\tt
cc3}, {\tt bccd} (CCD with Brueckner orbitals), {\tt bccd(t)} (CCD(T) with
Brueckner orbitals), {\tt eom-ccsd}, {\tt eom-cc2} (CC2 for excited states),
{\tt eom-cc3} (CC3 for excited states).  Response properties can be obtained
by calling the function {\tt response()} (instead of, for example, {\tt
energy()}), e.g., {\tt response(\qq{ccsd}\qq)}.  There are many sample
coupled cluster inputs provided in {\tt psi4/samples}.

\subsubsection{Basic Keywords}

A complete list of keywords related to coupled-cluster computations is
provided in Appendix \ref{keywords}, with the majority of the relevant
keywords appearing in Appendix \ref{kw-CCENERGY}.  For a standard ground-state 
CCSD or CCSD(T) computation, the following keywords are common:

\begin{tabular*}{\textwidth}[tb]{p{0.3\textwidth}p{0.7\textwidth}}
         \optionname{REFERENCE}{CCENERGY} & Reference wavefunction type \\
         & {\bf Possible Values:} RHF, ROHF, UHF \\
\end{tabular*}
\begin{tabular*}{\textwidth}[tb]{p{0.3\textwidth}p{0.35\textwidth}p{0.35\textwidth}}
           & {\bf Type:} string &  {\bf Default:} RHF\\
         & & \\
\end{tabular*}
\begin{tabular*}{\textwidth}[tb]{p{0.3\textwidth}p{0.7\textwidth}}
         \optionname{R-CONVERGENCE}{CCENERGY} 
         & Convergence criterion for wavefunction (change) in CC amplitude
         equations. \\
\end{tabular*}
\begin{tabular*}{\textwidth}[tb]{p{0.3\textwidth}p{0.35\textwidth}p{0.35\textwidth}}
           & {\bf Type:} double &  {\bf Default:} 1e-7\\
         & & \\
\end{tabular*}
\begin{tabular*}{\textwidth}[tb]{p{0.3\textwidth}p{0.7\textwidth}}
         \optionname{MAXITER}{CCENERGY}
         & Maximum number of iterations to solve the CC equations \\
\end{tabular*}
\begin{tabular*}{\textwidth}[tb]{p{0.3\textwidth}p{0.35\textwidth}p{0.35\textwidth}}
           & {\bf Type:} integer &  {\bf Default:} 50\\
         & & \\
\end{tabular*}
\begin{tabular*}{\textwidth}[tb]{p{0.3\textwidth}p{0.7\textwidth}}
         \optionname{BRUECKNER-ORBS-R-CONVERGENCE}{CCENERGY} \\
         & Convergence criterion for Breuckner orbitals. The convergence
           is determined based on the largest $T_1$ amplitude. \\
\end{tabular*}
\begin{tabular*}{\textwidth}[tb]{p{0.3\textwidth}p{0.35\textwidth}p{0.35\textwidth}}
           & {\bf Type:} double &  {\bf Default:} 1e-5\\
         & & \\
\end{tabular*}
\begin{tabular*}{\textwidth}[tb]{p{0.3\textwidth}p{0.7\textwidth}}
         \optionname{RESTART}{CCENERGY} 
         & Do restart the coupled-cluster iterations from old $t_1$ and
         $t_2$ amplitudes?  For geometry optimizations, Brueckner
         calculations, etc. the iterative solution of the CC amplitude 
         equations may benefit considerably by reusing old vectors as initial 
         guesses.  Assuming that the MO phases remain the same between 
         updates, the CC codes will, by default, re-use old vectors, unless 
         the user sets {\tt RESTART = false}. 
\end{tabular*}
\begin{tabular*}{\textwidth}[tb]{p{0.3\textwidth}p{0.35\textwidth}p{0.35\textwidth}}
           & {\bf Type:} boolean &  {\bf Default:} true\\
         & & \\
\end{tabular*}
\begin{tabular*}{\textwidth}[tb]{p{0.3\textwidth}p{0.7\textwidth}}
         \optionname{CACHELEVEL}{CCENERGY} 
         & Cacheing level for libdpd governing the storage of amplitudes,
    integrals, and intermediates in the CC procedure. A value of 0 retains
    no quantities in cache, while a level of 6 attempts to store all
    quantities in cache.  For particularly large calculations, a value of
    0 may help with certain types of memory problems.  The default is 2,
    which means that all four-index quantites with up to two virtual-orbital
    indices (e.g., $\langle ij | ab \rangle>$ integrals) may be held in the
    cache.
\end{tabular*}
\begin{tabular*}{\textwidth}[tb]{p{0.3\textwidth}p{0.35\textwidth}p{0.35\textwidth}}
           & {\bf Type:} integer &  {\bf Default:} 2\\
         & & \\
\end{tabular*}
\begin{tabular*}{\textwidth}[tb]{p{0.3\textwidth}p{0.7\textwidth}}
         \optionname{CACHETYPE}{CCENERGY} 
         & Selects the priority type for maintaining the automatic memory
    cache used by the libdpd codes. A value of LOW selects a "low priority"
    scheme in which the deletion of items from the cache is based on
    pre-programmed priorities. A value of LRU selects a "least recently used"
    scheme in which the oldest item in the cache will be the first one
    deleted. \\
         & {\bf Possible Values:} LOW, LRU \\
\end{tabular*}
\begin{tabular*}{\textwidth}[tb]{p{0.3\textwidth}p{0.35\textwidth}p{0.35\textwidth}}
           & {\bf Type:} string &  {\bf Default:} LOW\\
         & & \\
\end{tabular*}
\begin{tabular*}{\textwidth}[tb]{p{0.3\textwidth}p{0.7\textwidth}}
         \optionname{NUM-AMPS-PRINT}{CCENERGY}
         & Number of important $t_1$ and $t_2$ amplitudes to print
\end{tabular*}
\begin{tabular*}{\textwidth}[tb]{p{0.3\textwidth}p{0.35\textwidth}p{0.35\textwidth}}
           & {\bf Type:} integer &  {\bf Default:} 10\\
         & & \\
\end{tabular*}
\begin{tabular*}{\textwidth}[tb]{p{0.3\textwidth}p{0.7\textwidth}}
         \optionname{MP2-AMPS-PRINT}{CCENERGY}
         & Do print the MP2 amplitudes which are the starting guesses for RHF
           and UHF reference functions?
\end{tabular*}
\begin{tabular*}{\textwidth}[tb]{p{0.3\textwidth}p{0.35\textwidth}p{0.35\textwidth}}
           & {\bf Type:} boolean &  {\bf Default:} false\\
         & & \\
\end{tabular*}

\subsubsection{Larger Calculations}

Here are a few recommendations for carrying out large-basis-set coupled
cluster calculations with \PSIfour: 
\begin{enumerate}
\item In most cases it is reasonable to set the {\tt MEMORY} keyword to 90\% of 
the available physical memory, at most.  There is a small amount of overhead 
associated with the
coupled cluster modules that is not accounted for by the internal CC memory
handling routines.  Thus, the user should {\em not} sepcify the entire
physical memory of the system, or swapping is likely.  However, for especially large
calculations, it is better to set the {\tt MEMORY} keyword to a value less than 16 GB.
\item Set the \optionname{CACHELEVEL}{CCENERGY} keyword to \optionval{0}.
This will turn off cacheing, which, for very large calculations, can
lead to heap fragmentation and memory faults, even when sufficient
physical memory exists.
\item Set the \optionname{PRINT}{GLOBALS} keyword to \optionval{2}.  This 
will help narrow where memory bottlenecks or other errors exist in the 
event of a crash.
\end{enumerate}

\subsubsection{Excited State Coupled Cluster Calculations}

A complete list of keywords related to 
coupled cluster linear response is provided in Appendices \ref{kw-CCEOM}
and \ref{ekw-CCEOM}.
The most important keywords associated with EOM-CC calculations are:

\begin{tabular*}{\textwidth}[tb]{p{0.3\textwidth}p{0.7\textwidth}}
         \optionname{STATES-PER-IRREP}{CCEOM}
         & The number of electronic states to computed, per irreducible
         representation.  Irreps denote the final state symmetry, not the
         symmetry of the transtion.
\end{tabular*}
\begin{tabular*}{\textwidth}[tb]{p{0.3\textwidth}p{0.35\textwidth}p{0.35\textwidth}}
           & {\bf Type:} array &  {\bf Default:} none\\
         & & \\
\end{tabular*}
\begin{tabular*}{\textwidth}[tb]{p{0.3\textwidth}p{0.7\textwidth}}
         \optionname{E-CONVERGENCE}{CCEOM}
         & Convergence criterion for excitation energy (change) in the
         Davidson algorithm for CC-EOM.
\end{tabular*}
\begin{tabular*}{\textwidth}[tb]{p{0.3\textwidth}p{0.35\textwidth}p{0.35\textwidth}}
           & {\bf Type:} double &  {\bf Default:} 1e-8\\
         & & \\
\end{tabular*}
\begin{tabular*}{\textwidth}[tb]{p{0.3\textwidth}p{0.7\textwidth}}
         \optionname{SINGLES-PRINT}{CCEOM}
         & Do print information on the iterative solution to the
         single-excitation EOM-CC problem used as a guess to full EOM-CC?
\end{tabular*}
\begin{tabular*}{\textwidth}[tb]{p{0.3\textwidth}p{0.35\textwidth}p{0.35\textwidth}}
           & {\bf Type:} boolean &  {\bf Default:} false \\
         & & \\
\end{tabular*}
\begin{tabular*}{\textwidth}[tb]{p{0.3\textwidth}p{0.7\textwidth}}
         \optionname{SCHMIDT-ADD-RESIDUAL-TOLERANCE}{CCEOM} \\
         & Minimum absolute value above which a guess vector to a root is
         added to the Davidson algorithm in the EOM-CC iterative procedure.
\end{tabular*}
\begin{tabular*}{\textwidth}[tb]{p{0.3\textwidth}p{0.35\textwidth}p{0.35\textwidth}}
           & {\bf Type:} double &  {\bf Default:} 1e-3 \\
         & & \\
\end{tabular*}
\begin{tabular*}{\textwidth}[tb]{p{0.3\textwidth}p{0.7\textwidth}}
         \optionname{EOM-GUESS}{CCEOM}
         & Specifies a set of single-excitation guess vectors for the EOM-CC 
         procedure.  If \optionname{EOM-GUESS}{CCEOM} = \optionval{SINGLES}, 
         the guess will be taken from the singles-singles block of the 
         similarity-transformed Hamiltonian, Hbar.  If
         \optionname{EOM-GUESS}{CCEOM} = \optionval{DISK}, guess vectors from 
         a previous computation will be read from disk.  If 
         \optionname{EOM-GUESS}{CCEOM} = \optionval{INPUT}, guess vectors 
         will be specified in user input.  The latter method is not 
         currently available. \\
         & {\bf Possible Values:} SINGLES, DISK, INPUT \\
\end{tabular*}
\begin{tabular*}{\textwidth}[tb]{p{0.3\textwidth}p{0.35\textwidth}p{0.35\textwidth}}
           & {\bf Type:} string &  {\bf Default:} SINGLES \\
         & & \\
\end{tabular*}


\subsubsection{Linear Response (CCLR) Calculations}
Linear response computations are invoked like {\tt response(\qq{ccsd}\qq)}
or {\tt response(\qq{cc2}\qq)}.  A complete list of keywords related to
coupled cluster linear response is provided in Appendix \ref{kw-CCRESPONSE}.

The most important keywords associated with CC-LR calculations are as follows.

\begin{tabular*}{\textwidth}[tb]{p{0.3\textwidth}p{0.7\textwidth}}
         \optionname{PROPERTY}{CCRESPONSE}
         & The response property desired.  Acceptable values are {\tt
         POLARIZABILITY} (default) for dipole-polarizabilities,
         {\tt ROTATION} for specific rotations, {\tt ROA} for Raman
         Optical Activity, and {\tt ALL} for all of the above. \\
         & {\bf Possible Values:} POLARIZABILITY, ROTATION, ROA, ALL \\
\end{tabular*}
\begin{tabular*}{\textwidth}[tb]{p{0.3\textwidth}p{0.35\textwidth}p{0.35\textwidth}}
           & {\bf Type:} string &  {\bf Default:} POLARIZABILITY \\
         & & \\
\end{tabular*}
\begin{tabular*}{\textwidth}[tb]{p{0.3\textwidth}p{0.7\textwidth}}
         \optionname{OMEGA}{CCRESPONSE}
         & Array that specifies the desired frequencies of the incident 
         radiation field in CCLR calculations.  If only one element is given,
         the units will be assumed to be atomic units.  If more than one
         element is given, then the units must be specified as the final
         element of the array.  Acceptable units are {\tt HZ}, {\tt NM}, 
         {\tt EV}, and {\tt AU}.
\end{tabular*}
\begin{tabular*}{\textwidth}[tb]{p{0.3\textwidth}p{0.35\textwidth}p{0.35\textwidth}}
           & {\bf Type:} array &  {\bf Default:} 0 \\
         & & \\
\end{tabular*} 
\begin{tabular*}{\textwidth}[tb]{p{0.3\textwidth}p{0.7\textwidth}}
         \optionname{GAUGE}{CCRESPONSE}
         & Specifies the choice of representation of the electric dipole operator.
         Acceptable values are {\tt LENGTH} for the usual length-gauge representation,
         {\tt VELOCITY} for the modified velocity-gauge representation in which the 
         static-limit optical rotation tensor is subtracted from the frequency-dependent 
         tensor, or {\tt BOTH}.  Note that, for optical rotation calculations, only the
         choices of {\tt VELOCITY} or {\tt BOTH} will yield origin-independent results. \\
         & {\bf Possible Values:} LENGTH, VELOCITY, BOTH \\
\end{tabular*}
\begin{tabular*}{\textwidth}[tb]{p{0.3\textwidth}p{0.35\textwidth}p{0.35\textwidth}}
           & {\bf Type:} string &  {\bf Default:} LENGTH \\
         & & \\
\end{tabular*}




