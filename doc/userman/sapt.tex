\subsection{Symmetry-Adapted Perturbation Theory} \label{sapt}

Symmetry-adapted perturbation theory (SAPT) provides a means of directly
computing the noncovalent interaction between two molecules, that is, the
interaction energy is determined without computing the total energy of the
monomers or dimer. In addition, SAPT provides a decomposition of the
interaction energy into physically meaningful components: {\em i.e.}
electrostatic, exchange, induction, and dispersion terms. In SAPT, the 
Hamiltonian of the dimer is partitioned into contributions from each 
monomer and the interaction.
\begin{equation}
H=F_A+W_A+F_B+W_B+V
\end{equation}
Here, the Hamiltonian is written as a sum of the usual monomer Fock
operators, $F$, the fluctuation potential of each monomer, $W$, and the
interaction potential, $V$. The monomer Fock operators, $F_A+F_B$, are
treated as the zeroth-order Hamiltonian and the interaction energy is
evaluated through a perturbative expansion of $V$, $W_A$, and $W_B$. 
Through first-order in $V$, electrostatic and exchange interactions are
included; induction and dispersion first appear at second-order in $V$. For
a complete description of SAPT, the reader is refered to the excellent
review by Jeziorski, Moszynski, and Szalewicz \cite{Jeziorski:1994:1887}.

Several truncations of the SAPT expansion are available in the \PSIsapt\
module of \PSIfour. The simplest truncation of SAPT is denoted SAPT0.
\begin{equation}
\label{eqn:SAPT0}
E_{SAPT0} = E_{elst}^{(10)} + E_{exch}^{(10)} + E_{ind,resp}^{(20)} +
E_{exch-ind,resp}^{(20)} + E_{disp}^{(20)} + E_{exch-disp}^{(20)}.
\end{equation}
In this notation, $E^{(vw)}$ defines the order in $V$ and in $W_A+W_B$; the
subscript, $resp$, indicates that orbital relaxation effects are included.
\begin{equation}
\label{eqn:SAPT2}
E_{SAPT2} = E_{SAPT0} + E_{elst,resp}^{(12)} + E_{exch}^{(11)} +
E_{exch}^{(12)} +\/ ^{t}\!E_{ind}^{(22)} +\/ ^{t}\!E_{exch-ind}^{(22)}
\end{equation}
\begin{equation}
\label{eqn:SAPT2+}
E_{SAPT2+} = E_{SAPT2} + E_{disp}^{(21)} + E_{disp}^{(22)}
\end{equation}
\begin{equation}
\label{eqn:SAPT2+p3}
E_{SAPT2+(3)} = E_{SAPT2+} + E_{elst,resp}^{(13)} + E_{disp}^{(30)}
\end{equation}
\begin{equation}
\label{eqn:SAPT2+3}
E_{SAPT2+(3)} = E_{SAPT2+} + E_{ind,resp}^{(30)} + E_{exch-ind,resp}^{(30)}
+ E_{exch-disp}^{(30)} + E_{ind-disp}^{(30)} + E_{exch-ind-disp}^{(30)}
\end{equation}
A thorough analysis of the performance of these truncations of SAPT can be
found in a review by Hohenstein and Sherrill \cite{Hohenstein:2012:WIREs}.

The \PSIsapt\ module relies entirely on the density-fitting approximation
of the two-electron integrals. The factorization of the SAPT energy
expressions, as implemented in \PSIfour, assumes the use of density-fitted
two-electron integrals, therefore, the \PSIsapt\ module cannot be run with
exact integrals. In practice, we have found that the density-fitting
approximation introduces negligable errors into the SAPT energy and greatly
improves efficiency. 

\subsubsection{A First Example}

The following is the simplest possible input that will perform all
available SAPT computations (normally, you would pick one of these methods).
\begin{Snippet}

molecule water_dimer {
     0 1
     O  -1.551007  -0.114520   0.000000
     H  -1.934259   0.762503   0.000000
     H  -0.599677   0.040712   0.000000
     --
     0 1
     O   1.350625   0.111469   0.000000
     H   1.680398  -0.373741  -0.758561
     H   1.680398  -0.373741   0.758561

     units angstrom
     no_reorient
     symmetry c1
}

set globals {
    basis         aug-cc-pvdz
}

energy('sapt0')
energy('sapt2')
energy('sapt2+')
energy('sapt2+3')

\end{Snippet}
The \PSIsapt\ module uses the standard \PSIfour\ partitioning of the dimer
into monomers. Additionally, the \texttt{NO\_REORIENT} flag must be included
and the use of spatial symmetry disabled by setting the \texttt{SYMMETRY}
option to \texttt{C1}. A final note is that the \PSIsapt\ module is only 
capable of performing SAPT comuptations for interactions between closed-shell 
singlets. 

The example input shown above would not be used in practice.
To exploit the efficiency of the density-fitted SAPT implementation in
\PSIfour, the SCF computations should also be performed with density-fitted
(DF) integrals.
\begin{Snippet}

set globals {
    basis         aug-cc-pvdz
    df_basis_scf  aug-cc-pvdz-jkfit
    df_basis_sapt aug-cc-pvdz-ri
    guess         sad
    scf_type      df
}

set sapt {
    print         1
}

\end{Snippet}
These options will perform the SAPT computation with DF-HF and a 
superposition-of-atomic-densities guess. This is the preferred method of 
running the \PSIsapt\ module.

\subsubsection{SAPT0}

Generally speaking, SAPT0 should be applied to large systems or large data
sets. The performance of SAPT0 relies entirely on error cancellation, which
seems to be optimal with a truncated aug-cc-pVDZ basis, namely,
jun-cc-pVDZ (which we have referred to in previous work as
aug-cc-pVDZ$^{\prime}$).
The \PSIsapt\ module has been used to perform SAPT0 computations with over
200 atoms and 2800 basis functions; this code should be scalable to 4000
basis functions. Publications resulting from the use of the SAPT0 code 
should cite the following publications: \\[10pt]
E.~G. Hohenstein and C.~D. Sherrill, 
{\em J. Chem. Phys.}, {\bf 132}, 184111 (2010). \\[10pt]
E.~G. Hohenstein, R.~M. Parrish, C.~D. Sherrill, J.~M. Turney, and H.~F.
Schaefer, {\em J. Chem. Phys.}, {\bf 135}, 174107 (2011). 

\begin{flushleft}
{\bf Basic SAPT0 Keywords} \\[5pt]
\end{flushleft}
\begin{tabular*}{\textwidth}[tb]{p{0.3\textwidth}p{0.7\textwidth}}
         \texttt{BASIS} & The basis set used to describe the monomer molecular
orbitals. \\
\end{tabular*}
\begin{tabular*}{\textwidth}[tb]{p{0.3\textwidth}p{0.35\textwidth}p{0.35\textwidth}}
           & {\bf Type:} string &  {\bf Default:} none \\
         & & \\
\end{tabular*}
\begin{tabular*}{\textwidth}[tb]{p{0.3\textwidth}p{0.7\textwidth}}
         \texttt{DF\_BASIS\_SAPT} & The fitting basis to use for all
two-electron integrals in the SAPT computation. \PSIfour\ will attempt to
pick a reasonable fitting basis if one is not provided. \\
\end{tabular*}
\begin{tabular*}{\textwidth}[tb]{p{0.3\textwidth}p{0.35\textwidth}p{0.35\textwidth}}
           & {\bf Type:} string &  {\bf Default:} none \\
         & & \\
\end{tabular*}
\begin{tabular*}{\textwidth}[tb]{p{0.3\textwidth}p{0.7\textwidth}}
         \texttt{DF\_BASIS\_ELST} & Optionally, a different fitting basis
can be used for the $E_{elst}^{(10)}$ and $E_{exch}^{(10)}$ terms. This may
be important if heavier elements are involved. \\
\end{tabular*}
\begin{tabular*}{\textwidth}[tb]{p{0.3\textwidth}p{0.35\textwidth}p{0.35\textwidth}}
           & {\bf Type:} string &  {\bf Default:} none \\
         & & \\
\end{tabular*}
\begin{tabular*}{\textwidth}[tb]{p{0.3\textwidth}p{0.7\textwidth}}
         \texttt{FREEZE\_CORE} & Sets the number of core orbitals to freeze
in the evaluation of the $E_{disp}^{(20)}$ and $E_{exch-disp}^{(20)}$
terms. It is recommended to freeze core in all SAPT computations. \\

          & {\bf Possible Values:} TRUE, FALSE, SMALL, LARGE
\\
\end{tabular*}
\begin{tabular*}{\textwidth}[tb]{p{0.3\textwidth}p{0.35\textwidth}p{0.35\textwidth}}
           & {\bf Type:} string &  {\bf Default:} FALSE \\
         & & \\
\end{tabular*}
\begin{tabular*}{\textwidth}[tb]{p{0.3\textwidth}p{0.7\textwidth}}
         \texttt{D\_CONVERGE} & Convergence of the residual of the CPHF
coefficients needed for the $E_{ind,resp}^{(20)}$ term. \\
\end{tabular*}
\begin{tabular*}{\textwidth}[tb]{p{0.3\textwidth}p{0.35\textwidth}p{0.35\textwidth}}
           & {\bf Type:} double &  {\bf Default:} 1.0$\times 10^{-8}$\\
         & & \\
\end{tabular*}
\begin{tabular*}{\textwidth}[tb]{p{0.3\textwidth}p{0.7\textwidth}}
         \texttt{E\_CONVERGE} & Convergence of the energy change in the 
$E_{ind,resp}^{(20)}$ term during the solution of the CPHF equations (in
hartrees). \\
\end{tabular*}
\begin{tabular*}{\textwidth}[tb]{p{0.3\textwidth}p{0.35\textwidth}p{0.35\textwidth}}
           & {\bf Type:} double &  {\bf Default:} 1.0$\times 10^{-10}$\\
         & & \\
\end{tabular*}
\begin{tabular*}{\textwidth}[tb]{p{0.3\textwidth}p{0.7\textwidth}}
         \texttt{MAXITER} & The maximum number of CPHF iterations. \\
\end{tabular*}
\begin{tabular*}{\textwidth}[tb]{p{0.3\textwidth}p{0.35\textwidth}p{0.35\textwidth}}
           & {\bf Type:} integer &  {\bf Default:} 50 \\
         & & \\
\end{tabular*}
\begin{tabular*}{\textwidth}[tb]{p{0.3\textwidth}p{0.7\textwidth}}
         \texttt{PRINT} & The print level for the \PSIsapt\ module. If
\texttt{PRINT} is set to 0, only the header and final results are printed.
If \texttt{PRINT} is set to 1, some intermediate quantities are also
printed. For large SAPT computations, it is advisable to set \texttt{PRINT}
to 1 so the progess of the computation can be tracked. \\
\end{tabular*}
\begin{tabular*}{\textwidth}[tb]{p{0.3\textwidth}p{0.35\textwidth}p{0.35\textwidth}}
           & {\bf Type:} integer &  {\bf Default:} 1 \\
         & & \\
\end{tabular*}

\begin{flushleft}
{\bf Advanced SAPT0 Keywords} \\[5pt]
\end{flushleft}
\begin{tabular*}{\textwidth}[tb]{p{0.3\textwidth}p{0.7\textwidth}}
         \texttt{AIO\_CPHF} & Do disk I/O asynchronously during the
solution of the CPHF equations. This option may speed up the computation
slightly, however use of this option will cause \PSIfour\ to spawn an
additional thread. \\
\end{tabular*}
\begin{tabular*}{\textwidth}[tb]{p{0.3\textwidth}p{0.35\textwidth}p{0.35\textwidth}}
           & {\bf Type:} boolean &  {\bf Default:} FALSE \\
         & & \\
\end{tabular*}
\begin{tabular*}{\textwidth}[tb]{p{0.3\textwidth}p{0.7\textwidth}}
         \texttt{AIO\_DFINTS} & Do disk I/O asynchronously during the
formation of the DF integrals. This option may speed up the computation 
slightly, however use of this option will cause \PSIfour\ to spawn an 
additional thread. \\
\end{tabular*}
\begin{tabular*}{\textwidth}[tb]{p{0.3\textwidth}p{0.35\textwidth}p{0.35\textwidth}}
           & {\bf Type:} boolean &  {\bf Default:} FALSE \\
         & & \\
\end{tabular*}
\begin{tabular*}{\textwidth}[tb]{p{0.3\textwidth}p{0.7\textwidth}}
         \texttt{NO\_RESPONSE} & Don't solve the CPHF equations, evaluate
$E_{ind}^{(20)}$ and $E_{exch-ind}^{(20)}$ instead of their
response-including counterparts. Only turn on this option if you are not 
going to use the induction energy. \\
\end{tabular*}
\begin{tabular*}{\textwidth}[tb]{p{0.3\textwidth}p{0.35\textwidth}p{0.35\textwidth}}
           & {\bf Type:} boolean &  {\bf Default:} FALSE \\
         & & \\
\end{tabular*}
\begin{tabular*}{\textwidth}[tb]{p{0.3\textwidth}p{0.7\textwidth}}
         \texttt{INTEGRAL\_CUTOFF} & All three-index DF integrals and those
contributing to four-index integrals that fall below this Schwarz bound
will be neglected. The default is very conservative, however, there isn't much
to gain from loosening it. \\
\end{tabular*}
\begin{tabular*}{\textwidth}[tb]{p{0.3\textwidth}p{0.35\textwidth}p{0.35\textwidth}}
           & {\bf Type:} double &  {\bf Default:} 1.0$\times 10^{-12}$\\
         & & \\
\end{tabular*}
\begin{tabular*}{\textwidth}[tb]{p{0.3\textwidth}p{0.7\textwidth}}
         \texttt{DENOMINATOR\_DELTA} & The \PSIsapt\ module uses
approximate energy denominators for most of the $E_{disp}^{(20)}$ and
$E_{exch-disp}^{(20)}$ evaluation. This option controls the maximum
allowable error norm in the energy denominator tensor. \\
\end{tabular*}
\begin{tabular*}{\textwidth}[tb]{p{0.3\textwidth}p{0.35\textwidth}p{0.35\textwidth}}
           & {\bf Type:} double &  {\bf Default:} 1.0$\times 10^{-6}$\\
         & & \\
\end{tabular*}
\begin{tabular*}{\textwidth}[tb]{p{0.3\textwidth}p{0.7\textwidth}}
         \texttt{DENOMINATOR\_ALGORITHM} & Should the energy denominators
be approximated with Laplace transformations or a Cholesky decomposition?
We have found Laplace transformations to be slightly more efficient. \\

          & {\bf Possible Values:} LAPLACE, CHOLESKY
\\
\end{tabular*}
\begin{tabular*}{\textwidth}[tb]{p{0.3\textwidth}p{0.35\textwidth}p{0.35\textwidth}}
           & {\bf Type:} string &  {\bf Default:} LAPLACE \\
         & & \\
\end{tabular*}
\begin{tabular*}{\textwidth}[tb]{p{0.3\textwidth}p{0.7\textwidth}}
         \texttt{SCALE\_OS} & The \PSIsapt\ module will print a
decomposition of the $E_{disp}^{(20)}$ and $E_{exch-disp}^{(20)}$ terms
into same-spin and oppposite-spin contributions, in analogy to the SCS-MP2
method of Stefan Grimme. This option controls the scaling of the
oppposite-spin contributions. \\
\end{tabular*}
\begin{tabular*}{\textwidth}[tb]{p{0.3\textwidth}p{0.35\textwidth}p{0.35\textwidth}}
           & {\bf Type:} double &  {\bf Default:} $6.0/5.0$ \\
         & & \\
\end{tabular*}
\begin{tabular*}{\textwidth}[tb]{p{0.3\textwidth}p{0.7\textwidth}}
         \texttt{SCALE\_SS} & This option controls the scaling of the
same-spin contributions. \\
\end{tabular*}
\begin{tabular*}{\textwidth}[tb]{p{0.3\textwidth}p{0.35\textwidth}p{0.35\textwidth}}
           & {\bf Type:} double &  {\bf Default:} $1.0/3.0$ \\
         & & \\
\end{tabular*}
\begin{tabular*}{\textwidth}[tb]{p{0.3\textwidth}p{0.7\textwidth}}
         \texttt{DEBUG} & The \texttt{DEBUG} flag will print lots of
additional intermediate quantities that are not usually interesting. It 
will also do additional work (which is not optimized for large systems). Don't
turn on the \texttt{DEBUG} flag. \\
\end{tabular*}
\begin{tabular*}{\textwidth}[tb]{p{0.3\textwidth}p{0.35\textwidth}p{0.35\textwidth}}
           & {\bf Type:} integer &  {\bf Default:} 0 \\
         & & \\
\end{tabular*}

\subsubsection{Higher-order SAPT}

For smaller systems (up to the size of a nucleic acid base pair), more
accurate interaction energies can be obtained through higher-order SAPT
computations. The \PSIsapt\ module can perform density-fitted evaluations
of SAPT2, SAPT2+, SAPT2+(3), and SAPT2+3 energies. Publications resulting
from the use of the higher-order SAPT code should cite the following: \\[10pt]
E.~G. Hohenstein and C.~D. Sherrill, 
{\em J. Chem. Phys.}, {\bf 133}, 014101 (2010). 

A brief note on memory usage: the higher-order SAPT code assumes that
certain quantities can be held in core. This code requires sufficient
memory to hold $3o^2v^2+v^2N_{aux}$ arrays in core. With this requirement 
computations on the adenine-thymine complex can be performed with an
aug-cc-pVTZ basis in less than 64GB of memory.

In addition the SAPT0 keywords listed above, the following additional
keywords are relevant for Higher-order SAPT.
\begin{flushleft}
{\bf Additional SAPT Keywords for Higher-order SAPT} \\[5pt]
\end{flushleft}
\begin{tabular*}{\textwidth}[tb]{p{0.3\textwidth}p{0.7\textwidth}}
         \texttt{DO\_THIRD\_ORDER} & This option computes the following
corrections: $E_{ind}^{(30)}$, $E_{ind,resp}^{(30)}$, $E_{exch-ind}^{(30)}$,
$E_{exch-disp}^{(30)}$, $E_{ind-disp}^{(30)}$, and
$E_{exch-ind-disp}^{(30)}$. This is only relavent when the energy computed
is SAPT2+3; this option distinguishes SAPT2+(3) from SAPT2+3. A brief
technical note: the evaluation of $E_{disp}^{(30)}$ will be slower if this
option is activated. \\
\end{tabular*}
\begin{tabular*}{\textwidth}[tb]{p{0.3\textwidth}p{0.35\textwidth}p{0.35\textwidth}}
           & {\bf Type:} boolean &  {\bf Default:} FALSE \\
         & & \\
\end{tabular*}

\begin{flushleft}
{\bf Additional Advanced Keywords for Higher-order SAPT} \\[5pt]
\end{flushleft}
\begin{tabular*}{\textwidth}[tb]{p{0.3\textwidth}p{0.7\textwidth}}
         \texttt{INTEGRAL\_CUTOFF} & All three-index DF integrals and those
contributing to four-index integrals that fall below this Schwarz bound
will be neglected. The default is very conservative, however, there is
nothing to gain from loosening it in the case of higher-order SAPT. \\
\end{tabular*}
\begin{tabular*}{\textwidth}[tb]{p{0.3\textwidth}p{0.35\textwidth}p{0.35\textwidth}}
           & {\bf Type:} double &  {\bf Default:} 1.0$\times 10^{-12}$\\
         & & \\
\end{tabular*}
\begin{tabular*}{\textwidth}[tb]{p{0.3\textwidth}p{0.7\textwidth}}
         \texttt{SAPT\_MEM\_CHECK} & This flag can be used to disable
memory checking in the higher-order SAPT code; disabling it is ill advised. \\
\end{tabular*}
\begin{tabular*}{\textwidth}[tb]{p{0.3\textwidth}p{0.35\textwidth}p{0.35\textwidth}}
           & {\bf Type:} boolean &  {\bf Default:} TRUE \\
         & & \\
\end{tabular*}

\subsubsection{MP2 Natural Orbitals}

One of the unique features of the \PSIsapt\ module is it's ability to use
MP2 natural orbitals (NOs) to speed up the evaluation of the triples
contribution to disperison. By transforming to the MP2 NO basis, we can
throw away virtual orbitals that are expected to contribute little to the
dispersion energy. Speedups in excess of 50$\times$ are possible. In
practice, this approximation is very good and should always be applied.
Publications resulting from the use of MP2 NO-based approximations should 
cite the following: \\[10pt]
E.~G. Hohenstein and C.~D. Sherrill, 
{\em J. Chem. Phys.}, {\bf 133}, 104107 (2010).

\begin{flushleft}
{\bf Basic Keywords Controlling MP2 NO Approximations} \\[5pt]
\end{flushleft}
\begin{tabular*}{\textwidth}[tb]{p{0.3\textwidth}p{0.7\textwidth}}
         \texttt{NAT\_ORBS} & This flag activates MP2 NO approximations for
the triples correction to dispersion. \\
\end{tabular*}
\begin{tabular*}{\textwidth}[tb]{p{0.3\textwidth}p{0.35\textwidth}p{0.35\textwidth}}
           & {\bf Type:} boolean &  {\bf Default:} FALSE \\
         & & \\
\end{tabular*}
\begin{tabular*}{\textwidth}[tb]{p{0.3\textwidth}p{0.7\textwidth}}
         \texttt{OCC\_CUTOFF} & All virtual orbitals with smaller
occupation numbers (eigenvalues of the MP2 one-particle density matrix)
than this threshold will be discarded. See the above reference for a
discussion of acceptable values for \texttt{OCC\_CUTOFF}. \\
\end{tabular*}
\begin{tabular*}{\textwidth}[tb]{p{0.3\textwidth}p{0.35\textwidth}p{0.35\textwidth}}
           & {\bf Type:} double &  {\bf Default:} 1.0$\times 10^{-6}$\\
         & & \\
\end{tabular*}

\begin{flushleft}
{\bf Advanced Keywords Controlling MP2 NO Approximations} \\[5pt]
\end{flushleft}
\begin{tabular*}{\textwidth}[tb]{p{0.3\textwidth}p{0.7\textwidth}}
         \texttt{NAT\_ORBS\_T2} & This flag activates MP2 NO approximations for
the $v^4$ block of two-electron integrals in the evaluation of second-order
T2 amplitudes. At present, this approximation has not been rigorously
tested, however, initial results are promising for both accuracy and
computational savings. \\
\end{tabular*}
\begin{tabular*}{\textwidth}[tb]{p{0.3\textwidth}p{0.35\textwidth}p{0.35\textwidth}}
           & {\bf Type:} boolean &  {\bf Default:} FALSE \\
         & & \\
\end{tabular*}

\subsubsection{Charge-transfer in SAPT}

It is possible to obtain the stabilization energy of a complex due to
charge-transfer effects from a SAPT computation. The charge-transfer energy 
can be computed with the \PSIsapt\ module as described by Stone
and Misquitta \cite{Misquitta:2009:201}.

Charge-transfer energies can be obtained from the following calls to the
energy function.
\begin{Snippet}

energy('sapt0-ct')
energy('sapt2-ct')
energy('sapt2+-ct')
energy('sapt2+3-ct')

\end{Snippet}

A SAPT charge-transfer analysis will perform 5 HF computations: the dimer
in the dimer basis, monomer A in the dimer basis, monomer B in the dimer
basis, monomer A in the monomer A basis, and monomer B in the monomer B
basis. Next, it performs two SAPT computations, one in the dimer basis and
one in the monomer basis. Finally, it will print a summary of the
charge-transfer results:
\begin{Snippet}

  SAPT Charge Transfer Analysis
-----------------------------------------------------------------------------
  SAPT Induction (Dimer Basis)         -2.0970 mH       -1.3159 kcal mol^-1
  SAPT Induction (Monomer Basis)       -1.1396 mH       -0.7151 kcal mol^-1
  SAPT Charge Transfer                 -0.9574 mH       -0.6008 kcal mol^-1

\end{Snippet}
These results are for the water dimer geometry shown above computed with 
SAPT0/aug-cc-pVDZ. 

\subsubsection{Interpreting SAPT Results}

We will examine the results of a SAPT2+3/aug-cc-pVDZ computation on the
water dimer. This computation can be performed with the following 
input:
\begin{Snippet}

molecule water_dimer {
     0 1
     O  -1.551007  -0.114520   0.000000
     H  -1.934259   0.762503   0.000000
     H  -0.599677   0.040712   0.000000
     --
     0 1
     O   1.350625   0.111469   0.000000
     H   1.680398  -0.373741  -0.758561
     H   1.680398  -0.373741   0.758561
     units angstrom
     no_reorient
     symmetry c1
}

set globals {
    basis          aug-cc-pvdz
    guess          sad
    scf_type       df
}

set sapt {
    print          1
    nat_orbs       true
    freeze_core    true
    do_third_order true
}

energy('sapt2+3')

\end{Snippet}
To reiterate some of the options mentioned above: the \texttt{NAT\_ORBS} 
option will compute MP2 natural orbitals and use them in the evaluation of
the triples correction to dispersion, the \texttt{DO\_THIRD\_ORDER} option
will evaluate the additional third-order corrections that distinguish SAPT2+3
from SAPT2+(3). This SAPT2+3/aug-cc-pVDZ computation produces the following 
results:
\begin{Snippet}

  SAPT Results  
--------------------------------------------------------------------------
  Electrostatics            -13.06429805 mH      -8.19797114 kcal mol^-1
    Elst10,r                -13.37543274 mH      -8.39321111 kcal mol^-1
    Elst12,r                  0.04490253 mH       0.02817676 kcal mol^-1
    Elst13,r                  0.26623216 mH       0.16706321 kcal mol^-1

  Exchange                   13.41793548 mH       8.41988199 kcal mol^-1
    Exch10                   11.21823471 mH       7.03954885 kcal mol^-1
    Exch10(S^2)              11.13803867 mH       6.98922508 kcal mol^-1
    Exch11(S^2)               0.04558910 mH       0.02860760 kcal mol^-1
    Exch12(S^2)               2.15411167 mH       1.35172554 kcal mol^-1

  Induction                  -3.91333155 mH      -2.45565272 kcal mol^-1
    Ind20,r                  -4.57531220 mH      -2.87105187 kcal mol^-1
    Ind30,r                  -4.91715479 mH      -3.08556135 kcal mol^-1
    Ind22                    -0.83761074 mH      -0.52560870 kcal mol^-1
    Exch-Ind20,r              2.47828867 mH       1.55514969 kcal mol^-1
    Exch-Ind30,r              4.33916816 mH       2.72286924 kcal mol^-1
    Exch-Ind22                0.45370482 mH       0.28470409 kcal mol^-1
    delta HF,r (2)           -1.43240211 mH      -0.89884593 kcal mol^-1
    delta HF,r (3)           -0.85441547 mH      -0.53615383 kcal mol^-1

  Dispersion                 -3.62061213 mH      -2.27196851 kcal mol^-1
    Disp20                   -3.54292109 mH      -2.22321664 kcal mol^-1
    Disp30                    0.05959981 mH       0.03739945 kcal mol^-1
    Disp21                    0.11216179 mH       0.07038259 kcal mol^-1
    Disp22 (SDQ)             -0.17924270 mH      -0.11247650 kcal mol^-1
    Disp22 (T)               -0.47692549 mH      -0.29927528 kcal mol^-1
    Est. Disp22 (T)          -0.54385253 mH      -0.34127263 kcal mol^-1
    Exch-Disp20               0.64545652 mH       0.40503010 kcal mol^-1
    Exch-Disp30              -0.01823411 mH      -0.01144207 kcal mol^-1
    Ind-Disp30               -0.91816995 mH      -0.57616037 kcal mol^-1
    Exch-Ind-Disp30           0.76459013 mH       0.47978757 kcal mol^-1

  Total HF                   -5.68662366 mH      -3.56841037 kcal mol^-1
  Total SAPT0                -8.58408823 mH      -5.38659691 kcal mol^-1
  Total SAPT2                -6.72339084 mH      -4.21899163 kcal mol^-1
  Total SAPT2+               -7.26739725 mH      -4.56036082 kcal mol^-1
  Total SAPT2+(3)            -6.94156528 mH      -4.35589816 kcal mol^-1
  Total SAPT2+3              -7.11337921 mH      -4.46371303 kcal mol^-1

\end{Snippet}
At the bottom of this output are the total SAPT energies (defined above),
they are composed of subsets of the individual terms printed above. The
individual terms are grouped according to the component of the interaction
to which they contribute. The total component energies ({\em i.e.}
electrostatics, exchange, induction, and dispersion) represent what we
regard as the best estimate available at a given level of SAPT computed
from a subset of the terms of that grouping. The groupings shown above are
not unique and are certainly not rigorously defined. We regard the groupings 
used in \PSIfour\/ as a ``chemist's grouping'' as opposed to a more
mathematically based grouping, which would group all exchange terms ({\em
i.e.} $E_{exch-ind,resp}^{(20)}$, $E_{exch-disp}^{(20)}$, {\em etc.}) in
the exchange component. A final note is that both \texttt{Disp22(T)}
and \texttt{Est.Disp22(T)} results appear if MP2 natural orbitals are 
used to evaluate the triples correction to dispersion. The \texttt{Disp22(T)} 
result is the triples correction as computed in the truncated NO basis;  
\texttt{Est.Disp22(T)} is a scaled result that attempts to recover
the effect of the truncated virtual space. The \texttt{Est.Disp22(T)}
value used in the SAPT energy and dispersion component (see E.~G.
Hohenstein and C.~D. Sherrill, {\em J. Chem. Phys.}, {\bf 133}, 104107 
(2010) for details).
