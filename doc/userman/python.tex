\section{Python} \label{python}
\PSIfour\ has adopted the power of the Python scripting language
to drive calculations. In doing so, users are allowed to perform
complex calculations that would prove impossible with the old
IPV1 style of input.

\subsection{Preprocessor}
At first glace at \PSIfour 's new input style:
\begin{verbatim}
# cc-pVDZ H2O test single point
molecule {
o
h 1 0.957
h 1 0.957 2 104.5
}

set scf {
    basis cc-pVDZ
}

scf()
\end{verbatim}
does not look like Python. This is because to provide the user a simpler
format the input file is preprocessed into Python. Of course, if desired,
you are free to put any Python command into your input file. In some
cases you'll have to use Python (i.e. if you want to loop over a range of
geometrical parameters or basis sets).

The above input file does the exact same thing that the input file provided
in chapter \ref{tutorial}'s {\em Running a basic SCF calculation}. It provides
a simple single-point energy computation on water.

\begin{verbatim}
# cc-pVDZ H2O test single point
molecule {
o
h 1 roh
h 1 roh 2 ahoh

roh = 0.957
ahoh = 104.5
}

set scf {
    basis cc-pVDZ
}

scf()
\end{verbatim}

Both of these input files are automatically preprocessed to Python for you.

\subsection{Preprocessor Keywords}

\begin{description}
\item[molecule {\em name } { ... }]\mbox{}\\
Defines a molecule.

\end{description}
