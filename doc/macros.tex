
\newcommand{\PSI}{{\tt PSI}}
\newcommand{\PSItwo}{{\tt PSI2}}
\newcommand{\PSIthree}{{\tt PSI3}}
\newcommand{\PSIfour}{{\tt PSI4}}
\newcommand{\PSIversion}{4.0.0-alpha}
\newcommand{\PSImmmversion}{4.0.0-alpha}
\newcommand{\PSIemail}{psicode@users.sourceforge.net}

%
% Psi Modules
%
\def\module#1{{\tt #1}}
\newcommand{\libmints}{\module{libmints}}
\newcommand{\PSIdriver}{\module{psi4}}
\newcommand{\PSIinput}{\module{input}}
\newcommand{\PSIcints}{\module{cints}}
\newcommand{\PSIcderiv}{\module{cints --deriv1}}
\newcommand{\PSIdetci}{\module{detci}}
\newcommand{\PSIdetcas}{\module{detcas}}
\newcommand{\PSIdetcasman}{\module{detcasman}}
\newcommand{\PSIclag}{\module{clag}}
\newcommand{\PSIccenergy}{\module{ccenergy}}
\newcommand{\PSIccsort}{\module{ccsort}}
\newcommand{\PSIpsi}{\module{psi}}
\newcommand{\PSIcscf}{\module{cscf}}
\newcommand{\PSIoptking}{\module{optking}}
\newcommand{\PSItransqt}{\module{transqt}}
\newcommand{\PSInormco}{\module{normco}}
\newcommand{\PSIintder}{\module{intder95}}
\newcommand{\PSIgeom}{\module{geom}}
\newcommand{\PSIoeprop}{\module{oeprop}}
\newcommand{\PSIstable}{\module{stable}}

%
% Psi Library
%
\def\library#1{{\tt #1}}

%
% Psi and Unix Files
%
\def\FILE#1{{\tt file#1}}
\def\file#1{{\tt #1}}
\newcommand{\chkptfile}{\file{file32}}
\newcommand{\inputdat}{\file{input.dat}}
\newcommand{\outputdat}{\file{output.dat}}
\newcommand{\fconstdat}{\file{fconst.dat}}
\newcommand{\intcodat}{\file{intco.dat}}
\newcommand{\optaux}{\file{opt.aux}}
\newcommand{\basisdat}{\file{basis.dat}}
\newcommand{\pbasisdat}{\file{pbasis.dat}}
\newcommand{\geomdat}{\file{geom.dat}}
\newcommand{\geomout}{\file{geom.out}}

%
% Psi Keywords
%
\def\keyword#1{{\tt #1}}

%
% Psi C and Fortran Language elements
%
\def\celem#1{{\tt #1}}
\def\felem#1{{\tt #1}}

%
% Unix stuff
%
\def\unixid#1{{\em #1}} % names of groups and users
\def\shellvar#1{{\tt #1}}

%
% Nice output for function description
%
% Needs 4 arguments: function declaration,
%  description, arguments, and return values
%
% Call \initfuncdesc before using \funcdesc
%
\newcommand{\initfuncdesc}
{\newlength{\lcwidth}
\settowidth{\lcwidth}{Arguments:}
\newlength{\rcwidth}
\setlength{\rcwidth}{\linewidth}
\addtolength{\rcwidth}{-1.0\lcwidth}
\addtolength{\rcwidth}{-6.0\tabcolsep}
}

\newcommand{\funcdesc}[4]{
\celem{#1} \\
#2

\begin{tabular}{lp{\rcwidth}}
Arguments: & #3\\
Returns: & #4
\end{tabular}}

% This allows us to read in a sample file, needs the listings package
\newcommand \includesource[2] {
   \definecolor{XCodeComment}{rgb}{0.0, 0.4531, 0.0}
   \definecolor{XCodeKeyword}{rgb}{0.6641, 0.0508, 0.5664}
   \definecolor{XCodeString}{rgb}{0.7656, 0.1016, 0.0859}
   \lstset{
      language={#2}, basicstyle= \slshape \scriptsize \ttfamily,
      keywordstyle= \color{XCodeKeyword}, identifierstyle= \slshape \scriptsize,
      commentstyle= \color{XCodeComment},
      stringstyle=\ttfamily \color{XCodeString},
      showstringspaces=false,
      numbers=left,
      tabsize=4,
      title=\textit{#1},
      breaklines=true,
      showspaces=false,
      showtabs=false, frame=tb, framerule=0.5mm,
      aboveskip=5mm, belowskip=5mm,
   }
   \lstinputlisting{#1}
}

\newcommand \includeinput[1]{
   \lstdefinelanguage{psi}{
           morekeywords={geometry,label,wfn,refernce,subgroup,basis,docc,
                         print,do\_tei,ri\_basis,zmat,psi, reference,
                         df-mp2},
           sensitive=false,
           morecomment=[l]{\%}
   }
   \includesource{#1}{psi}
}
