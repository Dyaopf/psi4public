% Generated by Sphinx.
\def\sphinxdocclass{report}
\documentclass[letterpaper,10pt,english]{sphinxmanual}
\usepackage[utf8]{inputenc}
\DeclareUnicodeCharacter{00A0}{\nobreakspace}
\usepackage[T1]{fontenc}
\usepackage{babel}
\usepackage{times}
\usepackage[Bjarne]{fncychap}
\usepackage{longtable}
\usepackage{sphinx}
\usepackage{multirow}


\title{Psithon Documentation}
\date{February 13, 2012}
\release{4.01}
\author{Psi4 Project}
\newcommand{\sphinxlogo}{}
\renewcommand{\releasename}{Release}
\makeindex

\makeatletter
\def\PYG@reset{\let\PYG@it=\relax \let\PYG@bf=\relax%
    \let\PYG@ul=\relax \let\PYG@tc=\relax%
    \let\PYG@bc=\relax \let\PYG@ff=\relax}
\def\PYG@tok#1{\csname PYG@tok@#1\endcsname}
\def\PYG@toks#1+{\ifx\relax#1\empty\else%
    \PYG@tok{#1}\expandafter\PYG@toks\fi}
\def\PYG@do#1{\PYG@bc{\PYG@tc{\PYG@ul{%
    \PYG@it{\PYG@bf{\PYG@ff{#1}}}}}}}
\def\PYG#1#2{\PYG@reset\PYG@toks#1+\relax+\PYG@do{#2}}

\def\PYG@tok@gd{\def\PYG@tc##1{\textcolor[rgb]{0.63,0.00,0.00}{##1}}}
\def\PYG@tok@gu{\let\PYG@bf=\textbf\def\PYG@tc##1{\textcolor[rgb]{0.50,0.00,0.50}{##1}}}
\def\PYG@tok@gt{\def\PYG@tc##1{\textcolor[rgb]{0.00,0.25,0.82}{##1}}}
\def\PYG@tok@gs{\let\PYG@bf=\textbf}
\def\PYG@tok@gr{\def\PYG@tc##1{\textcolor[rgb]{1.00,0.00,0.00}{##1}}}
\def\PYG@tok@cm{\let\PYG@it=\textit\def\PYG@tc##1{\textcolor[rgb]{0.25,0.50,0.56}{##1}}}
\def\PYG@tok@vg{\def\PYG@tc##1{\textcolor[rgb]{0.73,0.38,0.84}{##1}}}
\def\PYG@tok@m{\def\PYG@tc##1{\textcolor[rgb]{0.13,0.50,0.31}{##1}}}
\def\PYG@tok@mh{\def\PYG@tc##1{\textcolor[rgb]{0.13,0.50,0.31}{##1}}}
\def\PYG@tok@cs{\def\PYG@tc##1{\textcolor[rgb]{0.25,0.50,0.56}{##1}}\def\PYG@bc##1{\colorbox[rgb]{1.00,0.94,0.94}{##1}}}
\def\PYG@tok@ge{\let\PYG@it=\textit}
\def\PYG@tok@vc{\def\PYG@tc##1{\textcolor[rgb]{0.73,0.38,0.84}{##1}}}
\def\PYG@tok@il{\def\PYG@tc##1{\textcolor[rgb]{0.13,0.50,0.31}{##1}}}
\def\PYG@tok@go{\def\PYG@tc##1{\textcolor[rgb]{0.19,0.19,0.19}{##1}}}
\def\PYG@tok@cp{\def\PYG@tc##1{\textcolor[rgb]{0.00,0.44,0.13}{##1}}}
\def\PYG@tok@gi{\def\PYG@tc##1{\textcolor[rgb]{0.00,0.63,0.00}{##1}}}
\def\PYG@tok@gh{\let\PYG@bf=\textbf\def\PYG@tc##1{\textcolor[rgb]{0.00,0.00,0.50}{##1}}}
\def\PYG@tok@ni{\let\PYG@bf=\textbf\def\PYG@tc##1{\textcolor[rgb]{0.84,0.33,0.22}{##1}}}
\def\PYG@tok@nl{\let\PYG@bf=\textbf\def\PYG@tc##1{\textcolor[rgb]{0.00,0.13,0.44}{##1}}}
\def\PYG@tok@nn{\let\PYG@bf=\textbf\def\PYG@tc##1{\textcolor[rgb]{0.05,0.52,0.71}{##1}}}
\def\PYG@tok@no{\def\PYG@tc##1{\textcolor[rgb]{0.38,0.68,0.84}{##1}}}
\def\PYG@tok@na{\def\PYG@tc##1{\textcolor[rgb]{0.25,0.44,0.63}{##1}}}
\def\PYG@tok@nb{\def\PYG@tc##1{\textcolor[rgb]{0.00,0.44,0.13}{##1}}}
\def\PYG@tok@nc{\let\PYG@bf=\textbf\def\PYG@tc##1{\textcolor[rgb]{0.05,0.52,0.71}{##1}}}
\def\PYG@tok@nd{\let\PYG@bf=\textbf\def\PYG@tc##1{\textcolor[rgb]{0.33,0.33,0.33}{##1}}}
\def\PYG@tok@ne{\def\PYG@tc##1{\textcolor[rgb]{0.00,0.44,0.13}{##1}}}
\def\PYG@tok@nf{\def\PYG@tc##1{\textcolor[rgb]{0.02,0.16,0.49}{##1}}}
\def\PYG@tok@si{\let\PYG@it=\textit\def\PYG@tc##1{\textcolor[rgb]{0.44,0.63,0.82}{##1}}}
\def\PYG@tok@s2{\def\PYG@tc##1{\textcolor[rgb]{0.25,0.44,0.63}{##1}}}
\def\PYG@tok@vi{\def\PYG@tc##1{\textcolor[rgb]{0.73,0.38,0.84}{##1}}}
\def\PYG@tok@nt{\let\PYG@bf=\textbf\def\PYG@tc##1{\textcolor[rgb]{0.02,0.16,0.45}{##1}}}
\def\PYG@tok@nv{\def\PYG@tc##1{\textcolor[rgb]{0.73,0.38,0.84}{##1}}}
\def\PYG@tok@s1{\def\PYG@tc##1{\textcolor[rgb]{0.25,0.44,0.63}{##1}}}
\def\PYG@tok@gp{\let\PYG@bf=\textbf\def\PYG@tc##1{\textcolor[rgb]{0.78,0.36,0.04}{##1}}}
\def\PYG@tok@sh{\def\PYG@tc##1{\textcolor[rgb]{0.25,0.44,0.63}{##1}}}
\def\PYG@tok@ow{\let\PYG@bf=\textbf\def\PYG@tc##1{\textcolor[rgb]{0.00,0.44,0.13}{##1}}}
\def\PYG@tok@sx{\def\PYG@tc##1{\textcolor[rgb]{0.78,0.36,0.04}{##1}}}
\def\PYG@tok@bp{\def\PYG@tc##1{\textcolor[rgb]{0.00,0.44,0.13}{##1}}}
\def\PYG@tok@c1{\let\PYG@it=\textit\def\PYG@tc##1{\textcolor[rgb]{0.25,0.50,0.56}{##1}}}
\def\PYG@tok@kc{\let\PYG@bf=\textbf\def\PYG@tc##1{\textcolor[rgb]{0.00,0.44,0.13}{##1}}}
\def\PYG@tok@c{\let\PYG@it=\textit\def\PYG@tc##1{\textcolor[rgb]{0.25,0.50,0.56}{##1}}}
\def\PYG@tok@mf{\def\PYG@tc##1{\textcolor[rgb]{0.13,0.50,0.31}{##1}}}
\def\PYG@tok@err{\def\PYG@bc##1{\fcolorbox[rgb]{1.00,0.00,0.00}{1,1,1}{##1}}}
\def\PYG@tok@kd{\let\PYG@bf=\textbf\def\PYG@tc##1{\textcolor[rgb]{0.00,0.44,0.13}{##1}}}
\def\PYG@tok@ss{\def\PYG@tc##1{\textcolor[rgb]{0.32,0.47,0.09}{##1}}}
\def\PYG@tok@sr{\def\PYG@tc##1{\textcolor[rgb]{0.14,0.33,0.53}{##1}}}
\def\PYG@tok@mo{\def\PYG@tc##1{\textcolor[rgb]{0.13,0.50,0.31}{##1}}}
\def\PYG@tok@mi{\def\PYG@tc##1{\textcolor[rgb]{0.13,0.50,0.31}{##1}}}
\def\PYG@tok@kn{\let\PYG@bf=\textbf\def\PYG@tc##1{\textcolor[rgb]{0.00,0.44,0.13}{##1}}}
\def\PYG@tok@o{\def\PYG@tc##1{\textcolor[rgb]{0.40,0.40,0.40}{##1}}}
\def\PYG@tok@kr{\let\PYG@bf=\textbf\def\PYG@tc##1{\textcolor[rgb]{0.00,0.44,0.13}{##1}}}
\def\PYG@tok@s{\def\PYG@tc##1{\textcolor[rgb]{0.25,0.44,0.63}{##1}}}
\def\PYG@tok@kp{\def\PYG@tc##1{\textcolor[rgb]{0.00,0.44,0.13}{##1}}}
\def\PYG@tok@w{\def\PYG@tc##1{\textcolor[rgb]{0.73,0.73,0.73}{##1}}}
\def\PYG@tok@kt{\def\PYG@tc##1{\textcolor[rgb]{0.56,0.13,0.00}{##1}}}
\def\PYG@tok@sc{\def\PYG@tc##1{\textcolor[rgb]{0.25,0.44,0.63}{##1}}}
\def\PYG@tok@sb{\def\PYG@tc##1{\textcolor[rgb]{0.25,0.44,0.63}{##1}}}
\def\PYG@tok@k{\let\PYG@bf=\textbf\def\PYG@tc##1{\textcolor[rgb]{0.00,0.44,0.13}{##1}}}
\def\PYG@tok@se{\let\PYG@bf=\textbf\def\PYG@tc##1{\textcolor[rgb]{0.25,0.44,0.63}{##1}}}
\def\PYG@tok@sd{\let\PYG@it=\textit\def\PYG@tc##1{\textcolor[rgb]{0.25,0.44,0.63}{##1}}}

\def\PYGZbs{\char`\\}
\def\PYGZus{\char`\_}
\def\PYGZob{\char`\{}
\def\PYGZcb{\char`\}}
\def\PYGZca{\char`\^}
\def\PYGZsh{\char`\#}
\def\PYGZpc{\char`\%}
\def\PYGZdl{\char`\$}
\def\PYGZti{\char`\~}
% for compatibility with earlier versions
\def\PYGZat{@}
\def\PYGZlb{[}
\def\PYGZrb{]}
\makeatother

\begin{document}

\maketitle
\tableofcontents
\phantomsection\label{index::doc}


\begin{notice}{note}{Note:}
Boolean arguments can be specified by \code{yes}, \code{on}, \code{true}, or \code{1}
for affirmative and \code{no}, \code{off}, \code{false}, or \code{0} for negative,
all irrespective of case.
\end{notice}
\phantomsection\label{index:module-wrappers2}\index{wrappers2 (module)}

\chapter{Database}
\label{index:welcome-to-psithon-s-documentation}\label{index:database}\index{database() (in module wrappers2)}

\begin{fulllineitems}
\phantomsection\label{index:wrappers2.database}\pysiglinewithargsret{\code{wrappers2.}\bfcode{database}}{\emph{name}, \emph{db\_name}\optional{, \emph{mode}, \emph{subset}, \emph{benchmark}, \emph{tabulate}, \emph{cp}, \emph{rlxd}, \emph{symm}, \emph{zpe}}}{}
Wrapper to access the molecule objects and reference energies of
popular chemical databases.
\begin{quote}\begin{description}
\item[{Returns}] \leavevmode
Mean absolute deviation of the database in kcal/mol

\end{description}\end{quote}

\textbf{Required Arguments}:
\begin{quote}\begin{description}
\item[{Parameters}] \leavevmode\begin{itemize}
\item {} 
\textbf{name} (\emph{string}) -- First argument, usually unlabeled.
Indicates the computational method to be applied to the database.
May be any valid argument to \code{energy()}.

\item {} 
\textbf{db\_name} (\emph{string}) -- Second argument, usually unlabeled.
Indicates the requested database name, matching the name of a python
file in \code{psi4/lib/databases}. Consult that directory for available
databases and literature citations.

\end{itemize}

\end{description}\end{quote}

\textbf{Optional Arguments}:
\begin{quote}\begin{description}
\item[{Parameters}] \leavevmode\begin{itemize}
\item {} 
\textbf{mode} (\emph{string}) -- \{\emph{`continuous'}, `sow', `reap'\}
Indicates whether the calculation required to complete the
database are to be run in one file (\code{'continuous'}) or are to be
farmed out in an embarrassingly parallel fashion
(\code{'sow'}/\code{'reap'}).  For the latter, run an initial job with
\code{'sow'} and follow instructions in its output file.

\item {} 
\textbf{cp} (\emph{bool}) -- \{\emph{`off'}, `on'\}
Indicates whether counterpoise correction is employed in computing
interaction energies.  Use this option and NOT the \code{cp()}
wrapper for BSSE correction in the \code{database()} wrapper.  Option
valid only for databases consisting of bimolecular complexes.

\item {} 
\textbf{rlxd} (\emph{bool}) -- \{\emph{`off'}, `on'\}
Indicates whether correction for the deformation energy is
employed in computing interaction energies.  Option valid only for
databases consisting of bimolecular complexes with non-frozen
monomers, e.g., HBC6

\item {} 
\textbf{symm} (\emph{bool}) -- \{\emph{`on'}, `off'\}
Indicates whether the native symmetry of the database molecules is
employed (\code{'on'}) or whether it is forced to c1 symmetry
(\code{'off'}). Some computational methods (e.g., SAPT) require no
symmetry, and this will be set by the database() wrapper.

\item {} 
\textbf{zpe} (\emph{bool}) -- \{\emph{`off'}, `on'\}
Indicates whether zero-point-energy corrections are appended to
single-point energy values. Option valid only for certain
thermochemical databases.  Disabled until Hessians ready.

\item {} 
\textbf{benchmark} (\emph{string}) -- \{\emph{`default'}, `S22A', etc.\}
Indicates whether a non-default set of reference energies, if
available, are employed for the calculation of error statistics.

\item {} 
\textbf{tabulate} (\emph{array of strings}) -- \{\emph{{[}{]}}, {[}'scf total energy', `natom'{]}, etc.\}
Indicates whether to form tables of variables other than the
primary requested energy.  Available for any PSI variable.

\item {} 
\textbf{subset} (\emph{string or array of strings}) -- 
Indicates a subset of the full database to run. This is a very
flexible option and can be used in three distinct ways, outlined
below. Note that two take a string and the last takes an array.
\begin{itemize}
\item {} \begin{description}
\item[{subset = \{`small', `large', `equilibrium'\}}] \leavevmode
Calls predefined subsets of the requested database, either
\code{'small'}, a few of the smallest database members,
\code{'large'}, the largest of the database members, or
\code{'equilibrium'}, the equilibrium geometries for a database
composed of dissociation curves.

\end{description}

\item {} \begin{description}
\item[{subset = \{`BzBz\_S', `FaOOFaON', `ArNe', etc.\}}] \leavevmode
For databases composed of dissociation curves, individual
curves can be called by name. Consult the database python
files for available molecular systems.  The choices for this
keyword are case sensitive and must match the database python file

\end{description}

\item {} \begin{description}
\item[{subset = \{{[}1,2,5{]}, {[}`1',`2',`5'{]}, {[}'BzMe-3.5', `MeMe-5.0'{]}, etc.\}}] \leavevmode
Specify a list of database members to run. Consult the
database python files for available molecular systems.  The
choices for this keyword are case sensitive and must match the
database python file

\end{description}

\end{itemize}


\end{itemize}

\item[{Todo }] \leavevmode
Make local options write to the generated input files in sow/reap mode.

\end{description}\end{quote}

\textbf{Example}:

\begin{Verbatim}[commandchars=\\\{\}]
\PYG{g+gp}{\textgreater{}\textgreater{}\textgreater{} }\PYG{c}{\PYGZsh{} [1] Two-stage SCF calculation on short, equilibrium, and long helium dimer}
\PYG{g+gp}{\textgreater{}\textgreater{}\textgreater{} }\PYG{n}{db}\PYG{p}{(}\PYG{l+s}{'}\PYG{l+s}{scf}\PYG{l+s}{'}\PYG{p}{,}\PYG{l+s}{'}\PYG{l+s}{RGC10}\PYG{l+s}{'}\PYG{p}{,}\PYG{n}{cast\PYGZus{}up}\PYG{o}{=}\PYG{l+s}{'}\PYG{l+s}{sto-3g}\PYG{l+s}{'}\PYG{p}{,}\PYG{n}{subset}\PYG{o}{=}\PYG{p}{[}\PYG{l+s}{'}\PYG{l+s}{HeHe-0.85}\PYG{l+s}{'}\PYG{p}{,}\PYG{l+s}{'}\PYG{l+s}{HeHe-1.0}\PYG{l+s}{'}\PYG{p}{,}\PYG{l+s}{'}\PYG{l+s}{HeHe-1.5}\PYG{l+s}{'}\PYG{p}{]}\PYG{p}{,} \PYG{n}{tabulate}\PYG{o}{=}\PYG{p}{[}\PYG{l+s}{'}\PYG{l+s}{scf total energy}\PYG{l+s}{'}\PYG{p}{,}\PYG{l+s}{'}\PYG{l+s}{natom}\PYG{l+s}{'}\PYG{p}{]}\PYG{p}{)}
\end{Verbatim}

\begin{Verbatim}[commandchars=\\\{\}]
\PYG{g+gp}{\textgreater{}\textgreater{}\textgreater{} }\PYG{c}{\PYGZsh{} [2] Counterpoise-corrected interaction energies for three complexes in S22}
\PYG{g+gp}{\textgreater{}\textgreater{}\textgreater{} }\PYG{c}{\PYGZsh{}     Error statistics computed wrt an old benchmark, S22A}
\PYG{g+gp}{\textgreater{}\textgreater{}\textgreater{} }\PYG{n}{database}\PYG{p}{(}\PYG{l+s}{'}\PYG{l+s}{dfmp2}\PYG{l+s}{'}\PYG{p}{,}\PYG{l+s}{'}\PYG{l+s}{S22}\PYG{l+s}{'}\PYG{p}{,}\PYG{n}{cp}\PYG{o}{=}\PYG{l+m+mi}{1}\PYG{p}{,}\PYG{n}{subset}\PYG{o}{=}\PYG{p}{[}\PYG{l+m+mi}{16}\PYG{p}{,}\PYG{l+m+mi}{17}\PYG{p}{,}\PYG{l+m+mi}{8}\PYG{p}{]}\PYG{p}{,}\PYG{n}{benchmark}\PYG{o}{=}\PYG{l+s}{'}\PYG{l+s}{S22A}\PYG{l+s}{'}\PYG{p}{)}
\end{Verbatim}

\begin{Verbatim}[commandchars=\\\{\}]
\PYG{g+gp}{\textgreater{}\textgreater{}\textgreater{} }\PYG{c}{\PYGZsh{} [3] SAPT0 on the neon dimer dissociation curve}
\PYG{g+gp}{\textgreater{}\textgreater{}\textgreater{} }\PYG{n}{db}\PYG{p}{(}\PYG{l+s}{'}\PYG{l+s}{sapt0}\PYG{l+s}{'}\PYG{p}{,}\PYG{n}{subset}\PYG{o}{=}\PYG{l+s}{'}\PYG{l+s}{NeNe}\PYG{l+s}{'}\PYG{p}{,}\PYG{n}{cp}\PYG{o}{=}\PYG{l+m+mi}{0}\PYG{p}{,}\PYG{n}{symm}\PYG{o}{=}\PYG{l+m+mi}{0}\PYG{p}{,}\PYG{n}{db\PYGZus{}name}\PYG{o}{=}\PYG{l+s}{'}\PYG{l+s}{RGC10}\PYG{l+s}{'}\PYG{p}{)}
\end{Verbatim}

\begin{Verbatim}[commandchars=\\\{\}]
\PYG{g+gp}{\textgreater{}\textgreater{}\textgreater{} }\PYG{c}{\PYGZsh{} [4] Optimize system 1 in database S22, producing tables of scf and mp2 energy}
\PYG{g+gp}{\textgreater{}\textgreater{}\textgreater{} }\PYG{n}{db}\PYG{p}{(}\PYG{l+s}{'}\PYG{l+s}{mp2}\PYG{l+s}{'}\PYG{p}{,}\PYG{l+s}{'}\PYG{l+s}{S22}\PYG{l+s}{'}\PYG{p}{,}\PYG{n}{db\PYGZus{}func}\PYG{o}{=}\PYG{n}{optimize}\PYG{p}{,}\PYG{n}{subset}\PYG{o}{=}\PYG{p}{[}\PYG{l+m+mi}{1}\PYG{p}{]}\PYG{p}{,} \PYG{n}{tabulate}\PYG{o}{=}\PYG{p}{[}\PYG{l+s}{'}\PYG{l+s}{mp2 total energy}\PYG{l+s}{'}\PYG{p}{,}\PYG{l+s}{'}\PYG{l+s}{current energy}\PYG{l+s}{'}\PYG{p}{]}\PYG{p}{)}
\end{Verbatim}

\begin{Verbatim}[commandchars=\\\{\}]
\PYG{g+gp}{\textgreater{}\textgreater{}\textgreater{} }\PYG{c}{\PYGZsh{} [5] CCSD on the smallest systems of HTBH, a hydrogen-transfer database}
\PYG{g+gp}{\textgreater{}\textgreater{}\textgreater{} }\PYG{n}{database}\PYG{p}{(}\PYG{l+s}{'}\PYG{l+s}{ccsd}\PYG{l+s}{'}\PYG{p}{,}\PYG{l+s}{'}\PYG{l+s}{HTBH}\PYG{l+s}{'}\PYG{p}{,}\PYG{n}{subset}\PYG{o}{=}\PYG{l+s}{'}\PYG{l+s}{small}\PYG{l+s}{'}\PYG{p}{,} \PYG{n}{tabulate}\PYG{o}{=}\PYG{p}{[}\PYG{l+s}{'}\PYG{l+s}{ccsd total energy}\PYG{l+s}{'}\PYG{p}{,} \PYG{l+s}{'}\PYG{l+s}{mp2 total energy}\PYG{l+s}{'}\PYG{p}{]}\PYG{p}{)}
\end{Verbatim}

\end{fulllineitems}



\chapter{Misc}
\label{index:misc}\index{corl\_xtpl\_helgaker\_2() (in module wrappers2)}

\begin{fulllineitems}
\phantomsection\label{index:wrappers2.corl_xtpl_helgaker_2}\pysiglinewithargsret{\code{wrappers2.}\bfcode{corl\_xtpl\_helgaker\_2}}{\emph{**largs}}{}
Extrapolation scheme for correlation energies with two adjacent zeta-level bases.
\begin{gather}
\begin{split}E_{corl}^X = E_{corl}^{\infty} + \beta X^{-3}\end{split}\notag\\\begin{split}\end{split}\notag
\end{gather}
\end{fulllineitems}



\chapter{Indices and tables}
\label{index:indices-and-tables}\begin{itemize}
\item {} 
\emph{genindex}

\item {} 
\emph{modindex}

\item {} 
\emph{search}

\end{itemize}


\renewcommand{\indexname}{Python Module Index}
\begin{theindex}
\def\bigletter#1{{\Large\sffamily#1}\nopagebreak\vspace{1mm}}
\bigletter{w}
\item {\texttt{wrappers2}}, \pageref{index:module-wrappers2}
\end{theindex}

\renewcommand{\indexname}{Index}
\printindex
\end{document}
